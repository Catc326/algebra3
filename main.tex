\documentclass[b5paper, openany]{ctexbook}
\usepackage{zref-abspage}
\usepackage[margin=2.5cm]{geometry}


\usepackage{pifont}
\usepackage[perpage,symbol*]{footmisc}
\DefineFNsymbols{circled}{{\ding{192}}{\ding{193}}{\ding{194}}
{\ding{195}}{\ding{196}}{\ding{197}}{\ding{198}}{\ding{199}}{\ding{200}}{\ding{201}}}
\setfnsymbol{circled}



\usepackage{amsmath,amsfonts,mathrsfs,amssymb}
\usepackage{graphicx}

\usepackage[font=bf,labelfont=bf,labelsep=quad]{caption}

\usepackage{tikz}


\usepackage{ntheorem}
\theoremseparator{\;}



\usepackage{blkarray}
\usepackage{bm}
\usepackage[colorlinks=true, linkcolor=black]{hyperref}



\theoremstyle{plain}
\theoremheaderfont{\normalfont\bfseries} 
\theorembodyfont{\normalfont}


\usepackage[framemethod=tikz]{mdframed}


\newtheorem{example}{\bf 例}[chapter]
\newenvironment{solution}{\noindent {\bf 解:}}{}
\newenvironment{analyze}{\noindent {\bf 分析:}}{}
\newenvironment{rmk}{ {\bf 注意:}}{}
\newenvironment{note}{ {\bf 评述:}}{}

\renewcommand{\emptyset}{\varnothing}

\renewcommand{\proofname}{\bf 证明:}
\newenvironment{proof}{{\noindent \bf 证明:}}{}%{\hfill $\square$\par}

\newcommand{\E}{\mathbb{E}}
\renewcommand{\Pr}{\mathbb{P}}
\newcommand{\EP}{\mathbb{E}^{\mathbb{P}}}
\newcommand{\EQ}{\mathbb{E}^{\mathbb{Q}}}
\newcommand{\dif}{\,{\rm d}}
\newcommand{\Var}{{\rm Var}}
\newcommand{\Cov}{{\rm Cov}}
\newcommand{\x}{\times}
\renewcommand{\Longrightarrow}{\;\Rightarrow\;}
\newcommand{\map}[3]{#1:\; #2\mapsto #3}



 \usepackage{tcolorbox}
 \tcbuselibrary{breakable}
 \tcbuselibrary{most}



\newtcolorbox{ex}[1][]
  {colback = white, colframe = cyan!75!black, fonttitle = \bfseries,
    colbacktitle = cyan!85!black, enhanced,
    attach boxed title to top center={yshift=-2mm},breakable, 
    title=练习, #1}

\newtcolorbox{blk}[1][]
  {colback = white, colframe = magenta!75!black, fonttitle = \bfseries,
    colbacktitle = magenta!85!black, enhanced,
    attach boxed title to top left={xshift=5mm, yshift=-2mm},breakable, 
    title=思考题, #1}

\newtcolorbox{thm}[2][]
  {colback = white, colframe = magenta!75!black, fonttitle = \bfseries,
    colbacktitle = magenta!85!black, enhanced,
    attach boxed title to top left={xshift=5mm, yshift=-2mm},breakable, 
    title=#2, #1}

% \newtcolorbox{note}[1][]
%   {colback = white, colframe = blue!75!black, fonttitle = \bfseries,
%     colbacktitle = blue!85!black, enhanced,
%     attach boxed title to top left={xshift=5mm, yshift=-2mm},breakable, 
%     title=说明, #1}



\setcounter{tocdepth}{1}

\setcounter{secnumdepth}{2}



% \ctexset {
% section = {
% 	name = {第,节},
%  	number = \chinese{section}},
% subsection = {
% 	name = {,、\hspace{-1em}},
% 	number = \chinese{subsection}
% },
% subsubsection = {
% 	name = {(,)\hspace{-1em}},
% 	number = \chinese{subsubsection},
% }
% }



\renewcommand{\contentsname}{目~~录}

\newcommand{\poly}{\polynomial[reciprocal]}
\newcommand{\Q}{\mathbb{Q}}
\newcommand{\R}{\mathbb{R}}
\newcommand{\N}{\mathbb{N}}
\newcommand{\Z}{\mathbb{Z}}



\usepackage{mathtools}

\setlength{\abovecaptionskip}{0.cm}
\setlength{\belowcaptionskip}{-0.cm}

\usetikzlibrary{decorations.pathmorphing, patterns}
\usetikzlibrary{calc, patterns, decorations.markings}
\usetikzlibrary{positioning, snakes, hobby}
\usetikzlibrary{decorations.pathreplacing}

\usepackage{lscape}

\usepackage{yhmath}
\usepackage{longdivision}
\usepackage{polynom}
\usepackage{polynomial}
\usepackage{cancel}

\renewcommand{\frac}{\dfrac}
\newcommand{\oc}{$^{\circ}{\rm C}$}
\newcommand{\blank}{\underline{\qquad}}

\usepackage{multicol}
\usepackage{cases}
% \usepackage{enumitem}
\usepackage{ulem}
\usepackage{enumerate}
\usepackage{polynomial}
\newcommand{\dd}{{\, \rm d}}

\newcommand{\Lim}[2]{\lim\limits_{#1\to #2}}
\newcommand{\LIM}[1]{\lim\limits_{#1\to \infty}}


\renewcommand{\ge}{\geqslant}
\renewcommand{\geq}{\geqslant}
\renewcommand{\le}{\leqslant}
\renewcommand{\leq}{\leqslant}


\usepackage{yhmath}
\usepackage{tkz-euclide}
\begin{document}



\title{\Huge\bfseries 北京四中高中教学讲义\\代数(第三册)}



\author{\Large 北京四中教学处~~编}
\date{\Large 1998年4月}

\maketitle




\frontmatter

\chapter{出版说明}

当前,中学教学改革已经深入到课程设置和教材改革领域。
我校数学教材的改革,以发展学生的数学思维为目标,以不改变现行教学大纲规定的教学闪容为前提,试图通过对知识结构及其展开方式的统盘考虑,实现整体优化。经多年反复探索、实验,编成了这套尝试融教材与教法、学法于一体的《北京四中高中教学讲义》。

这套讲义的产生可以上溯到 1982 年。从那时起,为了发展学生智能,提高数学素养,我校部分同志就开始对高中数学教学进行以教材改革为龙头,以学法教育为重点的“整体优化实验研究”。正是在这项研究的基础上,逐步形成了这套讲义编写的特色和风格。这就是:
\begin{enumerate}
    \item 为形成学生良好的认知结构,讲义的知识结构力求脉络分明,使学生能从整体上理解教材。
    \item 为了提高学生的数学素养,本讲义把数学思想的阐述放到了重要位置。数学思想既包含对数学知识点(概念、定理、公式、法则和方法)的本质认识,也包含对问题解决的数学基本观点。它是数学中的精华,对形成和发展学生的数学能力具有特别重要的意义。为此,讲义注重展现思维过程(概念、法则被概括的过程,教学关系被抽象的过程,解题思路探索形成的过程)。在过程中认识知识点的本质,在过程中总结思维规律,在过程中揭示数学思想的指导作用。力图使学生能深刻领悟教材。
    \item “再创造,再发现”在数学学习中对培养创造维能力至关重要,为引导学生积极参与“发现”,讲义在设计上做了某些尝试。
    \item 例题和习题的选配,力求典型、适量、成龙配套。习题分为 A 组(基本题)、 B 组(提高题)和 C 组(研究题)。教师可根据学生不同的学习水平适当选用。
    \item 教材是学生学习的依据。应有利于培养自学能力,本书注重启迪学法,并在书末附有全部习题的答案或提示,以供学习时参考。
 \end{enumerate}   

这套讲义在研究、试教和成书的过程中,始终得到了北京市和西城区教育部门有关领导的关怀和帮助,得到了北京师范大学数学系钟善基教授、曹才翰教授的热情指导,清华附中的瞿宁远老师也积极参与了我们的实验研究,并对这套教材做出了贡献,在此一并致以诚挚的谢意。

在编写过程中,北京四中数学组的教师们积极参加研讨,对他们的热情支持表示感谢。

这套讲义包括六册:高中代数第一、二、三册,三角、立体几何、解析几何各一册。

编写适应素质教育的教材,对我们来说是个尝试。由于水平所限,书中不当之处在所难免,诚恳希望专家、同行和同学们提出宝贵意见。

\begin{flushright}
    北京四中教学处\\
    1998年4月
\end{flushright}
\tableofcontents


\mainmatter

\setcounter{chapter}{7}

\chapter{行列式和线性方程组}

\section{二阶行列式和二元线性方程组}

\subsection{二阶行列式}
我们学过用消元法(代入消元或加减消元)解二元一次
方程组和三元一次方程组. 一次方程又叫\textbf{线性方程}, 一次方程
组又叫\textbf{线性方程组}. 本章学习线性方程组的行列式解法, 并对
解进行讨论。

一个二元线性方程组,当其中方程的个数与未知数的个
数相同时,它的一般形式可以写成
\[({\rm I})\begin{cases}
    a_1x+b_1y=c_1, & (1)\\
    a_2x+b_2y=c_2, & (2)
\end{cases}\]
其中$x,y$ 是未知数, $a_1,a_2,b_1,b_2$是未知数的系数, $c_1,c_2$是常
数项(在一般形式中, 我们把常数项写在方程的右边). 

若$x=x_1$, $y=y_1$适合方程组(I), 那么这一对有序实数
叫做方程组(I)的\textbf{一个解}, 记为
\[\begin{cases}
  x=x_1\\
y=y_1  
\end{cases}\]
或简记作$(x_1,y_1)$. 该方程组的所有的解构成的集合称为方程
组的\textbf{解集}. 

用加减消元法解这个方程组:
$(1)\x b_2-(2)\x b_1$,得
\begin{equation}
  (a_1b_2-a_2b_1)x=c_1b_2-c_2b_1\tag{3}
\end{equation}
$(2)\x a_1-(1)\x a_2$,得
\begin{equation}
  (a_1b_2-a_2b_1)y=a_1c_2-a_2c_1 \tag{4}
\end{equation}
应注意:方程组(3)(4)是方程组(I)的结果.因此,(I)的解一定适合(3)(4). 当$a_1b_2-a_2b_1\ne  0$时,方程组(3)(4)有唯一解.
\begin{equation}
\begin{cases}
  x=\frac{c_1b_2-c_2b_1}{a_1b_2-a_2b_1}\\
  y=\frac{a_1c_2-a_2c_1}{a_1b_2-a_2b_1}  
\end{cases}
\tag{5}
\end{equation}
可以验证:此时,(5)也适合方程组(I),所以当$a_1b_2-a_2b_1\ne  0$时,方程组(I)有唯一解(5).为了便于记忆(5),我们对它进行如下分析.

在公式(5)中,两个分母都是$a_1b_2-a_2b_1$,它只含有未知数的系数. 把未知数的系数按照它们在方程组中原来的位置排列成正方形(图8.1).
\begin{figure}[htp]
  \centering
\begin{tikzpicture}
  \node(A) at (0,0){$a_1$};
  \node(B) at (2,0){$b_1$};
  \node(C) at (0,-2){$a_2$};
  \node(D) at (2,-2){$b_2$};
  \draw[very thick](A)--(D);
  \draw[dashed, very thick](B)--(C);
\end{tikzpicture}
  \caption{}
\end{figure}

可以看出:$a_1b_2-a_2b_1$恰是这样两项之和:第一项为正方形中实线表示的对角线(叫做\textbf{主对角线})上两数的乘积添上正号;
第二项为虚线表示的对角线(叫做\textbf{副对角线})上两数的乘积添上负号.由此,我们引进符号
\begin{equation}
  \begin{vmatrix}
    a_1&b_1\\
    a_2&b_2\\
  \end{vmatrix}\tag{6}
\end{equation}
并且规定它表示
\begin{equation}
  a_1b_2-a_2b_1\tag{7}
\end{equation} 
这时,符号(6)叫做\textbf{二阶行列式},(7)叫做(6)的\textbf{展开式}. $a_1,a_2,b_1,b_2$叫做行列式(6)的\textbf{元素}.这四个元素排成二行二列(横排叫行,竖排叫列).例如,$a_2$处在第二行第一列上;$b_1$处在第一行第二列上. 把(6)写成(7)叫做\textbf{按对角线法则}展开.

\begin{example}
  按对角线法则展开下列行列式,并化简:
\begin{multicols}{3}
\begin{enumerate}[(1)]
  \item $\begin{vmatrix}
    10&-9\\ -3&7
  \end{vmatrix}$
  \item $\begin{vmatrix}
    m+1& m+2\\ m & m+1
  \end{vmatrix}$
  \item $\begin{vmatrix}
    \sin x& \cos x\\ \cos x& -\sin x
  \end{vmatrix}$
\end{enumerate}
\end{multicols}
\end{example}

\begin{solution}
  \begin{enumerate}[(1)]
  \item $\begin{vmatrix}
    10&-9\\ -3&7
  \end{vmatrix}=10\x 7-(-3)(-9)=43$
  \item $\begin{vmatrix}
    m+1& m+2\\ m & m+1
  \end{vmatrix}=(m+1)^2-m(m+2)=1$
  \item $\begin{vmatrix}
    \sin x& \cos x\\ \cos x& -\sin x
  \end{vmatrix}=\sin^2 x-\cos^2 x=-1$
  \end{enumerate}
\end{solution}

\begin{thm}{问1}
  按对角线法则展开、化简下列行列式:
\begin{multicols}{2}
\begin{enumerate}[(1)]
  \item $\begin{vmatrix}
    5&6\\3&7
  \end{vmatrix}$
  \item $\begin{vmatrix}
   -3&-7\\-2&1
  \end{vmatrix}$
  \item $\begin{vmatrix}
    6a-b&2b\\ 3a&b
  \end{vmatrix}$
  \item $\begin{vmatrix}
    \log_a x & \log_a x\\m&n
  \end{vmatrix}$
\end{enumerate}
\end{multicols}
\end{thm}

\subsection{二元线性方程组的解的行列式表示法}
利用二阶行列式,同样可以把公式(5)中的两个分子也写
成行列式的形式,即
\[c_1b_2-c_2b_1=\begin{vmatrix}
  c_1&b_1\\c_2&b_2
\end{vmatrix},\qquad a_1c_2-a_2c_1=\begin{vmatrix}
  a_1&c_1\\ a_2&c_2
\end{vmatrix}\]
这样,当$a_1b_2-a_2b_1\ne 0$时,二元线性方程组(I)的解可以写成
\begin{equation}
  \begin{cases}
    x=\frac{\begin{vmatrix}
      c_1&b_1\\c_2&b_2
    \end{vmatrix}}{\begin{vmatrix}
      a_1&b_1\\a_2&b_2
    \end{vmatrix}}\\
    y=\frac{\begin{vmatrix}
      a_1&c_1\\a_2&c_2
    \end{vmatrix}}{\begin{vmatrix}
      a_1&b_1\\a_2&b_2
    \end{vmatrix}}
  \end{cases}\tag{8}
\end{equation}

为简便,通常用$D$、$D_x$、$D_y$分别表示(8)式中作为分母与分子的行列式\footnote{$D$是英语单词Determinant(行列式)的词头字母,通常用来表示行列式.}:
\[D=\begin{vmatrix}
  a_1&b_1\\a_2&b_2
\end{vmatrix},\qquad D_x=\begin{vmatrix}
  c_1&b_1\\c_2&b_2
\end{vmatrix},\qquad D_y=\begin{vmatrix}
  a_1&c_1\\a_2&c_2
\end{vmatrix}\]
行列式$D$是由方程组中未知数$x$、$y$的系数组成的,叫做这个方程组的\textbf{系数行列式}. $D$中第一列的元素$a_1$、$a_2$(即$x$的系数)分别用方程组的常数项$c_1$、$c_2$替换,就得到行列式$D_x$;$D$中第二列的元素$b_1$、$b_2$(即$y$的系数)分别用常数项$c_1$、$c_2$替换,就得到行列式$D_y$.

于是,当$D\ne 0$时,二元线性方程组(I)的唯一解可以写成
\begin{equation}
  \begin{cases}
    x=\frac{D_x}{D}\\[1ex]
   y=\frac{D_y}{D}
  \end{cases}\tag{9}
\end{equation}
或$\left(\frac{D_x}{D},\frac{D_y}{D}\right)$,方程组的解集就是$\left\{\left(\frac{D_x}{D},\frac{D_y}{D}\right)\right\}$.

由此可见,用行列式表示二元线性方程组的解的确简单好记.

\begin{example}
  利用行列式法解方程组
$\begin{cases}
  11x-2y+5=0,\\3x+7y+24=0.
\end{cases}  $
\end{example}

\begin{solution}
  解先把方程组写成一般形式
\[\begin{cases}
  11x-2y=-5 \\ 3x+7y=-24
\end{cases}\]
(这一步很必要,否则$c_1$、$c_2$的符号容易搞错)
由
\[\begin{split}
  D&=\begin{vmatrix}
    11&-2\\3&7
  \end{vmatrix}=77-3(-2)=83\ne 0\\
  D_x&=\begin{vmatrix}
    -5&-2\\-24&7
  \end{vmatrix}=-35-48=-83\\
  D_y&=\begin{vmatrix}
    11&-5\\3&-24
  \end{vmatrix}=-264-(-15)=-249\\
\end{split}\]
得
\[x=\frac{D_x}{D}=\frac{-83}{83}=-1,\qquad y=\frac{D_y}{D}=\frac{-249}{83}=-3\]
$\therefore\quad $方程组的解集是$\{(-1,-3)\}$.
\end{solution}

\begin{thm}{问2}
  用行列式法解方程组
\begin{multicols}{2}
\begin{enumerate}[(1)]
  \item $\begin{cases}
    7x-8y=10\\6x-7y=11
  \end{cases}$
  \item $\begin{cases}
    14x-6y+1=0\\
    3x+7y-6=0
  \end{cases}$
\end{enumerate}
\end{multicols}
\end{thm}

\subsection{二元线性方程组的解的讨论}
上面我们看到,当$D\ne 0$时,方程组(I)有唯一解(9),它仅仅是由方程组中未知数的系数和常数项表示的. 这就向我们暗示了一个问题:不经过解方程组,而仅仅根据方程组的系数和常数项能否确定方程组是否有解?在有解的情况下有多少解呢?(类似的问题我们在一元二次方程解的讨论中已经见过,所以在此提出这个问题并非偶然)

下面,我们分情况进行讨论\footnote{这里,我们是对形如$\begin{cases}
  a_1x+b_1y=c_1\\
  a_2x+b_2y=c_2
\end{cases}$的方程组进行讨论,对其中的系数不加任何限制。}. 为此,先把方程组(3)(4)写成
\[({\rm II})\begin{cases}
  D\cdot x=D_x & (10)\\
  D\cdot x=D_y &(11)
\end{cases}\]
由于方程组(I)的解都是方程组(II)的解,所以我们可以从(II)出发讨论方程组(I)的解的情况.
\begin{enumerate}
  \item 当$D\ne 0$时,由(II)可知,方程组(I)有唯一解;
\item 当$D=0$时,由(II)可知需考虑$D_x$、$D_y$:
\begin{enumerate}[(1)]
  \item 若$D_x$、$D_y$中至少有一个不为零.则(II)无解,也就是方程组(I)无解;
  \item 若$D_x=D_y=0$,例如$\begin{cases}
    2x+3y=4\\ 4x+6y=8
  \end{cases}$,我们再分两种情况讨论:
  \begin{enumerate}[(i)]
    \item $a_1$、$a_2$、$b_1$、$b_2$不全为零时,不失一般性,设$b_1\ne 0$,则由
\begin{equation}
  \begin{cases}
    D=a_1b_2-a_2b_1 =0\\
    D_x=c_1b_2-c_2b_1=0
  \end{cases}\Longrightarrow a_2=\frac{a_1b_2}{b_1},\quad c_2=\frac{c_1b_2}{b_1}  \tag{12}
\end{equation}    
把(12)代入(2),有
\[\frac{a_1b_2}{b_1}x+b_2y=\frac{c_1b_2}{b_1}\]
即
\begin{equation}
  \frac{b_2}{b_1}(a_1x+b_1y)=\frac{b_2}{b_1}c_1 \tag{13}
\end{equation}
这说明方程(1)的解必定适合方程(2). 因为方程(1)有无穷多解,所以方程组(I)有无穷多解.
\item $a_1$、$a_2$、$b_1$、$b_2$全为零时,这时若不全为零,方程组(I)无解;若$c_1$、$c_2$也全为零,则的任意一组值都同时适合方程(1)和(2),因此方程组(I)有无穷多解.
  \end{enumerate}
\end{enumerate}
\end{enumerate}

综合上述,结论是

\begin{thm}{定理}
  二元线性方程组$\begin{cases}
    a_1x+b_1y=c_1\\
    a_2x+b_2y=c_2
  \end{cases}$,也就是$\begin{cases}
    D\cdot x=D_x\\
    D\cdot y=D_y
  \end{cases}$
\begin{enumerate}
  \item 当$D\ne 0$时,有唯一解$\left(\frac{D_x}{D},\frac{D_y}{D}\right)$;
  \item 当$D=0$时:
\begin{enumerate}[(1)]
  \item 若$D_x$、$D_y$不全为零时,无解,
  \item 若$D_x$、$D_y$全为零时,
\begin{enumerate}[(i)]
  \item $a_1$、$a_2$、$b_1$、$b_2$不全为零,有无穷多解;
  \item $a_1$、$a_2$、$b_1$、$b_2$全为零,
\begin{itemize}
  \item 当$c_1,c_2$不全为零,无解;
  \item 当$c_1,c_2$全为零,无穷多解.
\end{itemize}
\end{enumerate}
\end{enumerate}
\end{enumerate}
\end{thm}

\begin{example}
  解关于$x,y$的线性方程组,并讨论:
\[\begin{cases}
  mx+y=m+1\\
  x+my=2m
\end{cases}\]
\end{example}

\begin{solution}
\[\begin{split}
D&=\begin{vmatrix}
  m&1\\1&m
\end{vmatrix}=m^2-1=(m+1)(m-1)\\
D_x&=\begin{vmatrix}
  m+1&1\\2m&m
\end{vmatrix}=m(m+1)-2m=(m-1)\\
D&=\begin{vmatrix}
  m&m+1\\1&2m
\end{vmatrix}=2m^2-(m+1)=(2m+1)(m-1)\\
\end{split}\]
\begin{enumerate}
  \item 当$D\ne 0$时,即$m\ne\pm 1$时,方程组有唯一解,其解集是$\left\{\left(\frac{m}{m+1},\frac{2m+1}{m+1}\right)\right\}$;
  \item 当$D=0$,即$m=\pm 1$时,
\begin{enumerate}[(1)]
  \item 当$m=-1$,这时$D=0$, $D_x=2\ne 0$,方程组无解,即解集为$\emptyset$;
  \item 当$m=1$时,$D_x=0$, $D_y=0$, $a_1\ne 0$,方程组有无穷多解. 这时方程组是
\[\begin{cases}
  x+y=2\\
  x+y=2  
\end{cases}\]
  若令$x=t$($t$为任意常数),则$y=2-t$,方程组的解集可以写成$\{(t,2-t)\}$($t$为任意常数).
\end{enumerate}
\end{enumerate}
\end{solution}

\section*{习题一}
\begin{center}
  \bfseries A
\end{center}

\begin{enumerate}
  \item 用对角线法则,展开行列式并化简:
\begin{multicols}{2}
\begin{enumerate}[(1)]
  \item $\begin{vmatrix}
    x-1& x^3\\ 1& x^2+x+1
  \end{vmatrix}$
  \item $\begin{vmatrix}
    \sin x-\sin y& \cos x+\cos y\\
    \cos x-\cos y& \sin x+\sin y
  \end{vmatrix}$
  \item $\begin{vmatrix}
    1-\sqrt{2} &2-\sqrt{3}\\
    2+\sqrt{3} & 1+\sqrt{2}
  \end{vmatrix}$
  \item $\begin{vmatrix}
    \log_a b& 1\\2&\log_b a
  \end{vmatrix}$
  \item $\begin{vmatrix}
    a-b&a^2-ab+b^2\\ a+b& a^2+ab+b^2
  \end{vmatrix}$
  \item $\begin{vmatrix}
    e^{x+y} &e^x-1\\
    e^x+1 & e^{x-y}
  \end{vmatrix}$
\end{enumerate}
\end{multicols}
  \item 利用行列式解下列方程组:
\begin{multicols}{2}
\begin{enumerate}[(1)]
  \item $\begin{cases}
    13x-7y-10=0\\
    19x+15y-2=0
  \end{cases}$
  \item $\begin{cases}
    \frac{7}{s}+\frac{9}{t}=3\\[1ex]
    \frac{17}{s}+\frac{7}{t}=5
  \end{cases}$
\end{enumerate}
\end{multicols}
  \item 利用行列式解下列关于$x,y$的方程组:
\begin{multicols}{2}
\begin{enumerate}[(1)]
  \item $\begin{cases}
    mx+y=2m+1\\x-my=2-m
  \end{cases}$
  \item $\begin{cases}
   x\cos A-y\sin A=\cos B\\
   x\sin A +y\cos A=\sin B
  \end{cases}$
\end{enumerate}
\end{multicols}
\end{enumerate}

\begin{center}
  \bfseries B
\end{center}

\begin{enumerate}\setcounter{enumi}{3}
  \item 不解方程组,判定下列方程组有唯一解、无解、还是有无穷多解:
\begin{multicols}{2}
\begin{enumerate}[(1)]
  \item $\begin{cases}
    2x+3y=7\\ 5x-2y=1
  \end{cases}$
  \item $\begin{cases}
    6x+9y=7\\ 4x+6y=2
  \end{cases}$
  \item $\begin{cases}
    4x-3y=5\\ 8x+6y=22
  \end{cases}$
  \item $\begin{cases}
    5x-15y=10\\ 3x-9y=6
  \end{cases}$
\end{enumerate}
\end{multicols}
  \item 判断$m$取什么值时,下列关于$x,y$的方程组有唯一解:
\begin{multicols}{2}
\begin{enumerate}[(1)]
  \item $\begin{cases}
    (m^2-1)x-(m+1)y=m+1\\
    m^2x-(m+1)y=m-1
  \end{cases}$
  \item $\begin{cases}
    x-(m^2-5)y=-1\\
    (m+1)x- (m+1)^2 y=1
  \end{cases}$
\end{enumerate}
\end{multicols}
\item 解下列关于$x,y$的方程组,并进行讨论:
\begin{multicols}{2}
\begin{enumerate}[(1)]
  \item $\begin{cases}
    x+(m-1)y=1\\ (m-1)x+y=2
  \end{cases}$
  \item $\begin{cases}
    4x+my=m\\ mx+y=1
  \end{cases}$
\end{enumerate}
\end{multicols}
\end{enumerate}

\section{关于三元线性方程组的猜想}
二阶行列式的引入使二元线性方程组的求解和讨论大为简化.成功的喜悦和好奇心驱使我们进一步考虑能否用类似的办法去简化三元线性方程组的求解和讨论呢?具体讲,是否可以引入“三行三列”的行列式,由三元线性方程组
\begin{equation}
  \begin{cases}
    a_1x+b_1y+c_1z=d_1\\
    a_2x+b_2y+c_2z=d_2\\
    a_3x+b_3y+c_3z=d_3\\
  \end{cases}\mathop{\Longrightarrow}^{\text{推出}}\begin{cases}
    D\cdot x =D_x\\
    D\cdot y =D_y\\
    D\cdot z =D_z\\
  \end{cases}\tag{*}
\end{equation}
其中
\[D=\begin{vmatrix}
  a_1&b_1&c_1\\
  a_2&b_2&c_2\\
  a_3&b_3&c_3\\
\end{vmatrix},\quad D_x=\begin{vmatrix}
  d_1&b_1&c_1\\
  d_2&b_2&c_2\\
  d_3&b_3&c_3\\
\end{vmatrix}\]
\[D_y=\begin{vmatrix}
  a_1&d_1&c_1\\
  a_2&d_2&c_2\\
  a_3&d_3&c_3\\
\end{vmatrix},\quad D_z=\begin{vmatrix}
  a_1&b_1&d_1\\
  a_2&b_2&d_2\\
  a_3&b_3&d_3\\
\end{vmatrix}\]
当然必须恰当地规定$D$的展开式是什么.如果这一“猜想”得以实现,三元线性方程组的问题就基本解决了,就有可能进一步去考虑四元、五元以至$n$元线性方程组的问题了.

\begin{thm}
  {问1} 你能够把这一猜想付诸实现吗?这里关键是规定$D$的展开式(类比“二元”的线索去发现)和实现(*)式的改写.
\end{thm}

\section{三阶行列式及其性质}
引进符号
\begin{equation}
  \begin{vmatrix}
    a_1&b_1&c_1\\
    a_2&b_2&c_2\\
    a_3&b_3&c_3\\
  \end{vmatrix}\tag{1}
\end{equation}
并规定它表示
\begin{equation}
  a_1b_2c_3+a_2b_3c_1+a_3b_1c_2-a_3b_2c_1-a_2b_1c_3-a_1b_3c_2 \tag{2}
\end{equation}

(1)称为\textbf{三阶行列式}.它有三行三列,共有9个元素.(2)称为(1)的\textbf{展开式},共有六项.

三阶行列式按对角线法则展开,如图8.2.
\begin{figure}[htp]
  \centering
\begin{tikzpicture}[very thick, scale=1.2]
\node (A1) at (-1,1){$a_1$};
\node (B1) at (0,1){$b_1$};
\node (C1) at (1,1){$c_1$};
\node (A2) at (-1,0){$a_2$};
\node (B2) at (0,0){$b_2$};
\node (C2) at (1,0){$c_2$};
\node (A3) at (-1,-1){$a_3$};
\node (B3) at (0,-1){$b_3$};
\node (C3) at (1,-1){$c_3$};
\draw (A1)--(B2)--(C3);
\draw (A2)--(B3)--(1-.5,-2+.5)[bend left=-90]to (2,0)-- (C1);
\draw (B1)--(C2)--(1+.5,0-.5)[bend left=90]to (0,-2)--(A3);
\draw (A1)--(B2)--(C3);

\draw [dashed](A3)--(B2)--(C1);
\draw [dashed](C2)--(B3)--(-.5,-1.5)[bend left=90]to (-2,0)-- (A1);
\draw[dashed] (B1)--(A2)--(-1-.5,0-.5)[bend left=-90]to (0,-2)--(C3);

\node at (-2,-2){$(-)$};
\node at (2,-2){$(+)$};

\end{tikzpicture}
  \caption{}
\end{figure}


图中实线上三个元素的积,添上正号;虚线上三个元素的积,添上负号. 容易看出,三阶行列式的值就是这六项的和.

\begin{example}
  用对角线法则计算行列式$\begin{vmatrix}
    3&-2&1\\ -2&1&3\\2&0&-2
  \end{vmatrix}$
\end{example}

\begin{solution}
\[\begin{split}
  \begin{vmatrix}
    3&-2&1\\ -2&1&3\\2&0&-2
  \end{vmatrix}&=3\x1\x(-2)+(-2)\x0\x1+2\x(-2)\x3\\
  &\qquad 
  -2\x1\x1-(-2)\x(-2)\x(-2)-3\x0\x3\\
 & =-6+0-12-2+8-0\\
  &=-12
\end{split}\]
\end{solution}

\section*{习题二}
\begin{center}
  \bfseries A
\end{center}
\begin{enumerate}
  \item 用对角线法则计算:
\begin{multicols}{2}
\begin{enumerate}[(1)]
  \item $\begin{vmatrix}
    1&5&7\\2&0&-4\\-3&1&6
  \end{vmatrix}$
  \item $\begin{vmatrix}
    2&-3&1\\4&-1&7 \\-1&5&2
  \end{vmatrix}$
\end{enumerate}
\end{multicols}
  \item 用对角线法则展开下列行列式,并化简:
\begin{multicols}{2}
\begin{enumerate}[(1)]
  \item $\begin{vmatrix}
    0&a&b\\a&0&c\\b&c&0
  \end{vmatrix}$
  \item $\begin{vmatrix}
    x&y&z\\ z&x&y\\y&z&x
  \end{vmatrix}$
  \item $\begin{vmatrix}
    a&b&c\\2a&2b&2c\\m&n&l
  \end{vmatrix}$
  \item $\begin{vmatrix}
    1&-a&-b\\a&1&-c\\b&c&1
  \end{vmatrix}$
\end{enumerate}
\end{multicols}
\end{enumerate}

三阶行列式按对角线法则展开计算是较繁的.为了简化计算和理论研究的需要,我们以三阶行列式为例学习行列式
的性质.

\begin{thm}
  {性质1} 把各行变为相应的列(就是把第$i$行变为第$i$列,$i=1,\ldots,2,3$)所得行列式与原行列式等值(简述成:行列互换值不变). 即
\[\begin{vmatrix}
  a_1&b_1&c_1\\
  a_2&b_2&c_2\\
  a_3&b_3&c_3\\
\end{vmatrix}=\begin{vmatrix}
  a_1&a_2&a_3\\
  b_1&b_2&b_3\\
  c_1&c_2&c_3\\
\end{vmatrix}\]
\end{thm}

\begin{proof}
  按对角线法则分别把它们展开,比较相应项,即可得证.
\end{proof}

性质1的价值在于:对行成立的定理,对列也一定成立,反之亦然(这叫做行与列的对偶性). 因此,下面各条性质我们只对“行”叙证,对“列”的情况,由性质1保证,也就不述自明了.

\begin{thm}
 {性质2} 把任意两行对调,所得行列式与原行列式绝对值相等,符号相反(简述成:两行互换值相反). 
\end{thm}

\begin{proof}
  先证二、三两行互换值相反. 即
\[\begin{vmatrix}
  a_1&b_1&c_1\\
  a_2&b_2&c_2\\
  a_3&b_3&c_3\\
\end{vmatrix}=-\begin{vmatrix}
  a_1&b_1&c_1\\
  a_3&b_3&c_3\\
  a_2&b_2&c_2\\
\end{vmatrix}\]
按对角线法则展开比较两边即可获证.
\end{proof}

其他情况同理可证.

\begin{thm}
  {推论1} 若有两行对应元素相同,行列式的值等于零(简述成:两行相同值为零).
\end{thm}

\begin{proof}
  设$D$有两行对应元素相同,把这两行对调,所得仍是原行列式$D$,但据性质2,应有
$$D=-D\Longrightarrow D=0$$
\end{proof}

\begin{thm}
 {性质3} 某行元素都乘常数$k$,等于原行列式乘$k$(简述成:某行乘$k$,值$k$倍). 
\end{thm}

\begin{proof}
  先证第二行乘$k$等于原行列式乘$k$,即
\[\begin{vmatrix}
  a_1&b_1&c_1\\
  ka_2&kb_2&kc_2\\
  a_3&b_3&c_3\\
\end{vmatrix}=k\begin{vmatrix}
  a_1&b_1&c_1\\
  a_2&b_2&c_2\\
  a_3&b_3&c_3\\
\end{vmatrix}\]
两边都按对角线法则展开比较即明.
\end{proof}

\begin{thm}
{推论2} 某行有公因子,可以提到行列式外.
\end{thm}

\begin{example}
  计算$\begin{vmatrix}
    \frac{1}{2}&\frac{1}{2}&-1\\[1ex]
    \frac{1}{3}&\frac{2}{3}&-\frac{2}{3}\\[1ex]
    \frac{2}{5}&\frac{3}{5}&-\frac{1}{5}
  \end{vmatrix}$
\end{example}

\begin{solution}
\[\begin{split}
  \text{原行列式}&=(-1)\x \frac{1}{2}\x\frac{1}{3}\x\frac{1}{5}\x \begin{vmatrix}
    1 &1&2\\1&2&2\\2&3&1
  \end{vmatrix}\qquad \text{(推论2)}\\
  &=-\frac{1}{30}(2+6+4-8-1-6)=-\frac{1}{30}\times (-3)\\
  &=\frac{1}{10}
\end{split}\]
\end{solution}

\begin{rmk}
  由此可见,运用推论2提公因式,可简化计算.
\end{rmk}

\begin{thm}
{推论3} 若某行所有元素全为零,那么行列式的值为零(简述成:某行为零,值为零).  
\end{thm}

\begin{thm}
  {性质4} 若有两行对应元素成比例,那么行列式的值为
零(简述成两行成比例,值为零).
\end{thm}

\begin{proof}
  先证第一、二行成比例,行列式值为零.此时有
$$D=\begin{vmatrix}a_1&b_1&c_1\\ka_1&kb_1&kc_1\\a_3&b_3&c_3\end{vmatrix},$$
据推论 2 与推论 1 有
$$D=k\begin{vmatrix}a_1&b_1&c_1\\a_1&b_1&c_1\\a_3&b_3&c_3\end{vmatrix}=k\cdot0=0.$$
其他情况,同理可证。
\end{proof}

\begin{thm}
  {性质5} 若某行元素都是二项式,那么原行列式等于把这些二项式各取一项作成相应行,而其余行不变的两个行列式的和(简述成:某行都是二项式,可相应分成两个行列式之和).
\end{thm}

\begin{proof}
  我们先证明
\begin{equation}
  \begin{vmatrix}a_1+a_1'&b_1+b_1'&c_1+c_1'\\a_2&b_2&c_2\\a_3&b_3&c_3\end{vmatrix}=\begin{vmatrix}a_1&b_1&c_1\\a_2&b_2&c_2\\a_3&b_3&c_3\end{vmatrix}+\begin{vmatrix}{a_1}'&{b_1}'&{c_1}'\\{a_2}&{b_2}&{c_2}\\{a_3}&{b_3}&{c_3}\end{vmatrix}\tag{*}
\end{equation}
\[\begin{split}
  \text{左边}&=(a_{1}+a_{1}^{\prime})b_{2}c_{3}+a_{2}b_{3}(c_{1}+c_{1}^{\prime})+a_{3}(b_{1}+b_{1}^{\prime})c_{2}\\
&\qquad -a_{3}b_{2}(c_{1}+c_{1}^{\prime})-a_{2}(b_{1}+b_{1}^{\prime})c_{3}-(a_{1}+a_{1}^{\prime})b_{3}c_{2}\\
&=(a_{1}b_{2}c_{3}+a_{2}b_{3}c_{1}+a_{3}b_{1}c_{2}-a_{3}b_{2}c_{1}-a_{2}b_{1}c_{3}-a_{1}b_{3}c_{2})\\
&\qquad +(a_{1}^{\prime}b_{2}c_{3}+a_{2}b_{3}c_{1}^{\prime}+a_{3}b_{1}^{\prime}c_{2}-a_{3}b_{2}c_{1}^{\prime}-a_{2}b_{1}^{\prime}c_{3}-a_{1}^{\prime}b_{3}c_{2})\\
&=\begin{vmatrix}a_1&b_1&c_1\\a_2&b_2&c_2\\a_3&b_3&c_3\end{vmatrix}+\begin{vmatrix}{a_1}'&{b_1}'&{c_1}'\\{a_2}&{b_2}&{c_2}\\{a_3}&{b_3}&{c_3}\end{vmatrix}.
\end{split}\]

$\therefore\quad $(*)成立. 其余情况同理可证.
\end{proof}

\begin{example}
  求证
$D=\begin{vmatrix}1&x^{2}&a^{2}+x^{2}\\1&y^{2}&a^{2}+y^{2}\\1&z^{2}&a^{2}+z^{2}\end{vmatrix}=0.$
\end{example}

\begin{proof}
  $$D=\begin{vmatrix}1&x^2&a^2\\1&y^2&a^2\\1&z^2&a^2\end{vmatrix}+\begin{vmatrix}1&x^2&x^2\\1&y^2&y^2\\1&z^2&z^2\end{vmatrix}=0.$$
(据性质4, 第一个行列式为零, 据推论1, 第二个行列式也为零)
\end{proof}


由此可见,运用行列式性质确能简化计算.

\begin{thm}
{性质6} 某行各元素同乘以$k$, 加到另--行的对应元素上,所得行列式与原行列式等值(简述成:某行乘$k$加到另一行,值不变).  
\end{thm}

\begin{proof}
  先证明把第二行各元素同乘$k$, 加到第一行对应
元素上,行列式值不变,即
$$\begin{vmatrix}a_1+ka_2&b_1+kb_2&c_1+kc_2\\a_2&b_2&c_2\\a_3&b_3&\cdot&c_3\end{vmatrix}=\begin{vmatrix}a_1&b_1&c_1\\a_2&b_2&c_2\\a_3&b_3&c_3\end{vmatrix}.$$
\[\begin{split}
  \text{左边}&=\begin{vmatrix} a_1& b_1& c_1\\ a_2& b_2& c_2\\ a_3& b_3& c_3\end{vmatrix} + \begin{vmatrix} ka_2& kb_2& kc_2\\ a_2& b_2& c_2\\ a_3& b_3& c_3\end{vmatrix}\qquad  \text{(性质5)}\\
  &= \begin{vmatrix} a_1& b_1& c_1\\ a_2& b_2& c_2\\ a_3& b_3& c_3\end{vmatrix} \qquad  \text{(性质 4)}
\end{split}\]
其余情况同理可证.
\end{proof}

\begin{example}
  利用行列式性质,计算:
\begin{multicols}{2}
\begin{enumerate}[(1)]
  \item $\begin{vmatrix}3&2&6\\8&10&9\\6&-2&21\end{vmatrix}$
  \item $\begin{vmatrix} 10& - 2& 7\\ - 15& 3& 2\\ - 5& 4& 9\end{vmatrix} $
\end{enumerate}
\end{multicols}
\end{example}

\begin{solution}
\[\begin{split}
(1)\quad \begin{vmatrix}3&2&6\\8&10&9\\6&-2&21\end{vmatrix}&=3\times2\times\begin{vmatrix}3&1&2\\8&5&3\\6&-1&7\end{vmatrix}\qquad\text{(推论 2)}\\
&=6\times \begin{vmatrix} 3& 1+ 2& 2\\ 8& 5+ 3& 3\\ 6& - 1+ 7& 7\end{vmatrix} \qquad \text{(性质 6)} \\
&= 6\times \begin{vmatrix} 3& 3& 2\\ 8& 8& 3\\ 6& 6& 7\end{vmatrix} = 0 \qquad \text{(推论 1)}
\end{split}\]
\[\begin{split}
  (2)\quad \begin{vmatrix} 10& - 2& 7\\ - 15& 3& 2\\ - 5& 4& 9\end{vmatrix} &=5\times \begin{vmatrix} 2& - 2& 7\\ - 3& 3& 2\\ - 1& 4& 9\end{vmatrix} \qquad \text{(推论 2)}\\
  &=5\times \begin{vmatrix} 2& - 2+2& 7\\ - 3& 3+(-3)& 2\\ - 1& 4+(-1)& 9\end{vmatrix} \qquad \text{(性质6)}\\
  &=5\times \begin{vmatrix} 2& 0& 7\\ - 3& 0& 2\\ - 1& 3& 9\end{vmatrix} =5\times (-63-12)=-375.
\end{split}\]
\end{solution}

\begin{note}
在计算行列式时,为了简化计算,首先应注意使行列式值为零的几个定理或推论(推论 1、推论 3、性质 4)的条件是否满足,或者能否化到这三种情况——此时立刻可断定行列式值为零;其次,看看是否可提公因式,或用性质 6 把某行(或某列)的两个元素一起都变为零,从而简化计算;第三, 使行列式做等值变形的几个定理应熟练掌握.
\end{note}

\begin{example}
  利用行列式性质,证明
\begin{multicols}{2}
\begin{enumerate}[(1)]
  \item $\begin{vmatrix}0&a&b\\-a&0&c\\-b&-c&0\end{vmatrix}=0;$
  \item $\begin{vmatrix}a+b&c&-a\\a+c&b&-c\\b+c&a&-b\end{vmatrix}=\begin{vmatrix}b&a&c\\a&c&b\\c&b&a\end{vmatrix}.$
\end{enumerate}  
\end{multicols}
\end{example}

\begin{proof}
\[\begin{split}
  (1)\quad \begin{vmatrix} 0& a& b\\ - a& 0& c\\ - b& - c& 0\end{vmatrix} &= \begin{vmatrix} 0& - a& - b\\ a& 0& - c\\ b& c& 0\end{vmatrix}\qquad \text{(性质 1)}\\
  &=-\begin{vmatrix}0&a&b\\-a&0&c\\-b&-c&0\end{vmatrix}\qquad \text{(推论2)}
\end{split} \]
$\therefore\quad \begin{vmatrix} 0& a& b\\ - a& 0& c\\ - b& - c& 0\end{vmatrix}=0.$

\[\begin{split}
  (2)\quad \begin{vmatrix}a+b&c&-a\\a+c&b&-c\\b+c&a&-b\end{vmatrix}&=\begin{vmatrix}b&c&-a\\a&b&-c\\c&a&-b\end{vmatrix}\qquad \text{(性质 6)}\\
  &=- \begin{vmatrix} b& c& a\\ a& b& c\\ c& a& b\end{vmatrix}\qquad \text{(推论 2)} \\
  &= \begin{vmatrix} b& a& c\\ a& c& b\\ c& b& a\end{vmatrix}\qquad \text{(性质 2)}
\end{split}\]
\end{proof}

\section*{习题三}
\begin{center}
  \bfseries A
\end{center}
\begin{enumerate}
  \item 利用行列式的性质计算:
\begin{multicols}{3}
\begin{enumerate}[(1)]
  \item $\begin{vmatrix}3&2&6\\8&10&9\\6&-2&21\end{vmatrix}$
  \item $\begin{vmatrix}10&-2&7\\-15&3&2\\-5&4&9\end{vmatrix}$
  \item $\begin{vmatrix}1&3&4\\10&1&11\\7&1&8\end{vmatrix}$
  \item $\begin{vmatrix}3&49&4\\2&28&4\\4&35&8\end{vmatrix}$
  \item $\begin{vmatrix}
    \frac{2}{3}&\frac{2}{3}&3\\7&5&14\\
    \frac{1}{3}&\frac{1}{5}&\frac{4}{15}
  \end{vmatrix}$
  \item $\begin{vmatrix}
    1&4&7\\2&5&8\\3&6&9
  \end{vmatrix}$
\end{enumerate}
\end{multicols}

\item 利用行列式的性质计算:
\begin{multicols}{2}
\begin{enumerate}[(1)]
  \item $\begin{vmatrix}a&a&a\\-a&a&x\\-a&-a&x\end{vmatrix}$
  \item $\begin{vmatrix}1&a&b+c\\1&c&c+a\\1&c&a+b\end{vmatrix}$
  \item $\begin{vmatrix}1&1&1\\1&1+b&1\\1&1&1+c\end{vmatrix}$
  \item $\begin{vmatrix}a-b&b-c&c-a\\b-c&c-a&a-b\\c-a&a-b&b-c\end{vmatrix}$
\end{enumerate}
\end{multicols}

\end{enumerate}

\begin{center}
  \bfseries B
\end{center}

\begin{enumerate}\setcounter{enumi}{2}
  \item 不展开行列式,证明下列等式:
\begin{enumerate}[(1)]
  \item $\begin{vmatrix}
    1&1&1\\ p&q&p+q\\
    q&p&0
  \end{vmatrix}=0$
  \item $\begin{vmatrix}
    -a+b+c&a&-b\\ a-b+c& b&-c\\a+b-c&c&-a
  \end{vmatrix}=\begin{vmatrix}
    b&a&c\\c&b&a\\a&c&b
  \end{vmatrix}$
\end{enumerate}
\end{enumerate}

\section{按一行(或一列)展开行列式}
上一节我们看到,三阶行列式用性质去处理比用对角线法则展开计算上简便多了. 本节将学习按行(或列)展开行列式的方法. 它是展开行列式的简便计算法.

在三阶行列式的展开式中,如果把含$a_1$、$a_2$、$a_3$的项分别结合在一起,并提出公因子,就有
\begin{equation}
\begin{split}
  \begin{vmatrix}
    a_1&b_1&c_1\\
    a_2&b_2&c_2\\
    a_3&b_3&c_3\\
  \end{vmatrix}&=a_1b_2c_3+a_2b_3c_1+a_3b_1c_2-a_3b_2c_1-a_2b_1c_3-a_1b_3c_2\\
  &=a_1(b_2c_3-b_3c_2)+a_2(b_3c_1-b_1c_3)+a_3(b_1c_2-b_2c_1)\\
  &=a_1\begin{vmatrix}b_2&c_2\\b_3&c_3\end{vmatrix}-a_2\begin{vmatrix}b_1&c_1\\b_3&c_3\end{vmatrix}+a_3\begin{vmatrix}b_1&c_1\\b_2&c_2\end{vmatrix} 
\end{split}\tag{1}
\end{equation}
可以看出:(1)式中的$\begin{vmatrix}
  b_2&c_2\\
  b_3&c_3\\
\end{vmatrix}$
相当于在原三阶行列式中,划去$a_1$所在的行和列,剩下的元素按行、列顺序排列所组成的行列式. 把行列式中某一元素所在的行和列划去后,剩下的元素按原行、列顺序排列所组成的行列式,叫做原行列式中对应于这个元素的\textbf{余子式}. 例如在行列式$D=\begin{vmatrix}
  a_1&b_1&c_1\\
  a_2&b_2&c_2\\
  a_3&b_3&c_3\\
\end{vmatrix}$
中,对应于$a_2$的余子式是
$\begin{vmatrix}
  b_1&c_1\\
  b_3&c_3\\
\end{vmatrix}$.

若行列式中某元素位于第$i$行第$j$列,把对应于这个元素的余子式乘上$(-1)^{i+j}$后所得到的式子叫做原行列式中对应于这个元素的\textbf{代数余子式}. 例如在上面的行列式$D$中,元素$a_2$位于第二行第一列,$i+j=2+1=3$,所以对应于$a_2$的代数余子式为
$(-1)^{2+1}\begin{vmatrix}
  b_1&c_1\\
  b_3&c_3\\
\end{vmatrix}$
即$-\begin{vmatrix}
  b_1&c_1\\
  b_3&c_3\\
\end{vmatrix}$
三阶行列式各元素的代数余子式的符号$(-1)^{i+j}$可以用下图
帮助记忆
$$\begin{vmatrix}+&-&+\\-&+&-\\+&-&+\end{vmatrix}.$$

行列式$D=\begin{vmatrix}a_1&b_1&c_1\\a_2&b_2&c_2\\a_3&b_3&c_3\end{vmatrix}$
中某个元素的代数余子式通常用这个元素相应的大写字母并附加相同的下标来表示,例如元素$a_{1},b_{1},c_{1}$的代数余子式分别写成 $A_1,B_1,C_1$, 其中
\[\begin{split}
  A_{1}&=(-1)^{1+1}\begin{vmatrix}b_2&c_2\\b_3&c_3\end{vmatrix}=\begin{vmatrix}b_2&c_2\\b_3&c_3\end{vmatrix},\\
  B_{1}&=(-1)^{1+2}\begin{vmatrix}a_2&c_2\\a_3&c_3\end{vmatrix}=-\begin{vmatrix}a_2&c_2\\a_3&c_3\end{vmatrix},\\
  C_{1}&=(-1)^{1+3}\begin{vmatrix}a_2&b_2\\a_3&b_3\end{vmatrix}=\begin{vmatrix}a_2&b_2\\a_3&b_3\end{vmatrix}.\end{split}\]
这样,上面所得的(1)式就可写成
\begin{equation}
  \begin{vmatrix}a_1&b_1&c_1\\a_2&b_2&c_2\\a_3&b_3&c_3\end{vmatrix}=a_1A_1+a_2A_2+a_3A_3,\tag{2}
\end{equation}
它把一个三阶行列式表示成这个行列式第一列的元素与对应
于它们的代数余子式的乘积的和.

一般地,有如下定理:

\begin{thm}
{定理 1} 行列式等于它的任意一行(或一列)的所有元素
与它们各自对应的代数余子式的乘积的和.  
\end{thm}

这就是说,我们可以按任一行(或任一列)展开三阶行列
式$D$:
\[\begin{split}
  D=a_{1}A_{1}+b_{1}B_{1}+c_{1}C_{1},&\qquad D=a_{1}A_{1}+a_{2}A_{2}+a_{3}A_{3},\\
  D=a_{2}A_{2}+b_{2}B_{2}+c_{2}C_{2},&\qquad D=b_{1}B_{1}+b_{2}B_{2}+b_{3}B_{3},\\
  D=a_{3}A_{3}+b_{3}B_{3}+c_{3}C_{3},&\qquad D=c_{1}C_{1}+c_{2}C_{2}+c_{3}C_{3}
\end{split}\]
等式$D=a_1A_1+a_2A_2+a_3A_3$在前面已经证明过,其他五个等
式也可类似证明。

\begin{thm}
  {定理2} 行列式某一行(或一列)的各元素与另一行(或
一列)对应元素的代数余子式的乘积的和等于零。
\end{thm}

\begin{proof}
我们来证明行列式的第二行的各元素与第一行对
应元素的代数余子式的乘积的和等于零,即
\[a_2A_1+b_2B_1+c_2C_1=0\]

$\because\quad a_2\begin{vmatrix}b_2&c_2\\b_3&c_3\end{vmatrix}-b_2\begin{vmatrix}a_2&c_2\\a_3&c_3\end{vmatrix}+c_2\begin{vmatrix}a_2&b_2\\a_3&c_3\end{vmatrix}=\begin{vmatrix}a_2&b_2&c_2\\a_2&b_2&c_2\\a_3&b_3&c_3\end{vmatrix}=0$

$\therefore\quad a_{2}A_{1}+ b_{2}B_{1}+ c_{2}C_{1}= 0$

其他情况可类似证明。  
\end{proof}

\begin{example}
  把行列式
$D=\begin{vmatrix}3&1&-2\\5&-2&7\\3&4&2\end{vmatrix}$
按第一行展开,然后进行计算。
\end{example}

\begin{solution}
\[\begin{split}
  \begin{vmatrix}3&1&-2\\5&-2&7\\3&4&2\end{vmatrix}&=3\times\begin{vmatrix}-2&7\\4&2\end{vmatrix}-1\times\begin{vmatrix}5&7\\3&2\end{vmatrix}+(-2)\times\begin{vmatrix}5&-2\\3&4\end{vmatrix}\\
&=3\times(-32)-1\times(-11)-2\times26\\
&=-137.
\end{split}\]
\end{solution}


按一行(或--列)展开行列式来计算时,如果先根据行列式的性质把某一行(或一列)的两个元素变为零,就会使计算简便得多.如上题,把第二列乘以$-3$ 加到第一列,把第二列乘以 2 加到第三列,可得
\[\begin{split}&=
  \begin{vmatrix}3&1&-2\\5&-2&7\\3&4&2
  \end{vmatrix}
  =\begin{vmatrix}0&1&0\\11&-2&3\\-9&4&10
  \end{vmatrix}\\&=(-1)\times
  \begin{vmatrix}11&3\\-9&10\end{vmatrix}=-137
\end{split}\]

\begin{example}
  计算:
\begin{multicols}{2}
\begin{enumerate}[(1)]
  \item $\begin{vmatrix}4&-6&3\\5&2&7\\5&-2&8\end{vmatrix}$
  \item $\begin{vmatrix}8&-6&9\\5&4&6\\4&5&8\end{vmatrix}$
\end{enumerate}
\end{multicols}
\end{example}

\begin{solution}
\begin{enumerate}[(1)]
  \item $\begin{vmatrix}4&-6&3\\5&2&7\\5&-2&8\end{vmatrix}=\begin{vmatrix}19&0&24\\5&2&7\\10&0&15\end{vmatrix}=2\begin{vmatrix}19&24\\10&15\end{vmatrix}=90$
  \item $\begin{vmatrix}8&-6&9\\5&4&6\\4&5&8\end{vmatrix}=\begin{vmatrix}8&-6&9\\5&4&6\\-1&1&2\end{vmatrix}=\begin{vmatrix}2&-6&21\\9&4&-2\\0&1&0\end{vmatrix}=-\begin{vmatrix}2&21\\9&-2\end{vmatrix}=193$
\end{enumerate}
\end{solution}

\begin{rmk}
  为了使某行(或某列)的两个元素变为零,我们常常
是在这一行(或列)中“先变出 1,再变出零”,如第(2)题.
\end{rmk}

\begin{example}
  解方程
$\begin{vmatrix}15-2x&11&10\\11-3x&17&16\\7-x&14&13\end{vmatrix}=0$
\end{example}

\begin{solution}
  \[\begin{split}
    \begin{vmatrix}15-2x&11&10\\11-3x&17&16\\7-x&14&13\end{vmatrix}&= \begin{vmatrix}15-2x&1&10\\11-3x&1&16\\7-x&1&13\end{vmatrix}=  \begin{vmatrix}8-x&0&-3\\4-2x&0&3\\7-x&1&13\end{vmatrix}\\
    &=-\begin{vmatrix}8-x&-3\\4-2x&3\end{vmatrix} =9x-36=9(x-4)
  \end{split}\]

因为方程左边等于$9(x-4)$,所以原方程即$9(x-4)=0$,它的解集是$\{4\}$.
\end{solution}

\begin{example}
  求证
$\begin{vmatrix}a&b&c\\a^2&b^2&c^2\\b+c&c+a&a+b\end{vmatrix}=(a-b)(b-c)(c-a)(a+b+c)$
\end{example}

\begin{proof}
  \textbf{证法 1}
\[\begin{split}
  \begin{vmatrix}a&b&c\\a^2&b^2&c^2\\b+c&c+a&a+b\end{vmatrix}&=\begin{vmatrix}a-b&b-c&c\\a^2-b^2&b^2-c^2&c^2\\b-a&c-b&a+b\end{vmatrix}\\
  &=(a-b)(b-c)\begin{vmatrix}1&1&c\\a+b&b+c&c^2\\-1&-1&a+b\end{vmatrix}\\
  &=(a-b)(b-c)\begin{vmatrix}1&1&c\\a+b&b+c&c^2\\0&0&a+b+c\end{vmatrix}\\
  &=(a-b)(b-c)(a+b+c)\begin{vmatrix}1&1\\a+b&b+c\end{vmatrix}\\
  &=(a-b)(b-c)(c-a)(a+b+c)
\end{split}  \]

\textbf{证法 2} 若把第一行各元素加到第三行的相应元素上
\[\begin{split}
  \begin{vmatrix}a&b&c\\a^2&b^2&c^2\\b+c&c+a&a+b\end{vmatrix}&=\begin{vmatrix}a&b&c\\a^2&b^2&c^2\\a+b+c&a+b+c&a+b+c\end{vmatrix}\\
  &=(a+b+c)\begin{vmatrix}
    a&b&c\\a^2&b^2&c^2\\
    1&1&1
  \end{vmatrix}
\end{split}\]
(由于变出了“1”,以下极易变出“0”,由读者自己完成.)
\end{proof}

\section*{习题四}
\begin{center}
  \bfseries A
\end{center}
\begin{enumerate}
  \item 已知行列式:$\begin{vmatrix}
    3&6&7\\ 8&6&1\\ 2&-5&4
  \end{vmatrix}$
\begin{enumerate}[(1)]
  \item 求行列式中元素$-5$的余子式与代数余子式;
  \item 按第三列展开这一行列式;
  \item 验证行列式第一行的各元素与第三行对应元余子式的乘积的和等于零.
\end{enumerate}

\item 利用行列式的性质和本节的定理1,计算:
\begin{multicols}{3}
\begin{enumerate}[(1)]
  \item $\begin{vmatrix}
    5&0&-5\\3&2&7\\-4&3&9
  \end{vmatrix}$
  \item $\begin{vmatrix}
    -6&5&2\\2&1&-1\\1&7&4
  \end{vmatrix}$
  \item $\begin{vmatrix}
    2&6&7\\ -3&8&8\\ -5&2&3
  \end{vmatrix}$
\end{enumerate}
\end{multicols}

\item 解下列方程:
\begin{multicols}{2}
\begin{enumerate}[(1)]
  \item $\begin{vmatrix}
    2&x+2&6\\ 1&x&3\\1&3&x
  \end{vmatrix}=0$
  \item $\begin{vmatrix}
    a&a&x\\1&1&1\\b&x&b
  \end{vmatrix}=0$
\end{enumerate}
\end{multicols}
\end{enumerate}

\begin{center}
  \bfseries B
\end{center}
\begin{enumerate}\setcounter{enumi}{3}
  \item 求证:
\begin{enumerate}[(1)]
  \item $\begin{vmatrix}
    1&1&1\\a&b&c\\bc&ca&ab
  \end{vmatrix}=(a-b)(b-c)(c-a)$
  \item $\begin{vmatrix}
    a&b&b\\b&a&b\\b&b&a
  \end{vmatrix}=(a+2b)(a-b)^2$
  \item $\begin{vmatrix}
    1&p&p^2\\1&q&q^3\\1&r&r^3
  \end{vmatrix}=(p-q)(q-r)(r-p)(p+q+r)$
\end{enumerate}
\end{enumerate}

\section{三元线性方程组}
线性方程组,当其中方程的个数与未知数的个数相同时,它的一般形式是
\[({\rm III})\quad \begin{cases}
  a_1x+b_1y+c_1z=d_1, & (1)\\
  a_2x+b_2y+c_2z=d_2, & (2)\\
  a_3x+b_3y+c_3z=d_3, & (3)\\
\end{cases}\]

如果当$x=x_1$, $y=y_1$, $z=z_1$时,方程组(III)中的每个方程左右两边的值相等,那么$x=x_1$, $y=y_1$, $z=z_1$叫做\textbf{方程组(III)的一个解},简记为$(x_1,y_1,z_1)$. 方程组(III)的所有的解构成的集合叫做\textbf{方程组(III)的解集}\footnote{对一般$n$元线性方程组的解与解集,也可作相应定义。}。

我们用$D$表示方程组(III)的系数行列式,即
$D=\begin{vmatrix}a_1&b_1&c_1\\a_2&b_2&c_2\\a_3&b_3&c_3\end{vmatrix}$
用元素$a_1,a_2,a_3$对应的代数余子式$A_1,A_2,A_3$分别乘方程(1)(2)(3)的两边,得
\[\begin{split}
  a_1A_1x+b_1A_1y+c_1A_1z&=d_1A_1\\
a_2A_2x+b_2A_2y+c_2A_2z&=d_2A_2\\
a_3A_3x+b_3A_3y+c_3A_3z&=d_3A_3
\end{split}\]
以上三式等号两边分别相加,得
\begin{equation}
  \begin{split}
(a_1A_1+a_2A_2+a_3A_3)x&+(b_1A_1+b_2A_2+b_3A_3)y
+(c_1A_1+c_2A_2+c_3A_3)z\\
&    =d_1A_1+d_2A_2+d_3A_3
  \end{split}\tag{4}
\end{equation}
根据 8.4 节的定理 1 和定理 2, (4)中 $x$ 的系数等于 $D$, 而$y$、$z$
的系数都为零,等号右边合成一个行列式就是
$$\begin{vmatrix}d_1&b_1&c_1\\d_2&b_2&c_2\\d_3&b_3&c_3\end{vmatrix}\overset{\text{记为}}{\operatorname*{=}}D_x$$
这样,(4)式就可以写成$D\cdot x=D_x$.

用类似的方法,从方程组(III)中消去$x$、$z$, 或者$x$、$y$, 并分别标记
$$D_y=\begin{vmatrix}a_1&d_1&c_1\\a_2&d_2&c_2\\a_3&d_3&c_3\end{vmatrix},\qquad D_z=\begin{vmatrix}a_1&b_1&d_1\\a_2&b_2&d_2\\a_3&b_3&d_3\end{vmatrix},$$
则可得到
$D\cdot y=D_{y},\qquad D\cdot z=D_{z}$

以上三式合在一起就是
$$({\rm  IV})\quad \begin{cases}D\cdot  x=D_x,&(5)\\D\cdot y=D_y,&(6)\\D\cdot  z=D_z,&(7)\end{cases}$$

当$D\neq0$时,(IV)有唯一解,是
\begin{equation}
  \begin{cases}x=\frac{D_x}{D},\\[1ex] y=\frac{D_y}{D},\\[1ex] z=\frac{D_z}{D}.\end{cases}\tag{8}
\end{equation}


应该明确:我们并没有证明方程组(III)与(IV)是同解的. 从方程组(IV)的导出过程可以看出(IV)是(III)的“结果”, 因而(III)的解一定是(IV)的解. 在$D\neq0$时,(IV)只有唯一解(8), 此时,若(III)有解,其解又适合(IV), 必然就是(8).

现在来验证在$D\neq0$时,(8)确实就是(III)的解. 把(8)代入
(1)的左边,有
\[\begin{split}
  \text{左边}&=a_1\frac {D_x}D+ b_1\frac {D_y}D+ c_1\frac {D_z}D\\
&=\frac{a_{1}}{D}(d_{1}A_{1}+d_{2}A_{2}+d_{3}A_{3})+\frac{b_{1}}{D}(d_{1}B_{1}+d_{2}B_{2}+d_{3}B_{3})\\
&\qquad +\frac{c_1}D(d_1C_1+d_2C_2+d_3C_3)\\
&=\frac{1}{D}[(a_{1}A_{1}+b_{1}B_{1}+c_{1}C_{1})d_{1}+(a_{1}A_{2}+b_{1}B_{2}+c_{1}C_{2})d_{2}\\
&\qquad \quad+(a_{1}A_{3}+b_{1}B_{3}+c_{1}C_{3})d_{3}]\\
&=\frac{1}{D}[D\cdot d_{1}+0\cdot d_{2}+0\cdot d_{3}]\\
&=d_1=\text{右边}
\end{split}\]
即(8)式适合方程(1). 同样可以验证(8)式分别适合方程(2)和方程(3). 因此,(8)式是方程组(III)的解.

综上所述,可得以下结论:

\textbf{三元线性方程组(III),当它的系数行列式$D$不等于零时,有唯一解$\left(\frac{D_x}{D},\frac{D_y}{D},\frac{D_z}{D}\right)$,其中$D_x$、$D_y$、$D_z$是把系数行列式$D$中第一、二、三列分别换成方程组(III)的常数项列而得出的三个三阶行列式.}

我们已经知道,对二元线性方程组(I)已有类似的结论. 事实上,对$n$元线性方程组都有类似的结论. 这一结论称为克莱姆法则\footnote{克莱姆(Gabriel Cramer,1704—1752),瑞士数学家.},上面只是对$n=3$的情况进行了证明.

当方程组(III)的系数行列式$D=0$时,方程组(III)或者无解,或者有无穷多解(证明从略). 例如方程组
\[\begin{cases}x+y+z=1,\\x+y+2z=2,\\2x+2y+3z=5,\end{cases}\qquad \begin{cases}x+y+z=1,\\x+y+z=2,\\x+y+z=3.\end{cases}\]
都没有解, 而方程组
\[\begin{cases}x+y+z=1,\\x+2y+2z=1,\\y+z=0,\end{cases}\qquad \begin{cases}x+y+z=1,\\2x+2y+2z=2,\\4x+4y+4z=4,\end{cases}\]
都有无穷多解.

\begin{example}
  判断下列方程组是否有唯一解,如果有唯一解,根据克莱姆法则把解求出来.
\begin{multicols}{2}
\begin{enumerate}[(1)]
  \item $\begin{cases}
  2x+3y-5z=3,\\
  x-2y+z=0,\\
  3x+y+3z=7;
\end{cases}$
  \item $\begin{cases}
    x-3y+z=1,\\
    2x+y-z=0,\\
    4x-5y+z=2.
  \end{cases}$
\end{enumerate}
\end{multicols}
\end{example}

\begin{solution}
\begin{enumerate}[(1)]
  \item $D=\begin{vmatrix}2&3&-5\\1&-2&1\\3&1&3\end{vmatrix}=\begin{vmatrix}2&7&-7\\1&0&0\\3&7&0\end{vmatrix}=-\begin{vmatrix}7&-7\\7&0\end{vmatrix}=-49\ne 0$
所以方程组有唯一解. 由
\[\begin{split}
D_x&=\begin{vmatrix}
  3&3&-5\\ 0&-2&1\\7&1&3
\end{vmatrix}=\begin{vmatrix}
  3&-7&-5\\0&0&1\\7&7&3
\end{vmatrix}=-\begin{vmatrix}
  3&-7\\7&7
\end{vmatrix}=-70\\
D_y&=\begin{vmatrix}
  2&3&-5\\
  1&0&1\\
  3&7&3
\end{vmatrix}=\begin{vmatrix}
  7&3&-5\\0&0&1\\ 0&7&3
\end{vmatrix}=7\begin{vmatrix}
  0&1\\7&3
\end{vmatrix}=-49\\
D_z&=\begin{vmatrix}
  2&3&3\\1&-2&0\\ 3&1&7
\end{vmatrix}=\begin{vmatrix}
  2&7&3\\1&0&0\\3&7&7
\end{vmatrix}=-\begin{vmatrix}
  7&3\\7&7
\end{vmatrix}=-28\\
\end{split}\]
得:
\[\frac{D_x}{D}=\frac{-70}{-49}=\frac{10}{7},\quad \frac{D_y}{D}=\frac{-49}{-49}=1,\quad \frac{D_z}{D}=\frac{-28}{-49}=\frac{4}{7}.\]
$\therefore\quad $方程组的解集是$\left\{\left(\frac{10}{7},1,\frac{4}{7}\right)\right\}$

\item $D=\begin{vmatrix}
  1&-3&1\\2&1&-1\\4&-5&1
\end{vmatrix}=\begin{vmatrix}
  1&-3&1\\3&-2&0\\3&-2&0
\end{vmatrix}=0$

方程组或者无解,或者有无穷多解. 因此,方程组不可能有唯一解.
\end{enumerate}
\end{solution}

\section*{习题五}
\begin{center}
  \bfseries A
\end{center}

判断下列方程组是否有唯一解;如果有唯一解,根据克莱姆法则把解求出来:
\begin{multicols}{2}
\begin{enumerate}[(1)]
  \item $\begin{cases}x-2y+z=0,\\3x+y-2z=0,\\7x+6y+7z=100;\end{cases}$
  \item $\begin{cases}3x+2y+3z=15,\\4x-3y+2z=9,\\5x-4y+z=7;\end{cases}$
  \item $\begin{cases}2x+3y+4z=2,\\3x+5y+7z=-3,\\x+2y+3z=4;\end{cases}$
  \item $\begin{cases}x-3y+z=6,\\2x+y+2z=-2,\\4x-5y+6z=10\end{cases}$
\end{enumerate}
\end{multicols}

\section{三元齐次线性方程组}
常数项为零的三元线性方程组
\[({\rm V})\quad \begin{cases}
  a_1x+b_1y+c_1z=0,& (1)\\
  a_2x+b_2y+c_2z=0,& (2)\\
  a_3x+b_3y+c_3z=0,& (3)\\
\end{cases}\]
叫做\textbf{三元齐次线性方程组}. 显然,三元齐次线性方程组总有解$(0,0,0)$,它叫做\textbf{零解}.下面进一步讨论方程组(V)会不会有非零解的情况. 用$D$表示方程组(V)的系数行列式.
\begin{enumerate}
  \item $D\ne 0$. 方程组(V)有唯一解——零解.
  \item $D=0$. 我们来证明方程组(V)除零解外还有无穷多非零解\footnote{利用8.5节三元线性方程组(III)当$D=0$时或者无解或为有无穷多解的结论,容易得出三元齐次线性方程组(V)当$D=0$时一定有无穷多非零解.这里我们从头证明,并同时给出了求解的方法.}. 分两种情况:
\begin{enumerate}[(1)]
  \item $D$中至少有一个元素的代数余子式不等于零.不失一般性,设
$C_3=\begin{vmatrix}
  a_1&b_1\\ a_2&b_2
\end{vmatrix}\ne 0$, 
把方程(1)、(2)中含$z$的项移到等号右边,得
\[\begin{cases}
  a_1x+b_1y=-c_1z\\
  a_2x+b_2y=-c_2z\\
\end{cases}\]
把这个方程组看成关于$x,y$的线性方程组,解出
\[\begin{cases}
  x=\frac{\begin{vmatrix}
  b_1&c_1\\b_2&c_2
  \end{vmatrix}  }{ \begin{vmatrix}
    a_1&b_1\\a_2&b_2 
  \end{vmatrix} }z=\frac{A_3}{C_3}z\\
  y=\frac{-\begin{vmatrix}
    a_1&c_1\\a_2&c_2
    \end{vmatrix}  }{ \begin{vmatrix}
      a_1&b_1\\a_2&b_2 
    \end{vmatrix} }z=\frac{B_3}{C_3}z\\
\end{cases}\]
令$z=C_3 t$($t$为任意常数),得
\begin{equation}
  \begin{cases}
    x=A_3 t\\
    y=B_3 t\\
    z=C_3 t\\
  \end{cases}\tag{4}
\end{equation}
(4)式是方程(1)和(2)的所有公共解的一般表示形式. 把(4)代入(3)的左边,得
\[a_3x+b_3y+c_3z=a_3A_3t+b_3B_3t+c_3C_3t=D\cdot t=0\]
这说明(4)式又同时适合(3). 因此,(4)是方程组(V)的解,而且包括方程组(V)的所有的解.

对任意的一个$t$值,(4)式都可以确定方程组(V)的一个解,$t$值不同,确定的解也不同,而只有$t=0$时它才是零解,所以方程组(V)有无穷多非零解.
\item  $D$中每一个元素的代数余子式都为零.这时,若方程组(V)的每个系数都是零,那么任意一组的值都是(V)的解,当然它有无穷多非零解. 若系数不全为零,不失一般性,设$b_1\ne 0$,由
\[\begin{vmatrix}
  a_1&b_1\\a_2&b_2
\end{vmatrix}=0,\qquad \begin{vmatrix}
  b_1&c_1\\b_2&c_2
\end{vmatrix}=0\]
得:$a_1b_2=a_2b_1,\quad b_1c_2=b_2c_1$

$\therefore\quad a_2=\frac{a_1b_2}{b_1},\quad c_2=\frac{b_2c_1}{b_1}$
因此方程(2)就可由方程(1)两边同乘以常数$\frac{b_2}{b_1}$得出。同样,方程(3)可由方程(1)两边同乘以常数$\frac{b_3}{b_1}$得出。因此方程(1)的解就是
方程组(V)的解,所以方程组(V)除零解外还有无穷多非零解.

反过来,如果方程组(V)有非零解,那么它的系数行列式$D=0$. 不然的话,即如果$D\ne 0$,那么根据克莱姆法则,可推出方程组(V)只有零解,这和方程组(V)有非零解相矛盾.
\end{enumerate}
\end{enumerate}

综上所述,可以得出:

\begin{thm}
  {定理} 齐次线性方程组(V)有非零解的充要条件是它的系数行列式$D$等于零.
\end{thm}

\begin{example}
  解齐次线性方程组$\begin{cases}
    x+y+z=0\\
    2x+2y+3z=0\\
    4x+4y+5z=0\\
  \end{cases}$
\end{example}

\begin{solution}
  因为$D=\begin{vmatrix}
    1&1&1\\2&2&3\\4&4&5
  \end{vmatrix}=0$,所以方程组有无穷多解。

又因为
\[\begin{vmatrix}
  b_1&c_1\\b_2&c_2
\end{vmatrix}=\begin{vmatrix}
  1&1\\2&3
\end{vmatrix}=1\ne 0\]
把第一、第二两个方程中含$x$的项移到等号右边,得
\[\begin{cases}
  y+z=-x\\
  2y+3z=-2x
\end{cases}\]
把这个方程组看成关于$y,z$的线性方程组,解出
\[\begin{cases}
  y=-x\\ z=0
\end{cases}\]
令 $x=t$, 那么 $y=-t$, $z=0$. 不管 $t$ 取什么值, $(t,-t,0)$总适合第三个方程.

因此,原方程组的解集是$\{(t,-t,0),\; t \text{为任意常数}\}$.
\end{solution}

\begin{example}
  求方程组
$\begin{cases}a_1x+b_1y+c_1=0,\\a_2x+b_2y+c_2=0,\\a_3x+b_3y+c_3=0\end{cases}$
有解的必要条件.
\end{example}

\begin{solution}
如果这个方程组有解,那么至少存在一个有序数组
$(x_1,y_1)$, 使得
$$\begin{cases}a_1x_1+b_1y_1+c_1=0,\\a_2x_1+b_2y_1+c_2=0,\\a_3x_1+b_3y_1+c_3=0,\end{cases}$$
即
$$\begin{cases}a_1x_1+b_1y_1+c_1\cdot1=0,\\a_2x_1+b_2y_1+c_2\cdot1=0,\\a_3x_1+b_3y_1+c_3\cdot1=0.\end{cases}$$
也就是说,三元齐次线性方程组
$\begin{cases}a_1x+b_1y+c_1z=0,\\a_2x+b_2y+c_2z=0,\\a_3x+b_3y+c_3z=0.\end{cases}$
有一个非零解$(x_1,y_1,1)$. 根据齐次线性方程组有非零解的必
要条件是它的系数行列式等于零,从而推出
\[D=\begin{vmatrix}
  a_1&b_1&c_1\\
  a_2&b_2&c_2\\
  a_3&b_3&c_3\\
\end{vmatrix}=0\]
因此,原方程组$\begin{cases}
    a_1x_1+b_1y_1+c_1=0,\\
    a_2x_1+b_2y_1+c_2=0,\\
    a_3x_1+b_3y_1+c_3=0.
\end{cases}$有解的必要条件是$\begin{vmatrix}
  a_1&b_1&c_1\\
  a_2&b_2&c_2\\
  a_3&b_3&c_3\\
\end{vmatrix}=0$
\end{solution}

想一想,能否把题中的必要条件改为充要条件,为什么?

\section*{习题六}
\begin{center}
  \bfseries A
\end{center}

下列齐次线性方程组有没有非零解?如果有,把解集求出
来.
\begin{multicols}{2}
\begin{enumerate}[(1)]
  \item $\begin{cases}x+y+z=0,\\2x-y+3z=0,\\x-2y+z=0;\end{cases}$
  \item $\begin{cases}5x-6y-4z=0,\\x+2y+4z=0,\\3x+2y+6z=0.\end{cases}$
\end{enumerate}
\end{multicols}

\section{本章小结}

\subsection*{知识结构}
\begin{enumerate}
  \item 二阶、三阶行列式的定义
\[\begin{split}
  \begin{vmatrix}
    a_1&b_1\\a_2&b_2
  \end{vmatrix}&=a_1b_2-a_2b_1\\
  \begin{vmatrix}
    a_1&b_1&c_1\\a_2&b_2&c_2\\
    a_3&b_3&c_3
  \end{vmatrix}&=a_1b_2c_3+a_2b_3c_1+a_3b_1c_2-a_3b_2c_1-a_2b_1c_3-a_1b_3c_2
\end{split}\]
按照这两个定义,可以把二阶、三阶行列式展开.
\item 行列式的性质(参看8.3节)
\begin{enumerate}[(1)]
  \item 行列互换值不变;
(根据这一条性质,对行成立的性质对列也必然成立,因此以下只需对行讨论即可)
\item 两行互换值相反;

推论1:两行元素相同,行列式的值为零.
\item 某行元素同乘$k$,等于原行列式乘$k$;
推论2:某行有公因式可以提到行列式外:

推论3:某行元素全为零,行列式的值为零.
\item 两行对应元素成比例,行列式的值为零;
\item 某行元素都是二项式,原行列式可写成两个行列式之和.
\item 某行元素同乘$k$,对应加到另一行,行列式值不变.
\end{enumerate}
以上六条性质及其推论是化简、计算行列式的依据.
\item 行列式的展开(见8.4节)

行列式中某元素的\underline{余子式}与\underline{代数余子式}是两个很重要的概念.连同8.4节中的两个定理是展开与化简行列式的重要根据.
\item 二元线性方程组解的讨论

设$\begin{cases}
  a_1x+b_1y=c_1\\
  a_2x+b_2y=c_2\\
\end{cases}$\hfill (*)
\begin{enumerate}[(1)]
  \item 当系数行列式$D\ne 0$时,(*)有唯一解$\left(\frac{D_x}{D}, \frac{D_y}{D}\right)$;
  \item 当$D=0$,但$D_x,D_y$不全为零时,(*)无解;
  \item 当$D=D_x=D_y=0$时有两种情况:
\begin{itemize}
  \item $a_1,a_2,b_1,b_2$不全为零时或$a_1=a_2=b_1=b_2=c_1=c_2=0$时,有无穷多解;
  \item $a_1,a_2,b_1,b_2$全为零,但$c_1,c_2$不全为零时,(*)无解.
\end{itemize}
\end{enumerate}
  \item 三元线性方程组解的讨论

  设$\begin{cases}
  a_1x+b_1y+c_1z=d_1\\
  a_2x+b_2y+c_2z=d_2\\
  a_3x+b_3y+c_3z=d_3\\
  \end{cases}$ \hfill(*)
\begin{enumerate}[(1)]
  \item 当系数行列式$D\ne 0$时,(*)有唯一解$\left(\frac{D_x}{D}, \frac{D_y}{D}, \frac{D_z}{D}\right)$;
  \item 当$D=0$时,或者无解,或者有无穷多解.
\end{enumerate}
\item 三元齐次线性方程组(见8.6节).
\end{enumerate}

\section*{复习题八}

\begin{enumerate}
  \item 用对角线法则计算:
\begin{multicols}{2}
\begin{enumerate}[(1)]
  \item $\begin{vmatrix}
    3&-5&1\\ 2&3&-6\\ -7&2&4 
  \end{vmatrix}$
  \item $\begin{vmatrix}
    a&b&c\\ 0&d&e\\ 0&0&f
  \end{vmatrix}$
  \item $\begin{vmatrix}
    0& -\cos\alpha&-\cos\beta\\
    \cos\alpha & 0& -\cos\gamma\\
    \cos\beta &\cos\gamma &0
  \end{vmatrix}$
  \item $\begin{vmatrix}
    a&h&g\\ h&b&f\\g&f&c
  \end{vmatrix}$
\end{enumerate}
\end{multicols}
  \item 解方程:
\begin{multicols}{2}
\begin{enumerate}[(1)]
  \item $\begin{vmatrix}
    0&x-1&1\\ x-1&0&x-2\\1&x-2&0
  \end{vmatrix}=0$
  \item $\begin{vmatrix}
    x-1&1&1\\ 1&x-1&1\\1&1&x-1
  \end{vmatrix}=0$
\end{enumerate}
\end{multicols}
  \item 求证:
\begin{enumerate}[(1)]
  \item $\begin{vmatrix}
    1& \sin3\theta &\cos3\theta\\
    1& \sin2\theta &\cos2\theta\\
    1& \sin\theta &\cos\theta\\
  \end{vmatrix}=2\sin\theta(1-\cos\theta)$
  \item $\begin{vmatrix}
    2\cos\theta &1&0\\
    1&2\cos\theta &1\\
    0&1&2\cos\theta
  \end{vmatrix}=\frac{\sin 4\theta}{\sin\theta}\quad (\theta\ne 4k\pi,\; k\in\Z)$
\end{enumerate}
  \item 利用行列式的性质计算:
\begin{multicols}{3}
\begin{enumerate}[(1)]
  \item $\begin{vmatrix}
    10&8&-2\\ 15&12&-3\\ 20&32&7
  \end{vmatrix}$
  \item $\begin{vmatrix}
    \frac{1}{2}&\frac{1}{3}&\frac{1}{4}\\
    12&24&36\\
    -5&-4&-3
  \end{vmatrix}$
  \item $\begin{vmatrix}
    554&427&327\\
    586&443&343\\
    711&504&404
  \end{vmatrix}$
\end{enumerate}
\end{multicols}
  \item 利用行列式的性质计算:
\begin{multicols}{2}
\begin{enumerate}[(1)]
  \item $\begin{vmatrix}
    -ab&bd&bf\\ ac&-cd&cf\\
    ae&de&-ef
  \end{vmatrix}$
  \item $\begin{vmatrix}
    a&b&c\\ 2a&3a+2b&4a+3b+2c\\
    3a&6a+3b&10a+9b+3c
  \end{vmatrix}$
\end{enumerate}
\end{multicols}
  \item 不展开行列式,求证:
\begin{enumerate}[(1)]
  \item $\begin{vmatrix}
    a&a+3d&a+6d\\
    a+d&a+4d&a+7d\\
    a+2d&a+5d&a+8d\\
  \end{vmatrix}=0$
  \item $\begin{vmatrix}a_1&b_1&c_1\\a_2&b_2&c_2\\a_3&b_3&c_3\end{vmatrix}=\begin{vmatrix}c_3&b_3&a_3\\c_2&b_2&a_2\\c_1&b_1&a_1\end{vmatrix}$
  \item $\begin{vmatrix}0&am&-abn\\-e&0&bn\\e&-m&0\end{vmatrix}=0$
  \item $\begin{vmatrix}a_1&b_1&a_1x+b_1y+c_1\\a_2&b_2&a_2x+b_2y+c_2\\a_3&b_3&a_3x+b_3y+c_3\end{vmatrix}=\begin{vmatrix}a_1&b_1&c_1\\a_2&b_2&c_2\\a_3&b_3&c_3\end{vmatrix}$
  \item $\begin{vmatrix}0&(a-b)^3&(a-c)^3\\(b-a)^3&0&(b-c)^3\\(c-a)^3&(c-b)^3&0\end{vmatrix}=0$
\end{enumerate}

\item 利用行列式的性质和 8.4节中的定理 1, 计算:
\begin{multicols}{2}
\begin{enumerate}[(1)]
  \item $\begin{vmatrix}6&-4&3\\-3&3&-1\\18&7&5\end{vmatrix}$
  \item $\begin{vmatrix}8&3&-7\\5&0&-4\\-9&-2&3\end{vmatrix}$
\end{enumerate}
\end{multicols}
\item 解方程:
\begin{enumerate}[(1)]
  \item $\begin{vmatrix} x^2& x& 1\\ a^2& a& 1\\ b^2& b& 1\end{vmatrix} = 0,\quad ( a\neq b)$
  \item $\begin{vmatrix} {x}& {a}& {b+ c}\\ {x}& {a+ b}& {c}\\ {a+ b}& {b- c}& {a+ c}
  \end{vmatrix} = 0,\quad [b( a+ b) \neq 0]$
\end{enumerate}

\item 求证:
\begin{enumerate}[(1)]
  \item $\begin{vmatrix}a&a^2&1\\b&b^2&1\\c&c^2&1\end{vmatrix}=(a-b)(b-c)(c-a)$
  \item $\begin{vmatrix}a&a^2&bc\\b&b^2&ac\\c&c^2&ab\end{vmatrix}=(a-b)\left(b-c\right)\left(c-a\right)\left(ab+bc+ca\right)$
  \item $\begin{vmatrix}ax&a^{2}+x^{2}&1\\ay&a^{2}+y^{2}&1\\az&a^{2}+z^{2}&1\end{vmatrix}=a(x-y)(y-z)(z-x)$
  \item $\begin{vmatrix}\cos\theta&\cos3\theta&\sin3\theta\\\cos\theta&\cos\theta&\sin\theta\\\sin\theta&\sin\theta&\cos\theta\end{vmatrix}=\mathrm{sin}\theta\mathrm{sin}4\theta$
\end{enumerate}

\item 已知直线方程为:
$\begin{vmatrix}x&y&1\\3&5&1\\-2&3&1\end{vmatrix}=0$

问点$P_1\left(\frac{1}{2},4\right)$与$P_2(4,7)$是否在这条直线上?

\item 利用克莱姆法则解下列关于$x,y,z$的方程组:
\begin{multicols}{2}
\begin{enumerate}[(1)]
  \item $\begin{cases}4x-y-2z=4\\2x+y-4z=8\\
    x+2y+z=1\end{cases}$
  \item $\begin{cases}5x-8y+3z=0\\15x+12y-15z=11\\10x-4y-6z=1\end{cases}$
\end{enumerate}
\end{multicols}

\item 求下列关于$x,y,z$的方程组有唯一解的条件:
\begin{multicols}{2}
\begin{enumerate}[(1)]
  \item $\begin{cases}\lambda x+y+z=1\\x+\lambda y+z=\lambda\\x+y+\lambda z=\lambda^{2}\end{cases}$
  \item $\begin{cases}ay+bz=c\\cx+az=b\\bx+cy=a\end{cases}$
\end{enumerate}
\end{multicols}
\end{enumerate}


\chapter{一元多项式和高次方程}

\section{一元$n$次多项式}

\subsection{一元$n$次多项式}

形如$a_nx^n+a_{n-1}x^{n-1}+a_{n-2}x^{n-2}+\cdots+a_1x+a_0$.(其中$a_n\ne 0$)的代数式,叫做以$x$为元的一元$n$次多项式. $n$为确定的自然数.

我们把系数$a_i\; (i=0,1,\ldots,n)$都是复数的一元$n$次多项式叫做\textbf{复系数一元$n$次多项式}. 类似地,把系数都是实数(或有理数、整数等)的一元$n$次多项式叫做\textbf{实系数(或有理系数、整系数等)一元$n$次多项式},它们都是复系数一元$n$次多项式的特殊情形. 在本章中提到的多项式,如果不特别说明,都是指复系数多项式.

单独的一个非零数$a_0$,可以看作\textbf{零次多项式}(事实上,当$x\ne 0$时,$a_0=a_0x^0$). 系数都是零的多项式叫做\textbf{零多项式},零多项式没有确定的项数与次数.

\subsection{两个多项式相等的充要条件}

两个多项式相等的充要条件是它们的对应项系数相等(或两个多项式的差为零多项式). 这是待定系数法的理论依据.

\subsection{多项式的值}
一元$n$次多项式$f(x)=a_nx^n+a_{n-1}x^{n-1}+a_{n-2}x^{n-2}+\cdots+a_1x+a_0\; (a_n\ne 0)$,当$x$取复数集$\mathbb{C}$上的某一个值$a+bi\; (a,b\in\R)$时,都有其确定的值,记作$f(a+bi)$. $f(a+bi)$称为当$x=a+bi$时多项式$f(x)$的值. 很明显,无论$x$在$C$上取什么值,零多项式的值恒为零.故零多项式可记作数0.

\subsection{多项式的根}
如果$x=a$时,$f(a)=0$. 则称$a$为多项式$f(x)$的根(或零点).

\section{综合除法}
\subsection{带余除法}
一元多项式相加(包括相减)、相乘的结果仍是一元多项式,并且加、乘运算满足交换律、结合律以及乘法对加法的分配律.

一个一元多项式除以另一个一元多项式,并不是总能整除,当被除式$f(x)$除以除式$g(x)$(不是零多项式),得商式
$q(x)$及余式$r(x)$时,就有下列等式:
\[f(x)=g(x)q(x)+r(x)\]
其中$r(x)$的次数小于$g(x)$的次数,或者$r(x)$是零多项式. 当$r(x)$是零多项式时,就是$f(x)$能被$g(x)$整除.

\subsection{综合除法}
一个一元多项式除以一个一元一次式,有一种简便的计算方法——综合除法.

先用一般的除法来计算$a_3x^3+a_2x^2+a_1x+a_0$除以$x-b$:

\begin{center}
\begin{tikzpicture}[yscale=.6]

\node at (3,6){$a_3x^2+(a_2+a_3b)x+[a_1+(a_2+a_3b)b]$};
\foreach \x/\y in {5/a_3x^3+a_2x^2, 4/a_3x^3-a_3bx^2, 3/(a_2+a_3b)x^2, 2/(a_2+a_3b)x^2}
{
    \node at (0,\x){$\y$};
}
\foreach \x/\y in {5/+a_1x, 3/+a_1x, 2/-(a_2+a_3b)bx, 1/[a_1+(a_2+a_3b)b]x}
{
    \node at (4,\x)[left]{$\y$};
}
\foreach \x/\y in {5/+a_0, 1/+a_0, 0/[a_1+(a_2+a_3b)b]x-[a_1+(a_2+a_3b)b]b, -1/a_0+[a_1+(a_2+a_3b)b]b}
{
    \node at (7.5,\x)[left]{$\y$};
}

\foreach \x in {5.5,3.5,1.5,-.5}
{
    \draw(-1.3,\x)--(8,\x);
}
\node at (-1.7,5){$x-b\; \Big)$};
\end{tikzpicture}
\end{center}

这里所得的商式是$a_3x^2+(a_2+a_3b)x+[a_1+(a_2+a_3b)b]$;余式是$a_0+[a_1+(a_2+a_3b)b]b$,
它不含$x$,所以它是一个常数,下面把它叫做余数.

商式中各项的系数及余数分别是
\[a_3,\quad a_2+a_3b,\quad a_1+(a_2+a_3b)b\quad 
a_0+[a_1+(a_2+a_3b)b]b\]
其中第一个数就是被除式中第一项的系数,把这个数乘以$b$再加上被除式中下一项的系数就得第二个数,依此类推,最后得到余数.

因此,上面的除法可以用下面的简便算式来进行:
\begin{center}
\begin{tikzpicture}[yscale=.6, xscale=2]
\foreach \x/\y in {1/a_3,2/a_2,3.25/a_1,5/a_0,6.2/b}
{
    \node at (\x,3){$\y$};
}
\foreach \x/\y in {2/a_3b,3.25/(a_2+a_3b)b,5/[a_1+(a_2+a_3b)b]b}
{
    \node at (\x,2){$\y$};
}
\foreach \x/\y in {1/a_3,2/a_2+a_3b,3.25/a_1+(a_2+a_3b)b,5/a_0+[a_1+(a_2+a_3b)b]b}
{
    \node at (\x,1){$\y$};
}

\draw(.5,1.5)--(6,1.5)--(6,3.5);
\draw(4,1.5)--(4,.5)--(6,.5);
\end{tikzpicture}
\end{center}

这里,第一行是被除式按降幂排列时各项的系数,如果有缺项,必须用零补足. 移下第一个系数,乘以$b$,加上第二个系数,依次进行,算得的第三行就是商式各项的系数及余数. 用这种算式进行的除法叫做\textbf{综合除法}.

被除式的次数不是三次时,综合除法同样适用.

\begin{example}
用综合除法计算:
\begin{enumerate}[(1)]
    \item $(\polynomial[reciprocal]{1,8,-2,-14})\div (x+1)$
    \item $(\polynomial[reciprocal]{2,5,-24,0,15})\div(x-2)$
\end{enumerate}
\end{example}

\begin{solution}
\begin{enumerate}[(1)]
    \item $x+1$就是$x-(-1)$
\begin{center}
    \begin{tikzpicture}[yscale=.7]
\foreach \x/\y in {1/1,2/+8,3/-2,4/-14,5/-1}
{
    \node at (\x,3){$\y$};
}
\foreach \x/\y in {2/-1,3/-7,4/+9}
{
    \node at (\x,2){$\y$};
}
\foreach \x/\y in {1/1,2/+7,3/-9,4/-5}
{
    \node at (\x,1){$\y$};
}
\draw(.5,1.5)--(4.5,1.5)--(4.5,3.5);
\draw(3.5,1.5)--(3.5,.5)--(4.5,.5);

    \end{tikzpicture}
\end{center}
$\therefore\quad $商式是$\polynomial[reciprocal]{1,7,-9}$,余数是$-5$.
    \item 被除式缺一次项,用0补足,得
 \begin{center}
    \begin{tikzpicture}[yscale=.7]
\foreach \x/\y in {1/2,2/+5,3/-24,4/+0,5/+15, 6/2}
{
    \node at (\x,3){$\y$};
}
\foreach \x/\y in {2/+4,3/+18,4/-12, 5/-24}
{
    \node at (\x,2){$\y$};
}
\foreach \x/\y in {1/2,2/+9,3/-6,4/-12, 5/-9}
{
    \node at (\x,1){$\y$};
}
\draw(.5,1.5)--(5.7,1.5)--(5.7,3.5);
\draw(4.5,1.5)--(4.5,.5)--(5.7,.5);

    \end{tikzpicture}
\end{center}
$\therefore\quad $商式是$\polynomial[reciprocal]{2,9,-6,-12}$,余数是$-9$. 

\end{enumerate}
\end{solution}

\begin{example}
用综合除法计算下列各式,并把结果写成“$f(x)=g(x)q(x)+r(x)$”的形式:
\begin{enumerate}[(1)]
    \item $(\polynomial[reciprocal]{4,0,-7,7,5})\div \left(x-\frac{3}{2}\right)$
    \item $(\polynomial[reciprocal]{6,-5,-3,-1,4})\div (2x+1)$
\end{enumerate}
\end{example}

\begin{solution}
\begin{enumerate}[(1)]
    \item 
\begin{center}
    \begin{tikzpicture}[yscale=.7]
\foreach \x/\y in {1/4,2/+0,3/-7,4/-7,5/-5, 6/\frac{3}{2}}
{
    \node at (\x,3){$\y$};
}
\foreach \x/\y in {2/+6,3/+9,4/+3, 5/-6}
{
    \node at (\x,2){$\y$};
}
\foreach \x/\y in {1/4,2/+6,3/+2,4/-4, 5/-11}
{
    \node at (\x,1){$\y$};
}
\draw(.5,1.5)--(5.7,1.5)--(5.7,3.5);
\draw(4.5,1.5)--(4.5,.5)--(5.7,.5);

    \end{tikzpicture}
\end{center}
$\therefore\quad \polynomial[reciprocal]{4,0,-7,7,5}=\left(x-\frac{3}{2}\right)(\polynomial[reciprocal]{4,6,2,-4})-11$.

\item $2x+1$就是$2\left(x+\frac{1}{2}\right)$,先将
$\polynomial[reciprocal]{6,-5,-3,-1,4}$除以$x+\frac{1}{2}$.
\begin{center}
    \begin{tikzpicture}[yscale=1]
\foreach \x/\y in {1/6,2/-5,3/-3,4/-1,5/+4, 6/-\frac{1}{2}}
{
    \node at (\x,3){$\y$};
}
\foreach \x/\y in {2/-3,3/+4,4/-\frac{1}{2}, 5/+\frac{3}{4}}
{
    \node at (\x,2){$\y$};
}
\foreach \x/\y in {1/6,2/-8,3/+1,4/-\frac{3}{2}, 5/+\frac{19}{4}}
{
    \node at (\x,1){$\y$};
}
\draw(.5,1.5)--(5.7,1.5)--(5.7,3.5);
\draw(4.5,1.5)--(4.5,.5)--(5.7,.5);

    \end{tikzpicture}
\end{center}
\[\begin{split}
    \therefore\quad \polynomial[reciprocal]{6,-5,-3,-1,4}&=\left(x+\frac{1}{2}\right)\left(\polynomial[reciprocal]{6,-8,1,-\frac{3}{2}}\right)+\frac{19}{4}\\
    &=2\cdot \left(x+\frac{1}{2}\right)\cdot \frac{1}{2}\left(\polynomial[reciprocal]{6,-8,1,-\frac{3}{2}}\right)+\frac{19}{4}\\
    &=(2x+1)\left(\polynomial[reciprocal]{3,-4,\frac{1}{2},-\frac{3}{4}}\right)+\frac{19}{4}\\
\end{split}\]

\end{enumerate}
\end{solution}

由例9.2的第(2)小题可知,$f(x)$除以一般的一元一次式$px\pm q$,也可以使用综合除法:先将$f(x)$除以$x\pm\frac{q}{p}$,所得的商式除以$p$就是所求的商式,所得的余数就是所求的余数.

\begin{ex}
    用综合除法计算(第1—3题):
\begin{enumerate}
    \item $(x^{3}+6x^{2}-11x-14)\div(x-3).$
    \item $\left(x^{5}-4x^{3}-8\right)\div\left(x-2\right).$
    \item $(3x^{4}+7x^{3}-15x-20)\div(x+2).$
\end{enumerate}

用综合除法计算下列各式,并且把所得的结果写成
“$f(x)=g(x)q(x)+r(x)$”的形式(第4—6题):
\begin{enumerate}\setcounter{enumi}{3}
    \item $( x^{6}+ 1) \div ( x+ 1) .$
    \item $(27x^{3}-10)\div(3x-2).$
    \item $\left(20x^{5}+9x^{4}-8x^{3}+12x^{2}-35x-12\right)\div\left(5x+6\right).$
\end{enumerate}
\end{ex}





\section{余数定理}

设有多项式 $f(x)=x^3-7x^2+12x+27$, 那么 $f(5)=5^3 -7\times5^{2}+12\times5+27=125-175+60+27=37;$另一方面,如果把这个多项式除以 $x-5$, 求余数,那么用综合除法可得
\begin{center}
    \begin{tikzpicture}[yscale=.7]
\foreach \x/\y in {1/1,2/-7,3/+12,4/+27,5/5}
{
    \node at (\x,3){$\y$};
}
\foreach \x/\y in {2/+5,3/-10,4/+10}
{
    \node at (\x,2){$\y$};
}
\foreach \x/\y in {1/1,2/-2,3/+2,4/+37}
{
    \node at (\x,1){$\y$};
}
\draw(.5,1.5)--(4.7,1.5)--(4.7,3.5);
\draw(3.5,1.5)--(3.5,.5)--(4.7,.5);

    \end{tikzpicture}
\end{center}
我们发现,所得的余数正好也是 37. 这就是说,多项式$f(x)=
x^3-7x^2+12x+27$ 除以 $x-5$ 所得的余数正好等于$f(5)$.

对一般的多项式,有下面的重要定理:

\begin{thm}{余数定理}
    多项式$f(x)$除以$x-b$所得的余数等于$f(b)$.
\end{thm}

\begin{proof}
    设多项式$f(x)$除以$x-b$所得的商式为$q(x)$,余数为$r$,则有
\[f(x)=(x-b)\cdot q(x)+r\]
用$x=b$代入等式的两边,得
\[f(b)=(b-b)\cdot q(b)+r\]
由此即得余数$r=f(b)$.
\end{proof}

根据余数定理\footnote{此定理又叫做余式定理、剩余定理或裴蜀定理. 裴蜀(\'{E}tienne B\'{e}zout, 1730—1783),法国数学家.},既然多项式$f(x)$除以$x-b$所得的余数$r$等于$f(x)$在$x=b$时的值$f(b)$,那么$r$就可以由$f(b)$来求得,反过来,$f(b)$也可以由$r$来求得.

\begin{example}
    设$f(x)=x^8+3$,求$f(x)$除以$-1$所得的余数.
\end{example}

\begin{solution}
    根据余数定理,所求的余数等于$f(-1)=(-1)^8+3=4$.
\end{solution}

\begin{example}
    设$f(x)=x^5-12x^3+15x-8$,求$f(6)$.
\end{example}

\begin{solution}
    用综合除法求$f(x)$除以$x-6$所得的余数:
\begin{center}
    \begin{tikzpicture}[xscale=1.5, yscale=.7]
\foreach \x/\y in {1/1,2/+0,3/-12,4/+0,5/+15, 6/-8, 7/6}
{
    \node at (\x,3){$\y$};
}
\foreach \x/\y in {2/+6,3/+36,4/+144, 5/+864, 6/+5274}
{
    \node at (\x,2){$\y$};
}
\foreach \x/\y in {1/1,2/+6,3/+24,4/+144, 5/+879, 6/+5266}
{
    \node at (\x,1){$\y$};
}
\draw(.5,1.5)--(6.7,1.5)--(6.7,3.5);
\draw(5.5,1.5)--(5.5,.5)--(6.7,.5);
    \end{tikzpicture}
\end{center}

根据余数定理,余数5266等于$f(6)$,所以
$f(6)=5266$. 
\end{solution}

\begin{ex}
\begin{enumerate}
 \item 设$f(x)=5x^4-x^2+6$, 求$f(x)$除以$x-1$所得的余数.
\item 设$f(x)=x^4-3x^{3}+6x^{2}-10x+9$, 求$f(4)$.
\item 已知$f( x) = 16x^4- 14x^{3}- 15x^{2}- 24x+ 38$, 求$f\left ( \frac 32\right )$.
\item 设$f(x)=x^6+a^6$, 求$f(x)$除以$x-ai$所得的余数.
\end{enumerate}
\end{ex}

\section{因式定理}

从余数定理可以推出一个重要的定理——因式定理。

\begin{thm}
 {因式定理} 多项式$f(x)$有一个因式$x-b$的充要条件是
$f(b)=0$.   
\end{thm}

\begin{proof}
\begin{enumerate}[(1)]
\item 充分性. 设 $f(b)=0$, 则根据余数定理, $f(x)$除以$x-b$所得的余数也等于 0. 因此$f(x)$有一个因式$x-b$.
\item 必要性. 设 $f(x)$有一个因式 $x-b$, 则 $f(x)$除以 $x-b$
所得的余数等于 0. 根据余数定理,有$f(b)=0$.
\end{enumerate}
\end{proof}

\begin{example}
    求证$n$为任何正整数时,$x^n-a^n$都有因式$x-a$.
\end{example}

\begin{proof}
    设$f(x)=x^n-a^n$, 那么$f(a)=a^n-a^n=0$. 根据因
式定理, $x^*-a^*$有因式$x-a$.
\end{proof}

\begin{example}
    $m$为何值时,多项式$f(x)=x^5-3x^4+8x^3+11x+$
$m$能被$x-1$整除?
\end{example}

\begin{solution}
    $f(x)$能被 $x- 1$ 整除, 就是 $f(x)$有因式 $x-1$. 根据因式定理,充要条件是$f(1)=0$, 即$1-3+8+11+m=0$. 由此可得$m=-17$.
\end{solution}

\begin{ex}
\begin{enumerate}
\item 不用除法,求证多项式$x^4-5x^3-7x^2+15x-4$有因式
$x-1$.
\item 求证$n$为正偶数时, $x^n-a^n$有因式$x+a$; $n$为正奇数时, $x^n+a^{n}$有因式$x+a$.
\item 求证 $x^{in}-1\; (n\in \N)$有因式 $x-i$, 又有因式 $x+i$. 
\item 已知$f(x)=x^3-8x+\ell$有因式$x+2$, 确定$\ell$的值.
\end{enumerate}    
\end{ex}

\section{利用综合除法、因式定理来分解因式}

设有多项式 $x^6+x^4-x^2-1$, 我们把它在复数集$\mathbb{C}$ 中分
解因式,得
\[\begin{split}
    x^{6}+x^{4}-x^{2}-1&=x^{4}(x^{2}+1)-(x^{2}+1)\\
    &=(x^{4}-1)(x^{2}+1)\\
    &=(x^{2}-1)(x^{2}+1)(x^{2}+1)\\
    &=(x+1)(x-1)(x+i)^{2}(x-i)^{2}.
\end{split}\]

这个一元六次式有六个一次因式,其中有两个相同因式
$x+i$, 两个相同因式$x-i$.

关于复系数一元$n$次多项式的因式分解,有下面的定理: 

\begin{thm}
{定理1} 任何一个复系数一元 $n$ 次多项式 $f(x)$有且仅有
$n$ 个 一 次 因 式 $x- x_i\; ( i= 1, 2, \cdots , n)$, 把其中相同的因式的积用幂表示后, $f(x)$就具有唯一确定\footnote{这里所说的“唯一确定”,不考虑各一次因式的书写顺序,也不考虑常数因子. 例如,我们把$4x^2-16=(2x+4)(2x-4)$与$4x^2-16=4(x-2)(x+2)$等等看成同一种分解形式.}的因式分解的形式:
\begin{equation}
    f( x) = a_{x}( x- x_{1}) ^{k_{1}}( x- x_{2}) ^{k_{2}}\cdots ( x- x_{m}) ^{k_{m}} \tag{*}
\end{equation}   
其中 $k_1,k_2,\ldots,k_m\in \N$, 且 $k_1+k_2+\cdots+k_m=n$, 复数 $x_1,x_2,\ldots,x_{m}$两两不等.
\end{thm}

这个定理的证明超出中学数学范围,本书从略。

我们把分解结果(*)中的$x-x_i\; (i=1,2,\ldots,m)$叫做\textbf{多项式$f(x)$的$k_i$重一次因式}. 例如:多项式$x^2-6x+9$有 2 重一次因式 $x-3$; 多项式$(x-4)(x+2)^2(x-5)^3$ 有 1 重一次因式$x- 4$, 2重一次因式$x+ 2$, 3重一次因式$x-5$.

由定理 1 可以得到:

\begin{thm}
{推论} 如果 $x-a,x-b\; (a\neq0)$都是复系数一元$n$次多项式 $f(x)$的因式,那么它们的积$(x-a)(x-b)$也是 $f(x)$的因式.    
\end{thm}

\begin{proof}
因为 $f(x)$的分解结果(*)是唯一确定的,所以 $a$ 一定等于某个$x_i,b$一定等于某个$x_j\; (i=1,2,\ldots,m,\; j=1,2,\ldots,m, \text{ 且 } i\neq j)$, 即$(x-a)(x-b)=(x-x_{i})(x-x_j)$, 由此可见, $(x-x_i)(x-x_j)$是 $f(x)$的因式.    
\end{proof}

对于一个任意的复系数一元$n$次多项式$f(x)$, 要求出它的一次因式,没有一般的方法. 但是,如果 $f(x)$是整系数多项式,那么进一步运用下列定理,就能使我们较快地求得它的形如$x-\frac qp$(其中$p,q$是互质的整数)的因式,或者确定它没有这种形式的因式.

\begin{thm}
    {定理2} 如果整系数多项式$f(x)=a_nx^n+a_{n-1}x^{n-1}+\cdots+a_{1}x+a_{0}$有因式 $x-\frac qp$(其中 $p,q$ 是互质的整数),那么 $p$一定是首项系数$a$的约数,$q$一定是末项系数$a_0$的约数.
\end{thm}

例如,$15x^2-17x+4$ 有因式 $3x-1$, $5x-4$, 即$3\left(x-\frac{1}{3}\right)$, $5\left(x-\frac{4}{5}\right)$,3与5都是首项系数15的约数,1与4都是末项
系数4的约数. 又如,如果$2x^4-x^3-13x^2-x-15$有$x-\frac qp$形式的因式(其中$p,q$是互质的整数,下同),那么$p$只可能是$1,2$,$q$只可能是$\pm1,\pm3,\pm5,\pm15$.

要注意定理中“$p$ 是$a_n$的约数,$q$ 是$a_0$的约数”只是“整系数多项式$a_nx^n+a_{n-1}x^{n-1}+\cdots+a_1x+a_0$有因式$x-\frac qp$”的必要条件,而不是充分条件(为什么).

下面证明定理 2.

\begin{proof}
因为 $f(x)$有因式 $x-\frac qp$, 所以 $f\left(\frac qp\right)=0$, 即
$$a_{n}\Big(\frac{q}{p}\Big)^{n}+a_{n-1}\Big(\frac{q}{p}\Big)^{n-1}+\cdots+a_{1}\Big(\frac{q}{p}\Big)+a_{0}=0$$
把第二项起的各项移到右边,并将两边都乘以 $p^{n-1}$, 得
$$\frac{a_nq^n}p=-(a_{n-1}q^{n-1}+\cdots+a_1qp^{n-2}+a_0p^{n-1})$$

等式的右边是一个整数,所以$\frac{a_nq^n}p$也是一个整数,即$p$能整除$a_nq^n$. 但因$p,q$互质,所以$p$的任何一个质因数都不是$q$ 的约数,从而也不是$q^{n}$的约数\footnote{例如:$p=2\x 5=10$, $q=3\x 7=21$,$p$的任何一个质因数(2或5)都不是$q$的约数,从而也不是$q^n=3^n\x 7^n$的约数.}。由此可知,
$p$一定是$a_n$的约数。

同理,把上面的等式写成
$$\frac{a_{0}p^{n}}{q}=-\left(a_{n}q^{n-1}+a_{n-1}q^{n-2}p+\cdots+a_{1}p^{n-1}\right)$$
可以证明$q$一定是$a_0$的约数.
\end{proof}

\begin{thm}
    {推论} 如果首项系数为1的整系数多项式$f(x)=x^{n}+ a_{n-1}x^{n-1}+\cdots+a_1x+a_0$有因式 $x-q$, 其中 $q\in\mathbb{Z}$, 那么 $q$一定是常数项$a_0$的约数。
\end{thm}

利用定理 2 及其推论,我们可以较快地确定一个整系数
一元一次式是不是某整系数一元$n$次多项式的因式.

\begin{example}
    把$f(x)=x^3+x^2-10x-6$ 分解因式\footnote{如果没有特别说明,本章中所说的因式分解,都是指在复数集$\mathbb{C}$中的因
    式分解.}
\end{example}

\begin{analyze}
    先考虑 $x-q\; (q\in \Z)$形式的因式,因为 $f(x)$是首项系数为 1 的整系数多项式,根据定理 2 的推论,可能出现的$x-q$这样的因式有 $x\pm1$, $x\pm2$, $x\pm3$, $x\pm6$.

判断$x-1$, $x+1$是不是$f(x)$的因式时,只要根据因式定理,计算 $f(1)$, $f(-1)$是不是等于零就可以了. 因为 $f(1)=-14\neq0$, $f(-1)=4\neq0$, 所以 $x-1$, $x+1$ 都不是 $f(x)$的因式.

判断$x-2,x+2,\ldots$是不是$f(x)$的因式时,可以计算
$f(2),f(-2),\ldots$是不是等于零. 用综合除法,由于
\begin{center}
\begin{tikzpicture}[yscale=.7]
\begin{scope}
    \foreach \x/\y in {1/1,2/+1,3/-10,4/-6,5/2}
{
    \node at (\x,3){$\y$};
}
\foreach \x/\y in {2/+2,3/+6}
{
    \node at (\x,2){$\y$};
}
\foreach \x/\y in {1/1,2/+3,3/-4}
{
    \node at (\x,1){$\y$};
}
\draw(.5,1.5)--(4.5,1.5)--(4.5,3.5);
\end{scope}
\begin{scope}[xshift=6cm]
\foreach \x/\y in {1/1,2/+1,3/-10,4/-6,5/-2}
{
    \node at (\x,3){$\y$};
}
\foreach \x/\y in {2/-2,3/+2}
{
    \node at (\x,2){$\y$};
}
\foreach \x/\y in {1/1,2/-1,3/-8}
{
    \node at (\x,1){$\y$};
}
\draw(.5,1.5)--(4.5,1.5)--(4.5,3.5);
\end{scope}
\end{tikzpicture}
\end{center}
(上面左式中$-4\x2$不是$-6$的相反数,右式中$-8\x(-2)$不是$-6$的相反数,已经说明相应的余数都不是零,所以不必继续演算了.)可见$x-2$, $x+2$都不是$f(x)$的因式. 但
\begin{center}
    \begin{tikzpicture}[yscale=.7]
\foreach \x/\y in {1/1,2/+1,3/-10,4/-6,5/3}
{
    \node at (\x,3){$\y$};
}
\foreach \x/\y in {2/+3,3/+12, 4/+6}
{
    \node at (\x,2){$\y$};
}
\foreach \x/\y in {1/1,2/+4,3/+2, 4/0}
{
    \node at (\x,1){$\y$};
}
\draw(.5,1.5)--(4.7,1.5)--(4.7,3.5);
\draw(3.5,1.5)--(3.5,.5)--(4.7,.5);
\end{tikzpicture}
\end{center}
可知$x-3$是$f(x)$的因式. 所以
\[x^3+x^2-10x-6=(x-3)(x^2+4x+2)\]

因为方程$x^2+4x+2=0$的两个根是$-2\pm\sqrt{2}$,于是,
\[x^3+x^2-10x-6=(x-3)\left(x+2+\sqrt{2}\right)\left(x+2-\sqrt{2}\right)\]

解答时,只需写出结果是因式的试验过程,其他过程不必写出.
\end{analyze}

\begin{solution}
    \begin{center}
        \begin{tikzpicture}[yscale=.7]
    \foreach \x/\y in {1/1,2/+1,3/-10,4/-6,5/3}
    {
        \node at (\x,3){$\y$};
    }
    \foreach \x/\y in {2/+3,3/+12, 4/+6}
    {
        \node at (\x,2){$\y$};
    }
    \foreach \x/\y in {1/1,2/+4,3/+2, 4/0}
    {
        \node at (\x,1){$\y$};
    }
    \draw(.5,1.5)--(4.7,1.5)--(4.7,3.5);
    \draw(3.5,1.5)--(3.5,.5)--(4.7,.5);
    \end{tikzpicture}
    \end{center}
\[\begin{split}
\therefore\quad x^3+x^2-10x-6&=(x-3)(x^2+4x+2)\\
&=(x-3)\left(x+2+\sqrt{2}\right)\left(x+2-\sqrt{2}\right)
\end{split}\]
\end{solution}

\begin{example}
把$f(x)=\polynomial[reciprocal]{2,-1,-13,-1,-15}$分解因式。
\end{example}

\begin{analyze}
$f(x)$首项系数不是1,根据定理2,可试验$x\pm1$, $x\pm3$, $x\pm5$, $x\pm15$, $x\pm\frac{1}{2}$, $x\pm\frac{3}{2}$, $x\pm\frac{5}{2}$, $x\pm\frac{15}{2}$. 

因为$f(1)\ne 0$, $f(-1)\ne 0$,所以$x+1$, $x-1$不是$f(x)$的因式. 但
\begin{center}
    \begin{tikzpicture}[yscale=.7]
\foreach \x/\y in {1/2,2/-1,3/-13,4/-1,5/-15, 6/3}
{
    \node at (\x,3){$\y$};
}
\foreach \x/\y in {2/+6,3/+15, 4/+6, 5/+15}
{
    \node at (\x,2){$\y$};
}
\foreach \x/\y in {1/2,2/+5,3/+2, 4/+5, 5/0}
{
    \node at (\x,1){$\y$};
}
\draw(.5,1.5)--(5.7,1.5)--(5.7,3.5);
\draw(4.5,1.5)--(4.5,.5)--(5.7,.5);
\end{tikzpicture}
\end{center}

所以,$f(x)=(x-3)(2x^3+5x^2+2x+5)$.

继续分解$2x^3+5x^2+2x+5$. 这个多项式的首项系数是2,末项系数是5,所以只要试验$x\pm 1$, $x\pm 5$, $x\pm \frac{1}{2}$, $x\pm \frac{5}{2}$ 就可以了.但因$x\pm 1$不是原来多项式$f(x)$的因式,所以也不是这个多项式的因式. 由
\begin{center}
    \begin{tikzpicture}[yscale=.7]
\foreach \x/\y in {1/2,2/+5,3/+2,4/+5,5/-\frac{5}{2}}
{
    \node at (\x,3){$\y$};
}
\foreach \x/\y in {2/-5,3/+0, 4/-5}
{
    \node at (\x,2){$\y$};
}
\foreach \x/\y in {1/2,2/+0,3/+2, 4/0}
{
    \node at (\x,1){$\y$};
}
\draw(.5,1.5)--(4.7,1.5)--(4.7,3.5);
\draw(3.5,1.5)--(3.5,.5)--(4.7,.5);
\end{tikzpicture}
\end{center}
得:
\[\begin{split}
\polynomial[reciprocal]{2,5,2,5}&=\left(x+\frac{5}{2}\right)(2x^2+2)\\
&=(2x+5)(x^2+1)
\end{split}\]
(实际上,利用分组分解法也容易得到这个结果.)

$x^2+1$在复数集$\mathbb{C}$中还能继续分解因式,所以
\[\begin{split}
    \polynomial[reciprocal]{2,-1,-13,-1,-15}&=(x-3)(2x+5)(x^2+1)\\
    &=(x-3)(2x+5)(x+i)(x-i)
\end{split}\]
\end{analyze}

\begin{solution}
\begin{center}
    \begin{tikzpicture}[yscale=.7]
\begin{scope}
\foreach \x/\y in {1/2,2/-1,3/-13,4/-1,5/-15, 6/3}
{
    \node at (\x,3){$\y$};
}
\foreach \x/\y in {2/+6,3/+15, 4/+6, 5/+15}
{
    \node at (\x,2){$\y$};
}
\foreach \x/\y in {1/2,2/+5,3/+2, 4/+5, 5/0}
{
    \node at (\x,1){$\y$};
}
\draw(.5,1.5)--(5.7,1.5)--(5.7,3.5);
\draw(4.5,1.5)--(4.5,.5)--(5.7,.5);    
\end{scope}
\begin{scope}[xshift=7cm]
    \foreach \x/\y in {1/2,2/+5,3/+2,4/+5,5/-\frac{5}{2}}
{
    \node at (\x,3){$\y$};
}
\foreach \x/\y in {2/-5,3/+0, 4/-5}
{
    \node at (\x,2){$\y$};
}
\foreach \x/\y in {1/2,2/+0,3/+2, 4/0}
{
    \node at (\x,1){$\y$};
}
\draw(.5,1.5)--(4.7,1.5)--(4.7,3.5);
\draw(3.5,1.5)--(3.5,.5)--(4.7,.5);
\end{scope}
\end{tikzpicture}
\end{center}
\[\begin{split}
\therefore\quad \polynomial[reciprocal]{2,-1,-13,-1,-15}&=(x-3)(\polynomial[reciprocal]{2,5,2,5})\\
&=(x-3)(2x+5)(x^2+1)\\
&=(x-3)(2x+5)(x+i)(x-i)
\end{split}\]
\end{solution}

\begin{ex}
\begin{enumerate}
    \item 判断下列各命题的真假,并说明理由:
\begin{enumerate}[(1)]
\item 如果整数$p,q$互质,且$p$为整系数多项式
    $$f(x)=a_nx^n+a_{n-1}x^{n-1}+\cdots +a_1x+a_0$$
    的首项系数$a_n$的约数,$q$为末项系数$a_0$的约数,那么$x-\frac{q}{p}$一定是$f(x)$的因式;
\item 如果多项式$f(x)=g(x)q(x)$,其中$g(x)$, $q(x)$也是多项式,且$x-a$不是$f(x)$的因式,那么$x-a$也不是$q(x)$的因式.
\end{enumerate}
\item 把下列各式分解因式:
\begin{enumerate}[(1)]
    \item $\polynomial[reciprocal]{1,1,-10,8}$
    \item $\polynomial[reciprocal]{2,-9,1,12}$
    \item $\polynomial[reciprocal]{1,3,-3,-12,-4}$
    \item $\polynomial[reciprocal]{4,-13,10,-42,20}$
\end{enumerate}
\end{enumerate}
\end{ex}

\section*{习题一}
\begin{enumerate}
    \item 设$f(x)=\polynomial[reciprocal]{1,-i,2,-2i}$,求:
\begin{multicols}{3}
\begin{enumerate}[(1)]
    \item $f(0)$
    \item $f(2)$
    \item $f(i)$
    \item $f\left(\sqrt{2}i\right)$
    \item $f\left(-\sqrt{2}i\right)$
\end{enumerate}
\end{multicols}

\item 用综合除法求商式及余数:
\begin{enumerate}[(1)]
    \item $(\polynomial[reciprocal]{3,0,-50,0,14})\div (x-4)$
    \item $(\polynomial[reciprocal]{1,0,-15,-10,28})\div (x+2)$
    \item $(\polynomial[reciprocal]{3,-5,8,-1,-40})\div \left(x-\frac{2}{3}\right)$
    \item $(\polynomial[reciprocal]{1,6,9,0,-14,8})\div (x+4)$
    \item $(\polynomial[reciprocal]{5,-6,7,8})\div (5x+4)$
    \item $(\polynomial[reciprocal]{4,6,-8,5,0})\div (2x+5)$
    \item $(\polynomial[reciprocal, var=y]{1,-8,16,-25})\div (y-6)$
    \item $(\polynomial[reciprocal, var=a]{2,0,-1,9,-12})\div (a-2)$
    \item $(\polynomial[reciprocal, var=t]{8,14,-3,-35,18})\div (4t-3)$
    \item $(\polynomial[reciprocal, var=m]{25,0,36,-14,8,8})\div (5m+2)$
    \item $(x^3-8x^2y+8y^3)\div (x-2y)$ 
    
    (提示:把$x$看作多项式的元,$y$看作系数)
    \item $(10m^5-7m^4n-16m^3n^2+23mn^4-21n^5)\div (2m-3n)$
\end{enumerate}

\item \begin{enumerate}[(1)]
    \item 用综合除法求$f(x)=x^5-2x^4-5x^3+9x^2-14x+29$
    除以$x-3$所得的余数;
    \item 对于上题中的$f(x)$,用代入法求$f(3)$;
    \item 比较第(1),(2)小题所得的结果.
\end{enumerate}

\item \begin{enumerate}[(1)]
    \item 用综合除法求$f(x)=x^8-23x^4+19$除以$x+1$所得的余数;
\item 对于第(1)小题中的$f(x)$,用代入法求$f(-1)$;
\item 比较第(1),(2)小题所得的结果.
\end{enumerate}

\item \begin{enumerate}[(1)]
    \item 已知$f(x)=\polynomial[reciprocal]{1,2,-19,19,-25,-70}$,利用综合除法求$f(3)$;
    \item 已知$f(x)=\polynomial[reciprocal]{9,3,-32,10,27,-6}$,利用综合除法求$f\left(\frac{4}{3}\right)$;
\end{enumerate}

\item 不用除法,求下列各式除以$x-y$所得的余式以及除以$x+y$所得的余式:
\begin{multicols}{2}
\begin{enumerate}[(1)]
    \item $x^7+y^7$
    \item $x^7-y^7$
    \item $x^8+y^8$
    \item $x^8-y^8$
\end{enumerate}
\end{multicols}
\item 不用除法,求证:
\begin{enumerate}[(1)]
    \item $\polynomial[reciprocal]{5,0,4,-11,9,-3}$有因式$x-1$
    \item $\polynomial[reciprocal]{1,-8,-6,9,6}$有因式$x+1$
    \item $(x-1)^5-1$有因式$x-2$
    \item $(x+3)^{2n}-(x+1)^{2n}$(其中$n\in\N$)有因式$x+2$
\end{enumerate}

\item 用因式定理证明$(2a+b)^n-a^n$(其中$n\in\N$)有因式$a+b$.
\item 用因式定理证明$x^{4n+2}+a^{4n+2}$(其中$n\in\N$)有因式$x-ai$,又有因式$x+ai$.
\item 用因式定理证明$(a-b)^3+(b-c)^3+(c-a)^3$有一次因式$a-b$, $b-c$, $c-a$.
\item \begin{enumerate}[(1)]
\item 已知$f(x)=x^4+5x^3-mx-28$有因式$x-2$. 确定$m$的值;
\item 已知$f(x)=2x^3-9x^2+n$有因式$2x+3$,确定$n$的值.    
\end{enumerate}
\item 已知$f(x)=a_nx^n+a_{n-1}x^{n-1}+\cdots+a_1x+a_0$, 求证:
\begin{enumerate}[(1)]
    \item $x-1$成为$f(x)$的一次因式的充要条件是$a_n+a_{n-1}+\cdots +a_1+a_0=0$;
    \item $x+1$成为$f(x)$的一次因式的充要条件是$a_n-a_{n-1}+\cdots +(-1)^{n-1}a_1+(-1)^n a_0=0$.
\end{enumerate}

\item 已知$n\; (n\ge 1)$次多项式$f(x)=a_nx^n+a_{n-1}x^{n-1}+\cdots+a_1x+a_0$,且所有$a_i\; (i=0,1,\ldots,n)$都是非负实数,求证$x-b$(其中$b\in\R$)成为$f(x)$的一次因式的必要条件是$b\le 0$.
\item 把下列各式分解因式:
\begin{multicols}{2}
\begin{enumerate}[(1)]
    \item $\polynomial[reciprocal]{1,-4,-17,60}$
    \item $\polynomial[reciprocal]{1,0,-8,8}$
    \item $\polynomial[reciprocal]{6,1,7,4}$
    \item $\polynomial[reciprocal]{3,1,4,-4}$
    \item $\polynomial[reciprocal]{4,4,-9,-1,2}$
\end{enumerate}
\end{multicols}

\item 把下列各式分解因式(在有理数集$\Q$中):
\begin{enumerate}[(1)]
    \item $\polynomial[reciprocal]{4,-3,15,0,-1}$
    \item $3a^5-5a^4b+a^3b^2-8a^2b^3+3ab^4+2b^5$
\end{enumerate}
\end{enumerate}

\section{一元$n$次方程的根的个数}
如果$f(x)=a_nx^n+a_{n-1}x^{n-1}+\cdots+a_1x+a_0\quad (a_n\ne 0)$
是复系数一元$n$次多项式,那么方程$f(x)=0$,即
\[a_nx^n+a_{n-1}x^{n-1}+\cdots+a_1x+a_0=0\]
叫做\textbf{复系数一元$n$次方程}. 当$n>2$时,通常也叫做\textbf{复系数高次方程}. 我们过去学过的二项方程是复系数高次方程的特殊情形.

类似地,如果$f(x)$是实系数(或有理系数、整系数等)一元$n$次多项式,那么方程$f(x)=0$叫做\textbf{实系数}(或\textbf{有理系数}、
\textbf{整系数}等)\textbf{一元$n$次方程}. 当$n>2$时,通常也叫做\textbf{实系数}(或\textbf{有理系数}、\textbf{整系数}等)\textbf{高次方程}. 很明显,实系数、有理系数、整系数一元$n$次方程都是复系数一元$n$次方程的特殊情形. 在本章中所提到的一元$n$次方程,如果不特别说明,都是指复系数一元$n$次方程.

复系数一元$n$次方程$f(x)=0$的根与多项式$f(x)$的一次因式之间有着极为密切的关系. 首先,根据因式定理,我们有:

\begin{thm}
    {定理1} 一元$n$次方程$f(x)=0$有一个根$x=b$的充要条件是多项式$f(x)$有一个一次因式$x-b$.
\end{thm}

在第9.5节中,我们还知道,任何一个复系数一元$n$次多项式$f(x)$具有唯一确定的因式分解的形式:
\[f(x)=a_n(x-x_1)^{k_1}(x-x_2)^{k_2}\cdots (x-x_m)^{k_m}\]
其中$k_1,k_2,\ldots,k_m\in\N$,且$k_1+k_2+\cdots +k_m=n$,复数$x_1,x_2,\ldots,x_m$两两不等.由定理1,可知$x_1,x_2,\ldots,x_m$都是方程$f(x)=0$的根,且$f(x)=0$没有其他的根. 由于$x-x_i\; (i=1,2,\ldots,m)$是多项式$f(x)$的$k_i$重一次因式,我们相应地把$x_i$叫做\textbf{方程
$f(x)=0$的$k_i$重根}.

例如:方程$x^2-6x+9=0$,即$(x-3)^2=0$有2重根3;方程$(x-4)(x+2)^2(x-5)^3=0$有1重根4, 2重根$-2$, 3重根5.这两个方程的解集可以分别表示为$\{3_{(2)}\}$, $\{4,-2_{(2)},5_{(3)}\}$其中元素右边下标括号中的数$k\;(k\ge 2)$表示这个元素是相应方程的$k$重根.例如,元素5右边的下标(3),表示5是方程$(x-4)(x+2)^2(x-5)^3=0$的3重根,即此方程有3个相等的根5,但在解集中5只能算一个元素.

复系数一元$n$次方程有多少个根呢?由第9.5节的定理1,容易得到:

\begin{thm}
{定理2} 复系数一元$n$次方程在复数集$\mathbb{C}$中有且仅有$n$个根($k$重根算作$k$个根).    
\end{thm}

\begin{example}
    求方程
$f(x)=x^4+3x^3-2x^2-9x+7=0$
在复数集$\mathbb{C}$中的解集.
\end{example}

\begin{solution}
    方程$f(x)=0$的系数$1,3,-2,-9,7$的和为0,即$f(1)=0$,可知1是原方程的根,从而$x-1$是多项式$f(x)$的一次因式. 利用综合除法,得
\begin{center}
\begin{tikzpicture}[yscale=.7]
\foreach \x/\y in {1/1,2/+3,3/-2,4/-9,5/+7, 6/1}
{
    \node at (\x,3){$\y$};
}
\foreach \x/\y in {2/1,3/+4, 4/+2, 5/-7}
{
    \node at (\x,2){$\y$};
}
\foreach \x/\y in {1/1,2/+4,3/+2, 4/-7, 5/1}
{
    \node at (\x,1){$\y$};
}
\foreach \x/\y in {2/1,3/+5, 4/+7}
{
    \node at (\x,0){$\y$};
}
\foreach \x/\y in {1/1,2/+5,3/+7}
{
    \node at (\x,-1){$\y$};
}
\draw(.5,1.5)--(5.7,1.5)--(5.7,3.5);
\draw(.5,-.5)--(4.7,-.5)--(4.7,1.5);
% \draw(3.5,1.5)--(3.5,.5)--(4.7,.5);
\end{tikzpicture}
\end{center}

(说明:这里第一次除以$x-1$,所得商式的系数$1,4,2,-7$的和又为0,可知1又是方程$x^3+4x^2+2x-7=0$的根,所以利用综合除法,再将商式除以$x-1$,得到$x^2+5x+7$)\footnote{实际解题时,括号中的说明都可以省去.}.即
\[f(x)=(x-1)^2(x^2+5x+7)=0\]

这时商式$x^2+5x+7$已降为二次式了,解方程$x^2+5x+7=0$,得原方程的另外两个根
\[x=\frac{-5\pm\sqrt{3}i}{2}\]

由定理2,原方程有且仅有四个根.从而原方程在复数集$\mathbb{C}$中的解集是
\[\left\{1_{(2)}, \;\frac{-5+\sqrt{3}i}{2},\; \frac{-5-\sqrt{3}i}{2} \right\}\]
\end{solution}

由第9.5节的定理2及其推论,我们还可以得到:

\begin{thm}{定理3}
    如果既约分数$\frac{q}{p}$是整系数一元$n$次方程
\[a_nx^n+a_{n-1}x^{n-1}+\cdots +a_1x+a0=0\]
的根,那么$p$一定是$a_n$的约数,$q$一定是$a_0$的约数.
\end{thm}

\begin{thm}
 {推论1} 如果整系数一元$n$次方程的首项系数是1,那么这个方程的有理数根只可能是整数.   
\end{thm}

\begin{thm}
{推论2} 如果整系数一元$n$次方程有整数根,那么它一定是常数项的约数.    
\end{thm}

\begin{example}
    求方程$f(x)=2x^6+x^5-16x^4-6x^3+25x^2+20x+4=0$在复数集$\mathbb{C}$中的解集.
\end{example}

\begin{solution}
    原方程是一个整系数一元六次方程.由定理3,如果它有有理数根,只可能是:$\pm 1$, $\pm2$, $\pm4$, $\pm\frac{1}{2}$. 因为它的系数之和不为0,可知1不是它的根.利用综合除法,得
 \begin{center}
\begin{tikzpicture}[yscale=.6]
\foreach \x/\y in {1/2,2/+1,3/-16,4/-6,5/+25, 6/+20, 7/+4, 8/-1}
{
    \node at (\x,3){$\y$};
}
\foreach \x/\y in {2/-2,3/+1, 4/+15, 5/-9, 6/-16, 7/-4}
{
    \node at (\x,2){$\y$};
}
\foreach \x/\y in {1/2,2/-1,3/-15, 4/+9, 5/+16, 6/+4, 7/2}
{
    \node at (\x,1){$\y$};
}
\foreach \x/\y in {2/+4,3/+6, 4/-18, 5/-18, 6/-4}
{
    \node at (\x,0){$\y$};
}
\foreach \x/\y in {1/2,2/+3,3/-9,  4/-9, 5/-2, 6/2}
{
    \node at (\x,-1){$\y$};
}
\foreach \x/\y in {2/+4,3/+14, 4/+10, 5/+2}
{
    \node at (\x,-2){$\y$};
}
\foreach \x/\y in {1/2, 2/+7,3/+5, 4/+1}
{
    \node at (\x,-3){$\y$};
}
\foreach \x/\y in {2/-1,3/-3,  4/-1}
{
    \node at (\x,-4){$\y$};
}
\foreach \x/\y in {1/2,2/+6,3/+2}
{
    \node at (\x,-5){$\y$};
}

\draw(.5,1.5)--(7.7,1.5)--(7.7,3.5);
\draw(.5,-.5)--(6.7,-.5)--(6.7,1.5);
\draw(.5,-2.5)--(5.7,-2.5)--(5.7,-.5);
\draw(.5,-4.5)--(4.7,-4.5)--node[right]{$-\frac{1}{2}$}(4.7,-2.5);
\end{tikzpicture}
\end{center}

(说明:这里先除以$x+1$,得商式$2x^5-x^4-15x^3+9x^2+16x+4$,余数为0.方程$2x^5-x^4-15x^3+9x^2+16x+4=0$的有理数根只可能是$-1$,$\pm 2$, $\pm4$, $\pm\frac{1}{2}$.用心算可知$-1$不是它的根. 用综合除法除以$x-2$后,得商式$2x^4+3x^3-9x^2-9x-2$,
余数为0.方程$2x^4+3x^3-9x^2-9x-2=0$的有理数根只可能是$\pm 2$, $\pm\frac{1}{2}$. 用综合除法除以$x-2$后,得商式$2x^3+7x^2+5x+1$,余数为0.方程$2x^3+7x^2+5x+1=0$的系数都是正数,所以它没有正数根,它的有理数根只可能是$-\frac{1}{2}$. 用综合除法除以$x+\frac{1}{2}$后,得商式$2x^2+6x+2$,余数为0. $2x^2+6x+2$已经是二次式了.)即
\[f(x)=(x+1)(x-2)^2\left(x+\frac{1}{2}\right)(2x^2+6x+2)=0\]

解方程$2x^2+6x+2=0$,得原方程的另外两个根
\[x=\frac{-3\pm\sqrt{5}}{2}\]
从而原方程在复数集$\mathbb{C}$中的解集是
\[\left\{-1,\; 2_{(2)},\; -\frac{1}{2},\; \frac{-3+\sqrt{5}}{2},\; \frac{-3-\sqrt{5}}{2}\right\}\]
\end{solution}

\begin{example}
    求最简整系数方程(就是求一个整系数方程,并使最高次项系数取尽可能小的自然数)$f(x)=0$,已知它在复数集$\mathbb{C}$中的解集为$\left\{\frac{1}{2}_{(2)},\; i,\; -i\right\}$.
\end{example}

\begin{solution}
设所求的方程是
\[a\left(x-\frac{1}{2}\right)^2(x-i)(x+i)=0\quad (a\in\N, \text{ 且 } a\ne 0)\]
即
\[a\left(x^2-x+\frac{1}{4}\right)(x^2+1)=0\]

因为要求上式具有最简单的整系数,所以取$a=4$. 代入上式,得
\[4\left(x^2-x+\frac{1}{4}\right)(x^2+1)=0\]
即
\[\polynomial[reciprocal]{4,-4,5,-4,1}=0\]
\end{solution}

\begin{ex}
\begin{enumerate}
    \item (口答)在复数集$\mathbb{C}$中,下列方程有且仅有多少个根?
\begin{multicols}{2}
   \begin{enumerate}[(1)]
    \item $x^4+ 3x^2+ 4x+ 5= 0$
    \item $x^7= 1$
    \item $( x+ 1) ^{4}- ( x- 1) ^{4}= 0$
    \item $(x-1)^{2}(x-2)^{3}(x+3)^{4}=0$
\end{enumerate} 
\end{multicols}

\item \begin{enumerate}[(1)]
    \item 用综合除法验证 3是方程 $2x^3-5x^2-9x+18=0$ 的一个根;
    \item 把方程$2x^3-5x^2-9x+18=0$ 先化成
    $(x-x_{1})(ax^{2}+bx+c)=0$
    的形式,再化成
    $a(x-x_{1})(x-x_{2})(x-x_{3})=0$
    的形式.
\end{enumerate}
\item 求下列方程在复数集$\mathbb{C}$中的解集:
\begin{enumerate}[(1)]
    \item $3x^{3}- 11x^{2}+ 5x+ 3= 0$
    \item $6x^{4}+31x^{3}+25x^{2}-39x+9=0$
    \item $3x^{5}+4x^{4}-10x^{3}-14x^{2}+3x+6=0$
\end{enumerate}

\item 求最简整系数方程 $f(x)=0$, 已知它在复数集$\mathbb{C}$ 中的解集是:
\begin{multicols}{2}
\begin{enumerate}[(1)]
    \item $\left\{-1,-2,3\right\}$
    \item $\left\{-\frac{1}{2},\frac{2}{3},1\right\}$
    \item $\{-2,2_{(2)}\}$
    \item $\left\{1+i, 1-i, -\frac{\sqrt{3}}{2}, \frac{\sqrt{3}}{2}\right\}$
\end{enumerate}
\end{multicols}

\end{enumerate}
\end{ex}

\section{一元$n$次方程的根与系数的关系}
我们知道,如果一元二次方程
$ax^2+bx+c=0$
的两个根是$x_1,x_2$,那么根与系数之间有下列关系:
\[\begin{cases}
    x_1+x_2=-\frac{b}{a}\\
    x_1x_2=\frac{c}{a}
\end{cases}\]

一般地说,我们有如下的定理\footnote{此定理又叫韦达定理。韦达(Franciscus Vi\`{e}ta, 1540—1603),法国数学家.}:

\begin{thm}{定理}
如果一元$n$次方程$a_nx^n+a_{n-1}x^{n-1}+\cdots+a_1 x+a_0=0$在复数集$\mathbb{C}$中的根是$x_1,x_2,\ldots, x_n$,那么
\begin{equation}
\begin{cases}
x_1+x_2+\cdots+x_n=-\frac{a_{n-1}}{a_n}\\
x_1x_2+x_1x_3+\cdots+x_{n-1}x_n=\frac{a_{n-2}}{a_n}\\
x_1x_2x_3+x_1x_2x_4+\cdots+x_{n-2}x_{n-1}x_n=-\frac{a_{n-3}}{a_n}\\
\cdots\cdots\cdots\cdots\\
x_1x_2\cdots x_n=(-1)^n \frac{a_0}{a_n}
\end{cases}    \tag{*}
\end{equation}
\end{thm}

例如:当$n=3$时,有
\[\begin{cases}
    x_1+x_2+x_3=-\frac{a_2}{a_3}\\
    x_1x_2+x_1x_3+x_2x_3=\frac{a_1}{a_3}\\
    x_1x_2x_3=-\frac{a_0}{a_3}\\
\end{cases}\]
下面我们证明上述定理.

\begin{proof}
因为方程$a_nx^n+a_{n-1}x^{n-1}+\cdots +a_1x+a_0=0$的根是$x_1,x_2,\ldots ,x_n$.由上节定理1,可以把$f(x)=a_nx^n+a_{n-1}x^{n-1}+\cdots +a_1x+a_0$分解成$n$个一次因式与$a_n$的积:
\begin{equation}
    a_nx^n+a_{n-1}x^{n-1}+\cdots +a_1x+a_0=a_n(x-x_1)(x-x_2)\cdots (x-x_n) \tag{1}
\end{equation}

因为
\[\begin{split}
    (x-x_1)(x-x_2)\cdots (x-x_n) &=x^n-(x_1+x_2+\cdots +x_n)x^{n-1}\\
    &\qquad +(x_1x_2+x_1x_3+\cdots +x_{n-1}x_n)x^{n-2}\\
    &\qquad +\cdots +(-1)^n x_1x_2\cdots x_n
\end{split}\]
代入上面(1)式后,把每一项与$a_n$相乘,并将等号左边的多项式减去等号右边的多项式,所得的差$F(x)$是一个零多项式. 对$F(x)$进行整理,可知
\[\begin{split}
  F(x)&=[a_{n-1}+a_n(x_1+x_2+\cdots +x_n)]x^{n-1}\\
  &\qquad +\left[a_{n-2}-a_n(x_1x_2+x_1x_3+\cdots +x_{n-1}x_n)\right]x^{n-2}\\
  &\qquad +\cdots +\left[a_0-(-1)^n a_n x_1x_2\cdots x_n\right]
\end{split}\]

根据零多项式的定义,$F(x)$的系数都是0,所以   
\[\begin{cases}
    a_{n-1}+a_n(x_1+x_2+\cdots +x_n)=0\\
    a_{n-2}-a_n(x_1x_2+x_1x_3+\cdots +x_{n-1}x_n)=0\\
    \cdots \cdots \cdots \cdots \\
    a_0-(-1)^n a_n x_1x_2\cdots x_n=0
\end{cases}\]
由此即得
\[\begin{cases}
    x_1+x_2+\cdots +x_n=-\frac{a_{n-1}}{a_n}\\
   x_1x_2+x_1x_3+\cdots +x_{n-1}x_n=\frac{a_{n-2}}{a_n}\\
   \cdots \cdots \cdots \cdots \\
    x_1x_2\cdots x_n=(-1)^n \frac{a_{0}}{a_n}
\end{cases}\]
\end{proof}

这个定理的逆命题也成立,即对于任何一元$n$次方程
\begin{equation}
f(x)=a_nx^n+a_{n-1}x^{n-1}+\cdots +a_1x+a_0=0 \tag{*}
\end{equation}
如果有$n$个数$x_1,x_2,\ldots,x_n$满足(*)式,那么$x_1,x_2,\ldots,x_n$一定是方程$f(x)=0$的根.

\begin{example}
    已知方程$2x^3-5x^2-4x+12=0$有2重根,利用一元$n$次方程的根与系数的关系,求这个方程在复数集$\mathbb{C}$中的解集.
\end{example}

\begin{solution}
设原方程在$\mathbb{C}$中的解集为$\{\alpha_{(2)},\beta\}$,那么
\[\begin{cases}
    2\alpha +\beta =\frac{2}{5} & (1)\\
    \alpha^2 +2\alpha \beta =-2 & (2)\\
    \alpha^2 \beta =-6 & (3)\\
\end{cases}\]
(说明:这里有两个未知数、三个方程.我们可以选其中两个方程,求出满足这两个方程的,再代入另一个方程,如能满足,就是方程组的解,否则不是.)解(1),(2)两式组成的方程组,得
\[\begin{cases}
    \alpha=2\\ \beta =-\frac{3}{2}
\end{cases},\quad \text{或} \quad \begin{cases}
    \alpha=-\frac{1}{3}\\[1ex] \beta=\frac{19}{6}
\end{cases}\]

第一个解满足(3)式;第二个解不满足(3)式,应舍去. 所以原方程在$\mathbb{C}$中的解集为$\left\{2_{(2)},-\frac{3}{2}\right\}$.
\end{solution}

\begin{example}
    当且仅当$k$是什么数的时候,方程$x^3-6x^2+3x+k=0$的三个根成等差数列\footnote{本书中涉及等差、等比数列的问题,都限于在实数集内讨论.}?
\end{example}

\begin{solution}
设原方程在$\mathbb{C}$中的三个根成等差数列,并分别记作$a-d$, $a$, $a+d\; (d\ge 0)$,那么
\[\begin{cases}
    (a-d)+a+(a+d)=6\\
    a(a-d)+a(a+d)+(a+d)(a-d)=3\\
    a(a-d)(a+d)=-k
\end{cases}\]
整理后,得
\[\begin{cases}
    3a=6& (1)\\
    3a^2-d^2=3& (2)\\
    a^3-ad^2=-k & (3)\\
\end{cases}\]
由(1)式,得$a=2$;代入(2)式,得$d=3$; 再代入(3)式,便得
$k=10$.

这就是说,要使原方程的三个根成等差数列,$k$必须等于10.反过来,容易验证,当方程中的$k=10$时,$-1(=a-d)$,$2(=a)$,$5(=a+d)$三个数确实是原方程的根,且成等差数列. 所以当且仅当$k=10$时,原方程的三个根成等差数列.
\end{solution}

\begin{ex}
利用一元$n$次方程的根与系数的关系解下列各题:
\begin{enumerate}
\item 已知方程$6x^4+7x^3-36x^2-7x+6=0$的根中有三个是$-\frac{1}{2},\frac{1}{3},2$,求这个方程在复数集$\mathbb{C}$中的解集.
\item 已知方程$2x^3+x^2-8x-4=0$的根都是实数,且有两个互为相反数,求这个方程的解集.
\item 已知方程$x^3-9\sqrt{2}x^2+46x-30\sqrt{2}=0$的三个根成等差数列,求这个方程的解集.
\end{enumerate}
\end{ex}

\section{一元$n$次方程的根的基本对称函数}

上节韦达定理中出现的$x_1+x_2+\cdots +x_n$, $x_1x_2+x_1x_3+\cdots +x_{n-1}x_n$, $x_1x_2x_3+x_1x_2x_4+\cdots +x_{n-2}x_{n-1}x_n,\ldots,x_1x_2\cdots x_n$, 称作一元$n$次方程的根的基本对称函数. 为了书写简便,我们把前$n-1$个根的基本对称函数分别记作$\Sigma x_1$,$\Sigma x_1x_2,\ldots, \Sigma x_1x_2\cdots x_{n-1}$. 最后一个只有一项,不再用“$\Sigma$”. 于是有下列的性质:

\begin{enumerate}
    \item $\Sigma x_1x_2\cdots x_k$共有${\rm C}^k_n=\frac{n(n-1)\cdots (n-k+1)}{k(k-1)\cdots 2\cdot 1}$项.
    \item 令$S_1=\Sigma x_1$, $S_2=\Sigma x^2_1$, $S_3=\Sigma x^3_1,\ldots ,S_k=\Sigma x^k_1$,它们是一元$n$次方程的根的同次幂的和,也都是根的对称函数.
\end{enumerate}


设$x_1,x_2,\ldots ,x_n$是一元$n$次方程$f(x)=x^n+b_1x^{n-1}+b_2x^{n-2}+\cdots +b_{n-1}x+b_n=0$的$n$个根(最高次项系数为1),那么令$k=1,2,\ldots ,n-1$时,有
\[S_k+b_1S_{k-1}+b_2S_{k-2}+\cdots +b_{k-1}S_1+kb_k=0\]

当 $k\geqslant n$ 时,有 $S_k+b_1S_{k-1}+\cdots+b_nS_{k-n}=0$(这个公式叫
做牛顿公式,它可以利用多项式的导数加以证明)

\begin{example}
    求方程$x^3-2x^2+5x-3=0$的根的平方和,立方和
及四次幂的和.
\end{example}

\begin{solution}
    设方程根的平方和为$S_2$, 立方和为 $S_3$, 四次幂的和
为$S_{4}$

$\because\quad b_1=-2, \; b_2=5,\; b_3=-3$,由韦达定理可直接得出$S_{1}=2$. 利用牛顿公式,有
\[\begin{cases}
    S_{2}+ b_{1}S_{1}+ 2b_{2}= 0\\
    S_{3}+ b_{1}S_{2}+ b_{2}S_{1}+ 3b_{3}= 0\\
    S_4+ b_1S_3+ b_2S_2+ b_3S_1= 0
\end{cases}\Longrightarrow \begin{cases}
    S_{2}= - b_{1}S_{1}- 2b_{2}=2\times2-2\times5=-6\\
    S_3=- b_{1}S_{2}- b_{2}S_{1}- 3b_{3}
    =-13\\
    S_4= b_1S_3- b_2S_2- b_3S_1 =10
\end{cases}\]

$\therefore\quad $根的平方和为$-6$,立方和为$-13$. 
四次幂的和为10.
\end{solution}

\begin{rmk}
    对于高次(3 次以上)方程没有一般的求根方法。而牛顿公式告诉我们不必求方程的根,即可求出方程所有根的和、根的平方和,……,根的同次幂的和的方法.
\end{rmk}

\section{实系数方程虚根成对定理}

我们知道,如果 $\Delta=b^2-4ac<0$, 那么实系数一元二次方
程$ax^{2}+bx+c=0$ 有一对虚数根,它们互为共轭虚数,即
$$x=\frac{-b\pm\sqrt{4ac-b^{2}}i}{2a}$$

一般地说,关于实系数一元n次方程的虚数根,有下面的性质:

\begin{thm}
   {定理} 如果虚数$a+bi$是实系数一元$n$次方程$f(x)=0$的根,那么$a-bi$也是这个方程的根,并且它们的重数相等. 
\end{thm}

\begin{proof}
由$a+bi$是实系数一元$n$次方程$f(x)=0$的根,可知$f(a+bi)=0$. 我们先来证明也是方程$f(x)=0$的根,为此只需证明$f(a-bi)=0$.

考虑多项式
\begin{equation}
\begin{split}
    g(x)&=[x-(a+bi)][x-(a-bi)]\\
&=(x-a)^2-(bi)^2\\
&=x^2-2ax+(a^2+b^2)
\end{split}\tag{1}
\end{equation}
这是一个实系数二次三项式. 用$g(x)$除$f(x)$,设商式为$q(x)$,那么余式的次数不大于1,可以表示为$mx+n$(其中$m,n$为实数). 于是
\begin{equation}
    f(x)=g(x)\cdot q(x)+mx+n \tag{2}    
\end{equation}

把$x=a+bi$代入上式,左边$f(a+bi)=0$,又由(1)式知右边的$g(a+bi)=0$,从而右边的$m(a+bi)+n=0$,即
\[am+n+bmi=0\]

根据复数等于零的条件,得
\[am+n=0,\qquad bm=0\]
由于$b\ne 0$(否则$a+bi$不是虚数),所以$m=0$,由此$n=0$. 又由(1)式知$g(a-bi)=0$,所以由(2)式,
\[\begin{split}
    f(a-bi)&=g(a-bi)\cdot q(a-bi)+m(a-bi)+n\\
  &=0\cdot q(a-bi)+0\cdot (a-bi)+0=0.  
\end{split}\]
即$a-bi$是方程$f(x)=0$的根.

现在再证明$a+bi$与$a-bi$的重数相等,由上面的证明可
知$m=0$, $n=0$,代入(2)式,得
\[f(x)=g(x)\cdot q(x)\]
这说明$g(x)$整除$f(x)$. 因为$f(x)$, $g(x)$的系数都是实数,非零实系数多项式除以实系数多项式,商式仍然是实系数多项式,所以$q(x)$的系数也都是实数. 如果$a+bi$是方程$f(x)=0$的重根,那么它必然是方程$q(x)=0$的根,根据上面的证明,$a-bi$也必然是方程$q(x)=0$的根. 这样也是方程$f(x)=0$的重根. 设$a+bi$与分别是方程$f(x)=0$的$s$重根与$t$重根,重复运用这个推理方法,可知$s\le t$;同理可证$t\le s$. 所以$s=t$.
\end{proof}


由上面的定理可知,在实系数一元n次方程中,虚数根总是成对出现的.

\begin{example}
    求方程$2x^4-6x^3+21x^2+14x+39=0$在复数集$\mathbb{C}$中的解集,已知它的根中有一个是$2-3i$.
\end{example}

\begin{solution}
\textbf{解法一:} 这是一个一元四次方程,在复数集$\mathbb{C}$中有且仅有四个根. 因为它的系数都是实数,且是它的根,可知$2+3i$也是它的根.

把$2x^4-6x^3+21x^2+14x+39$除以
$[x-(2-3i)][x-(2+3i)]$, 
也就是除以$x^2-4x+13$,得商式$2x^2+2x+3$. 因此原方程可以化为
\[[x-(2-3i)][x-(2+3i)](2x^2+2x+3)=0\]

解方程$2x^2+2x+3=0$,得$x=\frac{-1\pm\sqrt{5}i}{2}$,所以原方程的解集是
\[\left\{2-3i,\; 2+3i,\; \frac{-1+\sqrt{5}i}{2},\; \frac{-1-\sqrt{5}i}{2}\right\}\]

\textbf{解法二:} 原方程有两个根$2-3i$, $2+3i$,设另外两个根为$\alpha, \beta$,由根与系数的关系,有
\[\begin{cases}
    \alpha+\beta+(2-3i)+(2+3i)=3\\
    \alpha\cdot \beta\cdot (2-3i)\cdot (2+3i)=\frac{39}{2}
\end{cases}\]
即
\[\begin{cases}
    \alpha+\beta=-1 \\ \alpha\beta=\frac{3}{2}
\end{cases}\]
所以$\alpha,\beta$是一元二次方程$2x^2+2x+3=0$的根. 解这个一元二次方程,得两个根$x=\frac{-1\pm\sqrt{5}i}{2}$. 从而原方程的解集是
\[\left\{2-3i,\; 2+3i,\; \frac{-1+\sqrt{5}i}{2},\; \frac{-1-\sqrt{5}i}{2}\right\}\]
\end{solution}

\begin{example}
    求次数最低的实系数方程
$f(x)=0$,已知它在复
数集$\mathbb{C}$中的解集含有$i$,$-1+i$,0
这三个数.
\end{example}

\begin{solution}
    根据实系数方程虚根成对定理,如果$i$,$-1+i$是所求实系数方程$f(x)=0$的根,那么它们的共轭虚数$-i$,$-1-i$也是这个方程的根,所以所求的实系数方程至少有五个根$\pm i$,$-1\pm i$,0,也就是说,$f(x)$至少有五个一次因式$x\mp i$, $x+1\mp i$,$x$,把$f(x)$写成这五个一次因式与一个常数$a\; (a\in\mathbb{C}, \text{ 且 } a\ne 0)$的积
\[f(x)=a(x-i)(x+i)(x+1-i)(x+1+i)x\]

取$a=1$,那么,实系数一元五次方程
\[(x-i)(x+i)(x+1-i)(x+1+i)x=0\]
即
\[x^5+2x^4+3x^3+2x^2+2x=0\]
就是所求的方程.
\end{solution}

\begin{ex}
\begin{enumerate}
    \item 已知方程$3x^4-2x^3+10x^2-2x+7=0$的根中有一个是$i$,求它在复数集$\mathbb{C}$中的解集.
    \item 求次数最低的实系数方程$f(x)=0$,已知它在复数集$\mathbb{C}$中的解集含有下列数:
\begin{multicols}{2}
\begin{enumerate}[(1)]
    \item $3+2i$
    \item $-2,\; 1-i$
\end{enumerate}
\end{multicols}
    \item 已知虚数$-1+\sqrt{2}i$是实系数方程$x^3+3x^2+ax+b=0$的根,求$a$,$b$的值以及这个方程在复数集$\mathbb{C}$中的解集.
\end{enumerate}
\end{ex}

\section{有理系数方程$f(x)=0$的有关无理根的定理}

\begin{thm}
{定理} 设$f(x)=0$是有理系数方程
\begin{enumerate}[(1)]
\item 若方程$f(x)=0$有无理根$a+b\sqrt{d}\; (a,b,d\in\Q,\; \sqrt{d}\in\overline{\Q} \text{ 且 } b\ne 0)$,则方程必还有另一个无理根$a-b\sqrt{d}$;
\item 若方程$f(x)=0$有无理根$a\sqrt{c}+b\sqrt{d}\; (a,b,c,d\in\Q,\; \sqrt{c},\sqrt{d}\in\overline{\Q}\text{ 且 }ab\ne 0)$,则方程必还有另外三个无理根:$a\sqrt{c}-b\sqrt{d}$, $-a\sqrt{c}+b\sqrt{d}$和$-a\sqrt{c}-b\sqrt{d}$;
\item 若$方程f(x)=0$有一个根是$a\sqrt{c} +bi\; (a,b,c\in\Q,\; \sqrt{c}\in\overline{\Q},\text{ 且 }ab\ne 0)$则方程必还有三个根:$a\sqrt{c}-bi$,$-a\sqrt{c}+bi$和$-a\sqrt{c}-bi$;
\item 若$方程f(x)=0$有一个根$a\sqrt{c}+b\sqrt{d}i\; (a,b,c,d\in\Q,\; \sqrt{c},\sqrt{d}\in\overline{\Q}\; \text{ 且 } ab\ne 0)$,则方程必还有三个根:$a\sqrt{c}-b\sqrt{d}i$, $-a\sqrt{c}+b\sqrt{d}i$和$-a\sqrt{c}-b\sqrt{d}i$.    
\end{enumerate}

\end{thm}

\begin{example}
求作含有根$\sqrt{3}+i$和$1-\sqrt{3}i$的次数最低的最简
有理系数的整式方程.
\end{example}

\begin{solution}
\textbf{解法一:} 由上述两个定理可知,方程有以下 6 个根:
$1\pm\sqrt{3}i$, $\sqrt{3}\pm i$ 和$-\sqrt3\pm i$
故所求的方程应为    
\[\begin{split}
    [x-(1+\sqrt{3}i)]&\cdot [x-(1-\sqrt{3}i)]\cdot [x-(\sqrt{3}+i)]\cdot\left[x-(\sqrt{3}-i)\right]\\
&\cdot \left[x-(-\sqrt{3}+i)\right]\cdot\left[x-(-\sqrt{3}-i)\right]=0\\
\end{split}\]
化简, 即为$x^6-2x^5+8x^3-32x+64=0$

\textbf{解法二:}先作含根$\sqrt{3}+i$的次数最低的有理系数的最
简整式方程,它有四个根,所以是四次方程

设 $x= \sqrt {3}+i$, 则$x-i=\sqrt{3}$

$\therefore\quad (x-i)^2=3$,即$x^2-2ix-4=0$

$\therefore\quad x^2-4=2ix \Longrightarrow (x^2-4)^2=(2ix)^2$

即$x^4-4x^2+16=0$

再作含根$1-\sqrt{3}i$的最简有理系数方程,该方程有两个
根,所以是二次方程。

设 $x= 1- \sqrt {3}i$, 则$x-1=-\sqrt{3}i$

$\therefore\quad (x-1)^2=(-\sqrt{3}i)^2$


化简,即 $x^2-2x+4=0$. 故所求的方程是
$$(x^4-4x^2+16)(x^2-2x+4)=0$$
即$$x^6-2x^5+8x^3-32x+64=0$$
\end{solution}

\section*{习题二}
\begin{enumerate}
    \item 求证任何复条数一元$n$次方程都可化为$x^{n}+b_{x-1}x^{n-1}+\cdots+b_{1}x+b_{0}=0$的形式,其中$b_0,b_1,\ldots,b_{n-1}\in \mathbb{C}$.
    \item 求下列方程在复数集$\mathbb{C}$中的解集:
\begin{enumerate}[(1)]
    \item $x^{3}-8x^{2}+20x-16=0$ 
    \item $x^4+ x^3- 5x^2+ x- 6= 0$ 
    \item $ 2x^4+ 9x^3- 27x^2+ 53x- 21= 0$ 
    \item $5x^{4}+6x^{3}-5x-6=0$
\end{enumerate}

\item 求最简整系数方程 $f(x)=0$, 已知它在复数集$\mathbb{C}$ 中的解集是:
\begin{multicols}{2}
\begin{enumerate}[(1)]
    \item $\left\{0, 2- \sqrt {3}, 2+ \sqrt {3}, 2i, - 2i\right\}$
    \item $\left\{\frac12_{(2)},-\frac23_{(3)}\right\}$
\end{enumerate}    
\end{multicols}

\item 求证:
\begin{enumerate}[(1)]
\item 如果一元$n$次方程 $f(x)=0$ 各项的余数都是正数,那
么它没有正数根;
\item 如果一元$n$次方程$f(x)=0$各奇次项的条数都是正
数,各偶次项(包括常数项$a_0$)的系数都是负数,那么它
没有负数根;
\item 方程$2x^6+3x^4+5x^2+7=0$ 没有实数根。
\end{enumerate}

\item 利用第 4 题的结论,求下列方程在复数集$\mathbb{C}$中的解集:
\begin{multicols}{2}
\begin{enumerate}[(1)]
    \item $ x^{3}+ \frac 72x^{2}+ \frac 52x+ \frac 12= 0$ 
    \item $x^3-\frac23x^2+3x-2=0$
\end{enumerate}    
\end{multicols}

利用一元$n$次方程根与系数的关系解下列各题(第6—9
题):

\item \begin{enumerate}[(1)]
    \item 已知方程 $18x^3+9x^2-74x+40=0$ 的根中有一个是另一个的 2倍,求这个方程在复数集$\mathbb{C}$中的解集;
    \item 已知方程 $x^4+4x^3+10x^2+12x+9=0$ 在复数集$\mathbb{C}$中
    的四个根是 2 重根$a$, 2重根$b$. 求$a,b$的值.
\end{enumerate}

\item \begin{enumerate}[(1)]
\item 已知方程$x^4-4x^3-34x^2+ax+b=0$的四个根成等差
    数列,求$a,b$的值,并且求这个方程的解集;
    \item 已知方程 $8x^3-14x^2+kx+27=0$ 的三个根成等比数
    列,求$k$的值,并且求这个方程的解集。
\end{enumerate}

\item 已知方程 $x^3+px^2+qx+r=0\; (p,q,r\in \mathbb{C})$在复数集$\mathbb{C}$中的根是 $x_1,x_2,x_3$, 求下列各式的值:
\begin{multicols}{2}
\begin{enumerate}[(1)]
    \item $\frac{1}{x_1x_2}+\frac{1}{x_1x_3}+\frac{1}{x_2x_3}$
    \item $\frac{1}{x_1}+\frac{1}{x_2}+\frac{1}{x_3}$
    \item $x_1^2+x_2^2+x_3^2$
    \item $x_1^2x_2^2+x_1^2x_3^2+x_2^2x_3^2$
\end{enumerate}
\end{multicols}
    
\item 设方程$x^3+2x^2-x+3=0$在复数集$\mathbb{C}$中的根是$x_1,x_2,x_3$, 求一元三次方程,使它在$\mathbb{C}$中的根是:
\begin{multicols}{3}
\begin{enumerate}[(1)]
    \item $2x_1,\; 2x_2,\; 2x_3$
    \item $-x_1,\; -x_2,\; -x_3$
    \item $\frac{1}{x_{1}},\; \frac{1}{x_{2}},\; \frac{1}{x_{3}}$
\end{enumerate}
\end{multicols}

\item 根据已知条件,求下列方程在复数集$\mathbb{C}$中的解集:
\begin{enumerate}[(1)]
\item $x^4-3x^3+10x^2+42x-20=0$, 已知它的根中有一个
    是 $3+i$ 
    \item $x^4-3x^3+5x^2+4x+2=0$, 已知它的根中有一个是
    $i- 1$
    \item $x^{4}-4x^{3}+11x^{2}-14x+10=0$, 已知它的根中有两个
    是 $a+bi$, $a+2bi$, 其中 $a,b\in \R$, 且 $b\neq0$
\end{enumerate}

\item 求次数最低的实系数方程$f(x)=0$,已知它在复数集$\mathbb{C}$中的解集含有下列数:
\begin{multicols}{2}
\begin{enumerate}[(1)]
    \item $1,\; \frac{-1+\sqrt{3}i}{2}$
    \item $2+i,\; -1+i$
    \item $\pm 1,\; i$
    \item $\sqrt{2},\; \sqrt{2}i$
\end{enumerate}
\end{multicols}

\item 求证实系数一元$n$次方程在$n$为奇数时,有奇数个实根;在$n$为偶数时,有偶数个实根,或者没有实根.
\item 已知虚数$a+bi\; (a,b\in\R)$ 是实系数方程的$x^3+px+q=0$的根,求证$2a$是方程$x^3+px-q=0$的根.
\item 一个长方体的长、宽、高分别是12cm,5cm,6cm.要使各度(即长、宽、高)都增加一个相同的长度,体积增加186${\rm cm}^3$,这个增加的长度应是多少?
\item 把边长为6dm的正方形铁板的四角各截去一个相同的小正方形,然后把各边折起来做成一个无盖的长方体盒. 已知这个长方体盒的容积(铁板厚度不计)是16${\rm dm}^3$,求截去的小正方形每边的长.
\end{enumerate}

\section{本章小结}

\subsection*{知识结构分析}

\begin{enumerate}
    \item 复系数一元$n$次多项式及有关概念
\begin{enumerate}[(1)]
    \item 复系数一元$n$次多项式的标准形式是
    $$a_nx^n+a_{n-1}x^{n-1}+\cdots+a_1x+a_0$$ 其中$a_n\ne 0$, $a_k$为复数$(k=0,1,\ldots ,n)$. 根据需要,该多项式也常常看作是定义在复数集$\mathbb{C}$上的函数,并记作$f(x)$, $g(x)$等.
    \item 零次多项式与零多项式
    
单独一个非零复数,可看作零次多项式;系数都是零的多项式称作零多项式. 显然对于$\mathbb{C}$上的每一个$x$值,零多项式的值恒为零.
\end{enumerate}

\item 余数定理和因式定理

多项式$f(x)$除以$x-b\; (b\in\mathbb{C})$所得的余数为$f(b)$. 余数定理的一个重要推论是多项式$f(x)$有一个因式$x-b$的充要条件是$f(b)=0$,此即因式定理.

\item 任何一个复系数一元$n$次多项式$f(x)$有唯一确定的因式分解形式:
\[f(x)=a_n(x-x_1)^{k_1}(x-x_2)^{k_2}\cdots (x-x_m)^{k_m}\]
其中$k_1,k_2,\ldots,k_m\in\N$,且$k_1+k_2+\cdots+k_m=n$,复数$x_1,x_2,\ldots,x_m$互不相等,$x-x_i\; (i=1,2,\ldots,m)$叫做多项式$f(x)$的$k_i$重因式.

此即多项式因式分解唯一性定理. 这里不考虑各个一次因式的书写顺序及常数因子.

\item 若整系数多项式$f(x)=a_nx^n+a_{n-1}x^{n-1}+\cdots+a_1x+a_0$有因式$x-\frac{q}{p}$($p,q$为互质的整数),则$p$一定是$a_n$的约数,$q$一定是$a_0$的约数.由此我们可借助综合除法,求出整系数多项式的整系数一次因式,或可以证明它没有这个因式.
\item 多项式因式分解与解方程密切相关.如果$x-x_i\; (i=1,,2,\ldots,m)$是多项式$f(x)$的$k_i$重因式,那么$x_i$叫做方程$f(x)=0$的$k_i$重根. 由此可得出“复系数一元$n$次方程在复数集$\mathbb{C}$中有且仅有$n$个根($k$重根算作$k$个根)”. 这个重要定理揭示出复系数一元$n$次方程在$\mathbb{C}$中根的个数.

显然实系数一元$n$次方程在$\R$中的根没有这个性质.
\item 韦达定理给出了复系数一元$n$次方程的根与其系数的关系,即$x_1,x_2,\ldots,x_n$是复系数一元$n$次方程$f(x)=0$的根,则有
\[\begin{cases}
x_1+x_2+\cdots+x_n=-\frac{a_{n-1}}{a_n}\\
x_1x_2+x_1x_3+\cdots+x_{n-1}x_n=\frac{a_{n-2}}{a_n}\\
x_1x_2x_3+x_1x_2x_4+\cdots+x_{n-2}x_{n-1}x_n=-\frac{a_{n-3}}{a_n}\\
\cdots\cdots\cdots\cdots\\
x_1x_2\cdots x_n=(-1)^n \frac{a_0}{a_n}
\end{cases} \]
其逆定理也是成立的.

\item 实系数一元$n$次方程的虚根成对出现定理,揭示了实系数一元$n$次方程的根的特点,与此类似的是,有理系数一元$n$次方程的无理根$a+b\sqrt{c}$或$a\sqrt{c}+b\sqrt{d}$(其中$a,b,c,d\in\Q$, $\sqrt{c},\sqrt{d}\in\overline{\Q}$)也是成对出现的.

由此可在解方程的过程中,根据方程的条件选择较简便的方法,如首先用综合除法求出有理根,再根据若$a+bi\; (a,b\in\R)$是实系数方程$f(x)=0$的根. 便可知$(x-a-bi)(x-a+bi)$必为$f(x)$的两个因式. 从而可降低方程$f(x)=0$的次数.
\end{enumerate}

\subsection*{几点说明}
\begin{enumerate}
    \item 零多项式的定义是用“待定系数法”确定多项式的理论依据:即两个一元同次多项式恒等的充要条件是对应项(次
    数相同的项)系数相等.
    \item 综合除法是求一元$n$次多项式$f(x)$的“$x-b$”型(或$x-\frac{q}{p}$,$p,q$互质)因式的简便方法. 简便之处在于只是利用了它的系数就解决了问题.
    
    它是通过竖式除法分离系数,改变运算程序得出来的.它提示我们去探求其它运算问题,以简化运算过程和改变运算程序,提高运算速度的途径.
    \item 易混淆的概念,如零次多项式与零多项,要给予足够的重视.
    \item 应用定理时,定理的条件不能忽视.如方程$x^3+2x^2-3x+2-4i=0$有一根为$i$时,误认为$-i$也是方程的根.
    \item 运用综合除法时,对所缺的项应予以补零.否则将导致错误.
    \item 韦达定理本身并不能帮助我们直接解方程,但是已知方程的一部分根,或方程的根之间的某些关系时,用韦达定理可列出方程(组),解这个方程(组). 可求得方程的其余根或全部根.
    
    利用韦达定理,可以不解方程而求出根的对称式的值. 
    
    利用韦达定理还可以求出以已知数为根的方程.
\end{enumerate}


\section*{复习题九}
\begin{center}
    \bfseries A
\end{center}

\begin{enumerate}
    \item 计算$(5x^2-2x^3+6x^4-18)\div(2x^2+1)$, 并把结果写成
    “$f(x)=g(x)q(x)+r(x)$”的形式。
    \item 一个多项式除法的除式是 $2x^2+3x-5$, 商式是 $3x-5$,余
    式是$-7$,求被除式。
    \item 用综合除法求商式及余数。其中哪些能够整除,哪些不能整除?
\begin{multicols}{2}
\begin{enumerate}[(1)]
\item $( a^{3}- b^{3}) \div ( a- b) $
\item $( a^{4}- b^{4}) \div ( a- b)$  
\item $( x^{6}- y^{6}) \div ( x+ y)$ 
\item  $( x^{5}+ y^{5}) \div ( x+ y)$ 
\item $( m^{5}- n^{5}) \div ( m+ n) $ 
\item $( m^{6}+ n^{6}) \div ( m- n)$  
\item $( u^{6}+ v^{6}) \div ( u+ v)$  
\item $( u^{7}+ v^{7}) \div ( u- v) $
\end{enumerate}
\end{multicols}

    \item 用综合除法求商式及余数:
\begin{enumerate}[(1)]
\item  $( 2x^{3}- 3x^{2}+ 8x- 12) \div ( 2x- 3)$ 
\item $( 4a^{3}+ 2a^{2}b- 8ab^{2}- 12b^{3}) \div ( 2a+ 3b)$ 
\item $( 3x^{4}+ 2x^{2}- 5x) \div ( 3x- 1)$ 
\item  $( 3x^{4}- 2x^{3}y+ 3x^{2}y^{2}- 2xy^{3}+ 3y^{4}) \div ( 3x+ 2y)$
\end{enumerate}

    \item \begin{enumerate}[(1)]
    \item 设 $f(x)=x^n+a^n$(其中$n\in \N$), 求$f(x)$除以$x-a$ 所
    得的余式,又求$f(x)$除以$x+a$所得的余式
    \item 设 $f(x)=x^n-a^n$(其中$n\in \N$), 求 $f(x)$除以$x-a$ 所得
    的余式,又求$f(x)$除以$x+a$所得的余式
    \item 通过第(1),(2)小题,说出在什么情况下,$x^n+a^n$或
    $x^n-a^n$(其中$n\in \N$)有因式 $x-a$ 或 $x+a$
    \end{enumerate}

\item 求证$x^3-4ax^2-10bx+16$有因式$x+2$的充要条件是
$a=\frac{1}{4}(2+5b)$
\item 把下列多项式分解因式:
\begin{enumerate}[(1)]
\item $y^{4}+ 5y^{3}+ 22y^{2}+ 80y+ 96$ 
\item $a^{4}- 4a^{3}b- 7a^{2}b^{2}+ 22ab^{3}+ 24b^{4}$ 
\end{enumerate}

\item 在复数集$\mathbb{C}$中解下列方程或方程组。
\begin{enumerate}[(1)]
    \item $x^{3}+ \frac 72x^{2}- 1= 0$ 
    \item $4x^{4}+ 7x^{3}- 22x^{2}- 35x+ 10= 0$
    \item $5x^{4}- 29x^{3}+ 14x^{2}- 116x- 24= 0$
    \item $x^{4}- x^{3}+ \frac 14x^{2}- 4x- 15= 0$
    \item $\begin{cases} x^3+ y^3= 19 \\ y= 2x+ 7  \end{cases}$
    \item $\begin{cases}x^{2}+y^{2}=52\\y=x^{2}-11x+34\end{cases}$
\end{enumerate}

\item \begin{enumerate}[(1)]
\item 已知方程$x^4-x^3+mx^2+nx-6=0$在复数集$\mathbb{C}$中有两
个根的和为 3,积为 2,求$m,n$的值,并且求方程在$\mathbb{C}$中
的解集;
\item 已知方程 $x^3+3x^2+mx+n=0$ 的三个根成等差数列,
方程$x^3-(m-2)x^2+(n-3)x-8=0$的三个根成等比
数列,求$m,n$的值。
\end{enumerate}

\item 设方程$2x^3-4x^2+x-6=0$在复数集$\mathbb{C}$中的根是$x_1,x_2,x_3$, 求在$\mathbb{C}$中的根是 $x_1+1,x_2+1,x_3+1$ 的一元三次方
程。
\item 已知方程$f(x)=2x^4-3x^3+ax^2+bx+c=0$在复数集$\mathbb{C}$
中有3重根$-1$,另一个根是$x_4$,求$a,b,c,x_4$的值.
\item 求方程$x^3+ix^2-4x-4i=0$在复数集$\mathbb{C}$中的解集.
\item 已知方程$x^3+2x^2-3x+2-4i=0$的根中有一个是$-i$,求这个方程在复数集$\mathbb{C}$中的解集.
\item 已知方程$x^3-9x^2+33x-65=0$的根中有一个虚根的模等于$\sqrt{13}$,求这个方程在复数集$\mathbb{C}$中的解集.
\item 一个长方体的长是宽的2倍,高比宽少2厘米,它的体积是490厘米$^3$,求它的长、宽、高.
\item 某厂一种轻工产品第一年的产值为200万元,以后三年逐年按同样的百分数递增,四年的总产值为928.2万元,求产值每年比上一年增加的百分数.
\end{enumerate}

\begin{center}
    \bfseries B
\end{center}
\begin{enumerate}\setcounter{enumi}{16}
\item 用综合除法求$(a^3-b^3+c^3+3abc)\div (a-b+c)$的商式及余数.
\item \begin{enumerate}[(1)]
\item 证明$x^3+y^3+z^3-3xyz$有因式$x+y+z$,并把它分解因式.
\item 利用第(1)小题的结果把下列各式分解因式:
\begin{enumerate}[(i)]
    \item $a^3-b^3+c^3+3abc$;
    \item $8a^3+b^3+c^3-6abc$.
\end{enumerate}
\end{enumerate}

\item \begin{enumerate}[(1)]
\item 证明$a(b-c)^3+b(c-a)^3+c(a-b)^3$有因式$a-b$, $b-c$, $c-a$,并把它分解因式;
\item 证明$(ay+bx)^3+(ax+by)^3-(a^3+b^3)(x^3+y^3)$有因式$x+y$,也有因式$a+b$,并把它分解因式.
\end{enumerate} 

\item 把$y^7+2y^6-y^5-2y^4+4y^3+8y^2-4y-8$分解因式.
\item 已知$x^4+ax^3-4x^2+bx-12$有因式$x-2$,又有因式$x+3$,确定$a,b$的值,并把这个多项式分解因式.

\item 已知$x^4+4x^{2}+ax+b$有一个因式$x^2+x+1$, 求$ a,b$的值,并把这个多项式分解因式。
\item 已知多项式$f(x)$除以$x+2$所得的余数为 1, 除以$x+3$ 所得的余数为$-1$. 求 $f(x)$除以$(x+2)(x+3)$所得的余式.
\item 求证多项式 $f(x)$除以$(x-a)(x-b)$(其中 $a\neq b$)所得的
余式是
$$\frac{f(a)-f(b)}{a-b}x+\frac{af(b)-bf(a)}{a-b}$$
\item 设方程$x^3-x^2+3x-2=0$在复数集$\mathbb{C}$中的根是$x_1,x_2,x_3$
\begin{enumerate}[(1)]
\item 求证$x_{1},x_{2},x_{3}$都不是有理数;
\item 求证$x_{1},x_{2},x_{3}$中有两个是虚数(提示:先求出$x_1^{2}+x_{2}^{2}+x_3^2$的值);
\item 求证对任何实数$k$, 方程$x^3-x^2+3x+k=0$有两个虚
数根。
\end{enumerate}


\item 在复数集$C$中解下列关于$x$的方程:
\begin{enumerate}[(1)]
    \item $x^{3}- ( a- 1) x^{2}- a^{2}= 0,\quad \left ( a> \frac 14\right )$
    \item $x^{3}+(k^{2}-2)x=2k(x^{2}-1).$
\end{enumerate}

\item 利用二项方程的解法解下列方程:
\begin{enumerate}[(1)]
  \item $(x^{3}+1)^{2}+3=0;$
\item $( x- 2) ^{2}( x^{2}+ 2x+ 4) ^{2}- 49= 0$
\end{enumerate}
\item 在复数集$\mathbb{C}$中解方程组
$\begin{cases}x=y^{2}+4y+1\\y=x^{2}+2x-3\end{cases}$
\item 求方程$x^4+x^3+x^2+x+1=0$在复数集$\mathbb{C}$中的解集(提示:在方程两边都乘以$x-1$)
\item 求证$\cos\frac{\pi}{7}+\cos\frac{3\pi}{7}+\cos\frac{5\pi}{7}=\frac{1}{2}$(提示:考虑方程$x^7-1=0$在复数集$\mathbb{C}$中的解集).

\end{enumerate}

\chapter{概率的初步知识}
概率论是一门从数量方面研究随机现象规律性的学科. 随着科学技术的迅猛发展,概率论有了严格的理论基础和丰硕的成果,并成为一门应用非常广泛的数学学科,它的理论和方法在自然科学、社会科学、工程技术、经济管理、工农业生产和国民经济的各个部门都得到了广泛的应用,并取得了很好的效果. 

本章仅就概率的初步知识作一介绍. 

\section{随机现象}
随机现象就是我们通常所说的偶然现象它在现实世界中大量地存在着. 它的对立面就是必然现象或确定性现象. 为了弄清随机现象的概念,我们先来看一看确定性现象的例子. 如纯水在一个大气压下加热到$100^{\circ}{\rm C}$必然沸腾;向上抛一重物必然下落;同性电荷必然相斥;异性电荷必然相吸等等都是确定性的现象. 所以确定性现象是在一定条件下,只有一种结果的现象. 即在一定条件下必然发生或必然不发生的现象. 

我们还经常遇到另一类现象.例如:在混有2件次品的10件产品中任意抽取1件,可能得到正品,也可能得到次品.抽取的结果不能在抽取以前确切地知道;又如向上抛掷一枚
硬币,落下后,可能是正面朝上,也可能是反面朝上,而抛掷的结果事先不能预言,这类现象的特点是:在一定的条件下,可能出现的结果不止一种,至于出现哪一种,事先又无法确定,我们把这类现象称为随机现象. 

随机现象具有不确定性,然而它还是有规律可循的,即它还有确定性的一面. 比如在相同条件下,多次重复抛掷同一枚硬币,就会发现“出现正面”的次数与抛掷次数的比接近$\frac{1}{2}$,又如掷骰子,可能出现1点,出现2点,……,出现6点.掷一次,不能预言出现的是几点,但多次重复掷时,就会发现它的规律,即出现$1,2,\ldots,6$各点的次数大约都是抛掷次数的$\frac{1}{6}$,也就是说还是有规律可循的,即有确定性的一面,而这种确定性(规律)就是我们要研究的内容. 

\section{随机试验和随机事件}

无论是随机现象或是确定性的现象,常常在人们所进行的试验或观察中呈现出来,我们把呈现随机现象的试验或观察叫做随机试验,简称试验. 在关于概率的理论中,总是通过研究随机试验来研究随机现象的,而所研究的随机试验具有以下特点:
\begin{enumerate}
\item 试验在相同的条件下,可以重复进行;
\item 每次试验的结果,具有多种可能性,并且能在试验之前就明确知道试验的所有可能结果;
\item 在每次试验之前,不能肯定这次试验将出现哪种结果,但可以肯定每次试验总是出现所有可能结果中的某一个. 
\end{enumerate}

我们所讨论的每一个随机试验中,试验的所有可能结果都应是明确知道的. 它的每一个结果就叫做一个样本点(或基本事件),全体样本点组成的集合叫做样本空间(或基本事件空间). 样本点常用$\omega$表示,样本空间常用$\Omega$表示. 这个集合的每一个子集叫做这个样本空间的一个随机事件,简称事件,用符号$A,B,C,\ldots$表示. 显然,基本事件(样本点)也是随机事件. 

我们看下面的例子:

\begin{example}
    抛一枚硬币,观察正、反面出现的情况.这个随机试验的所有可能结果有两个:正(抛得正面朝上),反(抛得反面朝上). 所以样本空间$=\{\text{正, 反}\}$,样本点有两个,一个是“正”,一个是“反”. 

这里,记$\omega_1=\text{正}$,$\omega_2=\text{反}$. 样本空间记$\Omega=\{\omega_1,\omega_2\}$. 
\end{example}
  
\begin{example}
    随机试验:“连续两次抛一枚硬币,观察它们正、反面出现的情况. ”写出这个随机试验的样本空间. 
\end{example}

\begin{solution}
    在这个随机试验中,所有的可能结果有4个:(正,正),(反,反),(反,正),(正,反). 
    
\[\therefore\quad \Omega=\{(\text{正,正}), (\text{反,反}), (\text{反,正}), (\text{正,反})\}\]
若记
\[\begin{split}
    \omega_1=(\text{正, 正})&\qquad \omega_2=(\text{反, 反})\\
    \omega_3=(\text{反, 正})&\qquad \omega_4=(\text{正, 反})\\
\end{split}\]
则样本空间也可抽象地记为
\[\Omega =\{\omega_1,\omega_2,\omega_3,\omega_4\}\]
\end{solution}

\begin{example}
    随机试验:“一个袋中装有两个红球和一个白球,从袋中任意取出两球,观察它们的颜色”. 试指出这个随机试验的样本点和样本空间. 
\end{example}

\begin{solution}
因为袋中共有3球,两红一白,每次取出两球,只关
  心这两球的颜色,而不用管是哪两个球,又两球同时被取出,所以不必考虑顺序. 故试验的所有可能结果只有两个,即有两个样本点. 若记$\omega_1=$“取出的两球都红球”,$\omega_2=$“取出的两球=一个红一个白”. 则样本空间$\Omega=\{\omega_1,\omega_2\}$. 
\end{solution}

\begin{example}
    从$1,2,3,4,5,6,7,8,9$这9个数字中任意取出两个数字. 
\begin{enumerate}[(1)]
\item 有几种取法?
\item 其中和是奇数的取法有几种?
\end{enumerate}
并说明在这种取法下“和是奇数”是否为随机事件?
\end{example}

\begin{solution}
\begin{enumerate}[(1)]
    \item 这是从9个不同的元素中每次取出2个不同元素的所有组合的种数,所以共有取法
    \[{\rm C}^2_9=\frac{9\x8}{2\x 1}=36\text{(种)}\]
    \item 和是奇数,两个加数须一奇,一偶.所以和是奇数的取法有
\[{\rm C}^1_5 \cdot {\rm C}^1_4=20\text{(种)}\]
\end{enumerate}


在这个问题里,一次试验就是从9个数中,任意取出两个数. 这种试验满足随机试验的三个特点:
\begin{enumerate}[(i)]
\item 试验在相同的条件下,可以重复进行;    
\item 每次试验(从9个不同的元素中取出2个不同的元素)具有多种可能性,并且在试验之前就明确知道试验的所有可能结果(36种取法);
\item 在每次取出2个数字之前,不能肯定它们的和是奇数还是偶数,但可以肯定,每一种取法的结果都在36种取法之内.
\end{enumerate}
所以“和是奇数”是一个随机事件. 
\end{solution}

特别,在一个随机试验中,每次试验一定发生的事情称为必然事件;每次试验中一定不会发生的事情称为不可能事件. 
例如在例10. 4的条件下,“和小于18的两个数”是必然事件,“和不小于18的两个数”是不可能事件,又如在前面提到的掷骰子试验中,“点数小于7”是必然事件,“点数不小于7”是不可能事件. 

应该看到,必然事件和不可能事件有着紧密的联系. 如果每一次试验中,某一结果必然发生(如上例中“点数小于7”),那么这一结果的反面(即“点数不小于7”)就一定不发生.不论必然事件、不可能事件,还是随机事件,都是相对于一定的试验条件而言的. 如果试验条件变了,事件的性质也会跟着发生变化. 例如在掷骰子的试验中,掷一颗骰子(条件)时,“点数小于7”是必然事件,掷两颗骰子(条件)时,“点数之和小于7”是随机事件,而掷十颗骰子(条件)时,“点数之和小于7”就是不可能事件了. 概率是研究随机事件的. 但是为使研究方便,我们把必然事件和不可能事件也看作随机事件(尽管它们不符合随机事件的含义),作为随机事件的极端情况. 

\section{随机事件的概率}
对于一个随机事件来说,虽然在一次试验中它是否发生,不能事先知道,但是如果大量地重复这一试验,就会发现,不同事件发生的可能性是有大小之分的. 这种可能性的大小,是事件本身固有的一种属性. 例如:掷一枚骰子,我们凭经验可以知道:\{出现偶数点\}与\{出现奇数点\}这两个事件发生的可能是相同的,而\{出现奇数点\}和\{出现3点\}这两个事件发生的可能性就不同,\{出现奇数点\}这个事件发生的可能性比\{出现三点\}发生的可能性要大. 为了定量地描述随机事件发生的
可能性的大小,我们先介绍频率的概念.

在相同的条件下,重复进行$n$次试验,若在$n$次试验中,事件$A$发生的次数(称为频数)为$\mu_A$,我们把比值$\mu_A/n$称为事件$A$在$n$次试验中发生的频率. 即
\[A_{\text{发生的频率}}=\frac{\text{频数}}{\text{试验次数}}=\frac{\mu_A}{n}\]

显然$0\le \frac{\mu_A}{n}\le 1$,它在一定程度上反映了事件$A$发生的可能性的大小.

我们再来看一下抛硬币的试验. 我们把前人的一些试验记录列成下表:
\begin{center}
\begin{tabular}{l|ccc}
\hline
    试验者 & 抛掷次数$n$ & 正面出现的次数$\mu_A$ & 频率$\mu_A/n$\\
\hline
棣莫根(de Morgan)&2048&1061&0.518\\
蒲丰(Buffon)&4040&2048&0.5069\\
皮尔逊(K. Pearson)&12000&6019&0.5016\\
皮尔逊(K. Pearson)&24000&12012&0.5005\\
维尼&30000&14994&0.4998\\
\hline
\end{tabular}
\end{center}

由上表可以看出,虽然一个事件的频率在一定程度上反映了这个事件发生可能性的大小,但它不是一个完全确定的数,因而无法用它来定量地描述这个事件发生的可能性的大小. 不过从上表还可以看到:同一事件“抛得正面出现”($A$)发生的频率虽然各不相同,但却都在一固定的数值0.5附近摆动,并且随着抛掷次数的增加,这种摆动的幅度越小,从而呈现出一定的稳定性,我们说事件A发生的频率稳定在0.5.于
是,0.5这个确定的数值就可以作为事件$A$发生可能性大小的一个客观的度量,我们称0.5这个数值为事件$A$的概率,记为$\Pr(A)=0.5$.

一般地,在相同条件下,重复进行$n$次试验,如果当$n$充分大时,事件$A$发生的频率$\mu_A/n$稳定在某一数值$P$附近摆动,而且一般说来,随着$n$的增大,这种摆动的幅度越来越小,则称数值$P$为事件$A$的概率,记作
\[\Pr(A)=P\]

由于对任何事件$A$,都有$0\le \frac{\mu_A}{n}\le 1$,所以$0\le \Pr(A)\le 1$. 显然,必然事件的概率是1,不可能事件的概率是0.

由定义知道,事件$A$的概率$\Pr(A)$,就是事件$A$在多次试验中,随着试验次数的增加,事件$A$的频率逐渐逼近,趋于稳定的那个数,它是事件$A$发生的可能性大小的定量的客观的描述. 这个定义直观地说明了概率的来源,但无法用这个定义来直接计算概率$\Pr(A)$. 实际上,人们采用一次大量实验的频率或一系列频率的平均值作为$\Pr(A)$的近似值(或估计值).


\section*{习题一}
\begin{center}
    \bfseries A
\end{center}

\begin{enumerate}
    \item 试举出两个随机现象的例子和其中的若干个随机事件,并且指出其中的必然事件和不可能事件.
    \item 随机试验:一个大箱子中装有5个型号相同的杯子,其中3个是一等品,2个是二等品,从中任取两个,观察它们是一等品,还是二等品,试写出它的样本空间$\Omega$.
    \item 指出下列事件是必然事件,不可能事件还是随机事件.
\begin{enumerate}[(1)]
\item 从54张扑克牌中任取一张,取出的恰好是方块5.
\item 从实数中任取两个数,取出的数$a$,$b$具有性质:
    $a\cdot b=b\cdot a$.
    \item 从三角形集合中任取一个三角形,这三角形的内角和大于$180^{\circ}$.
    \item 某电话总机在一分钟内接到5次呼唤. 
\end{enumerate}

    \item 对一批产品进行抽查,结果如下表所示.
\begin{enumerate}[(1)]
    \item 计算表中优等品的频率.
\item 写出优等品的频率接近的并在它附近摆动的那个常数.
\end{enumerate}

\begin{center}
\begin{tabular}{l|cccccc}
\hline
    抽取产品数$n$&50&    100&    200&    500&  1000
&    2000\\ \hline 
优等品数$\mu_A$&45&92&194&470&954&1902\\
优等品频率$\mu_A/n$\\
\hline
\end{tabular}
\end{center}

\end{enumerate}


\section{等可能性事件的概率}
随机试验的形式多种多样,内容可以千差万别,对此,我们可根据其特征的不同,来建立不同的数学模型,从而分类进行研究. 下面我们来研究最简单的一类随机试验,先看两个试验例子.
\begin{enumerate}[(1)]
\item 一盒灯泡一百个,要抽取一个检查灯泡的使用寿命,任意取一个,则一百个灯泡被抽取的机会相同.
\item 抛掷一枚匀称的硬币,可能出现正面与反面两种结果,显然这两种结果出现的可能性是相同的.
\end{enumerate}


这两个试验的共同特点是:
\begin{enumerate}[(1)]
\item 每次试验,只有有限个可能的试验结果,或说样本点总数为有限个.
\item 每次试验中,每个可能结果(即样本点)出现的可能性是相同的.
\end{enumerate}

凡是具有这两个特点的随机试验我们称之为古典的概率模型. 由于它只有有限个试验结果,且这些试验结果出现的可能性相同,所以也可以不通过重复试验,根据推理分析就能够求出某些事件的概率.

如上述例(1)中,从一百个灯泡中任取一个,由于一百个灯泡被抽取的机会相同.因此,可以认为,一个灯泡被抽取的可能性是$\frac{1}{100}$,就说一个灯泡被抽取的概率是$\frac{1}{100}$.

上例(2)中所说的掷硬币问题.因为掷一枚均匀的硬币,出现正面和出现反面这两种结果的可能性是相等的,所以可以认为,抛掷一次,正面出现的概率是$\frac{1}{2}$,反面出现的概率也是$\frac{1}{2}$,这样的分析和大量重复试验的结果是一致的.

又如,一个大箱子中装有5个型号相同的杯子,其中4个是一等品,1个是二等品,从中取出1个,取到一等品的概率是多少?

\begin{analyze}
    从箱子中任取一个,取到各个杯子的可能性是相同的,由于从5个杯子中任取1个,共有5种等可能的结果.又由于其中有4个一等品,从这5个杯子中取到一等品的结果有4种.因此可以认为取到一等品的概率是$\frac{4}{5}$.
\end{analyze}

这样,对这类随机试验我们有如下结论:

如果一次试验中共有$n$种等可能出现的结果,其中事件$A$包含的结果有$m$种,那么事件$A$的概率$\Pr(A)$是$\frac{m}{n}$.

也就是说,若随机试验属于古典概率模型,如果它的样本空间含有$n$个样本点,事件$A$含有$m$个样本点,则事件$A$的概率定义为
\begin{equation}
    \Pr(A)=\frac{m}{n}=\frac{\text{$A$包含的样本点数}}{\text{样本点总数}}\tag{1}
\end{equation}

这个定义与用频率来定义事件的概率是一致的. 如果我们进行大量重复试验,我们将会看到事件$A$发生的频率是稳定于$\frac{m}{n}$的.

\begin{example}
    在100个相同的球中,混有4个假货.现在从这100个球中任意取出一个球,这球是假货的概率是多少?
\end{example}

\begin{solution}
    从100个球中,任意取出1个,所以样本点的总数是100.每个球被取出的可能性是相同的,又因为100个球中混有4个假的,所以“从100个球中取出一个球是假货”这一事件包含有4个样本点,即$n=100$, $m=4$,所以所求的概率是
\[P=\frac{m}{n}=\frac{4}{100}=\frac{1}{25}\]
\end{solution}

\begin{example}
    在20件产品中,有15件一等品,5件二等品.从中任取4件,计算2件是一等品,2件是二等品的概率.
\end{example}

\begin{solution}
    由于是从20件中任取,每取4件都是等可能性的,所以样本点总数是${\rm C}_{20}^4$,记“任取4件,2件是一等品,2件是二等品”为事件$A$,则事件$A$含有的样本点数是${\rm C}_{15}^2\cdot {\rm C}_{5}^2$. 事件$A$的概率是
\[\Pr(A) = \frac{{\rm C}_{15}^2\cdot {\rm C}_{5}^2}{{\rm C}_{20}^4}=\frac{70}{323}\]

答:所求事件的概率是$\frac{70}{323}$.
\end{solution}

\section*{习题二}

\begin{center}
    \bfseries A
\end{center}

\begin{enumerate}
    \item 在45名同学中,有9名三好学生,从中任选一名学生,选到三好学生的概率是多少?
    \item 在10个乒乓球中,有8个是正品,2个是副品,从中任取两个,恰好都取到正品的概率是多少?
    \item 一个正方体的木块,各个侧面都涂有颜色,把它锯成1000个体积相同的小正方体.将这些小正方体充分地混合,求随机选取的一个小正方体恰有两个侧面涂有颜色的概率.在100张已编号的卡片(从1号到100号),从中任取1张,计算:
\begin{enumerate}[(1)]
    \item 卡片号是偶数的概率;
    \item 卡片号是7的倍数的概率.
\end{enumerate}

    \item  在7张卡片中,有4张负数卡片和3张正数卡片.从中任取2张作乘法计算,其积是负数的概率是多少?
    \item  某城市的电话号码由五个数字组成,每个数字可以是0到9这十个数字中的任一个,计算电话号码由五个不同数字组成的概率.
\end{enumerate}

\begin{center}
    \bfseries B
\end{center}

\begin{enumerate}\setcounter{enumi}{6}
    \item 4张同样的卡片上,分别写上数字$1,2,3,4$,从中任意抽出两张,求这两张卡片上两个数字是连续整数的概率.
    \item 号码锁有自左至右5个拨盘,每个拨盘上有1到9共9个数字,当这5个拨盘上的数字,组成某一个5位数(开锁号码)时,锁才能打开,如果不知道这个号码,求一次就能把锁打开的概率.
    \item 一枚骰子抛掷两次,求事件“两次掷得点数的和是8”的概率.
    \item 一枚硬币抛掷3次作为一次试验,试求
\begin{enumerate}[(1)]
    \item “恰有一次出现正面”的概率;
    \item “至少有一次出现正面”的概率.
\end{enumerate}

    \item $A$、$B$、$C$、$D$、$E$5个人站成一排,谁站在第几个位置是任意的. 计算:
\begin{enumerate}[(1)]
    \item $A$恰好站在正中间的概率;
    \item $B$、$C$两人恰好站在两端的概率.
\end{enumerate}

    \item 从数字1,2,3,4,5中任取3个,组成没有重复数字的三位数,计算:
\begin{enumerate}[(1)]
\item 这个三位数是5的倍数的概率;
\item 这个三位数是偶数的概率;
\item 这个三位数大于400的概率.
\end{enumerate}

\end{enumerate}

\section{事件间的关系}

在任何一个随机试验中,总有许多随机事件,其中有些比
较简单,有些就比较复杂. 它们之间又有着这样或那样的联系. 正确分析事件之间的联系,对计算事件的概率尤其是对复杂事件概率的计算是非常重要且关键的一步. 所以本节专门来研究事件间的一些关系,为进一步计算事件的概率作好准备.

\subsection{事件的包含关系}
设$A$、$B$为两个事件,若事件$A$发生时,事件$B$必发生,则称事件$B$包含事件$A$,或称事件$A$包含于事件$B$,记作$A\subseteq B$或$B\supseteq A$.

例如:连续两次抛一枚硬币,观察正、反面出现的情况. 我们知道在这个试验中,所有可能的结果有4个.它们是:$\omega_1=(\text{正}, \text{正})$,$\omega_2=(\text{正}, \text{反})$, $\omega_3=(\text{反}, \text{正})$, $\omega_4=(\text{反}, \text{反})$.

若令事件$A$为(两次出现正面)则$A=\{\omega_1\}$

令事件$B$为(第一次出现正面),则$B=\{\omega_1,\omega_2\}$. 这时,事件$A$发生时,事件$B$必然发生.

$\therefore\quad A\subseteq B$(或$B\supseteq A$)

\subsection{事件的等价关系}
若$A\subseteq B$且$B\subseteq A$(或说事件$A$发生,当且仅当事件$B$发生)则称事件$A$和事件$B$相等. 记作$A=B$.

在上例中令事件$A=$(两次抛掷,至少一次是正面),令事件$B=$(两次抛掷,出现反面不多于一次)则
\[A=\{\omega_1, \omega_2, \omega_3\},\qquad B=\{\omega_1, \omega_2, \omega_3\}\]

$\therefore\quad A=B$

\subsection{两事件的和}

例如:甲、乙两只猫去捉同一只老鼠,如果用$A$表示事件“猫甲捉到老鼠”,用$B$表示事件“猫乙捉到老鼠”,用$C$表示事件“猫捉到老鼠”.我们来看事件$C$如何用事件$A$和事件$B$ 表示.

事件$C$ 表示“猫捉到老鼠”,可能是猫甲捉到的,也可能是猫乙捉到的,或者,甲乙两猫同时捉到老鼠,可见事件$C$ 发生就是事件$A$发生或事件$B$发生. 记作$C=A+B$(或$A\cup B$).

我们把事件$C$叫做事件$A$与事件$B$的和. 它含有下面 3
部分内容.
\begin{enumerate}[(1)]
\item 事件$A$发生而$B$不发生;
\item 事件$B$发生而$A$不发生;
\item 事件$A,B$同时发生。  
\end{enumerate}

$A+B$也可以说成是:“事件$A$与事件$B$至少有一个发
生”.

事件的和可以推广到$n$个事件的情形:

如果事件$A_1,A_2,\ldots,A_n$至少有一个发生,这一事件记作
$B$, 则
$$B=A_{1}+A_{2}+\cdots+A_{n}$$

例如:连续两次抛一枚硬币的试验中,
\[\begin{split}
    A_{1}&=\{\omega_{1}\}\quad (\text{其中}\omega_1=(\text{正},\text{正}))\\ 
    A_{2}&=\{\omega_{2}\}\quad (\text{其中}\omega_2=(\text{正},\text{反}))\\ 
    A_{3}&=\{\omega_{3}\}\quad (\text{其中}\omega_3=(\text{反},\text{正}))\\ 
    A_{4}&=\{\omega_{4}\}\quad (\text{其中}\omega_4=(\text{反},\text{反}))\\ 
\end{split}\]
在这试验中“有正面向上”这一事件若用$B$表示. 则
$$B=A_{1}+A_{2}+A_{3}=\{\omega_{1},\omega_{2},\omega_{3}\}$$

\subsection{两事件的差}

设$A,B$为两个事件,$A$发生而$B$不发生也是一个事件,
称做事件$A$与事件$B$的差. 记作$A-B$.

在前面猫捉老鼠的例子中,$C-B$表示事件:猫甲提到老
鼠而且猫乙没有捉到老鼠”.

\subsection{事件的积}

设$A,B$为两个事件, $A,B$同时发生也是一个事件,称做
事件$A$与事件$B$的积(或交)记作$A\cdot B$或$AB$.

例:在射击比赛时,有一射手连续向一目标射击 2次. 我们把“第一次射击命中目标”叫做事件$A_1$, “第二次击中目标” 记作$A_2$,把“两次都击中目标”叫做事件$B$. 则
$$B=A_{1}\cdot A_{2}$$

\subsection{事件的逆}

“事件$A$不发生”也是一个事件,称为事件$A$的逆.(又称
做$A$的对立事件)记作$\overline{A}$.

例如:在上面射击比赛的例子中,“第二次射击命中目标” 叫做事件$A_2$. 若把事件“第二次射击未击中目标”叫做事件$C$,则 $C=\overline{A}_{2}$, 而事件 $A_1\overline{A_2}$ 表示事件“第一次射击命中目标, 第二次射击未命中目标.”

\subsection{互不相容}

在一次试验中,如果事件$A$和事件$B$不能同时发生(或
者说$AB$是不可能事件),则称事件$A$和事件$B$互不相容,记为$AB=\emptyset$. 这时,事件$A$和事件$B$也叫做互斥事件.

若事件$A$、$B$是互斥事件,$B$、$C$是互斥事件,$A$、$C$是互斥事件,换句话说,事件$A$,$B$,$C$中,任何两个都是互斥事件,这时我们说事件$A$,$B$,$C$彼此互斥(或互不相容). 一般地,如果事件$A_1,A_2,\ldots,A_n$中任何两个都是互斥事件,那么就说事件$A_1,A_2,\ldots,A_n$ 彼此互斥(或互不相容).

\section*{习题三}
\begin{center}
\bfseries A
\end{center}

\begin{enumerate}
    \item 一种圆柱形产品,只有当产品的长度和直径都合格时才算正品,否则就为次品,如果用$A_1$表示事件“长度合格”;$A_2$表示事件“直径合格”,试用$A_1,A_2$表示下列事件:
\begin{multicols}{2}
  \begin{enumerate}[(1)]
\item “产品为正品”.
\item “产品为次品”.
\end{enumerate}  
\end{multicols}

\item     10件产品中有7件正品3件次品,从中任意取出6件产品,若用$A$表示事件“取出的6件产品中至少有一件次品”;用$B$表示“取出的6件产品中次品不少于两件”.那么$\overline{A}$,$\overline{B}$各表示什么事件?
  \item   一枚硬币投掷两次,令$A_i=\text{“第i次正面朝上”}\; (i=1,2)$,试用$A_i\; (i=1,2)$表示下列事件:
\begin{enumerate}[(1)]
\item “两次都正面朝上”;
\item “至少有一次正面朝上”;
\item “至多有一次正面朝上”
\end{enumerate}

\item 制造某一零件需经过三道工序加工,只有当三道工序加工
均合格时,此零件才算正品,否则就为次品. 若用$A_i$表示事件“第$i$道工序加工合格” $(i=1,2,3)$. 
\begin{enumerate}[(1)]
    \item 试用$A_i\; (i=1,2,3)$表示“零件是次品”;
    \item 叙述事件$A_1A_2A_3$的含义;
    \item 说明在什么条件下$A_1A_2\subseteq A_3$.
\end{enumerate}

\item 某仪器由三个元件组成,用$A_i\; (i=1,2,3)$表示事件“第$i$个元件合格”,试用$A_i\; (i=1,2,2,3)$表示下列事件:
\begin{enumerate}[(1)]
\item “仪器合格”;
\item “仪器至多有一个元件不合格”;
\item “仪器仅有一个元件合格”;
\item “仪器至少有一个元件不合格”.
\end{enumerate}

\item 在某校高二年级中任选一名学生去参加一个会议.用$A$表示事件“被选学生是男生”,用$B$表示事件“被选学生是三好学生”,用$C$表示事件“被选学生是运动员”.
\begin{enumerate}[(1)]
    \item 叙述事件$AB\overline{C}$的意义;
    \item $ABC=C$在什么条件下成立?
    \item 什么时候关系式$C\subseteq B$是正确的?
    \item 什么时候$\overline{A}=B$成立?
\end{enumerate}

\item 设$A$,$B$,$C$是某个随机试验中的三个事件,试将下列事件用上面三个事件表示出来:
\begin{enumerate}[(1)]
\item 事件$A$发生;
\item 恰好事件$A$发生;
\item 事件$A$和$B$都发生而事件$C$不发生;
\item $A$,$B$,$C$三个事件都发生;
\item $A$,$B$,$C$三个事件中至少有一个事件发生;
\item $A$,$B$,$C$三个事件中至少有两个事件发生;
\item $A$,$B$,$C$三个事件中恰有一个事件发生;
\item $A$,$B$,$C$三个事件中恰有两个事件发生;
\item $A$,$B$,$C$三个事件中不多于一个事件发生;
\item $A$,$B$,$C$三个事件中不多于两个事件发生;
\item $A$,$B$,$C$三个事件都不发生.
\end{enumerate}
\end{enumerate}

\section{互斥事件有一个发生的概率}
\begin{example}
    100件产品中,有75件一等品,20件二等品,5件三等品.从其中任取1件,这件是一等品或二等品的概率是多少?
\end{example}

\begin{solution}
从100件产品中,任取一个产品有100种取法.即样本点总数为100.

从100件产品中,任取一个产品,这产品是一等品叫做事件$A$,因为有75件一等品,所以事件$A$含有的样本点数是75.

同理,把从100件产品中,任取一个产品,这产品是二等品叫做事件$B$,是三等产品叫做事件$C$,所以事件$A$、$B$、$C$彼此互斥,且事件B含有样本点数是20,事件C含有样本点的数是5.
因此,
\[\Pr(A)=\frac{75}{100},\qquad \Pr(B)=\frac{20}{100},\qquad \Pr(C)=\frac{5}{100}\]

事件“任意取出一件产品,该产品是一等品或二等品,是事件$A$与事件$B$的和:$A+B$.我们知道,从100件产品中,任意取出一件产品,这件产品是一等品或二等品”这个事件包含的样本点数是$75+20$. 所以$\Pr(A+B)=\frac{75+20}{100}=0.95$
\end{solution}

由$\frac{75+20}{100}=\frac{75}{100}+\frac{20}{100}$,我们看到:
\begin{equation}
\Pr(A+B)=\Pr(A)+\Pr(B) \tag{2}
\end{equation}

一般地,如果事件$A$,$B$互斥,那么事件“$A+B$”发生的概率,等于事件$A$和事件$B$分别发生的概率的和.

推广到$n$个彼此互斥事件的情形就是:

如果事件$A_1,A_2,\ldots,A_n$彼此互斥,那么事件“$A_1+A_2+\cdots+A_n$”发生的概率,等于这$n$个事件分别发生的概率的和,即
\begin{equation}
    \Pr(A_1+A_2+\cdots+A_n)=\Pr(A_1)+\Pr(A_2)+\cdots +\Pr(A_n)\tag{2$'$}
\end{equation}

\begin{example}
    在10件产品中,有7件一等品,3件二等品,从其中任取3件,至少有1件为一等品的概率是多少?
\end{example}

\begin{solution}
    把从10件产品中任取3件,其中恰有1件一等品记为事件$A_1$,恰有2件一等品记为事件$A_2$,恰有3件一等品记为事件$A_3$.则事件$A_1$, $A_2$, $A_3$的概率分别是:
\[\begin{split}
 \Pr(A_1)&=\frac{{\rm C}_7^1{\rm C}_3^2}{{\rm C}_{10}^3}=\frac{21}{120}\\
\Pr(A_2)&=\frac{{\rm C}_7^2{\rm C}_3^1}{{\rm C}_{10}^3}=\frac{63}{120}\\
\Pr(A_3)&=\frac{{\rm C}_7^3}{{\rm C}_{10}^3}=\frac{35}{120}  
\end{split}\]

根据题意,事件$A_1$, $A_2$, $A_3$彼此互斥.由公式($2'$)3件产品中至少1件为一等品的概率是:
\[\Pr(A_1+A_2+A_3)=\Pr(A_1)+\Pr(A_2)+\Pr(A_3)=\frac{21}{120}+\frac{63}{120}+\frac{35}{120}=\frac{119}{120}\]
答:从10件中任取3件,至少有1件为一等品的概率是$\frac{119}{120}$.
\end{solution}


因为对立事件是互斥事件,由(2)得
\begin{equation}
  \Pr(A)+\Pr(\overline{A})=\Pr(A+\overline{A})=1  \tag{3}
\end{equation}
即两个对立事件概率的和是1.

(3)式还可写做:
\begin{equation}
    \Pr(\overline{A})=1-\Pr(A)\quad \text{或}\quad \Pr(A)=1-\Pr(\overline{A}) \tag{$3'$}
\end{equation}

上述例(2)中,“任取3件,至少有1件为一等品”记作事件$B$,这个事件的对立事件是:“任取3件,3件全不是一等品”(应记作事件$B$).

$\because\quad \Pr(\overline{B})=\frac{{\rm C}_3^3}{{\rm C}_{10}^3}=\frac{1}{120}$

$\therefore\quad \Pr(B)=1-\Pr(\overline{B})=1-\frac{1}{120}=\frac{119}{120}$

\begin{rmk}
    对于两个对立事件,利用公式($3'$),可以把求其中一个事件的概率,转化为求它的对立事件的概率,往往比较简单.
\end{rmk}

\section*{习题四}
\begin{center}
    \bfseries A
\end{center}
\begin{enumerate}
    \item 在某一时期内,一条河流在某处的年最高水位在各个范围内的概率如下:
\begin{center}
\begin{tabular}{c|ccccc}
    \hline
    年最高水位 &低于10米& 10~12米&12~14米&14~16米&不低于16米\\
    \hline
    概率& 0.1&0.28&0.38&0.16&0.08\\
    \hline
\end{tabular}
\end{center}
计算在同一时期内,河流在此处的年最高水位在下列范围内的概率:
\begin{multicols}{2}
\begin{enumerate}[(1)]
\item 10~16米;
\item 低于12米;
\item 不低于14米;
\item 不低于12米.
\end{enumerate}
\end{multicols}

\item 某射手射击由三个区域组成的目标,击中第一个区域的概率是0.51,击中第二个区域的概率是0.32,求该射手在一次射击中,击中第一个区域或第二个区域的概率.
\item 一个袋子中有红球5个,白球4个,从中任取2个球,至少有一个为红球的概率是多少?
\item 100件商品中混有5件伪劣商品,
\begin{enumerate}[(1)]
\item 从这100件商品中任意取出50件,其中没有一件伪劣商品的概率是多少?
\item 从全部商品中任意取出50件,其中恰有一件伪劣商品的概率是多少?
\item 从全部商品中任意取出50件,伪劣商品不多于1件的概率是多少?  
\end{enumerate}

\item 50个产品中有46个合格品与4个废品,从中一次抽取3个,求其中有废品的概率.
\item 一部五卷本的文集,按任意的次序放到书架上,求
\begin{enumerate}[(1)]
\item 第一卷在左边或在右边的概率;    \item 第一卷和第五卷都在边上的概率;    \item 第三卷正好在正中间位置的概率.
\end{enumerate}

\item 在1000张彩券中,中奖的有10张,问持有5张彩券的人中奖的概率是多少?
\item 在$n$张彩券中,中奖的有$m$($m<n$)张,问持有$k$($k<n$)张彩券的人中奖的概率是多少?
\end{enumerate}

\section{相互独立事件同时发生的概率}
先看这样一个问题.甲袋里装有6个红球4个白球,乙袋里装有3个红球5个白球.从这两个袋里分别任意取出一个球,这两个球都是红球的概率是多少?

因为从一个袋里取出的球的颜色,对从另一个袋子取出的球的颜色没有影响,所以当我们把“从甲袋里取出的一个球是红球”叫做事件$A$,“从乙袋里取出一个球是红球”叫做事件$B$时,事件$A$(或$B$)是否发生对事件$B$(或$A$)发生没有影响,我们就说这两个事件在概率意义下是互相独立的,这样的两个事件叫做相互独立事件.

在这个问题里,事件$\overline{A}$是指“从甲袋里取出一个球是白球”,事件$\overline{B}$是指“从乙袋里取出一个球是白球”. 显然,事件$A$与$\overline{B}$,事件$\overline{A}$与$B$,事件$\overline{A}$和$\overline{B}$也都是相互独立的. 一般地,当事件$A$与事件$B$相互独立时,$A$与$\overline{B}$,$\overline{A}$与$B$,$\overline{A}$与$\overline{B}$也都是相互独立的.

“从两个袋子里分别任意取出一个球,这两个球都是红球”是一个事件,就是事件$AB$. 这个问题就是要求$A$、$B$同时发生的概率$\Pr(A\cdot B)$.从甲袋里取出一个球,有10种等可能的结果,从乙袋里取出一个球,有8种等可能的结果,于是从两个袋里分别取出一个球共有$10\x8$种结果,其中同时是红球的结果有$6\x3$种,所以从两袋里各取出一个球,都是红球的概率$\Pr(A\cdot B)=\frac{6\x3}{10\x 8}=\frac{6}{10}\cdot \frac{3}{8}$.

另一方面,从甲袋里任意取出一个球为红球的概率是
$\Pr(A)=\frac{6}{10}$,从乙袋里任意取出一个球为红球的概率是$\Pr(B)=\frac{3}{8}$.

从这例子中,我们看到
\begin{equation}
    \Pr(A\cdot B)=\Pr(A)\cdot \Pr(B)\tag{4}
\end{equation}

对于一般情况,当事件$A$与事件$B$相互独立时,这个等式也是成立的. 这就是说,\textbf{两个相互独立事件同时发生的概率,等于每个事件发生的概率的积}.

这个规律可以推广到$n$个事件的情况,就是:

\textbf{如果事件$A_1,A_2,\ldots,A_n$两两相互独立,那么这几个事件同时发生的概率,等于每个事件发生的概率的积,即}
\begin{equation}
    \Pr(A_1\cdot A_2\cdots A_n)=\Pr(A_1)\cdot \Pr(A_2)\cdots\Pr(A_n)  \tag{$4'$}    
\end{equation}

\begin{example}
甲、乙两人定点投篮各投一次,如果两人一次投进的概率都是0.7,计算
\begin{enumerate}[(1)]
 \item 两人都投进的概率;  
 \item 恰有一人投进的概率;
\item 至少有一人投进的概率.    
\end{enumerate}
 
\end{example}

\begin{analyze}
(1)甲、乙两人各投一次,甲是否投进对乙是否投进没有影响.反过来也是一样,所以甲(或乙)是否投进对乙(或甲)投进的概率没有影响. 即“甲投篮一次投进”与“乙投篮一次投进”是相互独立事件,依据公式可求出两个事件同时发生的概率.    
\end{analyze}

\begin{solution}
(1)把“甲投篮一次,投进”记为事件$A$,“乙投篮一次,投进”记为事件$B$. 所以“两人各投篮一次,都投进”就是事件$A\cdot B$,由题意知,$A$,$B$是相互独立事件. 

$\therefore\quad \Pr(A\cdot B)=\Pr(A)\cdot \Pr(B)
=0.7\x0.7=0.49$.

答:两人各投篮一次,都投进的概率是0.49.
\end{solution}

\begin{analyze}
(2)“两人各投一次,恰有一人投进“包括两种情况:一种是甲投进,乙未投进,即事件$A\overline{B}$发生;另一种情况是甲末投进,乙投进,即事件$\overline{A}B$发生,根据题意,这两种情况在各投篮一次的情况下不可能同时发生,即事件$A\cdot \overline{B}$和$\overline{A}\cdot B$互斥,所以根据公式和公式(4)可求出“恰有一人投进”的概率.
\end{analyze}

\begin{solution}
(2)
\[\begin{split}
    \Pr(AB+AB)&=\Pr(AB)+\Pr(AB)\\
&=\Pr(A)\cdot \Pr(\overline{B})+\Pr(\overline{A})\cdot \Pr(B)\\
&=0.7\x(1-0.7)+(1-0.7)\x0.7\\
&=0.21+0.21=0.42.    
\end{split}\]
答:恰有一人投进的概率是0.42.
\end{solution}

\begin{analyze}
(3)设“甲、乙两人各投篮一次,至少有一人投进”为事件$C$. 则$C$是事件$A\cdot B$, $\overline{A}\cdot B$, $A\cdot \overline{B}$的和,而后面这三个事件又是互斥的,根据公式($2'$)可求$\Pr(C)$.

又“两人各投篮一次,至少有一人投进”与事件“两人各投篮一次,都未投进”是对立事件,即事件$C$和事件$\overline{A}\cdot \overline{B}$是对立事件,所以又可用公式($3'$)求$\Pr(C)$.

从而有下面两种解法.    
\end{analyze}

\begin{solution}
\textbf{解法一:}\[\begin{split}
    \Pr(C)&=\Pr(A\cdot  B+\overline{A}\cdot B+A\cdot\overline{B})\\
    &=\Pr(A\cdot B)+\Pr(\overline{A}\cdot B)+\Pr(A\cdot \overline{B})\\
    &=\Pr(A)\cdot \Pr(B)+\Pr(\overline{A})\cdot \Pr(B)+\Pr(A)\cdot \Pr(\overline{B})\\
&=0.7\times0.7+0.3\times0.7+0.7\times0.3\\
&=0.49+0.21+0.21 =0.91.
\end{split}\]
\textbf{解法二:}
\[\begin{split}
    \Pr(C)=1-\Pr(\overline{A}\cdot\overline{B})
    &=1-\Pr(\overline{A})\cdot \Pr(\overline{B})\\
    &=1-0.3\times0.3=1-0.09=0.91
\end{split}\]
答:至少有一人投进的概率是0.91.
\end{solution}

\noindent
\begin{minipage}{.57\textwidth}
    \begin{example}
    在一段线路中并联着3个自动控制的开关(图10.1),只要其中一个开关能够闭合,线路就能正常工作,假定在某段时间内每个开关能够闭合的概率都是0.7,计算在这段时间内线路正常工作的概率.
\end{example}
\end{minipage}\hfill
\begin{minipage}{.4\textwidth}
    \centering
\begin{tikzpicture}
\draw(-.25,1.5)--(-.25,1)--(-2,1);
\draw(.25,1.5)--(.25,1)--(2,1);
\draw(-.25,.5)--(-.25,0)--(-2.5,0);
\draw(.25,.5)--(.25,0)--(2.5,0);
\draw(-.25,-.5)--(-.25,-1)--(-2,-1);
\draw(.25,-.5)--(.25,-1)--(2,-1);
\end{tikzpicture}
\captionof{figure}{}
\end{minipage}


\begin{analyze}
“这段时间内线路正常工作”这一事件即 3个开关
至少有一个能够闭合.

设这 3 个开关为$J_{A1}$, $J_{A2}$, $J_{A3}.$
$J_{A1}$闭合为事件 $B_1$, $J_{A2}$闭合为事件 $B_2$, $J_{A3}$闭合为事件$B_{3}$, 则 $\overline{B_{1}}$, $\overline{B_{2}}$, $\overline{B_{3}}$ 分别为 $J_{A1}$不闭合, $J_{A2}$不闭合, $J_{A3}$不闭合. 令“这段时间内线路正常工作”为事件$B$.
则
$$B=B_{1}B_{2}B_{3}+B_{1}B_{2}\overline{B_{3}}+B_{1}\overline{B_{2}}B_{3}+\overline{B_{1}}B_{2}B_{3}+B_{1}\overline{B_{2}}\overline{B_{3}}+\overline{B_1}B_2\overline{B_3}+\overline{B_1}\overline{B_2}B_3$$
那么 $\Pr(B)$等于右端各事件概率的和(因为它们互斥)

这样作太麻烦了. 能否有其他作法呢?

我们知道“3个开关中至少有一个能够闭合”这一事件的
对立事件是“3个开关都不闭合”. 即$\overline{B_1}\overline{B_2}\overline{B_3}$.

$\therefore\quad \Pr( B) = 1- \Pr( \overline {B_{1}}) P( \overline {B_{2}}) P( \overline {B_{3}}) $

从而得解法如下:
\end{analyze}


\begin{solution}
记“这段时间内线路正常工作”为事件$B$, 开关 $J_{A1}$,
$J_{A2}$, $J_{A3}$能够闭合分别记为事件$B_1$, $B_2$, $B_3$. 则
\[\begin{split}
    \Pr(B)&=1-\Pr(\overline{B_1})\Pr(\overline{B_2})\Pr(\overline{B_3})\\
    &=1-[1-\Pr(B_1)][1-\Pr(B_2)][1-\Pr(B_3)]\\
    &=1-0.3\times0.3\times0.3=1-0.027=0.973.
\end{split}\]
答:在这段时间内线路正常工作的概率是 0.973.
\end{solution}


\section*{习题五}
\begin{center}
    \bfseries A
\end{center}

\begin{enumerate}
    \item 甲、乙二人生产合格产品的概率分别是0.8和0.9,从他们生产的产品中各抽取一件,都抽到合格品的概率是多少?
    \item 某射手射击一次,击中目标的概率是0.8,他连续射击3次,第一次未击中,其余2次都击中的概率是多少?
    \item 电器$K_1,K_2,K_3$并联在电路中,已知电流在$K_1$处断路(事件$A$)的概率是0.3,在$K_2$处断路(事件$B$)的概率为0.4,在$K_3$处断路(事件$C$)的概率为0.6.求全电路断路的概率.
    \item 将一枚硬币连续抛掷4次,4次都出现反面的概率是多少?
    \item 电器$K_1,K_2,\ldots,K_9$串联在电路中,每个电器使用3000小时的概率为0.99,计算这个电路能使用3000小时的概率.
    \item 有三箱产品,每箱100个,每箱有一个产品不合格,从三箱中各抽取一个,计算
\begin{enumerate}[(1)]
    \item 3个中恰有一个不合格的概率;
    \item 3个中至少有一个不合格的概率.
\end{enumerate}
\item 四门大炮中,每门大炮在一次射击中命中目标的概率是0.7.求四门大炮在一次射击中,至少有一门炮命中目标的概率.
\end{enumerate}

\section{独立重复试验}
在实际问题中,常常会遇到这样的随机试验:它是一系列重复试验,其中每次试验结果与其它次试验结果无关,并且每次试验只有两个结果:一个结果的概率总是$P$,另一个的概率总是$1-P$.例如进行一系列射击,每次射击只有两个结果——命中目标与没有命中目标. 在射击条件不变时,可以认为命中目标的概率总是$P$,不命中目标的概率总是$1-P$.下面,我们就来研究这类试验(即通常所说的贝努利型).

\begin{example}
某射手向一目标连续射击3次,已知每次命中目标的概率都是0.8,试求恰好命中两次的概率.    
\end{example}

\begin{solution}
这个试验相当于下列试验进行 3次:“射手向目标射击一次,观察目标命中与否”. 这个试验只有两个可能的结果: $A$( “ 命 中 目 标 ” ) 和$\overline {A}$( “ 未 命 中 目 标 ” ) , 又 已 知$\Pr( A) = 0. 8$,

$\therefore\quad \Pr(\overline{A})=1-0.8=0.2$

因此射手向一目标连续射击 3 次所有可能的结果可表示为:
\begin{center}
\begin{tabular}{p{.6\textwidth}c}
    $\overline{A}_1\overline{A}_2\overline{A}_3$, & (${\rm C}_3^0$个)\\
   $A_{1} \overline{A}_{2} \overline{A}_{3}, \overline{A}_{1}A_{2} \overline{A}_{3}, \overline{A}_{1} \overline{A}_{2}A_{3}$,  & (${\rm C}_3^1$个)\\
   $A_{1}A_{2} \overline{A}_{3},A_{1} \overline{A}_{2}A_{3}, \overline{A}_{1} \overline{A}_{2}A_{3}$,
& (${\rm C}_3^2$个)\\
$A_1A_2A_3$ & (${\rm C}_3^3$个)
\end{tabular}
\end{center}
其中,$A_i\; (i=1,2,3)$表示第$i$ 次射击命中目标。

由于 3 次射击是独立进行的,即试验结果$A_1,A_2,A_3$是相互独立的,从而上面列举的每个可能结果的概率都可求出. 如
$$P(A_{1}\overline{A}_{2}A_{3})=P(A_{1})\cdot P( \overline{A}_{2})\cdot P(A_{3})=0.8\times0.2\times0.8=0.128$$

令$B=$“恰好命中目标两次”,

$\because\quad  P=0.8,\quad 1-P=0.2$

则$D=A_{1}A_{2}\overline{A}_{3}+A_{1}\overline{A}_{2}A_{3}+\overline{A}_{1}A_{2}A_{3}$
\[\begin{split}
    \therefore\quad P(B)&=P(A_{1}A_{2} \overline{A}_{3}+A_{1} \overline{A}_{2}A_{3}+ \overline{A}_{1}A_{2}A_{3})\\
    &=P(A_{1}A_{2} \overline{A}_{3})+P(A_{1} \overline{A}_{2}A_{3})+P( \overline{A}_{1}A_{2}A_{3})\\
    &=P^{2}(1-P)+P^{2}(1-P)+P^{2}(1-P)\\
    &={\rm C}_{3}^{2}P^{2}(1-P)\\
    &=3\times0.8^{2}\times0.2=0.384
\end{split}\]
\end{solution}

一般地,\textbf{如果在一次试验中某事件发生的概率是$P$,那么在$n$次独立重复试验中这个事件恰好发生$K$次的概率}
\begin{equation}
    P_n(K)={\rm C}^K_n P^K (1-P)^{n-K}\tag{5}
\end{equation}

\begin{example}
    某一批蚕豆种籽,如果一粒发芽的概率为80\%,播下5粒种籽,计算
\begin{enumerate}[(1)]
\item 其中恰有4粒发芽的概率;
\item 至少有4粒发芽的概率.
\end{enumerate}
\end{example}

\begin{analyze}
    播下5粒种籽,每一粒种籽只有两个结果:发芽或不发芽.所以本题属于独立重复试验,可用公式(5)解决.
\end{analyze}

\begin{solution}
\begin{enumerate}[(1)]
    \item 把“播下一粒种籽,发芽”记为事件A.播下5粒种籽相当于作5次独立重复试验,根据公式(5),播下5粒种籽恰有4粒发芽的概率是:
$$P_{5}(4)={\rm C}_{5}^{4}P^{4}(1-P)^{5-4}=5\times0.8^{4}\times0.2\approx0.41$$
答:播下5粒种籽恰有 4 粒发芽的概率约为 0.41.

\item 播下5粒种籽,至少有4粒发芽的概率,就是恰有 4
粒发芽的概率和 5 粒都发芽的概率之和.
\[\begin{split}
 P&=P_{5}(4)+P_{5}(5)\\
&\approx0.410+C_{5}^{5}\times0.8^{5}(1-0.8)^{5-5}\\
&\approx0.410+0.328\approx0.74
\end{split}\]
答:播下5粒种籽,至少有 4 粒发芽的概率约为 0.74
\end{enumerate}
\end{solution}

\section*{习题六}
\begin{center}
    \bfseries A
\end{center}

\begin{enumerate}
\item 生产一件产品,合格产品的概率是0.96,问生产这种产品
4件,其中恰有1件次品,恰有2件次品,至少有一件次品
的概率各是多少?
\item 4门大炮向同一目标各射击一次,每门炮击中目标的概率
都是 $P$, 试求恰有 3门大炮击中目标的概率。
\item 把一枚硬币连续地掷5次,试求5次中至多有两次正面朝
上的概率.
\item 某射手打靶,每次命中的概率为 0.7, 现在连续射击五次,
分别写出射手恰好击中5次,4次,3次,2次,1次,0次的概率的计算式子,并将它们与$(0.7+0.3)^{5}$的展开式的各项进行比较,你有什么结论?
\item 某气象站天气预报准确率为90\%, 计算
\begin{enumerate}
    \item 5次预报中恰有4次准确的概率;
    \item 5次预报中恰有2次不准确的概率.
\end{enumerate}
\end{enumerate}

\section{本章小结}
\subsection*{知识结构分析}
\begin{enumerate}
    \item 概念
\begin{center}
\begin{tikzpicture}[>=stealth, thick]
\node(A) at (-1,0)[right]{随机试验};
\node(A1) at (1,1)[right]{样本点};
\node(A2) at (3.5,1)[right]{随机事件};
\node(A3)[text width=.13\textwidth, align=center] at (6,1)[right]{随机事件\\的频率};
\node(A4)[text width=.13\textwidth, align=center] at (9,1)[right]{随机事件\\的概率};
\node(B1) at (1,-1)[right]{样本空间};
\node(B2)[text width=.5\textwidth] at (3,-1)[right]{随机事件的关系(包含关系;等价关系;\\事件的和,差,积,逆;互斥事件)};

\draw[->](A1)--(A2);
\draw[->](A2)--(A3);
\draw[->](A3)--(A4);

\draw[->](A2)--+(0,-1.5);

\draw[decorate, decoration={brace, amplitude=5pt}](1,-1)--(1,1);
\end{tikzpicture}
\end{center}

    \item 几个概率公式
\begin{enumerate}[(1)]
\item 古典概型:$\Pr(A)=\frac{m}{n}$.(其中$m$为事件$A$包含的样本点数,$n$为样本点总数)
\item $A$,$B$为互斥事件时,$\Pr(A+B)=\Pr(A)+\Pr(B)$.
\item $A$,$B$为对立事件时,$\Pr(A)+\Pr(B)=1$,
或$\Pr(B)=1-\Pr(A)$.
\item $A$,$B$为相互独立事件时,$\Pr(A\cdot B)=\Pr(A)\Pr(B)$.
\item $n$次独立重复试验. $P_n(K)={\rm C}^K_n P^K (1-P)^{n-K}$

(其中$P$为在一次试验中某事件发生的概率,$P_n(K)$为在$n$次独立重复试验中这个事件恰好发生$K$次的概率.)
\end{enumerate}
\end{enumerate}

\subsection{几点说明}
\begin{enumerate}
\item 在解具体问题时,首先要根据题意判断问题属于哪个类型,选择适合的概率公式;
\item 解决问题的关键是把所求事件用已知概率的事件去
表示. 所以正确分析事件之间的关系是非常重要的一步.
\item 恰当地利用对立事件的概率公式,往往使计算大大简化.
\end{enumerate}


\section*{复习题十}
\begin{center}
    \bfseries A
\end{center}
\begin{enumerate}
\item 袋中有5个白球,6个黑球,7个红球,从中任取一球,求取出的球是白球或红球的概率.
\item 用数字$1,2,3,5,6$任意组成没有重复数字的五位数,求这数为偶数的概率.
\item 100支电子元件中有6支次品,从中任意抽取20支,求20支中有次品的概率.
\item 靶子由1—10环组成,某射手射击时,命中1—4环的概率是0.2,命中5—8环的概率为0.4,脱靶的概率是0,求他命中9—10环的概率.
\item 在试验中四个两两互斥事件出现的概率分别是0.01,0.011, 0.007, 0.002.求试验中这四个事件之一出现的概率.
\item 某军训团的学员实弹射击中,有一批学员中,每一学员一次射击命中目标的概率都是0.3,要找几个这样的学员,同时射击一次,就可以使击中目标的概率大于0.95?
\item 某工人照看三台机器,如果在十分钟内,机器不需要工人照看的概率,第一台是0.9,第二台是0.8,第三台是0.7,假设各个机器是否需要照着相互之间没有影响.计算在这十分钟内,至少有一台机器需要工人照看的概率.
\item 在一副扑克牌(52张)中,有“黑桃,红桃,梅花,方块”这四种花色的牌各13张,从中任取4张,这4张牌的花色相同的概率是多少?这4张牌的花色没有任何两张相同的概率是多少?
\end{enumerate}

\begin{center}
    \bfseries B
\end{center}

\begin{enumerate}
   \setcounter{enumi}{8} 
\item 甲袋有10个白球5个红球,乙袋有5个白球,10个红球,从两个袋子内各任意摸出一个,
\begin{enumerate}[(1)]
    \item 得到一个白球,一个红球的概率是多少?
    \item 得到两个白球的概率是多少?
\end{enumerate}
\item 八把钥匙中有一把可以开锁,从中任取三把,求可以打开锁的概率.
\item 某计算中心备有4部计算机,在一天内每部机器被使用的概率为0.4.求在一天内至少有三部机器被使用的概率.
\item 将10人任意分成两组,每组5人,求甲、乙两人正好分在同一组的概率.
\item 甲、乙两个篮球运动员在罚球线投球的命中率分别是0.7和0.6,
\begin{enumerate}[(1)]
    \item 现在两人各投一次,求至少一人命中的概率.
    \item 每人投球3次,计算两人都恰好投进两球的概率.
\end{enumerate}
\item 如图,将6个相同的元件先两两串联成3组,再把这3组并联成一个系统,已知每个元件损坏的概率为$P$,且各个
元件损坏与否相互之间没有影响,求这个系统能正常工作的概率.   
\end{enumerate}

\begin{figure}[htp]
    \centering
\begin{tikzpicture}
\draw(0,0)--(1,0)--(5,0);
\draw(1,-1) rectangle (4,1);
\draw[fill=white](1.5,-1.1) rectangle (2,-.9);    
\draw[fill=white](2.5,-1.1) rectangle (3,-.9);    


\end{tikzpicture}
    \caption*{(第14题)}
\end{figure}


































\chapter{函数的极限}


在学习数列极限的基础上,本章进一步学习以实数$x$为自变量的函数$f(x)$的极限.

\section{函数的极限}
在研究无穷数到$\{a_n\}$的极限时,如果数列有通项公式,那么这个通项表达式就是一个关于自然数$n$的函数式,例如
$a_n=\frac{n+1}{n}$. 求数列$\{a_n\}$的极限,也
就是求这个函数的极限. 即
\[\lim_{n\to\infty}a_n = \lim_{n\to\infty}\frac{n+1}{n}=\lim_{n\to\infty}\left(1+\frac{1}{n}\right)=1\]

而对于函数$f(x)=\frac{x+1}{x}\; (x\ne 0)$,当自变量$x\to \infty$时函数的变化趋势是什么含义呢?

由于$f(x)=\frac{x+1}{x}=1+\frac{1}{x}$,观察函数$y=f(x)=\frac{x+1}{x}$的图象(如图11.1)可以看出:
\begin{figure}[htp]
    \centering
\begin{tikzpicture}[>=stealth, scale=.5]
\draw[->](-5,0)--(5,0)node[below]{$x$};
\draw[->](0,-3)--(0,5.5)node[left]{$y$};
\draw[thick](-4.5,1)--(4.5,1)node[right]{$y=1$};
\node [below right]{$O$};
\draw[domain=.25:4, smooth, samples=100, very thick]plot(\x, {1+1/\x});
\draw[domain=-4:-.25, smooth, samples=100, very thick]plot(\x, {1+1/\x});
\node at (3,4){$y=\frac{x+1}{x}$};
\end{tikzpicture}
    \caption{}
\end{figure}



若 $x>0$, 则$\frac{x+1}x>1$, 且当 $x$ 的值无限增大(记作 $x\to+\infty$)时,有$\frac{1}{x}$的值无限趋近于$0\; \left(\text{记作}\frac{1}{x}\to 0\right)$, 所以$1+\frac{1}{x}$的值
无限趋近于$1\; \left(\text{记作 }1+\frac1x\to1\right)$. 我们称,当 $x\to+\infty$时,$\frac{x+1}x$ 的极限值为 1, 记作$\lim\limits_{x\to+\infty}\frac{x+1}{x}=1$.

若 $x< 0$, 则$\frac{x+1}x<1$,且当 $x$ 的绝对值无限增大(记作 $x\to -\infty$)时 , 有 $\frac 1x\to 0$, 所 以$1+ \frac 1x\to 1$, 我 们 称 当 $x\to -\infty$ 时, $\frac{x+1}{x}$的极限值为 1, 记作$\lim\limits_{x\to-\infty}\frac{x+1}{x}=1$.

若$x$的值可以为正、也可以为负,且当$x$的绝对值无限增大(记作 $x\to \infty$ )时 , 有$\frac 1x\to 0$, 所 以$1+ \frac 1x\to 1$. 我们称$x\to\infty$ 时,$\frac{x+1}x$的极限值为 1. 记作$\lim\limits_{x\to\infty}\frac{x+1}x=1$.

一般地,对于函数 $y=f(x)$, 当自变量 $x$ 的绝对值无限增大时,若 $f(x)$的值无限趋近于一个常数 $A$, 则称 $x\to\infty$时,函数 $f(x)$的极限是 $A$, 记作$\lim\limits_{x\to\infty} f(x)=A$.

另外,当$x\to+\infty$时,若$f(x)\to A$($A$为常数),记作$\lim\limits_{x\to+\infty}f(x)=A$; 当 $x\to-\infty$时,若 $f(x)\to A$($A$ 为常数),记作$\lim\limits_{x\to-\infty}f(x)=A$.

根据函数极限的定义,若$f(x)=C$($C$为常数),显然$\Lim{x}{\infty}f(x)=C$.

\begin{example}
根据函数极限的定义及函数的图象,说出下列极限.
\begin{multicols}{2}
\begin{enumerate}[(1)]
    \item $\Lim{x}{-\infty}(2^x+1)$
    \item $\Lim{x}{-\infty}(2^{-x}-1)$
    \item $\LIM{x}\frac{-x}{x+2}$
    \item $\Lim{x}{+\infty}(2^x+1)$
\end{enumerate}    
\end{multicols}
\end{example}

\begin{solution}
\begin{enumerate}[(1)]
\begin{minipage}{.45\textwidth}
\item 作函数$y=2^x+1$的图象(如图11.2).

    $\because\quad$当$x\to - \infty$时, $2^{x}\to 0$,

$\therefore\quad 2^{x}+1\to1$

$\therefore\quad \lim\limits_{n\to -\infty }( 2^{x}+ 1) = 1$
\end{minipage}
\hfill
\begin{minipage}{.45\textwidth}
\centering
\begin{tikzpicture}[>=stealth, scale=.5]
\draw[->](-3,0)--(3,0)node[below]{$x$};
\draw[->](0,-1)--(0,5)node[left]{$y$};
\node[below left]{$O$};
\draw(-3,1)--(2.5,1)node[right]{$y=1$};
\draw[domain=-2.5:2, samples=100, smooth, very thick]plot(\x, {2^(\x)+1})node[right]{$y=2^{x}+1$};
\foreach \x in {-1,-2,1,2}
{
    \draw(\x,0)--(\x,.1);
}
\foreach \x in {1,2,3,4}
{
    \draw(0,\x)--(.1,\x);
}
\end{tikzpicture}
\captionof{figure}{}
\end{minipage}

\begin{minipage}{.45\textwidth}
\item 作函数$y=2^{-x}-1$的图象(如图11.3).

$\because\quad $当$x\to + \infty$ 时, $2^{- x}\to 0$,

$\therefore\quad 2^{-x}-1\to-1$

$\therefore\quad \lim\limits_{x\to + \infty }( 2^{- x}- 1) = - 1$
\end{minipage}
\hfill
\begin{minipage}{.45\textwidth}
\centering
\begin{tikzpicture}[>=stealth, scale=.5]
\draw[->](-3,0)--(3,0)node[right]{$x$};
\draw[->](0,-2)--(0,4)node[right]{$y$};
\node[below left]{$O$};
\draw(-3,-1)--(2.5,-1)node[right]{$y=-1$};
\draw[domain=2:-2.25, samples=100, smooth, very thick]plot(\x, {2^(-\x)-1})node[above]{$y=2^{-x}-1$};
\foreach \x in {-1,-2,1,2}
{
    \draw(\x,0)--(\x,.1);
}
\foreach \x in {1,2,3}
{
    \draw(0,\x)--(.1,\x);
}
\end{tikzpicture}
\captionof{figure}{}
\end{minipage}
    
\item 作函数$y=\frac{-x}{x+2}$的图象(如图11.4).

$\because\quad f(x)=\frac{-x}{x+2}=-1+\frac{2}{x+2}$,
当 $x\to \infty$ 时 , $\frac 2{x+ 2}\to 0$

$\therefore\quad - 1+ \frac 2{x+ 2}\rightarrow - 1$

$\therefore\quad \lim\limits_{x\to \infty }\frac {- x}{x+ 2}= - 1$

\begin{figure}[htp]
    \centering
\begin{tikzpicture}[>=stealth, scale=.3]
\draw[->](-9,0)--(7,0)node[right]{$x$};
\draw[->](0,-8)--(0,8)node[right]{$y$};
\node[below left]{$O$};
\draw(-9,-1)--(6.5,-1)node[right]{$y=-1$};
\draw(-2,-8)node[below]{$x=-2$}--(-2,8);
\draw[domain=-1.75:6.5, samples=100, smooth, very thick]plot(\x, {-\x/(\x+2)});
\draw[domain=-9:-2.3, samples=100, smooth, very thick]plot(\x, {2/(\x+2)-1});

\foreach \x in {-7,-6,...,6}
{
    \draw(\x,0)--(\x,.2);
    \draw(0,\x)--(.2,\x);
}
\node at (-4,-4)[left]{$y=\frac{-x}{x+2}$};

\end{tikzpicture}
    \caption{}
\end{figure}


\item $\because\quad $当 $x\to + \infty$ 时 , $2^{x}+ 1$ 的值无限增大,即 $2^x+1\to+\infty$(如图11.2).

$\because\quad $当 $x\to+\infty$时, $f(x)=2^x+1$ 的值不是无限趋近一个常数,

$\therefore\quad \lim\limits_{x\to + \infty }\left ( 2^x+ 1\right )$不存在. 
\end{enumerate}
\end{solution}


一般地,对于函数 $y=f(x)$, 当 $x\to\infty$时,若$|f(x)|\to+\infty$或 $f(x)$的值不是无限趋近一个
常数 $A$, 那么称当 $x\to\infty$时, $f( x)$ 的 极 限 不 存 在.

在本节开始,对于函数$y=\frac{x+1}{x}\; (x\neq0)$, 我们已经解决了
当 $x\to\infty$时,$f(x)$的变化趋势. 而当 $x\to x_0$ ($x_{0}$ 为给定的常数)
时,$f(x)$的变化趋势如何?

对于函数 $f(x)=2x+1$, 当 $x\to3$ 时,有 $2x+1\to7$, 我们
称 $f(x)$的极限值为 7, 即$\lim\limits_{x\to3}f(x)=7$.

下面再研究函数$y=\frac{x^{2}-1}{x-1}$ 在点 $x=1$ 处的变化趋势.

$\because\quad$ 当$x\neq1$时, $y=\frac{x^{2}-1}{x-1}=x+1$.

$\therefore\quad$ 观察函数 $y=\frac{x^{2}-1}{x-1}$ 的图象(如图 11.5)可知, 此函数图象是一条在点 $A(1,2)$处有个“洞”的直线.

\begin{figure}[htp]
    \centering
\begin{tikzpicture}[>=stealth]
\draw[->](-2,0)--(3.5,0)node[right]{$x$};    
\draw[->](0,-1)--(0,3)node[right]{$y$}; 
\draw[domain=-1.5:2, smooth, very thick]plot(\x, {\x+1})node[right]{$y=\frac{x^{2}-1}{x-1}$};
\draw[dashed](1,0)node[below]{1}--(1,2)node[right]{$A(1,2)$}--(0,2)node[left]{2};
\draw[fill=white](1,2)circle(2pt);
\node[below left]{$O$};
\node at (0,1)[left]{1};
\end{tikzpicture}
    \caption{}
\end{figure}

若$x> 1$, 则$\frac {x^{2}- 1}{x- 1}> 2$, 且当 $x$的值 无 限 趋 近 于 1(记作 $x\to 1^{+ }$)时 , 有 $x+ 1\to 2$, 所 以 $\frac {x^{2}- 1}{x- 1}\to 2$. 我 们 称 当  $x\to 1^{+ }$时 , $\frac {x^{2}- 1}{x- 1}$的 极 限 值 为  2, 记 作
$\lim\limits_{x\to1^+}\frac{x^2-1}{x-1}=2$.

若 $x<1$, 则$\frac{x^{2}-1}{x-1}<2$, 且当 $x$ 的值无限趋近于 1(记作
$x\to 1^{- }$) 时 , 有$x+ 1\to 2$, 所 以$\frac {x^{2}- 1}{x- 1}\to 2$. 我们称当$x\to1^-$时, 
$\frac{x^2-1}{x-1}$的极限值为 2, 记作$\lim\limits_{x\to1^-}\frac{x^2-1}{x-1}=2$.

当$x\neq1$且$x$的值无限趋近于 1(这时$x$的值大于1 或小于1,记作 $x\to1$)时,因为 $x+1\to2$, 所以$\frac{x^{2}-1}{x-1}\to2$. 我们称当
$x\to1$ 时, $\frac{x^{2}-1}{x-1}$的极限值为 2, 记作$\lim\limits_{x\to1}\frac{x^{2}-1}{x-1}=2$.

一般地,已知函数 $y=f(x)$在点 $x=x_0$ 的附近有定义,当自变量$x$取不同于$x_0$的值且无限地趋近于$x_0$时,若$f(x)$的值无限地趋近于一个常数 $A$, 则称当 $x\to x_0$ 时,函数 $f(x)$的极限值是 $A$, 记作 $\lim_{x\to x_0}\limits f(x)=A$.

另外,当$x\to x_0^-$时,若$f(x)\to A$($A$为常数), 我们称函数$f(x)$在点 $x=x_0$ 处有左极限,记作$\lim\limits_{x\to x_{0}^-}f(x)=A$; 当$x\to x_{0}^{+}$ 时,若$f(x)\to A$($A$为常数), 我们称函数$f(x)$在点$x=x_0$处有右极限,记作$\lim\limits_{x\to x^+_0}f(x)=A$.

\begin{example}
根据函数的极限定义及函数的图象,说出下列函
数指定的极限。
\begin{multicols}{2}
\begin{enumerate}[(1)]
    \item $ \lim\limits _{x\to 1^{- }}\left ( x^{2}- 1\right )$
    \item $\lim\limits _{x\to - 1^{+ }}\left ( x^{2}- 1\right )$
    \item $\lim\limits _{x\to 0}( x^{2}- 1)$
    \item $\lim\limits _{x\to \tfrac 12}( x^{2}- 1) $
\end{enumerate}
\end{multicols}
\end{example}

\begin{solution}
设$f( x) = x^2- 1\; (x\in [ - 1,1])$. 作函数 $y= x^2- 1\; ( x\in [ - 1, 1] )$的 图 象 (如 图  11.6) , 观 察 图 象 可 知 : 

\noindent
\begin{minipage}{.5\textwidth}
\begin{enumerate}[(1)]
    \item $\because\quad $当$x\to1^-$时,$x^2-1\to0$.
    
    $\therefore\quad \lim\limits_{x\to 1^{- }}( x^{2}- 1) = 0.$
    
\item      $\because\quad $当$x\to -1^+$时,$x^2-1\to0$.
    
   $\therefore\quad \lim\limits_{x\to 1^{+ }}( x^{2}- 1) = 0$

   \item      $\because\quad $当$x\to 0$时,$x^2-1\to-1$.
    
   $\therefore\quad \lim\limits_{x\to 0}( x^{2}- 1) = -1$

   \item      $\because\quad $当$x\to \frac{1}{2}$时,$x^2-1\to -\frac{3}{4}$.
    
   $\therefore\quad \lim\limits_{x\to \tfrac{1}{2}}( x^{2}- 1) = -\frac{3}{4}$
\end{enumerate}    
\end{minipage}
\hfill
\begin{minipage}{.4\textwidth}
    \centering
\begin{tikzpicture}[>=stealth]
\draw[->](-2,0)--(2,0)node[below]{$x$};
\draw[->](0,-2)--(0,1.5)node[left]{$y$};
\draw[domain=-1:1, smooth, very thick]plot(\x, {\x*\x-1});
\foreach \x in {-1,1}
{
    \draw[fill](\x,0)node[above]{$\x$} circle(1pt);
}
\draw(.5,-.75)--(.5,0)node[above]{$\frac{1}{2}$};
\node at (0,-1)[below left]{$-1$};
\node[below left]{$O$};
\end{tikzpicture}
\captionof{figure}{}
\end{minipage}

\end{solution}

\begin{example}
已知:函数$f(x)=\begin{cases}
    2-x,& x\ge 1\\
    -1,& -1<x<1\\
    1+x,& x\le -1
\end{cases}$

根据函数的极限定义及函数的图象,说出下列函数指定
的极限。
\begin{multicols}{3}
\begin{enumerate}[(1)]
    \item $\Lim{x}{1^+}f(x)$
    \item $\Lim{x}{1^-}f(x)$
    \item $\Lim{x}{-1^+}f(x)$
    \item $\Lim{x}{-1^-}f(x)$
    \item $\Lim{x}{-2}f(x)$
    \item $\Lim{x}{2}f(x)$
\end{enumerate}
\end{multicols}
\end{example}

\begin{solution}
作函数$y=f(x)$的图象(如图11.7)

\begin{multicols}{2}
    \begin{enumerate}[(1)]
\item $\because\quad$ 当  $x\to 1^{+ }$时, $2-x\to1$

$\therefore\quad \lim\limits _{x\to 1^{+ }}f( x) = 1$

\item $\because\quad$ 当$x\to1^-$时, $f(x)=-1$,

$\therefore\quad\lim\limits _{x\to 1^{-}} f(x)=-1.$

\item $\because\quad$ 当$x\to-1^+$时, $f(x)=-1$

$\therefore\quad\lim\limits_{x\to - 1^{+ }}f( x) = - 1.$ 

\item $\because\quad$ 当  $x\to -1^{- }$时, $1+x\to 0$

$\therefore\quad \lim\limits _{x\to -1^{- }}f( x) = 0$

\item $\because\quad$ 当$x\to -2$时, $1+x\to -1$,

$\therefore\quad\lim\limits _{x\to -2} f(x)=-1.$

\item $\because\quad$ 当$x\to 2$时, $2-x\to 0$

$\therefore\quad\lim\limits_{x\to 2}f( x) = 0.$ 
\end{enumerate}
\end{multicols}

\end{solution}

\begin{figure}[htp]
    \centering
\begin{tikzpicture}[>=stealth, scale=.7]
\draw[->](-4,0)--(5,0)node[below]{$x$};
\draw[->](0,-4)--(0,3)node[left]{$y$};
\draw[domain=-4:-1, smooth, very thick]plot(\x, {\x+1});
\draw[domain=1:4.5, smooth, very thick]plot(\x, {-\x+2})node[right]{$f(x)$};
\draw[domain=-1:1, smooth, very thick]plot(\x, -1);

\foreach \x/\y in {1/0, -1/-1, 1/-1, 1/1}
{
    \draw[fill](\x,\y) circle(2pt);
}
\node at (-2,-1)[left]{$D(-2,-1)$};
\node at (-1,-1)[below]{$C(-1,-1)$};
\node at (1,-1)[below right]{$B(1,-1)$};
\node at (1,1)[above right]{$A(1,1)$};

\foreach \x in {2,-1,-2}
{
    \node at (\x,0)[above]{$\x$};
}
\node at (1,0)[above left]{1};
\draw[dashed](-2,-1)--(-2,0);
\draw[dashed](1,-1)--(1,1);
\draw[dashed](-1,-1)--(-1,0);

\foreach \x in {-3,-2,-1,1,2}
{
    \draw(0,\x)--(.1,\x);
}

\node[below right]{$O$};
\end{tikzpicture}    
    \caption{}
\end{figure}

根据函数的极限的定义,不难得出:
\begin{enumerate}
    \item 若函数 $f(x)$在点 $x=x_0$ 处附近有定义则$\lim\limits_{x\to x_{0}}f(x)=A$的 充 分 必 要 条 件 是$\lim\limits _{x\to x^-_0 }f( x) = A$且$\lim\limits_{x\to x^+_0}f(x)=A$(其中$A$为常数).
    \item 对于函数$f(x)=C$($C$为常数)及任一给定的常数$x_0$, 恒有$\lim\limits_{x\to x_0}f(x)=c$.
    
现在我们看函数$f(x)=\frac{x+1}{x}$在点$x=0$处附近的变化趋势. 由函数$y=\frac{x+1}{x}$的图象(如图 11.1)可知:
\begin{itemize}
\item 当$x\to0^{+}$时, $\frac{x+1}x>0$且$\frac{x+1}x$的值无限增大,所以
$\lim\limits_{x\to0^+}\frac{x+1}x$不存在.
\item 
当$x\to0^{-1}$时, $\frac{x+1}x<0$且$\left|\frac{x+1}x\right|$的值无限增大,所以
$\lim\limits_{x\to0^-}\frac{x+1}x$也不存在.
\end{itemize}
我们也可以说当 $x\to0$ 时,$\lim\limits_{x\to0}\frac{x+1}x$不存在.
\end{enumerate}

\subsection*{练习}
\begin{enumerate}
    \item 根据函数的极限定义和函数的图象,写出下列各极限。
\begin{multicols}{2}
\begin{enumerate}[(1)]
    \item $\lim\limits_{x\to+\infty}(1-2^{-x})$
    \item $\lim\limits_{x\to+\infty}|2^{-x}-1|$
\item $\lim\limits_{x\to\infty}\frac{2x}{x-1}$
\item $\lim\limits _{x\to 0}|x|$
\item $\lim\limits _{x\to - 1}( - x+ 1)$
\item $\lim\limits_{x\to-1}\frac{x^2-1}{x+1}$
\item $\lim\limits _{x\to  1}( - x+ 1)$
\item $\lim\limits _{x\to  1}\frac{x^2-1}{x+1}$
\end{enumerate}
\end{multicols}

\begin{tikzpicture}[>=stealth, scale=.4]
\begin{scope}
    \draw[->](-4,0)--(5,0)node[below]{$x$};
    \draw[->](0,-4)node[below]{第(1)题}--(0,3)node[left]{$y$};
\draw (-4,1)--(5,1)node[above]{$y=1$};
\draw[domain=3:-2, smooth, very thick]plot(\x, {1-2^(-\x)})node[below left]{$y=1-2^{-x}$};
\node [below right]{$O$};
\foreach \x  in {1,2,3}
{
    \draw(\x,0)--(\x,.1);
    \draw(0,-\x)--(.1,-\x);
}
\end{scope}
\begin{scope}[xshift=13cm, yshift=-2cm]
    \draw[->](-4,0)--(5,0)node[below]{$x$};
\draw[->](0,-2)node[below]{第(2)题}--(0,5)node[left]{$y$};
\draw(-4,1)--(5,1)node[above]{$y=1$};
\draw[domain=0:3, smooth, very thick]plot(\x, {1-2^(-\x)});
\draw[domain=0:-2.2, smooth, very thick]plot(\x, {2^(-\x)-1});
\node at (-3,3)[above]{$y=|2^{-x}-1|$};
\node [below left]{$O$};
\foreach \x  in {-1,1,2,3,4}
{
    \draw(\x,0)--(\x,.1);
    \draw(0,\x)--(.1,\x);
}

\end{scope}
\end{tikzpicture}

\begin{tikzpicture}[>=stealth]
\begin{scope}[scale=.45]
    \draw[->](-4,0)--(5,0)node[below]{$x$};
    \draw[->](0,-4)node[below]{第(3)题}--(0,5)node[left]{$y$};
\draw (-4,2)--(5,2)node[below]{$y=2$};
\draw (1,-3.5)node[right]{$x=1$}--(1,5);
\draw[domain=1.7:5, smooth, very thick]plot(\x, {2+2/(\x-1)});
\draw[domain=.6:-4, smooth, very thick]plot(\x, {2+2/(\x-1)});
\node [below left]{$O$};
\foreach \x  in {-3,-1,-2,1,2,3}
{
    \draw(\x,0)--(\x,.1);
    \draw(0,\x)--(.1,\x);
}
\end{scope}
\begin{scope}[xshift=6cm, yshift=-1cm]
    \draw[->](-2.5,0)--(3,0)node[below]{$x$};
\draw[->](0,-1)node[below]{第(4)题}--(0,3)node[left]{$y$};
\draw[domain=0:2, smooth, very thick]plot(\x, \x);
\draw[domain=0:-2, smooth, very thick]plot(\x, -\x);
\node at (2,2)[right]{$y=|x|$};
\node [below left]{$O$};
\foreach \x  in {1,2}
{
    \draw(\x,0)--(\x,.1);
    \draw(-\x,0)--(-\x,.1);
    \draw(0,\x)--(.1,\x);
}
\end{scope}

\end{tikzpicture}

\begin{tikzpicture}[>=stealth, scale=.6]
\begin{scope}
    \draw[->](-4,0)--(4,0)node[below]{$x$};
    \draw[->](0,-1)node[below]{第(5)(7)题}--(0,5)node[left]{$y$};
\draw[domain=1.5:-2.5, smooth, very thick]plot(\x, {-\x+1})node[above]{$y=-x+1$};
\foreach \x  in {1,2,3}
{
    \draw(\x,0)--(\x,.1);
    \draw(-\x,0)--(-\x,.1);
    \draw(0,\x)--(.1,\x);
}
\foreach \x in {-1,1}
{
    \node at (\x,0)[below]{$\x$};
}
\node at (.2,1)[right]{1};
\node [below right]{$O$};

\draw[dashed](-1,0)--(-1,2);
\end{scope}
\begin{scope}[xshift=8cm, yshift=2cm]
    \draw[->](-2.5,0)--(4,0)node[below]{$x$};
\draw[->](0,-3)node[below]{第(6)(8)题}--(0,3)node[left]{$y$};
\draw[domain=-2:3, smooth, very thick]plot(\x, \x-1)node [right]{$y=\frac{x^2-1}{x+1}$};
\draw[fill=white](-1,-2)node[right]{$A(-1,-2)$} circle(2pt);
\draw[dashed](-1,0)node[above]{$-1$}--(-1,-2);
\node [below left]{$O$};
\foreach \x  in {1,2}
{
    \draw(\x,0)--(\x,.1);
    \draw(-\x,0)--(-\x,.1);
    \draw(0,\x)--(.1,\x);
}
\node at (1,0)[below]{1};
\node at (0,-1)[right]{$-1$};
\end{scope}

\end{tikzpicture}

\noindent
\begin{minipage}{.45\textwidth}
\item 已知:$f(x)=\arccos x$, 根据函数的极限的定义和函数的图象, 写 出 下 列 各 函 数 指 定 的 极 限 . 
\begin{multicols}{2}
\begin{enumerate}[(1)]
    \item $\lim\limits _{x\to 1^- }f( x)$
    \item $\lim\limits _{x\to -1^+ }f( x)$
    \item $\lim\limits _{x\to 0^+ }f( x)$
    \item $\lim\limits _{x\to 0^- }f( x)$
    \item $\lim\limits _{x\to 0 }f( x)$
    \item $\lim\limits _{x\to \tfrac{1}{2}^+ }f( x)$
    \item $\lim\limits _{x\to \tfrac{1}{2}^- }f( x)$
    \item $\lim\limits _{x\to \tfrac{1}{2} }f( x)$
\end{enumerate}
\end{multicols}
\end{minipage}\hfill
\begin{minipage}{.45\textwidth}
\centering
\begin{tikzpicture}[>=stealth]
    \draw[->](-2.5,0)--(2.5,0)node[below]{$x$};
    \draw[->](0,-1)node[below]{第3题}--(0,4)node[left]{$y$};
\draw[dashed](-1,0)node[below]{$-1$}--(-1,pi)node[above left]{$A(-1,\pi)$}--(0,pi)node[right]{$\pi$};
\draw[dashed](.5,0)node[below]{$\frac{1}{2}$}--(.5,pi/3)node[right]{$B\left(\frac{1}{2},\frac{\pi}{3}\right)$}--(0,pi/3)node[left]{$\frac{\pi}{3}$};
\node[below left]{$O$};
\draw[domain=0:pi, smooth, samples=100, very thick]plot({cos(\x r)}, \x);
\node at (0,pi/2)[above right]{$\frac{\pi}{2}$};

\end{tikzpicture}    
\end{minipage}

\item 已知:$f(x)=\begin{cases}
    x &x\ge 0\\
-x+1 & x<0
\end{cases}$,
根据函数的极限的定义和函数的图象, 写 出 下 列 各 函 数 指 定 的 极 限 . 
\begin{multicols}{2}
\begin{enumerate}[(1)]
    \item $\lim\limits _{x\to 0^+ }f( x)$
    \item $\lim\limits _{x\to 0^- }f( x)$
    \item $\lim\limits _{x\to 2 }f( x)$
    \item $\lim\limits _{x\to -2 }f( x)$
\end{enumerate}
\end{multicols}
\end{enumerate}

\section{函数极限的四则运算法则}
函数极限与数列极限有着类似的四则运算法则,利用它们可求一些较复杂的函数的极限.

如果$\lim\limits_{x\to x_{0}}f(x)=A$, $\lim\limits_{x\to x_{0}}g(x)=B$, 那么
\begin{enumerate}
    \item $\lim\limits_{x\to x_{0}}[f(x)\pm g(x)]=\lim\limits_{x\to x_{0}}f(x)\pm\lim_{x\to x_{0}}g(x)=A\pm B$
    \item $\lim\limits_{x\to x_{0}}[f(x)\cdot g(x)]=\lim\limits_{x\to x_{0}}f(x)\cdot\lim_{x\to x_{0}}g(x)=A  B$
\begin{thm}
   {推论} $\lim\limits_{x\to x_{0}}Cf( x) = C\lim\limits _{x\to x_{0}}f( x) = CA\quad (C\text{为常数})$
\end{thm}
\item $\lim\limits_{x\to x_0}\frac{f(x)}{g(x)}=\frac{\lim\limits_{x\to x_0}f(x)}{\lim\limits_{x\to x_0}g(x)}=\frac{A}{B}$

(当$x\to x_0$时,$g(x)\ne 0$且$B\ne 0$)
\end{enumerate}

\begin{rmk}
\begin{enumerate}
    \item 函数极限的四则运算法则的实质仍是改变运算顺序,
    \item 上述函数极限的四则运算法则还可以推广到有限个
函数的和、差、积、商的情况。
\item 上述函数极限的四则运算法则对于自变量$x$来说,当
$x\to+\infty$, $x\to-\infty$, $x\to\infty$, 或 $x\to x_0^+$, $x\to x_0^-$ 等情况也是成立
的.
\end{enumerate}
\end{rmk}

\begin{example}
已知:$f(x)=3x^2-2x-1$,求$\lim\limits_{x\to 2}f(x)$.
\end{example}

\begin{solution}
    \textbf{解法1:}
\[\begin{split}
    \lim\limits_{x\to 2}f(x)&=\lim\limits_{x\to 2}(3x^2-2x-1)\\
&=\lim\limits_{x\to 2}(3x^2)-\lim\limits_{x\to 2}(2x)-\lim\limits_{x\to 2}1\\
&=3\left(\lim\limits_{x\to 2} x\right)^2 -2\lim\limits_{x\to 2} x-1\\
&=3\x 2^2-2\x 2-1 =7
\end{split} \]
    
\textbf{解法2:}
\[\begin{split}
    \lim\limits_{x\to 2}f(x)&=\lim\limits_{x\to 2}[(3x+1)(x-1)]\\
&=\lim\limits_{x\to 2}(3x+1)\cdot \lim\limits_{x\to 2}(x-1)\\
&=\left[\lim\limits_{x\to 2}(3x)+\lim\limits_{x\to 2}1\right]\left(\lim\limits_{x\to 2}x- \lim\limits_{x\to 2}1\right)\\
&=(6+1)(2-1)=7
\end{split} \]
\end{solution}

\begin{example}
已知:$f(x)=\frac{x^2-2x-3}{x^2-9}$,求
\begin{multicols}{2}
\begin{enumerate}[(1)]
    \item $\lim\limits_{x\to 3}f(x)$
    \item $\lim\limits_{x\to -3}f(x)$
\end{enumerate}
\end{multicols}
\end{example}

\begin{analyze}
\[f(x)=\frac{(x-3)(x+1)}{(x-3)(x+3)},\qquad x\in(-\infty,-3)\cup(-3,3)\cup(3,+\infty)\]
\begin{enumerate}[(1)]
    \item $\because\quad $当$x\to3$时,$x^2-9\to 0$
    
    $\therefore\quad $不能直接运用函数极限的运算法则求极限.

    由于当$x\to 3$时,$x\ne 3$即$x-3\ne 0$. 若先约去$f(x)$的分子、分母中的公因式$x-3$,则可求$f(x)$的极限.

\item $\because\quad $当$x\to-3$时,$x\ne -3$且$x+3\to 0$,而$f(x)=\frac{x+1}{x+3}=1-\frac{2}{x+3}$,这时$\left|\frac{2}{x+3}\right|\to +\infty$

$\therefore\quad f(x)$无极限.
\end{enumerate}
\end{analyze}

\begin{solution}
\begin{enumerate}[(1)]
    \item $\because\quad $当$x\to 3$时,$x-3\ne 0$

\[\begin{split}
    \therefore\quad \lim\limits_{x\to 3}f(x)&=\lim\limits_{x\to 3}\frac{(x-3)(x+1)}{(x-3)(x+3)}
    =\lim\limits_{x\to 3}\frac{x+1}{x+3}\\
    &=\frac{\lim\limits_{x\to 3}(x+1)}{\lim\limits_{x\to 3}(x+3)}=\frac{3+1}{3+3}=\frac{2}{3}
\end{split} \]


\item $f(x)=\frac{x+1}{x+3}=1-\frac{2}{x+3}$

又$\because\quad $当$x\to -3$时,$x+3\to 0$

$\therefore\quad \frac{2}{x+3}\to \infty$

$\therefore\quad 1-\frac{2}{x+3}\to\infty$,即$\lim\limits_{x\to-3}f(x)$不存在.
\end{enumerate}
\end{solution}

\begin{example}
    已知$f(x)=\frac{3x^{2}+2x-1}{4x^{2}-100x}$, 求$\lim\limits_{x\to\infty}f(x)$.    
\end{example}

\begin{analyze}
    因为当$x\to\infty$时, $f(x)$的分子, 分母都无极限. 所以不能直接运用函数根据的运算法则求极限.

    又$x\neq0$, 故可像求数列的极限一样, 先将$f(x)$的分子、分母同除以$x$的最高次幂后, 再求极限.
\end{analyze}

\begin{solution}
\[\begin{split}
    \lim\limits_{x\to\infty}f(x)&=\lim\limits_{x\to\infty}\frac{3x^{2}+2x-1}{4x^{2}-100x}=\lim\limits_{x\to\infty}\frac{3+\frac{2}{x}-\frac{1}{x^{2}}}{4-\frac{100}{x}}\\
    &=\frac{\lim\limits_{x\to\infty}\left(3+\frac{2}{x}-\frac{1}{x^{2}}\right)}{\lim\limits_{x\to\infty}\left(4-\frac{100}{x}\right)}=\frac{3+\lim\limits_{x\to\infty}\frac{2}{x}-\lim\limits_{x\to\infty}\frac{1}{x^2}}{4-\lim\limits_{x\to\infty}\frac{100}{x}}=\frac{3}{4}.
\end{split}\]
\end{solution}

\begin{example}
已知:$f(x)=\frac{1}{x-2}-\frac{3}{x^3-8}$,求$\Lim{x}{2}f(x)$.
\end{example}

\begin{analyze}
$\because\quad $当$x\to 2$时,$\frac{1}{x-2}\to\infty$, $\frac{3}{x^3-8}\to\infty$

$\therefore\quad $不能直接运用函数极限的四则运算法法则求$f(x)$的极限,应将$f(x)$的解析式先合并成一个分式,看它的分子、分母能否约去$x-2$的因式.
\end{analyze}

\begin{solution}
\[\begin{split}
\because\quad f(x)&=\frac{1}{x-2}-\frac{3x+6}{x^3-8}=\frac{x^2+2x+4-(3x+6)}{(x-2)(x^2+2x+4)}\\
&=\frac{x^2-x-2}{(x-2)(x^2+2x+4)}=\frac{(x-2)(x+1)}{(x-2)(x^2+2x+4)}
\end{split}\]
而当$x\to 2$时,$x-2\ne 0$
\[\begin{split}
\therefore\quad \Lim{x}{2}f(x)&=\Lim{x}{2}\frac{(x-2)(x+1)}{(x-2)(x^2+2x+4)}=\Lim{x}{2} \frac{x+1}{x^2+2x+4}\\
&=\frac{\Lim{x}{2}(x+1)}{\Lim{x}{2}(x^2+2x+4)}=\frac{\Lim{x}{2}x+1}{\Lim{x}{2}x^2+\Lim{x}{2}2x+4}\\
&=\frac{2+1}{4+4+4}=\frac{1}{4}
\end{split}\]
\end{solution}

\section*{习题一}
\begin{center}
\bfseries A
\end{center}

\begin{enumerate}
    \item 根据函数的极限的定义及函数的图象,分别写出下列各极限:
\begin{multicols}{3}
\begin{enumerate}[(1)]
    \item $\Lim{x}{\infty}x^{-2}$
    \item $\Lim{x}{+\infty}\left(\frac{1}{2}\right)^x$
    \item $\Lim{x}{+\infty}\arctan x$
    \item $\Lim{x}{-\infty}\arctan x$
    \item $\Lim{x}{\tfrac{\pi}{2}}\sin x$
    \item $\Lim{x}{0} \cos x$
\end{enumerate}
\end{multicols}

\item 已知:$f(x)=\begin{cases}
    (x-1)^2 & x>0\\
    0& x=0\\
    -(x+1)^2 & x<0
\end{cases}$,作函数$y=f(x)$的图象,写出下列各极限:
\begin{multicols}{4}
\begin{enumerate}[(1)]
    \item $\Lim{x}{1}f(x)$
    \item $\Lim{x}{-1}f(x)$
    \item $\Lim{x}{0}f(x)$
    \item $\Lim{x}{2}f(x)$
\end{enumerate}
\end{multicols}

\item 利用函数极限的四则运算法则求下列极限:
\begin{multicols}{2}
\begin{enumerate}[(1)]
    \item $\LIM{x}\frac{3x+1}{x^2-4}$
    \item $\LIM{x}\frac{x^2+x-1}{2x^2+5}$
    \item $\LIM{x}\frac{3x^2-2x+4}{2x+1}$
    \item $\Lim{x}{1}(5x^2-2x+7)$
    \item $\Lim{x}{-1}\frac{x^3+1}{x+1}$
    \item $\Lim{x}{2}\frac{x^2-x-2}{x^2-2x}$
    \item $\Lim{x}{1}\frac{x^2+x-2}{x^2-3x+2}$
    \item $\Lim{x}{1}\left(\frac{2}{x-1}-\frac{x+3}{x^2-1}\right)$
    \item $\Lim{x}{0}\frac{x^2+x}{2x^2-x}$
    \item $\Lim{x}{0}\left(1-\frac{2x^2-3x}{x}\right)$
\end{enumerate}
\end{multicols}
\end{enumerate}

\section{函数的连续性}
在11.2节中求函数$f(x)=3x^2-2x-1$在点$x=2$处的极限值时,我们发现
\[\lim_{x\to 2}f(x)=\lim_{x\to 2}(3x^2-2x-1)=3\x2^2-2\x2-1=7\]
而$f(2)$的值也是7.

这是不是偶然的巧合?当任取$x_0\in\R$,求函数$f(x)=3x^2-2x-1$在点$x=x_0$处的极限值时,通过计算仍有$Lim{x}{x_0}f(x)=3x^2_0-2x_0-1=f(x_0)$. 具有这种性质的函数就是本节要研究的连续函数.

下面先给出函数在某一点处连续的定义.

已知函数$y=f(x)$及点$x=x_0$,如果函数$f(x)$满足以下3个条件:
\begin{enumerate}
\item $f(x)$在点$x=x_0$处及其附近有定义;
\item $\Lim{x}{x_0}f(x)$存在;
\item $\Lim{x}{x_0}f(x)=f(x_0)$.
\end{enumerate}

那么称函数$f(x)$在点$x=x_0$处连续.

根据这个定义,如果一个函数$y=f(x)$在点$x=x_0$处不能完全满足上述3个条件,那么此函数在这个点就不连续,这时称点$x=x_0$是函数$f(x)$的间断点. 例如:

点$x=1$是函数$y=\frac{x^2-1}{x-1}$的间断点(如图11.8).

点$x=\frac{\pi}{2}$是函数$y=\tan x$ 的一个间断点(如图11.9).

\noindent
\begin{minipage}{.45\textwidth}
    \centering
\begin{tikzpicture}[>=stealth]
\draw[->](-2,0)--(3,0)node[below]{$x$};
\draw[->](0,-1)--(0,4)node[left]{$y$};
\draw[domain=-1.5:2, smooth, very thick]plot(\x, \x+1)node[above]{$y=\frac{x^2-1}{x-1}$};
\draw[dashed](1,0)node[below]{1}--(1,2)--(0,2)node[left]{2};
  \node[below left]{$O$};
  \draw[fill=white](1,2)circle(1.5pt);
\end{tikzpicture}  
\captionof{figure}{}
\end{minipage}\hfill
\begin{minipage}{.45\textwidth}
    \centering
\begin{tikzpicture}[>=stealth, scale=.6]
    \draw[->](-2.5,0)--(6,0)node[below]{$x$};
    \draw[->](0,-4.5)--(0,4.5)node[left]{$y$};
\draw[domain=-pi/2+.25:pi/2-.25, very thick, smooth, samples=100]plot(\x, {tan(\x r)})node[above right]{$y=\tan x$};
\draw[domain=-pi/2+.25:pi/2-.25, very thick, smooth, samples=100]plot(\x+pi, {tan(\x r)});
\foreach \x in {-1,1,3}
{
    \draw(\x*pi/2,-4)--(\x*pi/2,4);
}
\node at (pi/2,0)[below left]{$\frac{\pi}{2}$};
\node at (pi/2,-4)[below]{$x=\frac{\pi}{2}$};
\node [below right]{$O$};
\end{tikzpicture}  
\captionof{figure}{}
\end{minipage}


点$x=0$是函数$y=\begin{cases}\frac{|x|}{x},& x\neq0 \\0,& x=0\end{cases}$的间断点(如图11.10).

\noindent
\begin{minipage}{.45\textwidth}
\centering
\begin{tikzpicture}[>=stealth]
    \draw[->](-3,0)--(3,0)node[below]{$x$};
    \draw[->](0,-2)--(0,2)node[left]{$y$};
\draw[very thick](0,1)node[left]{1}--(2.5,1)node[above]{$y=\frac{|x|}{x}$};
\draw[very thick](0,-1)node[right]{$-1$}--(-2.5,-1);
\node[below left]{$O$};
\foreach \x in {-1,1}
{
    \draw[fill=white](0,\x) circle(1.5pt);
}

\end{tikzpicture}  
\captionof{figure}{}
\end{minipage}\hfill
\begin{minipage}{.45\textwidth}
\centering
\begin{tikzpicture}[>=stealth, scale=.8]
    \draw[->](-2.5,0)--(2.5,0)node[below]{$x$};
    \draw[->](0,-1)--(0,4)node[left]{$y$};
\draw[domain=0:pi, samples=100, very thick, smooth]plot({cos(\x r)}, \x);
\foreach \x in {-1,1}
{
    \node at (\x,0)[below]{$\x$};
}
\draw[dashed](-1,0)--(-1,pi)--(0,pi)node[right]{$\pi$};
\node[below left]{$O$};
\node at (1.5,1.5){$y=\arccos x$};
\end{tikzpicture}  
\captionof{figure}{}
\end{minipage}

如果函数$y=f(x)$在点$x=x_0$处及其左侧附近有定义,并且$\Lim{x}{x^-_0}f(x)=f(x_0)$,那么称函数$f(x)$在点$x=x_0$处左连
续;如果函数$y=f(x)$在点$x=x_0$处及其右侧附近有定义,并且$\Lim{x}{x^+_0}f(x)=f(x_0)$,那么称函数$f(x)$在点$x=x_0$处右连续.
例如:函数$y=\arccos x$在点$x=1$处左连续;在点$x=-1$处右连续(如图11.11).

我们引入函数$f(x)$在一点连续和不连续的概念,反映了一个函数在一种条件下可能平稳地变化,而在另一种条件下就可能不是这样. $f(x)$在点$x=x_0$处连续的特点是当自变量$x$的值变化很小时,函数$f(x)$相应的值也变化很小;而$f(x)$在间断点$x=x_0$处的特点是自变量$x$的值变化很小时,函数$f(x)$相应的值发生很大的变化.

函数$y=f(x)$的间断点,通常是分段函数定义域区间的分界点,或者是使$f(x)$的解析式中分母的值为零的点等.

\begin{example}
已知函数$f(x)=\begin{cases}
    \sqrt{x-1},& x\geqslant1\\
    \sqrt{1-x^{2}},& 0\leqslant x<1\\
    -\sqrt{1-x^{2}},& -1\leqslant x<0
\end{cases}$
研究函数在下列各点处的连续性. 
\begin{multicols}{3}
\begin{enumerate}[(1)]
    \item $x=1$;
    \item $x=0$;
    \item $x=-1$.
\end{enumerate}
\end{multicols}
\end{example}

\begin{solution}
    函数 $y=f(x)$的图象(如图11.12)

\noindent
\begin{minipage}{.58\textwidth}
\begin{enumerate}[(1)]
    \item $\because\quad $函 数 $f( x)$在 点 $x= 1$ 处 及 附 近 有 定 义 , 并 且 $\Lim{x}{1}f( x) = f(1)=0$

$\therefore\quad $函数 $f(x)$在点 $x=1$ 处连续.
\item $\because\quad \Lim{x}{0^+}f(x)=1$, $\Lim{x}{0^-} f(x)=-1$

$\therefore\quad $函数$f(x)$在点$x=0$处不连续. 即点$x=0$是$f(x)$的间断点.
\item $\because\quad $函数$f(x)$在点$x=-1$处及右侧附近有定义,且$\Lim{x}{-1^+} f(x)=f(-1)=0$.

$\therefore\quad $函数在点$x=-1$处右连续.
\end{enumerate}    
\end{minipage}\hfill
\begin{minipage}{.4\textwidth}
\centering
\begin{tikzpicture}[>=stealth]
    \draw[->](-1.5,0)--(3.5,0)node[below]{$x$};
    \draw[->](0,-2)--(0,2.5)node[left]{$y$};
\draw[very thick](-1,0)node[above]{$-1$} arc (-180:-90:1)node[right]{$-1$};
\node[below left]{$O$};
\draw[very thick](0,1)node[left]{$1$} arc (90:0:1)node[below]{$1$};
\draw[domain=1:3.25, smooth, very thick]plot(\x, {sqrt(\x-1)});
\draw[fill=white](0,-1) circle(1.5pt);
\draw[fill](0,1) circle(1.5pt);
\end{tikzpicture}
\captionof{figure}{}
\end{minipage}    
\end{solution}

下面利用函数在某一点处连续的概念来定义连续函数.

如果函数$f(x)$在区间$(a,b)$内每一点处都连续,则称函数$f(x)$在区间$(a,b)$内连续,或者称函数$f(x)$是区间$(a,b)$内的连续函数.

如果函数$f(x)$在开区间$(a,b)$内连续,且在左端点$x=a$处右连续,在右端点$x=b$处左连续,则称函数$f(x)$在闭区间$[a,b]$上连续.

\begin{example}
    指出下列函数是否为给定区间上的连续函数?
\begin{enumerate}[(1)]
    \item $f(x)=2x^2-3x+5,\qquad x\in (-\infty,+\infty)$
    \item $f(x)=\log_2 x,\qquad x\in (0,+\infty)$
    \item $f(x)=\sqrt{2-x^2},\qquad x\in \left[-\sqrt{2},\sqrt{2}\right]$
\end{enumerate}
\end{example}

\begin{solution}
    (1)、(2)、(3)中的函数都是给定区间上的连续函数.
\end{solution}

\begin{ex}
\begin{enumerate}
    \item 已知函数$y=f(x)$的图象如下图所示,请将符合下述性质的函数图象的代号A、B、C、D分别填在指定位置。
\begin{enumerate}[(1)]
    \item $f(x)$在点$x=a$处有定义,但极限不存在的是\blank ;
    \item $f(x)$在点$x=a$处没有定义,且极限也不存在的是\blank ;
       \item $f(x)$在点$x=a$处有定义、有极限,但是$\Lim{x}{a}f(x)\ne f(a)$的是\blank ;
    \item $f(x)$在点$x=a$处没有定义,但极限存在的是\blank . 
\end{enumerate}
\begin{center}
\begin{tikzpicture}[>=stealth]
\begin{scope}
\draw[->](-1,0)--(3.5,0)node[below]{$x$};
\draw[->](0,-1)--(0,3)node[left]{$y$};
\draw[domain=0:2, smooth,very thick, samples=100]plot(\x+1, {sqrt(\x)});
\draw[domain=0:2, smooth,very thick, samples=100]plot(-\x+1, {sqrt(\x)});
\node [below left]{$O$};
\node at (1,0)[below]{$a$};
\draw[dashed](1,0)--(1,2)node[right]{$A(a,b)$}--(0,2)node[left]{$b$};
\draw[fill](1,2)circle (1.5pt);
\draw[fill=white](1,0)circle (1.5pt);
\node at (1,-1){(A)};
\end{scope}
\begin{scope}[xshift=5cm]
    \draw[->](-1,0)--(3.5,0)node[below]{$x$};
\draw[->](0,-1)--(0,3)node[left]{$y$};
\draw[domain=0:2, smooth,very thick, samples=100]plot(\x+1, {sqrt(\x)});
\draw[domain=0:2, smooth,very thick, samples=100]plot(-\x+1, {sqrt(\x)});
\node [below left]{$O$};
\node at (1,0)[below]{$a$};
\draw[fill=white](1,0)circle (1.5pt);
\node at (1,-1){(B)};

\end{scope}
\begin{scope}[yshift=-4.5cm]
    \draw[->](-1,0)--(3.5,0)node[below]{$x$};
    \draw[->](0,-1.5)--(0,2.5)node[left]{$y$};
\draw(1,-.8)node[below]{$x=a$}--(1,2.5);
\node at (1,-1.5){(C)};
\node [below left]{$O$};
\node at (1,0)[below left]{$a$};
\draw[domain=0.2:2.2, smooth,very thick, samples=100]plot(\x+1, {.5/\x});
\draw[domain=0.2:2.2, smooth,very thick, samples=100]plot(-\x+1, {.5/\x});


\end{scope}
\begin{scope}[yshift=-5cm, xshift=5cm]
    \draw[->](-1,0)--(3.5,0)node[below]{$x$};
\draw[->](0,-1)--(0,3)node[left]{$y$};
\draw[domain=0:2, smooth,very thick, samples=100]plot(\x+1, {sqrt(\x)});
\draw[domain=0:2, smooth,very thick, samples=100]plot(-\x+1, -{sqrt(\x)}+2);
\node [below left]{$O$};
\node at (1,0)[below]{$a$};
\draw[dashed](1,0)--(1,2)node[right]{$A(a,b)$}--(0,2)node[left]{$b$};
\draw[fill](1,2)circle (1.5pt);
\draw[fill=white](1,0)circle (1.5pt);
\node at (1,-1){(D)};
\end{scope}
\end{tikzpicture}
\captionof*{figure}{第1题}
\end{center}

\item 指出下列函数在给定点处是否连续.
\begin{enumerate}[(1)]
    \item $f(x)=\sqrt{x+1}$,点$x=-1$;
    \item $f(x)=x^2-3x+2$,点$x=2$;
    \item $f(x)=\log_{\tfrac{1}{2}}x$,点$x=1$;
    \item $f(x)=\frac{x-1}{\sqrt{x}-1}$,点$x=1$;
    \item $f(x)=ax^2+bx+c\; (a\ne 0)$,点$x=-\frac{b}{2a}$.
\end{enumerate}
\item 指出下列函数的间断点.
\begin{multicols}{2}
\begin{enumerate}[(1)]
    \item $f(x)=\frac{1}{x^2-3x+2}$
    \item $f(x)=\frac{2x+1}{x-3}$
    \item $f(x)=\frac{x^2-4}{x-2}$
    \item $f(x)=\begin{cases}
        2^x, & x\ge 0\\ 0,& x<0
    \end{cases}$
\end{enumerate}
\end{multicols}
\end{enumerate}
\end{ex}


\section{连续函数的性质}
在客观世界中存在许多连续变化的数量,连续函数反映了这类事物中变量之间的关系.

连续函数有以下性质:

\begin{thm}{性质1}
    若函数$y=f(x)$在区间$(a,b)$内连续,$x_0\in (a,b)$,则$\Lim{x}{x_0}f(x)=f(x_0)$
\end{thm}

利用函数$f(x)$在点$x=x_0$处连续的定义,求$\Lim{x}{x_0}f(x)$的问题可转化为求$f(x_0)$的值.

\begin{thm}{性质2}
    如果函数$f(x)$、$g(x)$在某一点$x=x_0$处连续,那么函数$f(x)\pm g(x)$, $f(x)\cdot g(x)$以及$\frac{f(x)}{g(x)}\; (g(x)\ne 0)$在点$x=x_0$处都连续(此性质利用函数极限的四则运算法则不难证明).
\end{thm}

这个性质告诉我们,连续函数的和、差、积、商仍是连续函数.

\begin{thm}{性质3}
    已知函数$y=f[g(x)]$是由函数$y=f(u)$、$u=g(x)$复合而成的函数,若函数$u=g(x)$在点$x=x_0$处连续、$u_0=g(x_0)$,且函数$y=f(u)$在点$u=u_0$处也连续,则函数$y=f[g(x)]$在点$x=x_0$处也连续. 即
\[\Lim{x}{x_0} f[g(x)]=f\left[\Lim{x}{x_0} g(x)\right]=f[g(x_0)]  \]
\end{thm}

这个性质给出了求复合函数极限的一种方法.

\begin{example}
    求下列各函数的极限.
\begin{multicols}{2}
\begin{enumerate}[(1)]
    \item $\Lim{x}{3}(2x^2+5x-1)$
    \item $\Lim{x}{2}\sqrt{x^2-2}$
\end{enumerate}
\end{multicols}
\end{example}

\begin{solution}
\begin{enumerate}[(1)]
    \item $\Lim{x}{3}(2x^2+5x-1)=2\x 3^2+5\x 3-1=32$
    \item $\Lim{x}{2}\sqrt{x^2-2}=\sqrt{\Lim{x}{2}(x^2-2)}=\sqrt{2^2-2}=\sqrt{2}$
\end{enumerate}
\end{solution}    

\begin{thm}{性质4}
    若函数$y=f(x)$在区间$(a,b)$内连续,则函数$y=f(x)$在区间$(a,b)$内的图象是一条连续曲线.
\end{thm}

\begin{example}
    作下列各函数的图象
\begin{enumerate}[(1)]
    \item $y= -(|x|-1)^2+2 ,\qquad x\in \left[-\frac{3}{2},\frac{3}{2}\right]$
    \item $y= 2^{x-1}-1 ,\qquad x\in (-1,2)$
    \item $y=x^2-4x+3  ,\qquad x\in \left[0,4\right]$
\end{enumerate}
\end{example}

\begin{solution}
所作函数的图象分别为图11.13中的(1)、(2)、(3)图.
\begin{figure}[htp]
    \centering
\begin{tikzpicture}[>=stealth, scale=.9]
\begin{scope}
    \draw[->](-2,0)--(2,0)node[below]{$x$};
    \draw[->](0,-1)node[below]{(1)}--(0,3)node[left]{$y$};
\node [below left]{$O$};
\draw[domain=0:1.5, smooth, very thick]plot(\x, -\x*\x+2*\x+1);
\draw[domain=0:1.5, smooth, very thick]plot(-\x, -\x*\x+2*\x+1);
\draw[dashed](1,0)node[below]{1}--(1,2)node[above]{$A(1,2)$};
\draw[dashed](-1,0)node[below]{$-1$}--(-1,2)node[above]{$B(-1,2)$};
\draw[dashed](1.5,0)node[below]{$\tfrac{3}{2}$}--(1.5,1.75);
\draw[dashed](-1.5,0)node[below]{$-\tfrac{3}{2}$}--(-1.5,1.75);
\node at (0,1)[left]{1};
\foreach \x in {1,2}
{
    \draw(0,\x)--(.1,\x);
}

\end{scope}
\begin{scope}[xshift=4cm]
    \draw[->](-1.5,0)--(2.5,0)node[below]{$x$};
    \draw[->](0,-1)--(0,2)node[left]{$y$};
\node [below left]{$O$};
\foreach \x in {1,2}
{
    \node at (\x,0)[below]{\x};
}
\draw[dashed](2,0)node[below]{2}--(2,1)node[above]{$A(2,1)$}--(0,1)node[left]{1};
\draw[dashed](-1,0)node[above]{$-1$}--(-1,-.75)node[below]{$B(-1,-\tfrac{3}{4})$};
\draw[domain=-1:2, smooth, very thick]plot(\x, {2^(\x-1)-1});
\draw[fill=white](2,1)circle (1.5pt);
\draw[fill=white](-1,-.75)circle (1.5pt);
\node at (.5,-1)[below]{(2)};

\end{scope}
\begin{scope}[xshift=8cm, scale=.6]
    \draw[->](-1.5,0)--(5,0)node[below]{$x$};
    \draw[->](0,-2)--(0,4.5)node[left]{$y$};
\node [below left]{$O$};
\draw[domain=0:4, smooth, very thick]plot(\x, {(\x-2)^2-1})node[above]{$A(4,3)$};
\foreach \x in {1,3,4}
{
    \node at (\x,0)[below]{\x};
}
\foreach \x in {-1,1,2}
{
    \draw(0,\x)--(.1,\x);
}
\node at (2,-2)[below]{(3)};


\draw[dashed](2,0)--(2,-1)node[below]{$O'(2,-1)$};
\draw[dashed](4,0)--(4,3);
\node at (0,3)  [left]{$3$};
\end{scope}
\end{tikzpicture}
    \caption{}
\end{figure}





    
\end{solution}

观察例11.11中3个函数的图象,我们发现,第(1)、(3)个函数在给定的闭区间上都存在着最大值和最小值,而第(2)个函
数没有最大值和最小值.

\begin{thm}{性质5}
    如果$f(x)$是\textbf{闭区间}$[a,b]$上的连续函数,那么在区间$[a,b]$上\textbf{至少存在}两个点$x=x_1$、$x=x_2$,使得$f(x_1)\ge f(x)$且$f(x_2)\le f(x)$.(最大值和最小值存在定理)
\end{thm}

这个性质告诉我们,定义在闭区间上的连续函数一定有最大值和最小值,但并未解决如何求这个函数$f(x)$的最大值点和最小值点的问题.

观察例11.11中3个函数的图象我们还可以发现,当自变量$x$的值在给定区间内变化时,若使对应的函数值$f(x)$由负变正(或由正变负),则在它们中间一定存在一个点$x_0$,使得$f(x_0)=0$.

\begin{thm}{性质6}
    如果函数$f(x)$在闭区间$[a,b]$上连续,且$f(a)\cdot f(b)<0$,则在\textbf{闭区间}$[a,b]$上\textbf{至少存在}一点$C$,使得$f(c)=0$.(中间值定理).
\end{thm}

像上一个性质一样,这个中间值定理也只告诉我们使$f(c)=0$的存在性,但又启发我们可反复利用这个“中间值定理”,用逼近的方法求方程$f(x)=0$根的近似值.







\begin{example}
    利用“中间值定理”解方程$x^3-3=0$(精确到
0.001)
\end{example}

\begin{solution}
设函数 $f(x)=x^3-3$, 它的零点为 $x=\alpha$, 则 $\alpha$ 是方程
$x^3-3=0$的解

$\because\quad f( 1) = - 2< 0,\quad  f( 2) = 5> 0$

$\therefore\quad 1< \alpha < 2$ 

又$\because\quad f( 1. 4) = - 0.256< 0,\quad  f(1.5) = 0.375> 0$ 

$\therefore\quad 1.4< \alpha < 1.5$ 

又$\because\quad f( 1. 44) \approx - 0. 014< 0,\quad  f( 1. 45) \approx 0. 048> 0$

$\therefore\quad 1.44<\alpha<1.45$

又$\because\quad f( 1. 442) \approx - 0. 00156< 0$, 
$f(1.443)\approx0.0045>0$

$\therefore\quad 1. 442< \alpha < 1. 443$

又$\because\quad f( 1. 4422) \approx - 0. 0003< 0$, 
$f(1.4423)\approx0.0003>0$

$\therefore\quad 1. 4422< \alpha < 1. 4423$

$\therefore\quad $ 方程 $x^3-3=0$ 的根 $x\approx1.442$(精确到 0.001)
\end{solution}

\begin{ex}
\begin{enumerate}
    \item 求下列函数的极限:
\begin{multicols}{2}
\begin{enumerate}[(1)]
    \item $\Lim{x}{7}\sqrt[3]{x+1}$
    \item $\Lim{x}{1}\log_2(x^2+3)$
    \item $\Lim{x}{3}2^{x-1}$
    \item $\Lim{x}{0}\frac{(1+x)^2-1}{x}$
\end{enumerate}
\end{multicols}
\item 判断下列命题是否为真命题:
\begin{enumerate}[(1)]
\item 若函数$f(x)$在点$x=x_0$处连续,则当$x\to x_0$时,$f(x)-f(x_0)\to 0$.
\item 若函数$f(x)$在点$x=x_0$处连续,函数$g(x)$虽有定义但在点$x=x_0$处不连续,则$f(x)+g(x)$在点$x=x_0$处不连续.
\item 若函数$f(x)$、$g(x)$在点$x=x_0$处都有定义且都不连续,则$f(x)+g(x)$在点$x=x_0$处不连续.
\item 若函数$f(x)$在区间$(a,b)$内连续,则函数$|f(x)|$在区间$(a,b)$内也连续.
\item 若函数$f(x)$在区间$(a,b)$内是连续函数,则$f(x)$在区间$(a,b)$内有最大值和最小值.
\item 对于函数$f(x)$,若在其定义域内有两点$x=a$和$x=b$,使得$f(a)\cdot f(b)<0$,则在区间$[a,b]$上必存在一点$C$,使得$f(C)=0$.
\end{enumerate}

\item 求证:方程$7x^3+19x^2+x-2=0$有一根在区间$(0,1)$内.
\end{enumerate}
\end{ex}

\section{初等函数及其极限}

我们学过的基本函数可以分为以下5类:
\begin{enumerate}
\item 幂函数$y=x^{a}$($a$是实数);
\item 指数函数$y=a^x$($a>0$且$a\ne 1$);
\item 对数函数$y=\log_a x$($a>0$且$a\ne 1$);
\item 三角函数$y=\sin x$, $y=\cos x$, $y=\tan x$, $y=\cot x$, $y=\sec x$和$y=\csc x$.
\item 反三角函数$y=\arcsin x$, $y=\arccos x$, $y=\arctan x$, $y={\rm arccot\, }x$等.
\end{enumerate}

上述5种函数统称基本初等函数.

由基本初等函数经过\textbf{有限次}四则运算和\textbf{有限次}复合所得到的函数,统称初等函数.

\begin{example}
    试用3个基本初等函数$\lg x$、$e^x$、$x^2$构造5个初等函数.
\end{example}

\begin{solution}
    例如:
\[\begin{split}
    f_1(x)&=\lg(e^x)^2\\
    f_2(x)&=e^{\lg x}-3x^2\\
    f_3(x)&=(\lg x)^2-3\lg x\\
    f_4(x)&=e^{x^2}(x^2+e^x)\\
    f_5(x)&=3\lg^4 x-5\lg^3x+2\lg^2x-\lg x+\frac{2}{\lg x}
\end{split}\]
\end{solution}

由基本初等函数的连续性和连续函数的性质,可以推出:\textbf{一切初等函数在它们定义区间上都分别是连续函数}.

有了初等函数是连续函数的结论,不仅使我们对函数的性质有了进一步的了解,而且利用连续函数的性质,为我们研究问题带来一些方便. 比如求初等函数的极限,若函数$y=f[g(x)]$是初等函数,点$x=x_0$是它的定义区间内的一点,则
\[\lim_{x\to x_0}f[g(x)]=f\left[\lim_{x\to x_0}g(x)\right]=f[g(x_0)]\]
例如:
\[\lim_{x\to\tfrac{\pi}{2}}(\ln \sin x)=\ln\left(\lim_{x\to\tfrac{\pi}{2}}\sin x\right)=\ln\sin\frac{\pi}{2}=\ln 1=0\]
\[\LIM{x}\frac{\sqrt{2+x}-\sqrt{2}}{x}=\LIM{x}\frac{\sqrt{\frac{2}{x^2}+\frac{1}{x}}-\sqrt{\frac{2}{x^2}}}{1}=0\]
\[\begin{split}
    \Lim{x}{0}\frac{\sqrt{2+x}-\sqrt{2}}{x}&= \Lim{x}{0}\frac{(2+x)-2}{x\left(\sqrt{2+x}+\sqrt{2}\right)}\\
    &= \Lim{x}{0} \frac{1}{\sqrt{2+x}+\sqrt{2}}=\frac{1}{\sqrt{ \Lim{x}{0}(2+x)}+\sqrt{2}}\\
    &=\frac{1}{\sqrt{2}+\sqrt{2}}=\frac{\sqrt{2}}{4}
\end{split}\]

求函数的极限时,有时还可用下面这个“两面夹”定理.

\begin{thm}{定理}
 如果函数$f(x)$、$g(x)$、$h(x)$在点$x=x_0$附近满足:
\begin{enumerate}[(1)]
\item $g(x)\le f(x)\le h(x)$;
\item $\Lim{x}{x_0}g(x)=\Lim{x}{x_0}h(x)=A$($A$为常数).
\end{enumerate}
那么,$\Lim{x}{x_0}f(x)=A$.   
\end{thm}

这就是说,如果直接求一个函数$f(x)$的极限有困难,那么可以用放缩的方法,使$f(x)$夹在另外两个简单函数之间,只要这两个函数有相同的极限,$f(x)$的极限也就随之得到.


\begin{example}
求$\LIM{x}\frac{\cos x}{x}$.
\end{example}

\begin{solution}
$\because\quad $当$x\to \infty$时, $-\frac{1}{x}\le \frac{\cos x }{x}\le \frac{1}{x}$

而$\LIM{x}\left(-\frac{1}{x}\right)=0$,$\LIM{x}\frac{1}{x}=0$

$\therefore\quad \LIM{x}\frac{\cos x}{x}=0.$
\end{solution}

\section*{习题二}
\begin{center}
    \bfseries A
\end{center}

\begin{enumerate}
    \item (口答)下列初等函数分别是由哪几个基本初等函数复合而成的?
\begin{multicols}{2}
  \begin{enumerate}[(1)]
 \item $y=\sqrt[5]{\ln^2\cos x}$
\item $y=\log_2 \frac{1}{x}$   
\item $y=\left(\frac{1}{3}\right)^{x^2}$
\item $y=\arctan(\sin x^3)$
\end{enumerate}  
\end{multicols}

    \item (口答)下列初等函数是由哪些基本初等函数经过哪些四则运算得出的?
\begin{multicols}{2}
\begin{enumerate}[(1)]
    \item $y=x^4-4x^3+6x^2-4x+1$
    \item $y=\frac{2\ln x+x^2}{x-\sqrt{x}}$
    \item $y=(\arcsin x+\arccos x)^2$
    \item $y=\frac{e^x-e^{-x}}{2}$
\end{enumerate}
\end{multicols}
\item 求下列函数的连续区间:
\begin{multicols}{2}
\begin{enumerate}[(1)]
    \item $y=\frac{x}{1-\sin x}$
    \item $y=\arccos\frac{1}{x}$
    \item $y=\sqrt{2x^2+5x-3}$
    \item $y=\lg(1-e^{x+2})$
\end{enumerate}
\end{multicols}

\item 求下列函数的极限:
\begin{multicols}{2}
\begin{enumerate}[(1)]
    \item $\Lim{x}{\tfrac{\pi}{6}}(1-\sin x)^2$
    \item $\Lim{x}{2}\log_3 (2x^2+1)$
    \item $\Lim{x}{0}\frac{9^x-1}{3^x-1}$
    \item $\Lim{x}{0}\frac{(1+x)^3-1}{x}$
\end{enumerate}
\end{multicols}


\end{enumerate}

\begin{center}
    \bfseries B
\end{center}
\begin{enumerate}\setcounter{enumi}{4}
    \item 求下列各极限:
\begin{multicols}{2}
\begin{enumerate}[(1)]
    \item $\Lim{x}{ 0}\frac{\sin^2 x}{1-\cos x}$
    \item $\Lim{x}{\tfrac{\pi}{2}}\frac{2\cos x}{\sin 2x}$
    \item $\Lim{x}{16 }\frac{\sqrt[4]{x}-2}{\sqrt{x}-4}$
    \item $\Lim{x}{2 }\frac{\sqrt{2x-2}-\sqrt{x}}{x^2-4}$
    \item $\Lim{x}{ x_0}\frac{x^3-x^3_0}{x-x_0}$
    \item $\Lim{x}{x_0 }\frac{\sqrt{x}-\sqrt{x_0}}{x-x_0}$
    \item $\Lim{x}{0 }\frac{\sqrt[3]{1+x}-1}{x}$
    \item $\Lim{x}{0 }\frac{(2+x)^{-2}-2^{-2}}{x}$
\end{enumerate}
\end{multicols}

\item \begin{enumerate}[(1)]
    \item 求证$f(x)=x^4-12x^3+46x^2-60x+9$在区间$[1,5]$上有最大值和最小值.
    \item 求证方程$x^3-6x-3=0$在区间$[2,3]$内至少有一实根.
\end{enumerate}

\item \begin{enumerate}[(1)]
 \item 求证:当$x\to 0$时,$\frac{3^x}{5}<\frac{2^x+3^x}{5}<2\left(\frac{3^x}{5}\right)$
    \item 计算$\Lim{x}{0}\left(\frac{2^x+3^x}{3}\right)^{\tfrac{1}{x}}$   
\end{enumerate}
\end{enumerate}

\section{两个重要极限}
本节介绍在微积分中经常用到的两个重要的极限.

\subsection{$\Lim{x}{0}\frac{\sin x}{x}=1$}


\noindent
\begin{minipage}{.55\textwidth}
    \CTEXindent
虽然当$x\to 0$时,$\sin x\to 0$且$x\to 0$,但不能用消去分子、分母的公因式的方法来求极限. 因此想到能否利用“两面夹”定理来求这个极限.

如图11.14,在单位圆中,以$Ox$为始边的圆心角$\angle XOA=x$(其中$0<|x|<\frac{\pi}{2}$)的终边交单位圆于$A$. 作$AC\bot Ox$轴于$C$,作切线$AD$交$Ox$轴于$D$,则
\[\overline{CA}=\sin x,\quad \wideparen{TA}=x,\quad \overline{DA}=\tan x\]
\end{minipage}\hfill
\begin{minipage}{.45\textwidth}
  \centering
\begin{tikzpicture}[>=stealth, scale=1.5]
\draw[->](-1.5,0)--(1.75,0)node[below]{$x$};
\draw[->](0,-1.5)--(0,1.5)node[left]{$y$};
\draw(0,0)node[below left]{$O$} circle (1);
\tkzDefPoint(35:1){A}
\tkzDefPoints{1.22/0/D, 0.82/0/C, 1/0/T, 0/0/O}
\tkzDefPoint(-35:1){A'}
\draw[dashed, thick](0,0)--(A')--(D);
\draw[thick](0,0)--(A)--(A');
\draw[thick](A)--(D);
\tkzLabelPoints[below](A',D)
\tkzLabelPoints[above](A)
\tkzLabelPoints[above right](T)
\tkzLabelPoints[below left](C)
\tkzMarkRightAngle[size=.1](A,C,O)
\end{tikzpicture}  
\captionof{figure}{}
\end{minipage}

延长$AC$交单位圆于$A'$点,连结$OA'$, $A'D$

$\because\quad |AA'|<\wideparen{ATA'}<|AD|+|A'D|$

$\therefore\quad 2\sin x<2x<2\tan x\quad (x>0)$
或$2\sin x>2x>2\tan x\quad (x<0)$.

$\because\quad 0<|x|<\frac{\pi}{2}$

$\therefore\quad 2\sin x\ne 0$,
不等式变为
\[
    1<\frac{x}{\sin x}<\frac{1}{\cos x}\qquad  1>\frac{\sin x}{x}>\cos x
\]

$\because\quad \Lim{x}{0}1=1,\quad \Lim{x}{0}\cos x=1$

$\therefore\quad \Lim{x}{0}\frac{\sin x}{x}=1$

\begin{rmk}
由于在微积分中三角函数的自变量都是实数,因此它对应于以\textbf{弧度}为单位的角或弧.
\end{rmk}

\begin{example}
求下列极限
\begin{multicols}{2}
\begin{enumerate}[(1)]
    \item $\Lim{x}{\infty}\left(x\sin\frac{1}{x}\right)$
    \item $\Lim{x}{0}\frac{\sin 5x}{3x}$
    \item $\Lim{x}{0}\frac{\tan x}{x}$
    \item $\Lim{x}{0}\frac{1-\cos 2x}{x^2}$
\end{enumerate}
\end{multicols}
\end{example}

\begin{solution}
\begin{enumerate}[(1)]
    \item $\Lim{x}{\infty}\left(x\sin\frac{1}{x}\right)=\Lim{\tfrac{1}{x}}{0}\frac{\sin\frac{1}{x}}{\frac{1}{x}}=1$
    \item $\Lim{x}{0}\frac{\sin 5x}{3x}=\Lim{5x}{0}\frac{5\sin 5x}{3(5x)}=\frac{5}{3}\Lim{5x}{0}\frac{\sin 5x}{5x}=\frac{5}{3}$
    \item $\Lim{x}{0}\frac{\tan x}{x}=\Lim{x}{0}\frac{\sin x}{x\cos x}=\Lim{x}{0}\frac{\sin x}{x}\cdot \Lim{x}{0}\frac{1}{\cos x}=1\x 1=1$
    \item $\Lim{x}{0}\frac{1-\cos 2x}{x^2}=\Lim{x}{0}\frac{2\sin^2 x}{x^2}=2\Lim{x}{0}\left(\frac{\sin x}{x}\right)^2=2\left(\Lim{x}{0}\frac{\sin x}{x}\right)^2=2\x 1^2=2$
\end{enumerate}
\end{solution}

\subsection{$\Lim{x}{\infty}\left(1+\frac{1}{x}\right)^x=e$}
我们从下表先观察当$x\to +\infty$时函数$\left(1+\frac{1}{x}\right)^x$的变化趋势.

\begin{center}
\begin{tabular}{ccc}
\hline
    $x$ & $\left(1+\frac{1}{x}\right)^x$ & 近似值\\
\hline
1   &  $\left(1+\frac{1}{1}\right)^1$ & 2\\
10   &  $\left(1+\frac{1}{10}\right)^{10}$ &2.59374\\
100   &  $\left(1+\frac{1}{100}\right)^{100}$ &2.70481\\
1000   &  $\left(1+\frac{1}{1000}\right)^{1000}$ &2.71692\\
10000   &  $\left(1+\frac{1}{10000}\right)^{10000}$ &2.71815\\
100000   &  $\left(1+\frac{1}{100000}\right)^{100000}$ &2.71827\\
……&……& $2.71828\cdots$\\
\hline
\end{tabular}
\end{center}


可以证明,当$x\to +\infty$时,$\left(1+\frac{1}{x}\right)^x$趋近于无理数$2.718281828459045\cdots$,记作$e$.即
\[\Lim{x}{+\infty}\left(1+\frac{1}{x}\right)^x=e\]
无理数$e$和$\pi$一样,都是数学中应用最广泛的两个超越常数,很多自然规律是以$e$为底的指数函数. 当无理数$e$作为对数的底数时,这个对数又称为自然对数,正数$x$的自然对数用$\ln x$表示.

还可以证明,$\Lim{x}{\infty}\left(1+\frac{1}{x}\right)^x=e$

\begin{example}
求下列极限: 
\begin{multicols}{2}
\begin{enumerate}[(1)]
    \item $\Lim{x}{0}(1+x)^{\tfrac{1}{x}}$
    \item $\Lim{x}{\infty}\left(1+\frac{1}{x}\right)^{-x}$
    \item $\Lim{x}{\infty}\left(1-\frac{2}{x}\right)^x$
    \item $\Lim{x}{\infty}\left(1+\frac{1}{x-2}\right)^x$
\end{enumerate}
\end{multicols}
\end{example}

\begin{solution}
\begin{enumerate}[(1)]
    \item $\Lim{x}{0}(1+x)^{\tfrac{1}{x}}=\Lim{\tfrac{1}{x}}{\infty}\left(1+\frac{1}{\frac{1}{x}}\right)^{\tfrac{1}{x}}=e$
    \item $\Lim{x}{\infty}\left(1+\frac{1}{x}\right)^{-x}=\frac{1}{\Lim{\tfrac{1}{x}}{\infty}\left(1+\frac{1}{x}\right)^x}=\frac{1}{e}$
    \item \[\begin{split}
        \Lim{x}{\infty}\left(1-\frac{2}{x}\right)^x&=\Lim{x}{\infty}\left(1+\frac{1}{-\frac{x}{2}}\right)^{-\tfrac{x}{2}\cdot (-2)}\\
        &=\left[\Lim{-\tfrac{x}{2}}{\infty}\left(1+\frac{1}{-\frac{x}{2}}\right)^{-\tfrac{x}{2}}\right]^{-2}=e^{-2}
    \end{split}\]
    \item \[\begin{split}
        \Lim{x}{\infty}\left(1+\frac{1}{x-2}\right)^x&=\Lim{x}{\infty}\left(1+\frac{1}{x-2}\right)^{(x-2)+2}\\
        &=\Lim{(x-2)}{\infty}\left(1+\frac{1}{x-2}\right)^{x-2}\cdot \Lim{x}{\infty}\left(1+\frac{1}{x-2}\right)^2\\
        &=e\x 1=e
    \end{split}\]
\end{enumerate}
\end{solution}

\begin{example}
求下列极限:
\begin{multicols}{2}
\begin{enumerate}[(1)]
    \item $\Lim{x}{0}\frac{\sin\left(\frac{\pi}{3}+x\right)-\sin\frac{\pi}{3}}{x} $
    \item $\Lim{x}{0}\frac{\ln(2+x)-\ln 2}{x} $
\end{enumerate}
\end{multicols}
\end{example}

\begin{solution}
\begin{enumerate}[(1)]
    \item \[\begin{split}
\text{原式}= \lim_{x\to 0}\frac{2\cos\left(\frac{\pi}{3}+\frac{\pi}{2}\right)\sin\frac{x}{2}}{x}&= \lim_{x\to 0}\left[\frac{\sin\frac{x}{2}}{\frac{x}{2}}\cdot \cos\left(\frac{\pi}{3}+\frac{x}{2}\right)\right]\\
&= \lim_{\tfrac{x}{2}\to 0}\frac{\sin\frac{x}{2}}{\frac{x}{2}}\cdot  \lim_{\tfrac{x}{2}\to 0}\cos\left(\frac{\pi}{3}+\frac{x}{2}\right)\\
&=1\x \cos\frac{\pi}{3}=\frac{1}{2}      
    \end{split}\]
\item \[\begin{split}
\lim_{x\to 0}\frac{\ln(2+x)-\ln 2}{x}&=\lim_{x\to 0}\left[\frac{1}{x}\ln\left(1+\frac{x}{2}\right)\right]=\lim_{x\to 0}\left[\ln\left(1+\frac{1}{\frac{2}{x}}\right)^{\tfrac{1}{x}}\right]\\
&=\lim_{\tfrac{2}{x}\to \infty}\left[\ln\left(1+\frac{1}{\frac{2}{x}}\right)^{\tfrac{2}{x}}\right]^{\tfrac{1}{2}}=\lim_{\tfrac{2}{x}\to \infty}\left[\frac{1}{2}\ln\left(1+\frac{1}{\frac{2}{x}}\right)^{\tfrac{2}{x}}\right]\\
&=\frac{1}{2}\ln\left[\lim_{\tfrac{2}{x}\to \infty}\left(1+\frac{1}{\frac{2}{x}}\right)^{\tfrac{2}{x}}\right]=\frac{1}{2}\ln e=\frac{1}{2}
\end{split}\]
\end{enumerate}
\end{solution}


\begin{example}
计算$\Lim{x}{0}\frac{e^x-1}{x}$
\end{example}

\begin{solution}
    令$t=e^x\; (t>0)$,则$x=\ln t$
\[\begin{split}
    \Lim{x}{0}\frac{e^x-1}{x}&=\lim_{t\to 1}\frac{t-1}{\ln t}=\lim_{t\to 1}\frac{1}{\frac{1}{t-1}\ln\left(1+\frac{1}{\frac{1}{t-1}}\right)}=\lim_{t\to 1}\frac{1}{\ln\left(1+\frac{1}{\frac{1}{t-1}}\right)^{\tfrac{1}{t-1}}}\\
    &=\frac{1}{\Lim{\tfrac{1}{t-1}}{\infty} \left[\ln \left(1+\frac{1}{\frac{1}{t-1}}\right)^{\tfrac{1}{t-1}} \right] }=\frac{1}{\ln \left[\Lim{\tfrac{1}{t-1}}{\infty} \left(1+\frac{1}{\frac{1}{t-1}}\right)^{\tfrac{1}{t-1}} \right] }\\
&=\frac{1}{\ln e}=1
\end{split}\]
\end{solution}

\begin{ex}
\begin{enumerate}
    \item 求下列各极限:
\begin{multicols}{2}
\begin{enumerate}[(1)]
    \item $\Lim{x}{0} \frac{\sin 3x}{\sin 4x}$
    \item $\Lim{x}{0} \frac{\tan 2x}{3x}$
    \item $\Lim{x}{0} \frac{1-\cos 2x}{x\sin x}$
    \item $\Lim{x}{0} \frac{\sin5x-\sin3x}{\sin x}$
    \item $\Lim{x}{\infty} x\sin\frac{5}{x}$
    \item $\Lim{x}{\tfrac{\pi}{2}} \frac{\cos x}{\frac{\pi}{2}-x}$
\end{enumerate}
\end{multicols}

\item 求下列各极限:
\begin{multicols}{2}
    \begin{enumerate}[(1)]
        \item $\Lim{x}{\infty}\left(1+\frac{m}{x}\right)^{mx} $
        \item $\Lim{x}{0}\left(1+\tan x\right)^{\cot x} $
        \item $\Lim{x}{\infty} \left(1+\frac{1}{x}\right)^{x+2}$
        \item $\Lim{x}{0} \left(1-\frac{1}{x}\right)^{2x}$
    \end{enumerate}
\end{multicols}

\item 求下列各极限:
\begin{multicols}{2}
    \begin{enumerate}[(1)]
        \item $\Lim{x}{0} \frac{\cos\left(\frac{\pi}{3}+x\right)-\cos\frac{\pi}{3}}{x}$
        \item $\Lim{x}{0} \frac{\sin(\alpha+x)-\sin\alpha}{x}$
        \item $\Lim{x}{0} \frac{\ln(3+x)-\ln 3}{x}$
    \end{enumerate}
\end{multicols}
\end{enumerate}    
\end{ex}

\section{本章小结}

\subsection{主要内容}
本章的主要内容是函数极限的概念及其运算法则;函数连续的概念及连续函数的性质;初等函数的连续性以及二个重要的极限. 即
\[\lim_{x\to 0}\frac{\sin x}{x}=1,\qquad \lim_{x\to \infty}\left(1+\frac{1}{x}\right)^x=e\]


\subsection{函数的极限}
极限是描述函数在无限过程中的变化趋势的重要概念. 对于一般函数$f(x)$,涉及的极限过程有两类,即$x\to +\infty$、$x\to-\infty$、$x\to \infty$和$x\to x_0$、$x\to x_0^+$、$x\to x_0^-$. 在上述的任一过程中,若$f(x)$的极限存在,即$f(x)\to A$($A$为常数),则$f(x)=A+a(x)$(其中$a(x)\to 0$). 反之,在上述的任一过程中,若$f(x)
=A+a(x)$(其中$a(x)\to 0$,$A$为常数),则$f(x)\to A$,即$f(x)$的极限存在.

函数$f(x)$在点$x=x_0$处连续是利用极限$\Lim{x}{x_0}f(x)$来定义的,但是二者有区别:

$f(x)$在点$x=x_0$处的极限可用来研究$f(x)$在点$x=x_0$处附近的变化趋势,与$f(x)$在点$x=x_0$¡处是否有定义或取什么值无关.

$f(x)$在点$x=x_0$处连续不仅要求$\Lim{x}{x_0}f(x)$存在,而且要求$\Lim{x}{x_0}f(x)=f(x_0)$. 换句话说,$f(x)$在点$x=x_0$处连续就是$x\to x_0$时,$f(x)-f(x_0)\to 0$.

\subsection{求函数的极限}
求函数的极限是本章的重点内容,除了根据极限的定义求函数的极限外,目前我们已掌握的方法主要有:

\begin{enumerate}
    \item  利用函数的连续性求极限.

    若函数$f(x)$在点$x=x_0$处连续,则$\Lim{x}{x_0}f(x)=f(x_0)$.

如果复合函数$y=f[g(x)]$在点$x=x_0$处连续,则$\Lim{x}{x_0}f[g(x)]=f\left[\Lim{x}{x_0}g(x)\right]=f[g(x_0)]$.
\item 利用“两面夹定理”求极限.

若$x\to x_0$时,$g(x)\le f(x)\le h(x)$,且$\Lim{x}{x_0}g(x)=\Lim{x}{x_0}h(x)=A$($A$为常数),则
$\Lim{x}{x_0}f(x)=A$.

\begin{rmk}
    若极限过程改为$x\to \infty$等,两面夹定理仍成立.
\end{rmk}

\item 利用“两个重要极限”求极限.
\item 利用极限的四则运算法则求极限.

函数的和、差、积、商的极限利用极限的运算法则可转化为函数极限的和、差、积、商.而转化的前提是各函数的极限存在且使变形后的解折式有意义.对于那些不能直接利用极限运算法则的式子(又称为未定式),需注意掌握常用的变形方法,使得变形后的式子能利用法则、定理求极限.
\item 利用换元法求极限

换元法是数学解题中常用的方法,通过变量代换可使问题的本质看得更清楚.
\end{enumerate}

\subsection{连续函数}
一切初等函数在它们的定义区间上是连续函数.

函数的连续性是函数的又一个重要性质,应重视对连续函数性质的研究.


\section*{复习题十一}
\begin{center}
    \bfseries A
\end{center}

\begin{enumerate}
    \item (口答)指出下列各命题是真命题,还是假命题.
\begin{enumerate}[(1)]
\item 如果函数$f(x)$在点$x=x_0$处连续,那么$\Lim{x}{x_0}f(x)$存在;
\item 如果$\Lim{x}{x_0}f(x)$存在,那么函数$f(x)$在点$x=x_0$处连续;
\item 若函数$f(x)$在点$x=x_0$处连续,则当$x\to x_0$时,$f(x)-f(a)\to 0$;
\item 若$\Lim{x}{x_0}f(x)=A$,则当$x\to x_0$时,$f(x)-A\to 0$;
\item 若函数$f(x)$的定义域是实数集$\R$,则$f(x)$是连续函数;
\item 函数$f(x)=|x-1|$在点$x=1$处连续,但$\Lim{x}{1}\frac{|x-1|}{x-1}$不存在;
\item 函数$f(x)=(x+1)^2$在点$x=1$处连续,且$\Lim{x}{1}\frac{(x+1)^2-2^2}{x-1}$存在;
\item 存在这样的函数$f(x)$,它在定义域内的每一点处都不连续.
\end{enumerate}

\item 求下列函数的极限:
\begin{multicols}{2}
\begin{enumerate}[(1)]
    \item $\Lim{x}{1}\sqrt{10-x} $
    \item $\Lim{x}{3}\frac{x}{\sqrt{1+x}} $
    \item $\Lim{x}{3}\frac{x^2-9}{x^2-3x} $
    \item $\Lim{x}{\infty}\frac{x^2-9}{x^2-3x} $
    \item $\Lim{x}{\infty}\frac{-1+\sqrt{1+x}}{x} $
    \item $\Lim{x}{0}\frac{-1+\sqrt{1+x}}{x} $
    \item $\Lim{x}{\infty}\frac{3x^2+x-2}{2x^2+5x+3} $
    \item $\Lim{x}{-1}\frac{3x^2+x-2}{2x^2+5x+3} $
    \item $\Lim{x}{3}\left(\frac{12}{x^2-9}-\frac{2}{x-3}\right) $
    \item $\Lim{x}{4}\frac{\sqrt{x-2}-\sqrt{2}}{\sqrt{2x+1}-3} $
    \item $\Lim{x}{\tfrac{3\pi}{4}} \frac{\cos 2x}{\cos x+\sin x}$
    \item $\Lim{x}{0} \frac{\sin 5x-\sin 3x}{\sin x}$
    \item $\Lim{x}{\infty}x\sin\frac{2}{x} $
    \item $\Lim{x}{0} \frac{\sqrt{x+4}-2}{\sin 5x}$
    \item $\Lim{x}{\infty}\left(\frac{x+1}{x-1}\right)^x $
\end{enumerate}
\end{multicols}
\end{enumerate}

\begin{center}
    \bfseries B
\end{center}

\begin{enumerate}\setcounter{enumi}{2}

\item 求下列函数的极限:
\begin{multicols}{2}
\begin{enumerate}[(1)]
\item $\Lim{x}{0} \frac{\sin\left(\sqrt{1+x}-1\right)}{x}$
\item $\Lim{x}{0} \frac{\tan x-\sin x}{\sin^3 x}$
\item $\Lim{x}{1} \frac{\sin \pi x}{x-1}$
\item $\Lim{x}{1} \frac{\ln x}{x-1}$
\end{enumerate}
\end{multicols}


\item 求下列函数的极限:
\begin{multicols}{2}
\begin{enumerate}[(1)]
    \item $\Lim{x}{0} \frac{(a+x)^4-a^4}{x}$
    \item $\Lim{x}{0} \frac{\sqrt{a+x}-\sqrt{a}}{x}$
    \item $\Lim{x}{0} \frac{\sin 2(\alpha+x)-\sin 2\alpha}{x}$
    \item $\Lim{x}{0} \frac{\ln(a+x)-\ln a}{x}$
\end{enumerate}
\end{multicols}

\item 求证:方程$x\cdot 3^x-2=0$在区间$(0,1)$内存在一实数根.
\item 已知$\Lim{x}{2} \frac{x^2+ax+b}{x-2}=3$. 求$a$、$b$的值.
\item 已知$f(x)=\begin{cases}
    x^2+1, & x<0\\
    x-1,& 0\le x\le 2\\
    (x-1)^2, & x>2
\end{cases}$

求此函数的连续区间,并作出其图象.








\end{enumerate}

\chapter{导数和微分}
本章将运用极限的方法给出导数和微分的概念,并对有关的基础知识,基本方法作简要介绍.

\section*{一、导数}

\section{平均变化率}

大家知道,若一辆汽车从上午8:30至10:00行驶了60千米,则在这段时间内汽车的平均速度为
\[\frac{\Delta S}{\Delta t}=\frac{60\text{千米}}{1.5\text{时}}=40\text{千米}/\text{时}\]

虽然汽车在行驶过程中的速度是不断变化的,但是这个平均速度表示了汽车在这段时间内,路程关于时间的平均变化率.

\noindent
\begin{minipage}{.48\textwidth}
    \CTEXindent
诸如像人口的出生率、导线的伸长系数、物体的膨胀系数等,都涉及了一个变量关于另一个变量的平均变化率.

对于函数$y=f(x)$,当自变量从$a$变化到$b$时,因变量的改变量$\Delta y=f(b)-f(a)$与自变量的改变量$\Delta x=b-a$的比,即
\[\frac{\Delta y}{\Delta x}=\frac{f(b)-f(a)}{b-a}\]
叫做$f(x)$从$a$到$b$之间的平均变化率.
\end{minipage}
\hfill
\begin{minipage}{.48\textwidth}
    \centering
\begin{tikzpicture}[>=stealth]
\draw[->](-.75,0)--(5,0)node[right]{$x$};
\draw[->](0,-1)--(0,3.5)node[left]{$y$};
\draw[domain=-.5:4.75, very thick, samples=100, smooth]plot(\x, {.15*(1*\x+.5)*(1*\x-2)*(1*\x-4)+1.5})node[left]{$y=f(x)$};
\tkzDefPoints{2.7/1.06/P, 4.25/1.9/Q}
\tkzDrawLines[add= .75  and .5](P,Q)
\tkzLabelPoints[above](P,Q)
\draw(4.25,1.06)--node[below]{$\Delta x$}(P)--(2.7,0)node[below]{$x_0$};
\draw(Q)--(4.25,0)node[below]{$x_0+\Delta x$};
\node [below left]{$O$};
\tkzDefPoints{2.7/0/A, 4.25/0/B, 4.25/1.06/C,0/0/O}
\tkzMarkRightAngles[size=.1](Q,C,P Q,B,A P,A,O)
\node at (4.25, 1.5)[right]{$\Delta y$};
\end{tikzpicture}
\captionof{figure}{}
\end{minipage}

如图12.1,设函数$y=f(x)$的图像是曲线$C$,对于曲线$C$上一点$P(x_0,y_0)$及点$P$邻近的任意一点$Q(x_0+\Delta x,y_0+\Delta y)$,不难得出割线的斜率就是函数$f(x)$从$x_0$到$x_0+\Delta x$之间的平均变化率,即    
\[K_{PQ}=\frac{\Delta y}{\Delta x}=\frac{f(x_0+\Delta x)-f(x_0)}{\Delta x}\]


\begin{example}
设一个质点在做非匀速运动中,位移$S$与时间$t$的函数关系是$S=4.9t^2+2t$,求此质点由时刻$t_0$到$t_0+\Delta t$的这段时间内运动的平均速度.
\end{example}

\begin{solution}
\[\begin{split}
    \Delta S&=[4.9(t_0+\Delta t)^2+2(t_0+\Delta t)]-(4.9t^2_0+2t_0)\\
&=9.8t_0\cdot \Delta t+2\Delta t+4.9(\Delta t)^2\\
\frac{\Delta S}{\Delta t}&=9.8t_0+2+4.9\Delta t
\end{split}\]
答:此质点由时刻$t_0$到$t_0+\Delta t$这段时间内运动的平均速度为$9.8t_0+2+4.9\Delta t$.
\end{solution}

\noindent
\begin{minipage}{.52\textwidth}
    \begin{example}
        如图12.2,设曲线$y=f(x)$与横轴$Ox$及直线$x=m$围成的面积为$S=g(m)=\frac{1}{2}m^3$, 
    \begin{enumerate}[(1)]
        \item 求函数$g(m)$从$m$到$m+\Delta m$之间的平均变化率;
        \item 当$|\Delta m|$的值愈来愈小,即$\Delta m\to 0$时,$g(m)$的平均变化率的极限是多少?
        \item 求$f(m)$.
    \end{enumerate}
    \end{example}
\end{minipage}\hfill
\begin{minipage}{.45\textwidth}
\centering
\begin{tikzpicture}[>=stealth, scale=2]
\draw[->](-.5,0)--(2,0)node[below]{$x$};    
\draw[->](0,-.5)--(0,2.5)node[left]{$y$};    
\draw[domain=0:1.7, very thick, smooth]plot(\x, {.7*\x^2})node[above]{$y=f(x)$};
\draw(1,0)node[below]{$C$}--(1,.7)node[above left]{$A(m,f(m))$};
\draw(1.5,0)node[below]{$D$}--(1.5,1.575)node[right]{$B$};
\node at (.65,.1){$S$};
\node at (1.25,.5){$\Delta S$};
\tkzDefPoints{0/0/O, 1.5/1.575/B, 1.5/0/D, 1/.7/A, 1/0/C}
\tkzMarkRightAngle[size=.1](A,C,O)
\node at (1.25,0)[above]{$\Delta m$};
\node [below left]{$O$};
\end{tikzpicture}
\captionof{figure}{}
\end{minipage}


\begin{solution}
\begin{enumerate}[(1)]
    \item 函数$g(m)$从$m$到$m+\Delta m$之间的平均变化率为
\[\begin{split}
\frac{\Delta S}{\Delta m}=\frac{g(m+\Delta m)-g(m)}{\Delta m}&=\frac{\frac{1}{2}(m+\Delta m)^3-\frac{1}{2}m^3}{\Delta m}\\
&=\frac{\frac{3}{2}m^2\cdot \Delta m+\frac{3}{2}m\cdot (\Delta m)^2+\frac{1}{2}(\Delta m)^3}{\Delta m}\\
&=\frac{3}{2}m^2+\frac{3}{2}m\cdot \Delta m+\frac{1}{2}(\Delta m)^2
\end{split}\]

\item $\Lim{\Delta m}{0}\frac{\Delta S}{\Delta m}=\Lim{\Delta m}{0}\left[\frac{3}{2}m^2+\frac{3}{2}m\cdot \Delta m+\frac{1}{2}(\Delta m)^2\right]=\frac{3}{2}m^2$

\item  当$|\Delta m|$充分小时,曲边梯形$ACDB$的面积$\Delta S$可用梯形$ACDB$的面积表示它的近似值,即$\Delta S\approx S_{\text{梯形}ACDB}$,因此$\frac{\Delta S}{\Delta m}\approx \text{梯形}ACDB$的中位线的长,

$\therefore\quad \Lim{\Delta m}{0}\frac{\Delta S}{\Delta m}=f(m)$,
即$f(m)=\frac{3}{2}m^2$.
\end{enumerate}
\end{solution}

\begin{ex}
\begin{enumerate}
    \item 已知函数$y=f(x)=x^2-x+2$的图象上有一动点$Q(x,y)$及定点$P(2,4)$.
\begin{enumerate}[(1)]
    \item 根据下表中$x_Q$的值,分别求出$f(x)$从2到$x_Q$之间的平均变化率;
\begin{center}
\begin{tabular}{c|ccccccc}
\hline
$x_Q$ & 4&3&2.5&2.1& 2.01& 2.001& 2.0001\\
\hline
$\Delta x$ \\
$\Delta y$ \\ [1.5ex]
$\frac{\Delta y}{\Delta x}$\\
\hline
\end{tabular}
\end{center}
\item 计算$\frac{f(2+\Delta x)-f(2)}{\Delta x}$
\item 计算$\Lim{\Delta x}{0}\frac{f(2+\Delta x)-f(2)}{\Delta x}$
\end{enumerate}

\item 对于12.1节中例12.1,计算$\Lim{\Delta t}{0}\frac{\Delta S}{\Delta t}$,并指出这个极限值的物理意义.
\item 已知$f(x)=\sqrt{25-x^2}$
\begin{enumerate}[(1)]
\item 计算$\frac{f(3+\Delta x)-f(3)}{\Delta x}$;
\item 计算$\Lim{\Delta x}{0}\frac{f(3+\Delta x)-f(3)}{\Delta x}$;
\item 试给出(2)中极限值的一种几何意义.
\end{enumerate}
\end{enumerate}
\end{ex}

\section{导数}
我们在研究函数$y=f(x)$的变量之间的关系时,不仅要
注意$f(x)$在$x_0$到$x_0+\Delta x$这一范围内的平均变化率,而且还要进一步研究当$\Delta x\to 0$时,$f(x)$在点$x=x_0$处的瞬时变化率. 例如,作非匀速直线运动的物体在某一时刻的瞬时速度;物体受热时,在某时刻温度的变化速度;导线在某时刻的电流强度;物质进行化学反应时,在某时刻的反应速度等.

设函数$y=f(x)$在点$x=x_0$及其附近有定义,$\Delta x$是自变量在点$x_0$的改变量,$\Delta y=f(x_0+\Delta x)-f(x_0)$是函数$f(x)$相应的改变量,若极限
\[\lim_{\Delta x\to 0}\frac{\Delta y}{\Delta x}=\lim_{\Delta x\to 0}\frac{f(x_0+\Delta x)-f(x_0)}{\Delta x}\]
存在,则称函数$f(x)$在点$x_0$处可导,且称此极限值为函数$f(x)$在点$x_0$的导数(即瞬时变化率,简称变化率),记作$f'(x_0)$或$y'\Big|_{x=x_0}$,于是
\[f'(x_0)=\lim_{\Delta x\to 0}\frac{\Delta y}{\Delta x}=\lim_{\Delta x\to 0}\frac{f(x_0+\Delta x)-f(x_0)}{\Delta x}\]

函数$f(x)$在点$x_0$处的导数$f'(x_0)$就是函数的平均变化率当$\Delta x\to 0$时的极限值,若极限值不存在,则称函数$f(x)$在点$x_0$处\textbf{不可导},即$f'(x_0)$不存在. 

若令$x=x_0+\Delta x$,代入上式,得
\[f'(x_0)=\lim_{x\to x_0}\frac{f(x)-f(x_0)}{x- x_0}\]
这是求$f'(x_0)$常用的另一种表达式.


\begin{example}
    已知$f(x)=\sqrt{x}$,求$f'(1)$, $f'(2)$
\end{example}

\begin{solution}
\[\begin{split}
f'(1)&=\Lim{x}{1}\frac{f(x)-f(1)}{x-1}=\Lim{x}{1}\frac{\sqrt{x}-1}{x-1}\\
&=\Lim{x}{1}\frac{\sqrt{x}-1}{(\sqrt{x}-1)(\sqrt{x}+1)}=\Lim{x}{1}\frac{1}{\sqrt{x}+1}=\frac{1}{2}\\
f'(2)&=\Lim{\Delta x}{0}\frac{f(2+\Delta x)-f(2)}{\Delta x}=\Lim{\Delta x}{0}\frac{\sqrt{2+\Delta x}-\sqrt{2}}{\Delta x}\\
&=\Lim{\Delta x}{0}\frac{\Delta x}{\Delta x\left(\sqrt{2+\Delta x}+\sqrt{2}\right)}\\
&=\Lim{\Delta x}{0}\frac{1}{\sqrt{2+\Delta x}+\sqrt{2}}=\frac{1}{2\sqrt{2}}
\end{split}\]
$\therefore\quad f'(1)=\frac{1}{2},\quad f'(2)=\frac{\sqrt{2}}{4}$
\end{solution}

\begin{example}
已知$f(x)=|x|$,求$f'(-1)$,$f'(0)$
\end{example}

\begin{solution}
\[\begin{split}
    f'(-1)&=\Lim{x}{-1}\frac{f(x)-f(-1)}{x-(-1)}= \Lim{x}{-1}\frac{|x|-1}{x+1}\\
    &=\Lim{x}{-1}\frac{-(x+1)}{x+1}=\Lim{x}{-1}(-1)=-1
\end{split} \]
$\therefore\quad f'(-1)=-1$

当$x\to 0$时,$\frac{\Delta y}{\Delta x}=\frac{|\Delta x|}{\Delta x}$

$\because\quad $当$\Delta x>0$时,$\frac{|\Delta x|}{\Delta x}=1$;当$\Delta x<0$时,$\frac{|\Delta x|}{\Delta x}=-1$

$\therefore\quad \Lim{\Delta x}{0}\frac{\Delta y}{\Delta x}=\Lim{\Delta x}{0}\frac{|\Delta x|}{\Delta x}$不存在,即$f(x)=|x|$在$x=0$时没有导数,$f'(0)$不存在.
\end{solution}

\begin{ex}
\begin{enumerate}
    \item 求$f'(2)$
\begin{multicols}{2}
\begin{enumerate}[(1)]
    \item $f(x)=x^3$
    \item $f(x)=\frac{1}{x}$
    \item $f(x)=-2x^2+3x+1$
\end{enumerate}
\end{multicols}
    \item 已知$f(x)=\sin x$,求:$f'\left(\frac{\pi}{6}\right)$、$f'\left(\frac{\pi}{2}\right)$
\end{enumerate}
\end{ex}

\section{导数的几何意义}
在12.1节中我们已经知道,若点$Q(x_0+\Delta x,y_0+\Delta y)$、$P(x_0,y_0)$分别是函数$y=f(x)$的图象上的两点,则割线的斜率为
\[K_{PQ}=\frac{\Delta y}{\Delta x}=\frac{f(x_0+\Delta x)-f(x_0)}{\Delta x}\]

若函数$f(x)$在点$x_0$处可导,则
\[f'(x_0)=\lim_{\Delta x\to 0}\frac{\Delta y}{\Delta x}=\lim_{\Delta x\to 0}\frac{f(x_0+\Delta x)-f(x_0)}{\Delta x}\]

当$\Delta x\to 0$时,如图12.3,点$Q(x_0+\Delta x,y_0+\Delta y)$沿着函数
$y=f(x)$的图象无限地趋近于$P(x_0,y_0)$点,这时割线$PQ$绕着点$P$转动,它的极限位置是函数$y=f(x)$的图象在点$P$处的切线. 因此,
\[K_{PT}=\lim_{Q\to P}K_{PQ}=\lim_{\Delta x\to 0}\frac{\Delta y}{\Delta x}=f'(x_0)\]

\begin{figure}[htp]
    \centering
\begin{tikzpicture}[>=stealth, scale=1.5]
\draw[->](-.75,0)--(5,0)node[right]{$x$};
\draw[->](0,-1)--(0,3.5)node[left]{$y$};
\draw[domain=-.5:4.75, very thick, samples=100, smooth]plot(\x, {.15*(1*\x+.5)*(1*\x-2)*(1*\x-4)+1.5})node[left]{$y=f(x)$};
\tkzDefPoints{3.5/1.05/P, 4.25/1.9/Q}
\tkzDrawLines[add= 2.5  and .75](P,Q)
\tkzLabelPoints[above](P)
\tkzLabelPoints[above left](Q)
\draw(4.25,1.06)--node[below]{$\Delta x$}(P)--(3.5,0)node[below]{$x_0$};
\draw(Q)--(4.25,0)node[below]{$x_0+\Delta x$};
\node [below left]{$O$};
\tkzDefPoints{3.5/0/A, 4.25/0/B, 4.25/1.05/S,0/0/O}
\tkzMarkRightAngles[size=.1](Q,S,P Q,B,A P,A,O)
\tkzLabelPoints[right](S)

\tkzDefPoints{4.25/1.7/Q1, 4.25/1.6/Q2, 4.25/1.5/Q3, 4.25/1.4/Q4}

\foreach \x in {1,2,3}
{
    \tkzDrawLines[add= 3  and .75](P,Q\x)
}

\tkzDrawLines[add= 4.5  and .75, very thick](P,Q4)
\node at (4.75,1.65)[right]{$T$};

\end{tikzpicture}
    \caption{}
\end{figure}

综上所述,函数$y=f(x)$在
点$x_0$处的导数$f'(x_0)$的几何意
义是曲线$y=f(x)$在点$(x_0,f(x_0))$处的切线的斜率.

\begin{example}
求与曲线$y=x^3$相切于点$A(-1,-1)$的切线方程,并作图.
\end{example}

\noindent
\begin{minipage}{.55\textwidth}
    \begin{analyze}
        欲求以点$A(-1,-1)$为切点的切线方程,只需求出切线的斜率,而$K_{\text{切}}=f'(-1)$.
    \end{analyze}
    
    \begin{solution}
\[\begin{split}
    K_{\text{切}}=f'(-1)&=\lim_{x\to -1}\frac{f(x)-f(-1)}{x-(-1)}\\
    &=\lim_{x\to -1}\frac{x^3+1}{x+1}\\
    &=\lim_{x\to -1}(x^2-x+1)=3
\end{split}\]
\[\begin{split}
    y-(-1)&=3[x-(-1)]\\
    y+1&=3x+3
\end{split}\]
$\therefore\quad y=3x+2$为所求的切线方程. 

作图如图12.4.    
\end{solution}
\end{minipage}
\begin{minipage}{.4\textwidth}
\centering
\begin{tikzpicture}[>=stealth, scale=.5]
\draw[->](-4,0)--(5,0)node[below]{$x$};    
\draw[->](0,-3)--(0,10)node[left]{$y$};
\draw[domain=-1.3:2.1, smooth, very thick, samples=100]plot(\x, {\x^3});
\node[below right]{$O$};
\tkzDefPoints{-1/-1/A, 2/8./B}
\tkzDrawLines[](A,B)
\node at (A)[left] {$A(-1,-1)$};
\node at (B)[right] {$B(2,8)$};
\tkzDrawPoints(A,B)
\node at (-.25,1.25)[left]{$y=3x+2$};
\node at (1.5,3.375)[right]{$y=x^3$};
\end{tikzpicture}
\captionof{figure}{}
\end{minipage}
    
\begin{example}
求过曲线$y^2=2px\; (y\ge 0)$上一点$P(x_0,y_0)$的切线方程.
\end{example}

\begin{solution}
$y=\sqrt{2px}\quad (y\ge 0,\; x\ge 0)$
\[\begin{split}
    y' \Big|_{x=x_{0}}=\lim_{x\to x_{0}}\frac{\sqrt{2px}-\sqrt{2px_{0}}}{x-x_{0}}&=\lim_{x\to x_{0}}\frac{\sqrt{2p}(\sqrt{x}-\sqrt{x_{0}})}{(\sqrt{x}-\sqrt{x_{0}})(\sqrt{x}+\sqrt{x_{0}})}\\
    &=\lim_{x\to x_{0}}\frac{\sqrt{2p}}{\sqrt{x}+\sqrt{x_{0}}}=\sqrt{\frac{p}{2x_{0}}}\\
\end{split}\]
以$P(x_0,y_0)$为切点做切线方程为
\[\begin{split}
    y-y_0&=\sqrt{\frac{p}{2x_0}}(x-x_0)\\
    y-\sqrt{2px_0}&=\sqrt{\frac{p}{2x_0}}x-\sqrt{\frac{px_0}{2}}\\
y&=\sqrt{\frac p{2x_{0}}}x+\sqrt{\frac{px_{0}}2}=\frac{p}{\sqrt{2px_{0}}}(x+x_{0})
\end{split}\]
$\therefore\quad y_{0}y= p( x+ x_{0})$为所求切线方程.
\end{solution}

\begin{ex}
\begin{enumerate}
    \item 已知曲线$y=x^2-x$,分别求出在下列点处的切线方程:
\begin{multicols}{2}
\begin{enumerate}[(1)]
    \item $A(1,0)$
    \item $B(0,0)$
\end{enumerate}
\end{multicols}
    \item 求曲线$y=6x^{-1}$在点$A(3,2)$处的切线方程.
    \item 若函数$f(x)$在点$x_0$处及其附近可导,当$|\Delta x|$很小时,$\Delta y\approx$\blank .
\end{enumerate}
\end{ex}

\section{函数可导与连续的关系}
由12.2节中的例12.4可知,函数$f(x)=|x|$在$x=0$时连续,但$f'(0)$不存在.因此.若函数$f(x)$在点$x_0$处连续,函数$f(x)$在该点不一定可导. 反过来,若函数$f(x)$在点$x_0$处可导,则可以证明函数$f(x)$在该点一定连续. 事实上,欲证$f(x)$在点$x_0$处连续,只需证
$\Lim{x}{x_0}f(x)=f(x_0)$.

\[\begin{split}
\because\quad \Lim{x}{x_0}f(x)&=\Lim{x}{x_0}[f(x)-f(x_0)+f(x_0)]=\Lim{x}{x_0}[f(x)-f(x_0)]+f(x_0)\\
&=\Lim{x}{x_0}\left[\frac{f(x)-f(x_0)}{x-x_0}\cdot (x-x_0)\right]+f(x_0)\\
&=\Lim{x}{x_0} \frac{f(x)-f(x_0)}{x-x_0}  \cdot \Lim{x}{x_0}(x-x_0)+f(x_0)\\
&=f'(x_0)\cdot 0+f(x_0) =f(x_0)
\end{split}\]

$\therefore\quad \Lim{x}{x_0}f(x)=f(x_0)$即$f(x)$在点$x_0$处连续.

\begin{thm}{定理}
    若函数$f(x)$在点$x_0$处可导,则函数$f(x)$在点$x_0$处也连续.
\end{thm}

\begin{example}
    已知 函数$f(x)=|x^2-1|$
\begin{enumerate}[(1)]
\item 函数在$x_1=-1,\; x_2=0,\; x_3=-1$是否连续?
\item 函数在$x_1=-1,\; x_2=0,\; x_3=-1$是否可导?
\end{enumerate}
\end{example}

\begin{solution}
\[f(x)=|x^2-1|=\begin{cases}
    x^2-1,& x\le -1\; \text{或}\; x\ge 1\\
    1-x^2,& -1\le x\le 1
\end{cases}\]
\begin{enumerate}[(1)]
    \item $\because\quad \Lim{x}{0}f(x)=\Lim{x}{0}(1-x^2)=1=f(0)$

    $\therefore\quad f(x)$在$x=0$处连续.

$\because\quad \Lim{x}{1^+}f(x)=\Lim{x}{1^+}(x^2-1)=0,\;  \Lim{x}{1^-}f(x)=\Lim{x}{1^-}(1-x^2)=0,\; f(1)=0$

$\therefore\quad f(x)$在$x=1$处连续. 同理可证$f(x)$在$x=-1$处连续.

\item $\because\quad \Lim{x}{0}\frac{f(x)-f(0)}{x}=\Lim{x}{0}\frac{1-x^2-1}{x}=\Lim{x}{0}(-x)=0$

$\therefore\quad f'(0)$存在,且$f'(0)=0$
\[\begin{split}
  \because\quad  \Lim{x}{1^+}\frac{f(x)-f(1)}{x-1}&= \Lim{x}{1^+}\frac{x^2-1}{x-1}= \Lim{x}{1^+}(x+1)=2\\
  \Lim{x}{1^-}\frac{f(x)-f(1)}{x-1}&=\Lim{x}{1^-}\frac{1-x^2}{x-1}=-\Lim{x}{1^-}(x+1)=-2
\end{split}\]
$\therefore\quad f(x)$在$x=1$处不可导. 同理可证,$f(x)$在$x=-1$处也不可导.
\end{enumerate}
\end{solution}

\begin{ex}
\begin{enumerate}
    \item 作函数$y=|x^2-x|$的图象,并说明它在哪个点不可导.
    \item 已知函数$f(x)=\begin{cases}
        x\sin\frac{1}{x},& x\ne 0\\
        0,& x=0
    \end{cases}$

    试说明此函数$f(x)$在点$x=0$处是否可导及连续.
\end{enumerate}
\end{ex}

\section*{习题一}
\begin{center}
    \bfseries A
\end{center}

\begin{enumerate}
    \item 分别求下列函数的平均变化率$\frac{\Delta y}{\Delta x}$:
\begin{enumerate}[(1)]
    \item $y=-2x^3+x^2-1$,当$x=1$, $\Delta x=0.1$时;
    \item $y=\frac{x}{3}$,当$x=2$, $\Delta x=0.01$时;
    \item $y=\sqrt{2x}$,当$x=4$, $\Delta x=0.001$时.
\end{enumerate}
    ((2)、(3)题精确到0.001)
    \item     根据导数定义,求下列函数在给定点的导数:
\begin{multicols}{2}
\begin{enumerate}[(1)]
    \item $y=-2x^3+x^2-1$,求$f'(1)$;
    \item $y=\frac{3}{x}$,求$f'(2)$;
    \item $y=\sqrt{2x}$,求$f'(4)$.
\end{enumerate}
\end{multicols}
    \item    一质点做直线运动,它所经过的路程和时间的关系是$s=3t^2+2t+1$,求$t=2$时的瞬时速度.
    \item    求曲线$y=x^3$在点$A(2,8)$处的切线方程.
\end{enumerate}

\begin{center}
    \bfseries B
\end{center}

\begin{enumerate}\setcounter{enumi}{4}
    \item 已知$f(x)=\begin{cases}
        x^2\sin\frac{1}{x},& x\ne 0\\
        0,& x=0
    \end{cases}$,求$f'(0)$.
\item 函数$y=|\sin x|$在$x=0$处是否存在导数?并说明你的理由.
\item 已知$f(x)=\begin{cases}
    2x^2,& x\le 1\\
-2x^2+8x-4,& x\ge 1
\end{cases}$
,求$f'(1)$
\item 设$f(x)$在点$x=1$处的导数为$f'(1)$,计算
\begin{multicols}{2}
\begin{enumerate}[(1)]
    \item $\Lim{h}{0}\frac{f(1+3h)-f(1)}{h}$
    \item $\Lim{h}{0}\frac{f(1+h)-f(1-h)}{2h}$
\end{enumerate}
\end{multicols}
\end{enumerate}

\section*{二、函数的导数}

\section{函数的导数}

我们已经学了函数$f(x)$在点$x_0$处的导数,例如$f(x)=\frac{1}{x}$在点$x_0$处的导数为:
\[\begin{split}
f'(x_0)&=\Lim{h}{0}\frac{f(x_0+h)-f(x_0)}{h}=\Lim{h}{0}\frac{\frac{1}{x_0+h}-\frac{1}{x_0}}{h}\\
&=\Lim{h}{0}\frac{x_0-(x_0+h)}{hx_0(x_0+h)}=\Lim{h}{0}\frac{-1}{x_0(x_0+h)}=-\frac{1}{x^2_0}
\end{split}\]
即:$f'(x_0)=-\frac{1}{x^2_0}$

如果$x_0\in (0,+\infty)$,不难看出,对于开区间$(0,+\infty)$内每一个确定的值$x_0$,都有唯一确定的值$f'(x_0)$与其相对应,这样,在开区间$(0,+\infty)$内,利用求导的方法又得到了一个新
的函数$f'(x)=-\frac{1}{x^2}$.

一般地,若函数$f(x)$在开区间$(a,b)$内每一点$x$处都可导,即.
\[f'(x)=\Lim{h}{0}\frac{f(x+h)-f(x)}{h}\qquad x\in(a,b)\]
则由此对应法则确定的函数$f'(x)$叫做函数$f(x)$的导函数(简称$f(x)$的导数)。这时又称$f(x)$在开区间$(a,b)$内可导.

下面根据函数的导数定义,求几种最常见的函数的导数.
\begin{enumerate}
    \item $f(x)=c$($c$为常数)的导数.

    $\because\quad f'(x)=\Lim{h}{0}\frac{f(x+h)-f(x)}{h}=\Lim{h}{0}\frac{c-c}{h}=0$

$\therefore\quad    f'(x)=c'=0$($c$为常数).
    \item $f(x)=x^n$($n$为自然数)的导数.
\[\begin{split}
    \because\quad f'(x)&=\Lim{h}{0}\frac{f(x+h)-f(x)}{h}=\Lim{h}{0}\frac{(x+h)^n-x^n}{h}\\
&=\Lim{h}{0}\left({\rm C}^1_n x^{n-1}+{\rm C}^2_nx^{n-2}\cdot h+\cdots+{\rm C}^k_nx^{n-k}\cdot h^{k-1}+\cdots +h^{n-1}\right)\\
&={\rm C}^1_nx^{n-1}=nx^{n-1}
\end{split}\]
$\therefore\quad f'(x)=(x^n)'=nx^{x-1}$($n$为自然数).
    \item $f(x)=\sin x$的导数.
    \[\begin{split}
        \because\quad f'(x)&=\Lim{h}{0}\frac{f(x+h)-f(x)}{h}=\Lim{h}{0}\frac{\sin (x+h)-\sin x}{h}\\
    &=\Lim{h}{0}\frac{2\cos \left(x+\frac{h}{2}\right)\sin \frac{h}{2}}{h}\\
    &=\Lim{h}{0}\cos \left(x+\frac{h}{2}\right)\cdot \Lim{\tfrac{h}{2}}{0}\frac{\sin\frac{h}{2}}{\frac{h}{2}}=\cos x
    \end{split}\]
    $\therefore\quad f'(x)=(\sin x)'=\cos x$.

\item $f(x)=\cos x$的导数.
\[\begin{split}
    \because\quad f'(x)&=\Lim{h}{0}\frac{f(x+h)-f(x)}{h}=\Lim{h}{0}\frac{\cos (x+h)-\cos x}{h}\\
&=\Lim{h}{0}\frac{-2\sin \left(x+\frac{h}{2}\right)\sin \frac{h}{2}}{h}\\
&=-\Lim{h}{0}\sin \left(x+\frac{h}{2}\right)\cdot \Lim{\tfrac{h}{2}}{0}\frac{\sin\frac{h}{2}}{\frac{h}{2}}=-\sin x
\end{split}\]
$\therefore\quad f'(x)=(\cos x)'=-\sin x$.

\item $f(x)=\ln x$的导数.
\[\begin{split}
    \because\quad f'(x)&=\Lim{h}{0}\frac{f(x+h)-f(x)}{h}=\Lim{h}{0}\frac{\ln (x+h)-\ln x}{h}\\
&=\Lim{h}{0}\frac{1}{h}\ln\left(1+\frac{h}{x}\right)=\Lim{h}{0}\left[\frac{1}{x}\cdot \frac{x}{h}\ln\left(1+\frac{1}{\frac{x}{h}}\right)\right]\\
&=\frac{1}{x}\lim_{\tfrac{x}{h}\to \infty}\ln\left(1+\frac{1}{\frac{x}{n}}\right)^{\tfrac{x}{h}}=\frac{1}{x}\ln\left[\lim_{\tfrac{x}{h}\to\infty}\left(1+\frac{1}{\frac{x}{h}}\right)^{\tfrac{x}{h}}\right]\\
&=\frac{1}{x}\ln e=\frac{1}{x}
\end{split}\]
$\therefore\quad f'(x)=(\ln x)'=\frac{1}{x},\quad (x>0)$.
\end{enumerate}

综上所述,目前我们得到几种最常见的函数的导数为(今后可当公式用):
\begin{thm}{}
   \begin{enumerate}
\item $c'=0$($c$为常数).
\item $(x^n)'=nx^{n-1}$($n$为自然数).
\item $(\sin x)'=\cos x$.
\item $(\cos x)'=-\sin x$.
\item $(\ln x)'=\frac{1}{x}$
\end{enumerate}     
\end{thm}

\begin{ex}
\begin{enumerate}
    \item 求下列函数的导数:
\begin{multicols}{2}
\begin{enumerate}[(1)]
\item $f(x)=x^{10}$;
\item $f(x)=x^{99}$;
\item $f(x)=\cos x$;
\item $f(x)=\ln x$.
\end{enumerate}
\end{multicols}
    \item 根据函数的导数定义,求下列函数的导数.
\begin{multicols}{2}
\begin{enumerate}[(1)]
\item $f(x)=3x^4+2x^3$;
\item $f(x)=\frac{1}{x^2}$;
\item $f(x)=\sqrt{x}$;
\item $f(x)=\sin3x$;
\item $f(x)=\cos(-2x)$.
\end{enumerate}
\end{multicols}
\end{enumerate}
\end{ex}

\section{函数的和、差、积、商的导数}
如同研究函数极限的四则运算一样,下面研究函数的四则运算的导数.

已知函数$f(x)$、$g(x)$的导数分别为$f'(x)$、$g'(x)$

\subsection{$y=f(x)+g(x)$}
\[\begin{split}
    \because\quad y' &=\lim_{\Delta x\to0}\frac{\Delta y}{\Delta x}=\lim_{\Delta x\to0}\frac{\left[f(x+\Delta x)+g(x+\Delta x)\right]-\left[f(x)+g(x)\right]}{\Delta x}\\
    &=\lim_{\Delta x\to0}\left[\frac{f(x+\Delta x)-f(x)}{\Delta x}+\frac{g(x+\Delta x)-g(x)}{\Delta x}\right]\\
    &=\lim_{\Delta x\to0}\frac{f(x+\Delta x)-f(x)}{\Delta x}+\lim_{\Delta x\to0}\frac{g(x+\Delta x)-g(x)}{\Delta x}\\
    &=f' (x)+g' (x)
\end{split}\]

$\therefore\quad y' =[f(x)+g(x)]' =f' (x)+g' (x)$

\begin{thm}{定理1}
\[[f(x)+g(x)]' =f' (x)+g' (x)\]
\end{thm}

\begin{thm}{推论}
\[[f(x)-g(x)]' =f' (x)-g' (x)\]
\end{thm}

\subsection{ $y=f(x)g(x)$}
\[\begin{split}
\because\quad \Delta y=&f(x+\Delta x)g(x+\Delta x)-f(x)g(x)\\
&=f(x+\Delta x)g(x+\Delta x)-f(x)g(x+\Delta x)+f(x)g(x+\Delta x)-f(x)g(x)\\
&=[f(x+\Delta x)-f(x)]g(x+\Delta x) +f(x)[g(x+\Delta x)-g(x)]
\end{split}\]
而
\[\begin{split}
    y' &=\lim_{\Delta x\to0}\frac{\Delta y}{\Delta x}=\lim_{\Delta x\to0}\frac{\left[f(x+\Delta x)-f(x)\right]g(x+\Delta x)+f(x)\left[g(x+\Delta x)-g(x)\right]}{\Delta x}\\
    &=\lim_{\Delta x\to0}\frac{f(x+\Delta x)-f(x)}{\Delta x}\cdot\lim_{\Delta x\to0}g(x+\Delta x)+f(x)\cdot\lim_{\Delta x\to0}\frac{g(x+\Delta x)-g(x)}{\Delta x}\\
    &=f' (x)g(x)+f(x)g' (x)
\end{split}\]

$\therefore\quad y' =[f(x)g(x)]' =f' (x)g(x)+f(x)g' (x)$

\begin{thm}{定理2}
\[[f(x)g(x)]' =f' (x)g(x)+f(x)g' (x)\]
\end{thm}

\begin{thm}{推论}
   \[ [cf(x)]'=cf'(x)\qquad \text{($c$为常数)}\]
\end{thm}

\begin{example}
求下列函数的导数
\begin{multicols}{2}
\begin{enumerate}[(1)]
    \item $y=x^3+\sin x$
    \item $y=x^3\sin x$
\end{enumerate}
\end{multicols}
\end{example}

\begin{solution}
\begin{enumerate}[(1)]
    \item $y'=(x^3)'+(\sin x)'=3x^2+\cos x$

$\therefore\quad y'=3x^2+\cos x$

\item $y'=(x^3)'\sin x+x^3(\sin x)'=3x^2\sin x+x^3\cos x$

$\therefore\quad y'=3x^2\sin x+x^3\cos x$
 \end{enumerate}   
\end{solution}

\subsection{$y=\frac{f(x)}{g(x)}$}
\[\begin{split}
    \because\quad  y&=\frac{f(x)}{g(x)}=f(x)\cdot \frac{1}{g(x)}\\
y'&=f'(x)\cdot \frac{1}{g(x)}+f(x)\cdot \left[\frac{1}{g(x)}\right]'
\end{split}\]

$\therefore\quad $欲求$y'$,只需求出$\left[\frac{1}{g(x)}\right]'$

\[\begin{split}
  \because\quad   \left[\frac{1}{g(x)}\right]' &=\lim_{h\to 0}\frac{\frac{1}{g(x+h)}-\frac{1}{g(x)}}{h}=\lim_{h\to 0}\frac{g(x)-g(x+h)}{hg(x+h)g(x)}\\
    &=\frac{1}{g(x)}\cdot \lim_{h\to 0}\frac{-[g(x+h)-g(x)]}{h}\cdot \lim_{h\to 0}\frac{1}{g(x+h)}\\
&=\frac{1}{g(x)}\cdot [-g'(x)]\cdot \frac{1}{g(x)}=-\frac{g'(x)}{[g(x)]^2}
\end{split}\]
\[\begin{split}
    \therefore\quad \left[\frac{1}{g(x)}\right]' &=-\frac{g'(x)}{[g(x)]^2}\\
    y'=\left[\frac{f(x)}{g(x)}\right]'&=f' \left(x\right)\cdot\frac{1}{g\left(x\right)}+f\left(x\right)\cdot\left[\frac{1}{g\left(x\right)}\right]' \\
    &=f' \left(x\right)\cdot\frac{1}{g\left(x\right)}-f\left(x\right)\cdot\frac{g' \left(x\right)}{\left[g\left(x\right)\right]^{2}}\\
    &=\frac{f' \left(x\right)g\left(x\right)-f\left(x\right)g' \left(x\right)}{\left[g\left(x\right)\right]^{2}}
\end{split}\]

\begin{thm}{定理3}
\[\left[\frac{f(x)}{g(x)}\right]'=\frac{f' \left(x\right)g\left(x\right)-f\left(x\right)g' \left(x\right)}{\left[g\left(x\right)\right]^{2}}\]
其中$g(x)\ne 0$.
\end{thm}

\begin{example}
    求下列函数的导数
\begin{multicols}{2}
\begin{enumerate}[(1)]
    \item $y=\frac{x^3}{\sin x}$
    \item $y=\frac{1}{x^3}$
\end{enumerate}
\end{multicols}
\end{example}

\begin{solution}
\begin{enumerate}[(1)]
    \item $y'=\left(\frac{x^3}{\sin x}\right)'=\frac{(x^3)'\sin x-x^3 (\sin x)'}{\sin^2 x}=\frac{3x^2\sin x-x^3\cos x}{\sin^2 x}$
    
    $\therefore\quad y'=\frac{3x^2\sin x-x^3\cos x}{\sin^2 x}$
    \item $y'=\left(\frac{1}{x^3}\right)'=\frac{0\cdot x^3-1\cdot (x^3)'}{(x^3)^2}=-\frac{3x^2}{x^6}=-\frac{3}{x^4}$
\end{enumerate}    
\end{solution}

\begin{example}
    求证:当$n$为负整数时,公式$(x^n)'=nx^{n-1}$仍然成立.
\end{example}

\begin{proof}
设$n=-k$($k$为自然数)
\[\begin{split}
 \because\quad (x^n)'=\left(\frac{1}{x^k}\right)'=\frac{0\cdot x^k-1\cdot (x^k)'}{(x^k)^2}&=-\frac{kx^{k-1}}{x^{2k}}\\
 &=\frac{-k}{x^{k+1}}=-kx^{-(k+1)}=nx^{n-1}   
\end{split}\]

$\therefore\quad (x^n)'=nx^{n-1}$($n$为负整数).
\end{proof}


\begin{rmk}
\begin{enumerate}
    \item 由例12.10可知,公式$(x^n)'=nx^{n-1}$可由$n$为自然数推广到$n$为任意整数.
    \item 若$f(x)$、$g(x)$可导,则$f(x)\pm g(x)$、$f(x)g(x)$、$\frac{f(x)}{g(x)}\; (g(x)\ne 0)$都可导,并且
\begin{enumerate}[(1)]
\item $[f(x)\pm g(x)]'=f'(x)\pm g'(x)$;
\item $[f(x)g(x)]'=f'(x)g(x)+f(x)g'(x)$;
\item $\left[\frac{f(x)}{g(x)}\right]'=\frac{f'(x)g(x)-f(x)g'(x)}{[g(x)]^2}$.
\end{enumerate}

    \item 利用数学归纳法,不难得出
\[ [f_1(x)+f_2(x)+\cdots +f_n(x)]'=f'_1(x)+f'_2(x)+\cdots +f'_n(x)\]
    对于$[f_1(x)f_2(x)\cdots f_n(x)]'=?$ 留给读者去证明自己的猜想.
    \item 利用上述函数四则运算的导数公式,可将某些初等函数的导数问题转化为最简单的函数的导数来解决.
\end{enumerate}
\end{rmk}


\begin{ex}
\begin{enumerate}
    \item 求下列各函数的导数:
\begin{enumerate}[(1)]
    \item $3x^4-4x^3+6x^2-7x+100-3x^{-1}+5x^{-2}-6x^{-3}$;
    \item $x\ln x$;
    \item $(3x^2+1)(x+\cos x)$;
    \item $\frac{2x-1}{x+1}$.
\end{enumerate}
    \item 求下列函数的导数(今后可当公式用):
\begin{multicols}{2}
\begin{enumerate}[(1)]
    \item $\tan x$
    \item $\cot x$
    \item $\sec x$
    \item $\csc x$
\end{enumerate}
\end{multicols}
\end{enumerate}
\end{ex}

\section*{习题二}
\begin{center}
    \bfseries A
\end{center}

\begin{enumerate}
    \item 求下列函数的导数:
\begin{multicols}{2}
\begin{enumerate}[(1)]
    \item $- 3x^{2}+ 2x+ 1$
    \item $( x+ 1) ^{2}( x- 1)$
    \item $\frac{x+1}{x-1}$
    \item $\frac{1}{x^{2}-3x+6}$
    \item $x\sin x+\cos x$
    \item $\frac{\sin x}{1+\cos x}$
    \item $\frac{\ln x}{x}$
    \item $x\tan x-\cot x$
\end{enumerate}    
\end{multicols}
\item 已知:$f(x)=\cos x-\sin x$ 求:$f' \left(\frac\pi3\right)$, $f' \left(\frac{3\pi}4\right)$.
\item 已知:$f(x)=\frac{3}{5-x}+\frac{x^{2}}{5}$, 求:$f' (0)$, $f' (2)$.
\end{enumerate}

\begin{center}
    \bfseries B
\end{center}

\begin{enumerate}\setcounter{enumi}{3}
    \item 求下列函数的导数:
    \begin{multicols}{2}
    \begin{enumerate}[(1)]
\item $x^{2}\tan x$
\item $\frac{x^{2}}{\tan x}$
\item $\frac{\cos x}{1-x^{2}}$
\item $x\sin x\ln x$
\item $x\tan x-\cot x$
\item $(x-a)(x-b)(x-c)$
    \end{enumerate}    
\end{multicols}

\item 设$f(x)=\frac{x^{3}}{3}+\frac{x^{2}}{2}-2x$, 求下列方程的解集:
\begin{multicols}{3}
\begin{enumerate}[(1)]
    \item $f' (x)=0$
    \item $f' (x)=-2$
    \item $f' (x)=10$
\end{enumerate}
\end{multicols}

\item 设$f(x+1)=x^{3}-1$, 求$f' (x)$.
\item 设$f(x)=\begin{cases}
    3-x,& x<1\\
    (3-x)(1-x),& 1\le x<2\\
    -(1-x),& x\ge 2
\end{cases}$,求$f'(x)$.
\item 已知$f(x)=x\sin x$.
\begin{enumerate}[(1)]
\item 求$f'(x)$;
\item 若令$F(x)=f'(x)$,求$F'(x)$.
\end{enumerate}

\item 已知$f(x)=(5x+2)^4$,求$f'(x)$.
\end{enumerate}

\section{复合函数的导数}
在求函数$f(x)=(5x+2)^4$的导数$f'(x)$时,可将
$(5x+2)^4$“展开”后再求导,也可运用函数乘积的导数运算定理求. 即
\[\begin{split}
[(5x+2)^{4}]' &=(5x+2)' (5x+2)^{3}+(5x+2)[(5x+2)^{3}]' \\
&=(5x+2)' (5x+2)^{3}+(5x+2)' (5x+2)^{3}+(5x+2)^{2}[(5x+2)^{2}]' \\
&=2(5x+2)' (5x+2)^{3}+(5x+2)^{2}[2(5x+2)' (5x+2)]\\
&=4(5x+2)^{3}(5x+2)'     
\end{split}\]
$\therefore\quad f' (x)=[(5x+2)^{4}]'=4(5x+2)^{3}(5x+2)=20(5x+2)^{3}$

由于函数$f(x)=(5x+2)^4$可以看成是由函数$y=u^4$和$u=5x+2$复合而成的. 从这个例子我们看到,函数$y=(5x+2)^4$的导数恰好等于对中间变量$u$的导数乘以中间变量对$x$的导数(即$4u^3\cdot u'_x$).

\begin{thm}
{定理} 由函数$y=f(u)$和$u=\varphi(x)$构成的复合函数$y=f[\varphi(x)]$对自变量$x$的导数$y'_x$等于$y$对中间变量$u$的导数$y'_u$ 乘以中间变量$u$对自变量$x$的导数$u'_x$,即
\[y'_x=y'_u\cdot u'_x\]    
\end{thm}

\begin{proof}
若函数$y=f(u)$在点$u$处可导,则
    $\Delta y=f' \left(u\right)\cdot \Delta u+\alpha\left(\Delta u\right)$(其中当$\Delta u\to0$时,$\alpha(\Delta u)\to0$)

\[\begin{split}
    \because\quad y'_x=\lim_{\Delta x\to0}\frac{\Delta y}{\Delta x}&=\lim_{\Delta x\to0}\frac{f' (u)\cdot\Delta u+\alpha(\Delta u)}{\Delta x}\\
    &=f' \left(u\right)\lim_{\Delta x\to0}\frac{\Delta u}{\Delta x}+\lim_{\Delta x\to0}\frac{\alpha\left(\Delta u\right)}{\Delta x}\\
    &=f(u)\varphi' (x)+0=y'_{u}\cdot  u'_{x}
\end{split}\]
$\therefore\quad y'_x=y'_{u}\cdot  u'_{x}$.
\end{proof}

\begin{rmk}
这个求复合函数导数的方法,可以推广到两个以上的中间变量.例如,若函数$y=f(x)$是由函数$y=h(u)$, $u=\varphi(t)$, $t=g(x)$复合而成的,则$y'_x=y'_u\cdot u'_t\cdot t'_x$.
\end{rmk}

在求复合函数的导数时,关键在于分析清楚该函数是由哪几个函数复合而成的,否则极容易出现计算错误,请读者特别注意.

\begin{example}
    求$y=\sin^2 3x$的导数.
\end{example}

\begin{analyze}
函数$y=\sin^2 3x$是由$y=u^2$, $u=\sin t$, $t=3x$复合而成的.
\end{analyze}

\begin{solution}
\textbf{解法1} 
\[\begin{split}
    \because\quad y'_x=y'_u\cdot u'_t\cdot t'_x &=(u^2)'_u\cdot (\sin t)'_t\cdot (3x)'\\
    &=2u\cdot \cos t\cdot 3 =6\sin 3x\cos 3x=3\sin 6x
\end{split}\]

$\therefore\quad y'_x=3\sin 6x$.

\textbf{解法2}
\[\begin{split}
    \because \quad y' _{x}=(\sin^{2}3x)' &=2\sin3x\cdot (\sin 3x)' \\
    &=2\sin 3x\mathrm{cos}3x\cdot (3x)' =\sin 6x\cdot 3 =3\sin 6x.
\end{split}\]

$\therefore\quad y'_x=3\sin 6x$.
\end{solution}

\begin{example}
    求函数$y=\frac{1}{(2-5x)^4}$的导数
\end{example}

\begin{analyze}
    函数$y=\frac{1}{(2-5x)^4}$是由函数$y=u^{-4}$, $u=2-5x$复合而成的.
\end{analyze}

\begin{solution}
    \textbf{解法1} 
\[\begin{split}
    \because\quad y'_x=y'_u\cdot u'_x &=(u^{-4})'_u\cdot (2-5x)'\\
    &=-4u^{-5}\cdot (-5) =20(2-5x)^{-5}=\frac{20}{(2-5x)^5}
\end{split}\]

$\therefore\quad y'_x=\frac{20}{(2-5x)^5}$.

\textbf{解法2}
\[\begin{split}
    \because \quad y' _{x}=[(2-5x)^{-4}]' &=-4(2-5x)^{-5}\cdot (2-5x)' \\
    &=\frac{-4}{(2-5x)^5}\cdot (-5)=\frac{20}{(2-5x)^5}
\end{split}\]

$\therefore\quad y'_x=\frac{20}{(2-5x)^5}$.
\end{solution}

\begin{example}
求函数$y=x^{\alpha}$ ($\alpha$为任意实数,$x\in\R^+$)的导数.
\end{example}

\begin{analyze}
    由于等式$y=x^{\alpha}$的右边是幂的形式且$x>0$,我们不妨对等式两边取对数.
\[\ln y=\alpha\ln x\]
这时等式两边就都能利用已有的知识对$x$求导了.
\end{analyze}

\begin{solution}
\[\begin{split}
    y&=x^{\alpha}\\
    \ln y&=\alpha\ln x\\
\frac{1}{y}\cdot y'_x &=\frac{\alpha}{x}\\
y'_x&=y\cdot\frac{\alpha}{x}=\frac{\alpha x^{\alpha}}{x}=\alpha x^{\alpha-1}
\end{split}\]
$\therefore\quad y'=\alpha x^{\alpha-1}$
\end{solution}

\begin{rmk}
\begin{enumerate}
    \item 例12.13最终解决了幂函数$x^{\alpha}$($\alpha$为任意实数,$x\in\R^+$)的导数.
\[    (x^{\alpha})'=\alpha x^{\alpha-1}\]
    作为它的特例,
\[\begin{split}
    (x^{n})'=n x^{n-1}&\qquad (n\in\Z)\\
    \left(\sqrt[n]{x}\right)'=\frac{1}{n\sqrt[n]{x^{n-1}}}&\qquad (n\in\Z\text{且}n\ge 2)\\
\end{split}\]

\item 欲求函数$y=f(x)$的导数,先将它的两边取对数,然后再对它的两边分别按复合函数求导,这对于能利用对数运算性质变形的一些函数来说,是可采用的一种求导方法. 例如,求指数函数$y=a^x\; (a>0\text{且}a\ne 1)$的导数.
\[\begin{split}
    \ln y&= x\ln a\\
    \frac{1}{y}\cdot y'_x&=\ln a\\
    y'_x&=y\ln a=a^x\ln a
\end{split}\]
$\therefore\quad (a^x)'=a^x\ln a$

特别地,当$a=e$时,$(e^x)'=e^x$.
\end{enumerate}
\end{rmk}




\begin{example}
    例4 求函数 $y= \sqrt [ 3] {\frac {x+ 2}{x- 1}}$ $( x> 1)$的导数


\end{example}

\begin{solution}
\textbf{解法1} 
\[\begin{split}
    \because\quad  y'=\left(\sqrt[3]{\frac{x+2}{x-1}}\right)' &=\frac{1}{3}\left(\frac{x+2}{x-1}\right)^{-\tfrac{2}{3}}\cdot\left(\frac{x+2}{x-1}\right)' \\
    &=\frac{1}{3}\left(\frac{x-1}{x+2}\right)^{\tfrac{2}{3}}\cdot\frac{-3}{(x-1)^{2}}=-\frac{\sqrt[3]{(x-1)^2(x+2)}}{(x-1)^2(x+2)}
\end{split}\]
$\therefore\quad y'=-\frac{\sqrt[3]{(x-1)^2(x+2)}}{(x-1)^2(x+2)}$

\textbf{解法2} $\because\quad y=\left(\sqrt[3]{\frac{x+2}{x-1}}\right)\quad (x>1)$

$\therefore\quad \ln y=\frac{1}{3}\left[\ln(x+2)-\ln(x-1)\right]$

两边对$x$求导,得
\[\begin{split}
    \frac{1}{y}\cdot y'_x&=\frac{1}{3}\left(\frac{1}{x+2}-\frac{1}{x-1}\right)\\
    y'_x&=\frac{y}{3}\cdot \frac{-3}{(x+2)(x-1)}=-\sqrt[3]{\frac{x+2}{x-1}}\cdot \frac{1}{(x+2)(x-1)}\\
    &=-\frac{\sqrt[3]{(x-1)^2(x+2)}}{(x-1)^2(x+2)}
\end{split} \]
即:$y'=-\frac{\sqrt[3]{(x-1)^2(x+2)}}{(x-1)^2(x+2)}$
\end{solution}

\begin{ex}
\begin{enumerate}
    \item 填空(写出下列函数$f(x)$的导数):
\begin{multicols}{2}
\begin{enumerate}[(1)]
    \item 若$c$为常数,则$c'=\blank$;
    \item $(x^{a})'=\blank$;
    \item $(a^x)'=\blank$;
    \item $\left(\frac{1}{x}\right)'=\blank$;
    \item $(\sqrt{x})'=\blank$;
    \item $(\ln x)'=\blank$;
    \item $(e^x)'=\blank$;
    \item $(\sin x)'=\blank$;
    \item $(\cos x)'=\blank$;
    \item $(\tan x)'=\blank$;
    \item $(\cot x)'=\blank$;
    \item $(\sec x)'=\blank$;
    \item $(\csc x)'=\blank$.
\end{enumerate}
\end{multicols}
    \item 填空(写出下列导数的运算法则):
\begin{enumerate}[(1)]
    \item 若$c$为常数,$[cf(x)]'=\blank$;
    \item $[u(x)\pm v(x)]'=\blank$;
    \item $[u(x)v(x)]'=\blank$;
    \item $[u(x)v(x)w(x)]'=\blank$;
    \item $\left[\frac{u(x)}{v(x)}\right]'=\blank$;
    \item 由函数$y=f(u)$, $u=\varphi(x)$复合而成的函数$y=f[\varphi(x)]$的导数$y'_x=\blank$.
\end{enumerate}
    \item 求下列函数的导数:
\begin{multicols}{2}
\begin{enumerate}[(1)]
    \item $y=(2x^{2}-x+3)^{2}$
    \item $y=\frac{1}{2x+1}$
    \item $y=\sqrt{x^{2}-5}$
    \item $y=\frac{1}{\sqrt{3x-2}}$
    \item $y=\sin x^{3}$
    \item $y=\cos\left(2x-\frac{\pi}{6}\right)$
    \item $y=e^{\sin x}+3^{x^{2}}$
    \item $y=\log_{2}(2x^{2}-3x+1)$
\end{enumerate}
\end{multicols}
\end{enumerate}
\end{ex}

\section{隐函数的导数}
在12.7节中,我们为了求函数$y=a^x$的导数,曾将等式两边取对数,得
\[\ln y=x\ln a\]
即
\[x\ln a-\ln y=0\]

像这种变量之间的函数关系是由某一方程$F(x,y)=0$所确定的函数叫做隐函数.

对隐函数求导数,应在等式的两边同时对$x$求导.

\begin{example}
    已知 $e^y=\sin x+2y$, 求 $y'_x$
\end{example}

\begin{solution}
    在方程两边同时对$x$ 求导,得
\[\begin{split}
    e^y\cdot y' _x&=\cos x+2y'_x\\
    (e^{y}-2)y'_{x}&=\cos x
\end{split}\]
当$e^{y}- 2\neq 0$时,$y'_{x}=\frac{\cos x}{e^{y}-2}$.
\end{solution}


\begin{example}
    求椭圆$\frac{x^2}{a^2}+\frac{y^2}{b^2}=1$上一点$P(x_0,y_0)$处的切线方程。
\end{example}

\begin{analyze}
    当$y_0\neq0$时,问题的关键是求点$P(x_0,y_0)$处切线
的斜率,这就需视方程$\frac{x^2}{a^2}+\frac{y^2}{b^2}=1$中的$y$为$x$的函数,利用隐
函数的求导方法求出$y'_x$.
\end{analyze}

\begin{solution}
    由已知,$b^2x^2+a^2y^2=a^2b^2,\quad 2b^2x+2a^2y\cdot y'_x=0$

    当 $y\neq0$ 时,$y'_{x}=-\frac{b_{2}x}{a_{2}y}$.

因此,当$y_0\ne 0$时,椭圆$\frac{x^{2}}{a^{2}}+\frac{y^{2}}{b^{2}}=1$ 在点 $P(x_0,y_0)$处的切
线斜率为 $K=-\frac{b^{2}x_{0}}{a^{2}y_{0}}$, 其切线方程为
\[\begin{split}
    y-y_{0}&=-\frac{b^{2}x_{0}}{a^{2}y_{0}}(x-x_{0})\\
    b^2x_0x+a^2y_0y&=b^2 x^2_0+a^2 y^2_0\\
    \frac{x_0x}{a^2}+\frac{y_0y}{b^2}&=\frac{x^2_0}{a^2}+\frac{y^2_0}{b^2}=1
\end{split}\]

$\therefore\quad $所求切线方程为$\frac{x_0x}{a^2}+\frac{y_0y}{b^2}=1$\hfill(1)

当$y_0=0$时,椭圆在点$(a,0)$和$(-a,0)$的切线分别为$x=a$和$x=-a$,也符合(1)式. 

$\therefore\quad $椭圆$\frac{x^2}{a^2}+\frac{y^2}{b^2}=1$上一点$P(x_0,y_0)$处的切线方程为$\frac{x_0x}{a^2}+\frac{y_0y}{b^2}=1$
\end{solution}

利用求隐函数导数的方法,我们还可求一些反三角函数的导数.

\begin{example}
求下列函数的导数(今后可当公式用)
\begin{multicols}{2}
\begin{enumerate}[(1)]
    \item $y=\arcsin x$
    \item $y=\arctan x$
\end{enumerate}
\end{multicols}
\end{example}

\begin{solution}
\begin{enumerate}[(1)]
    \item $\because\quad y=\arcsin x$

$\therefore\quad \sin y=x$(转化为$y=\arcsin x$的反函数)等式两边对$x$求导,得
\[\begin{split}
    \cos y\cdot y'_x&=1\\
y'_x&=\frac{1}{\cos y}=\frac{1}{\cos(\arcsin x)}=\frac{1}{\sqrt{1-x^2}}
\end{split}\]
$\therefore\quad y'=\frac{1}{\sqrt{1-x^2}}$

\item $\because\quad y=\arctan x$

$\therefore\quad \tan y=x$
\[\begin{split}
     \sec^2 y\cdot y'_x&=1\\
     y'_x&=\frac{1}{\sec^2 y}=\frac{1}{1+\tan^2 y}=\frac{1}{1+x^2}
\end{split}\]
$\therefore\quad y'=\frac{1}{1+x^2}$.
\end{enumerate}    
\end{solution}

\begin{ex}
\begin{enumerate}
    \item 求双曲线$3x^2-y^2=1$在点$A(1,-\sqrt{2})$处的切线的斜率.
    \item 求抛物线$y=2px\; (p>0)$上一点$p(x_0,y_0)$处的切线方程.
    \item 求下列函数的导数:
\begin{multicols}{2}
\begin{enumerate}[(1)]
    \item $y=\arccos x$
    \item $y={\rm arccot\;} x$
\end{enumerate}
\end{multicols}
\end{enumerate}
\end{ex}


\section{初等函数的导数}
从12.5——12.8节,我们陆续解决了求各基本初等函数的导数的问题,现将它们汇总如下,作为求函数导数的基本公式,并请读者回忆它们的推导过程.

\begin{thm}{常数的导数}
\[c'=0\qquad \text{($c$为常数)}\]    
\end{thm}

\begin{thm}
    {幂函数的导数}
\[(x^{\alpha})'=\alpha x^{\alpha-1}\qquad \text{($\alpha$为实数)}\]
特别地,$x'=1,\quad \left(\frac{1}{x}\right)'=-\frac{1}{x^2},\quad (\sqrt{x})^1=\frac{1}{2\sqrt{x}}$等.
\end{thm}

\begin{thm}
    {对数函数的导数}
\begin{enumerate}[(1)]
\item $(\ln x)' =\frac{1}{x}$;
\item  $( \log _{a}x) '= \frac 1{x\ln a},\qquad ( a> 0\text{且}a\neq1)$.
\end{enumerate}
\end{thm}

\begin{thm}{指数函数的导数}
\begin{enumerate}[(1)]
    \item $( e^x) '= e^x$;
\item $(a^{x})' =a^{x}\ln a,\qquad ( a> 0\text{且} a\neq 1)$.
\end{enumerate}
\end{thm}

\begin{thm}{三角函数的导数}
\begin{enumerate}[(1)]
    \begin{multicols}{2}
    \item $(\sin x)' =\cos x$
    \item $(\cos x)' =-\sin x$
    \item $(\tan x)'  = \frac 1{\cos ^{2}x}= \sec ^{2}x$
    \item $\left ( \cot x\right ) '= - \frac 1{\sin ^{2}x}= - \csc ^{2}x$
    \item $\left(\sec x\right)' =\frac{\sin x}{\cos^{2}x}=\tan x\sec x$
\end{multicols}
    \item $\left(\csc x\right)' =-\frac{\cos x}{\sin^{2}x}=-\cot x\csc x$
\end{enumerate}    
\end{thm}

\begin{thm}{反三角函数的导数}
\begin{multicols}{2}
  \begin{enumerate}[(1)]
    \item $(\arcsin x)' =\frac{1}{\sqrt{1-x^{2}}}$
    \item $(\arccos x)' =-\frac{1}{\sqrt{1-x^{2}}}$
    \item $(\arctan x)'=\frac{1}{1+x^2}$
    \item $({\rm arccot\;} x)'=-\frac{1}{1+x^2}$
\end{enumerate}  
\end{multicols}
\end{thm}

\begin{ex}
    求下列函数的导数:
 \begin{enumerate}
    \item $y=\sin 3x\cos 2x+\cos 3x\sin 2x$
    \item $y=\frac{2\tan 3x}{1-\tan^2 3x}$
    \item $y=\arctan\sqrt{x}$
    \item $y=e^{\cos 2x}$
\end{enumerate}   
\end{ex}

\section{二阶导数}
我们知道,一个运动的质点的位移$s=s(t)$对于时间$t$的导数是时刻$t$时的瞬时速度
\[v=v(t)=s'(t)\]
由于$v=v(t)$仍为$t$的函数,如果$v(t)$可导,那么称$v'(t)$是该质点在时刻$t$时的加速度.
\[ a=a(t)=v'(t)=[s'(t)]'  \]

\begin{thm}
{定义} 对于函数$y=f(x)$,若$f'(x)$可导,那么它的导数$[f'(x)]'$叫做$f(x)$的二阶导数,记作$f''(x)$或$y''$.
\end{thm}


\begin{example}
    已知$y=3x^4-2x^3+x+5$,求$y''$.
\end{example}

\begin{solution}
$\because\quad y=3x^4-2x^3+x+5,\quad y'=12x^3-6x^2+1$

$\therefore\quad y''=36x^2-12x$.
\end{solution}

\begin{example}
    设$y=e^x\sin x$,求$y'\big|_{x=0}$及$y''\big|_{x=0}$.
\end{example}

\begin{solution}
$\because\quad     y=e^x\sin x$

$\therefore\quad y'=e^x\sin x+e^x\cos x=e^x(\sin x+\cos x)\quad \Rightarrow\quad y'\big|_{x=0}=1$

$\because\quad y''=e^x(\sin x+\cos x)+e^x(\cos x-\sin x)=2e^x\cos x$

$\therefore\quad y''\big|_{x=0}=2$.
\end{solution}

\begin{ex}
\begin{enumerate}
    \item 某物体的运动方程为$s=v_0t-\frac{1}{2}gt^2$,求$t=2$时的速度和加速度.
    \item  求下列函数的二阶导数:
\begin{multicols}{2}
\begin{enumerate}[(1)]
 \item  $y=x\sin x$
\item  $y=\sqrt{x}$
\item  $y=\tan x$
\item  $y=ax^3+bx^2+cx+d$ 
\end{enumerate}
\end{multicols}
\end{enumerate}
\end{ex}

\section*{习题三}
\begin{center}
    \bfseries A
\end{center}

\begin{enumerate}
    \item 求下列函数的导数: 
\begin{multicols}{2}
\begin{enumerate}[(1)]
    \item $y=\ln\tan x$
    \item $y=\sin(2^x)$
    \item $y=e^{\sin^2 x}$
    \item $y=(\arcsin x)^2$
    \item $y=\ln[\ln(\ln x)]$
    \item $y=\sqrt{\cos x^2}$
    \item $y=\arctan \frac{x+1}{x-1}$
    \item $y=x^2\sin\frac{1}{x}$
\end{enumerate}
\end{multicols}
    \item 求下列隐函数的导数$y'_x$:
\begin{multicols}{2}
\begin{enumerate}[(1)]
    \item $y^2-2xy+10=0$
    \item $y-1=xe^y$
    \item $x^3+y^3=3xy$
    \item $x^y=y^x$
\end{enumerate}
\end{multicols}
    \item 在曲线$y=x^3+x-2$上哪一点的切线与直线$y=4x-1$平行?
    \item  写出双曲线$3x^2-y^2=1$在点$A(1,\sqrt{2})$处的切线方程.
    \item  求下列函数的二阶导数:
\begin{multicols}{2}
\begin{enumerate}[(1)]
    \item $y=\sqrt{a^2-x^2}$
    \item $y=e^{2x-1}$
    \item $y=\sin^4 x+\cos^4 x$
    \item $y=\ln\sin x$
\end{enumerate}
\end{multicols}
\end{enumerate}


\begin{center}
    \bfseries B
\end{center}

\begin{enumerate}\setcounter{enumi}{5}
    \item 求下列函数的导数: 
\begin{multicols}{2}
    \begin{enumerate}
        \item $y=x^x$
        \item $y=x^{\tfrac{1}{x}}$
        \item $y=\sqrt{\frac{x-1}{x^2+3x}}$
        \item $y=(\sin x)^{\cos x}$
    \end{enumerate}
\end{multicols}
\end{enumerate}


\section*{三、微分}

\section{微分的概念及其运算}
请先看两个例子:

对于函数$y=f(x)=3x^2$,它的导函数$f'(x)=6x$,在$x=1$处,函数的改变量为
\[\begin{split}
   \Delta y=f(1+\Delta x)-f(1)
&=3(1+\Delta x)^2-3\\
&=6\cdot \Delta x+3(\Delta x)^2=f'(1)\cdot \Delta x+3(\Delta x)^2 
\end{split}\]
即$\Delta y=f'(1)\cdot \Delta x+3(\Delta x)^2$. 当$\Delta x$很小时,有$\Delta y\approx f'(1)\cdot \Delta x$.

一个半径为$r$的球,它的表面积为$S=4\pi r^2$,当半径由$r$变为$r+\Delta r$时,其对应的表面积的改变量为
\[\begin{split}
\Delta S&=4\pi (r+\Delta r)^2-4\pi r^2\\
&=8\pi r\cdot \Delta r+4\pi (\Delta r)^2\\
&=S'_r\cdot \Delta r+4\pi (\Delta r)^2    
\end{split}\]
当$\Delta r$很小时,$\Delta S\approx S'_x\cdot \Delta r$.

从近似计算的角度来看,$f'(x)\cdot \Delta x$是函数$y=f(x)$的改变量$\Delta y$的主要部分,当$|\Delta x|$很小时,$\Delta y$的值可用$f'(x)\cdot \Delta x$近似地表示出来. 在误差不太大的情况下,可将求$\Delta y$的复杂问题转化为计算$f'(x)\cdot \Delta x$.

设函数$y=f(x)$在点$x$处可导.
\begin{enumerate}
    \item 自变量$x$的改变量$\Delta x$又称为自变量$x$的微分,记作$\dd x$,即$\dd x=\Delta x$.
    \item $f'(x)\cdot \dd x$称为$f(x)$在点$x$处的微分,记作$\dd y$,即$\dd y=f(x)\cdot \dd x$.
\end{enumerate}

    有了微分的定义,函数$y=f(x)$的导数$f'(x)$又可看成函数的微分$\dd y$与自变量的微分的商,即
\[\frac{\dd y}{\dd x}=f'(x)=\lim_{\Delta x\to 0}\frac{\Delta y}{\Delta x}\]

\begin{rmk}
    微分一词源于拉丁文“differentia”,表示“差”的意思.
\end{rmk}


\begin{example}
已知函数$y=\sqrt{\sin x}$,求
\begin{multicols}{2}
\begin{enumerate}[(1)]
    \item $\frac{\dd y}{\dd x}$
    \item $\dd y$
\end{enumerate}
\end{multicols}
\end{example}

\begin{analyze}
由微分的定义可知,求$\frac{\dd y}{\dd x}$就是求$f'(x)$,而求$\dd y$是求$f'(x)\dd x$.   
\end{analyze}

\begin{solution}
\begin{enumerate}[(1)]
    \item $\frac{\dd y}{\dd x}=\left(\sqrt{\sin x}\right)'=\frac{1}{2\sqrt{\sin x}}\cdot (\sin x)'=\frac{\cos x}{2\sqrt{\sin x}}$
    \item $\dd y=\left(\sqrt{\sin x}\right)'\dd x=\frac{\cos x}{2\sqrt{\sin x}}\cdot \dd x$
\end{enumerate}
\end{solution}

\begin{rmk}
    由于$(\sin x)'\dd x=\dd(\sin x)$,今后求$\dd y$时,又可写成
\[\dd y=\dd\left(\sqrt{\sin x}\right)=\frac{1}{2\sqrt{\sin x}}\cdot \dd (\sin x)=\frac{\cos x}{2\sqrt{\sin x}}\cdot \dd x\]
\end{rmk}

一般地,对于$y=f(u)$与$u=g(x)$复合而成的$y=f[g(x)]$
\[\begin{split}
    dy=f'(u)du&=f'(u)\cdot g'(x)\dd x\\
&=f'[g(x)]\cdot g'(x)\dd x
\end{split}\]

由于函数的微分是该函数的导数与自变量的积,即$\dd y=f'(x)\dd x$. 不难证明求微分的四则运算法则.

\begin{thm}{微分的四则运算法则}
\begin{multicols}{2}
\begin{enumerate}[(1)]
    \item $\dd (u\pm v)=\dd u\pm \dd v$
    \item $\dd(uv)=v\dd u+u\dd v$
    \item $\dd\left(\frac{u}{v}\right)=\frac{v\dd u-u\dd v}{v^2}$
\end{enumerate}        
\end{multicols}
\end{thm}

例如,
\[\begin{split}
    \dd\left(\frac{u}{v}\right)=\left(\frac{u}{v}\right)'\dd x&=\frac{vu'-uv'}{v^2}\cdot \dd x\\
    &=\frac{v(u'\dd x)-u(v'\dd x)}{v^2}=\frac{v\dd u-u\dd v}{v^2}
\end{split}\]

\begin{example}
    求函数$y=e^{3x}\sin 2x$的微分
\end{example}

\begin{solution}
\[\begin{split}
\dd y=\dd (e^{3x}\sin 2x)&=\sin 2x\cdot \dd(e^{3x})+e^{3x}\cdot \dd (\sin2x)\\
&=\sin 2x\cdot e^{3x}\dd (3x)+e^{3x}\cos 2x\cdot \dd(2x)\\
&=3\sin 2x\cdot e^{3x}\dd x+2e^{3x}\cdot \cos2x\dd x\\
&=\left(3e^{3x}\sin2x+2e^{3x}\cdot \cos x\right)\dd x
    \end{split}\]
\end{solution}

\begin{ex}
\begin{enumerate}
    \item 写出各基本初等函数的微分.
    \item 求下列函数的微分:
\begin{multicols}{2}
\begin{enumerate}[(1)]
    \item $y=x^3+2x^2-4x+5$
    \item $y=e^{\sin x}\cdot \cos(2x+1)$
    \item $y=\frac{\sin x}{x}$
    \item $y=\arctan(\ln x)$
\end{enumerate}
\end{multicols}

    \item 半径$R=$10mm的球,在冷却中缩短了0.01mm,求此时刻球体积的微分$\dd v$及$\Delta v-\dd v$的值.
\end{enumerate}
\end{ex}

\section{微分在近似计算中的应用}
我们知道,若函数$y=f(x)$在点$x_0$处可导,当$|\Delta x|$很小时,可用微分$\dd y$近似代替函数的改变量$\Delta y$,于是
\begin{align}
  f(x)-f(x_0)&=\Delta y\approx f'(x_0)\dd x=f'(x_0)(x-x_0)\nonumber\\
f(x)&\approx f(x_0)+f'(x_0)(x-x_0)  \tag{1}
\end{align}
所得的关系式(1)给出了求$f(x)$近似值的一个方法.

\begin{example}
    求$\sin31^{\circ}$的近似值(精确到0.0001)
\end{example}

\begin{analyze}
   $\sin31^{\circ}$的值不易直接求出,为了求出$\sin31^{\circ}$的近似值,可先构造函数$f(x)=\sin x$,而$f(30^{\circ})$和$f'(30^{\circ})$的值都容易求出,所以利用关系式
\[f(31^{\circ})\approx f(30^{\circ})+f'(30^{\circ})\cdot \Delta x\]
可求出$f(31^{\circ})$的近似值.

又$\because\quad f(30^{\circ})$、$f'(30^{\circ})$、$f(31^{\circ})$都是实数,

$\therefore\quad $必须将$\Delta x$化为弧度,即
$\Delta x=31^{\circ}-30^{\circ}=1^{\circ}=\frac{\pi}{180}$
\end{analyze}

\begin{solution}
令$f(x)=\sin x$, $x_0=30^{\circ}=\frac{\pi}{6}$,则
\[\Delta x=31^{\circ}-30^{\circ}=1^{\circ}=\frac{\pi}{180}\]

$\because\quad f(31^{\circ})\approx f\left(\frac{\pi}{6}\right)+f'\left(\frac{\pi}{6}\right)\cdot \dd x$

$\therefore\quad \sin31^{\circ}\approx \sin\frac{\pi}{6}+\cos\left(\frac{\pi}{6}\right)\cdot \frac{\pi}{180}\approx \frac{1}{2}+\frac{\sqrt{3}}{2}\x 0.01745\approx 0.5151$

即$\sin31^{\circ}\approx 0.5151$.
\end{solution}

\begin{rmk}
    利用微分进行近似计算的一般步骤是:
\begin{enumerate}[(1)]
\item 设适当的辅助函数$f(x)$;
\item 确定定点$x_0$及$\Delta x$,使$f(x_0)$易于计算,而且$|\Delta x|$相对较小;
\item 求$f'(x_0)$的值;
\item 按近似公式$f(x_0+\Delta x)\approx f(x_0)+f'(x_0)\Delta x$进行计算.
\end{enumerate}
\end{rmk}

对于关系式(1),若设$x_0=0$,则(1)式变为
\begin{equation}
   f(x)=f(0)+f'(0)\cdot x  \tag{2}
\end{equation}

利用(2)式,当$x$充分小时,可导出常用的一些近似计算公式.
\begin{itemize}
    \item $(1+x)^{\alpha}\approx 1+\alpha x$  \hfill (1)
    \item $e^x\approx 1+x$  \hfill (2)
    \item $\ln(1+x)\approx x$  \hfill (3)
    \item $\sin x\approx x$  \hfill (4)
    \item $\tan x\approx x$  \hfill (5)
\end{itemize}
(上述各近似公式的推导,请读者自己完成)

  \begin{example}
    求$\sqrt[5]{270}$的近似值(精确到0.0001).
\end{example}

\begin{analyze}
    为了能利用近似公式(1),可按照公式(1)的结构,将式子变形.
\[\sqrt[5]{270}=\sqrt[5]{243+27}=3\sqrt[5]{1+\frac{27}{243}}=3\left(1+\frac{27}{243}\right)^{\tfrac{1}{5}}\]
\end{analyze}

\begin{solution}
\[\begin{split}
    \sqrt[5]{270}=\sqrt[5]{243+27}&=3\left(1+\frac{27}{243}\right)^{\tfrac{1}{5}}\\
    &\approx 3\left(1+\frac{1}{5}\x\frac{27}{243}\right) \qquad \text{(近似公式(1))}\\
    &=3+\frac{1}{15}\approx 3.0667
\end{split}\]
即$\sqrt[5]{270}\approx 3.0667$.
\end{solution}

\begin{ex}
\begin{enumerate}
    \item 当$|x|$充分小时,求证:
\begin{multicols}{2}
\begin{enumerate}[(1)]
    \item $\sqrt[n]{1+x}\approx 1+\frac{x}{n}\qquad (n\in\N,\; n\ge 2)$
    \item $e^x\approx 1+x$
\end{enumerate}
\end{multicols}
\item 求下列各数的近似值(精确到0.0001):
\begin{multicols}{2}
    \begin{enumerate}[(1)]
        \item $\sin 46^{\circ}$
        \item $\sqrt{4.01}$
        \item $\ln 0.9998$
        \item $e^{-0.0002}$
    \end{enumerate}
    \end{multicols}
\end{enumerate}
\end{ex}

\section*{习题四}
\begin{center}
    \bfseries A
\end{center}

\begin{enumerate}
    \item 已知函数$y=x^3-x$,在$x=2$、$\dd x=0.01$时,计算$\Delta y$及$\dd y$.
    \item 当自变量$x$由$\frac{\pi}{6}$变到$\frac{61\pi}{360}$时,求函数$y=\sin\frac{x}{3}$的微分(精确到0.0001).
   \item 在下列图形中,标出相应的$\Delta y$及$\dd y$.
\begin{figure}[htp]
    \centering
\begin{tikzpicture}[>=stealth]
\begin{scope}
\draw[->](-1,0)--(4,0)node[above]{$x$};
\draw[->](0,-1)--(0,3.5)node[left]{$y$};
\draw[domain=-1:2.75, samples=100, very thick, smooth]plot(\x, {0.25*(\x-.7)*(\x+.5)*(\x-1.5)+1})node[left]{$y=f(x)$};
\node[below left]{$O$};
\draw(1.5,0)node[below]{$x_0$}--(1.5,1)node[below left]{$C$};
\draw(2.5,0)node[below]{$x_0+\Delta x$}--(2.5,2.35)node[right]{$A$};
\draw[thick](2.5,1.4)node[right]{$B$}--(1.5,1);
\end{scope}
\begin{scope}[xshift=6cm]
    \draw[->](-1,0)--(4,0)node[above]{$x$};
\draw[->](0,-1)--(0,3.5)node[left]{$y$};
\draw[domain=-1:2.75, samples=100, very thick, smooth]plot(\x, {-0.3*(\x-.5)*(\x+.5)*(\x-2.5)+2});
\node at (0,1.5)[right]{$y=f(x)$};
\node[below left]{$O$};
\draw(2,0)node[below]{$x_0$}--(2,2.56)node[above]{$A$};
\draw(2.5,0)node[below right]{$x_0+\Delta x$}--(2.5,3);
\node at (2.5,2)[right]{$C$};
\node at (2.5,2.3)[above right]{$B$};
\tkzDefPoints{2.5/2.3/B, 2/2.56/B'}
\tkzDrawLines[add= 1 and 1, thick](B,B')


\end{scope}
\end{tikzpicture}
    \caption*{第3题}
\end{figure}
\item 求下列函数的微分:
\begin{multicols}{2}
    \begin{enumerate}[(1)]
        \item $y=\frac{2}{x^2}$
        \item $y=5\sqrt[3]{x+1}$
        \item $y=(1+2x-x^2)^3$
        \item $y=\cos x^2$
        \item $y=e^x\sin^2 x$
        \item $y=\frac{2x-1}{(x-1)^2}$
        \item $y=\frac{1-x^2}{1+x^2}$
        \item $y=\frac{1}{3}\tan^3\theta+\tan\theta$
    \end{enumerate}
\end{multicols}

\item 计算$\tan 30^{\circ}30'$的近似值(精确到0.0001).
\item 计算$\sin29^{\circ}$的近似值(精确到0.0001).
\item 计算下列各数的近似值(精确到0.0001):
\begin{multicols}{2}
    \begin{enumerate}[(1)]
        \item $\sqrt[5]{1.002}$
        \item $\sqrt[3]{0.998}$
        \item $\ln1.0021$
        \item $\sin0.1^{\circ}$
    \end{enumerate}
\end{multicols}
\end{enumerate}

\begin{center}
    \bfseries B
\end{center}

\begin{enumerate}\setcounter{enumi}{7}
    \item 已知函数$f(x)=x^2$在点$x_0$处自变量$x$的微分$\dd x=0.2$, 对应函数的微分$\dd f(x)=-0.8$,试求$x_0$的值.
    \item   已知函数$y=f[f(x)]$,求$\frac{\dd y}{\dd x}$及$\dd y$.
        \item   已知函数$y=f(\sin x)$,求$\frac{\dd y}{\dd(\sin x)}$及$\frac{\dd y}{\dd x}$.
        \item  单摆的周期$T$(秒)与单摆的长度$\ell$(厘米)之间的函数关系$T=2\pi\sqrt{\frac{\ell}{980}}$,长度为20厘米的单摆加长1厘米后,它的周期大约增加多少?
\end{enumerate}

\section*{四、利用导数研究函数}
过去我们曾研究过函数的性质,现在当我们学习导数后,就可以用导数直接研究函数的性质了,并且方法要比过去简捷.

\section{函数的单调性}
首先我们介绍一个有关导数的十分重要的定理——拉格朗日中值定理.(这个定理的证明留给读者上大学后去完成)

\begin{thm}
    {定理} 如果函数$y=f(x)$在闭区间$[a,b]$上连续,在开区间$(a,b)$内可导,那么在$(a,b)$内至少有一点$\xi$,使得
    \[f'(\xi)=\frac{f(b)-f(a)}{b-a}\]  
\end{thm}

如图12.5,若$[a,b]$上的一段连续曲线,在$(a,b)$内每一点处有切线,则至少有一条切线平行于连结曲线两端点$(a,f(a))$、$(b,f(b))$所成的弦.

\begin{figure}[htp]
    \centering
\begin{tikzpicture}[>=stealth]
    \draw[->](-1,0)--(6,0)node[below]{$x$};
\draw[->](0,-1)--(0,5)node[left]{$y$};
\draw[domain=0:4, smooth, very thick, samples=100]plot(\x+.5, {0.2*(\x-1)*(\x-2)*(\x-3.5)+3.5});
\tkzDefPoints{.5/2.1/A, 4.5/4.1/B}
\node[below left]{$O$};
\draw[dashed](A)node[left]{$A$}--node[right]{$f(a)$}(.5,0)node[below]{$a$};
\draw[dashed](B)node[right]{$B$}--node[right]{$f(b)$}(4.5,0)node[below]{$b$};
\draw(A)--(B);
\tkzDefPoints{1.5/0/C}
\draw[dashed](1.5,0)node[below]{$\xi$}--(1.5,3.5);
\draw[domain=.5:3, thick]plot(\x, {0.5*\x+2.75});



\end{tikzpicture}
    \caption{}
\end{figure}


我们已经学过增函数与减函数的概念,下面学习如何利用导数来判断函数的单调性.

\begin{thm}
    {定理}设函数$f(x)$在闭区间$[a,b]$上连续,在开区间$(a,b)$内可导.
\begin{enumerate}[(1)]
\item 若在$(a,b)$内,$f'(x)>0$,则$f(x)$在$(a,b)$上是增函数;
\item 若在$(a,b)$内,$f'(x)<0$,则$f(x)$在$(a,b)$上是减函数.
\end{enumerate}
\end{thm}

\begin{proof}
设$x_1,x_2\in[a,b]$且 $x_1<x_2$, 则必存在一点 $\xi\; (x_1<\xi<x_2)$, 使得
$$\frac{f(x_{2})-f(x_{1})}{x_{2}-x_{1}}=f'(\xi)$$
即 $f( x_{2}) - f( x_{1}) = f'( \xi ) ( x_{2}- x_{1})$

若 $f'(x)$在$(a,b)$内恒为正,则 $f'({\xi})>0$ 而 $x_2-x_1>0$

$\therefore\quad f( x_{2}) - f( x_{1}) = f'( \xi ) ( x_{2}- x_{1}) > 0$ 即
$f(x_{2})>f(x_{1})$. 得出 $f(x)$在$(a,b)$上是增函数。

若$f'(x)$在$(a,b)$内恒为负,则 $f'(\xi)<0$, 而 $x_{2}-x_{1}>0$

$\therefore\quad f( x_{2}) - f( x_{1}) = f'( \xi ) ( x_{2}- x_{1}) < 0$即$f(x_{2})<f(x_1)$. 得出$f(x)$在$(a,b)$上是减函数.

另外,若在开区间$(a,b)$内 $f'(x)\equiv 0$,则 $f'({\xi})\equiv0,\; (x_{1}<\xi<x_{2})$, 因此有 $f(x_{2})\equiv f(x_{1})$, 这样 $f(x)$在区间$[a,b]$上是常数.
\end{proof}



\begin{example}
    求函数$f(x)=x^3-x^2-x+5$的单调区间
\end{example}

\begin{solution}
$f(x)=x^3-x^2-x+5,\qquad f'\left(x\right)=3x^{2}-2x-1=\left(3x+1\right)\left(x-1\right)$

$\because\quad $ 当 $x<-\frac13$或 $x>1$ 时,$f'(x)>0$.

$\therefore\quad f( x)$的递增区间是$\left(-\infty,\; -\frac13\right)$和$(1,+\infty)$.

$\because\quad $ 当$-\frac13<x<1$时,$f'(x)<0$. 

$\therefore\quad f( x)$的递减区间是$\left(-\frac13,\; 1\right).$
\end{solution}

\begin{example}
    求函数$f(x)=(x-2)^2(x+1)^3$的单调区间
\end{example}

\begin{solution}
\[\begin{split}
    f(x)&=(x-2)^2(x+1)^3\\
f'(x)&=2(x-2)(x+1)^{3}+3(x-2)^{2}(x+1)^{2}=(x-2)(x+1)^{2}(5x-4)
\end{split}\]
令$f'( x) = 0$ 得 $x= - 1,\; \frac 54,\; 2$.
\begin{center}
    \begin{tabular}{c|cccc}
\hline
& $(-\infty,-1)$ & $\left(-1,\frac{5}{4}\right)$ & $\left(\frac{5}{4},2\right)$ & $(2,+\infty)$ \\ 
\hline
$f'(x)$&  $+$&  $+$&  $-$&  $+$\\
$f(x)$ & $\nearrow$& $\nearrow$& $\searrow$& $\nearrow$\\
\hline
    \end{tabular}
\end{center}

$f(x)$的递增区间是$\left(-\infty,\; \frac{5}{4}\right)$, $(2,+\infty)$;递减区间是
$\left(\frac{5}{4},\; 2\right).$
\end{solution}

\begin{example}
    求证:若 $x\in\left(0,\; \frac\pi2\right)$, 则 $\tan x>\sin x$.
\end{example}

\begin{analyze}
    欲证$\tan x>\sin x$, 只需证$\tan x-\sin x>0$, 若设 $f(x) =\tan x-\sin x$,
    
$\because\quad  f(0)=0$, 只需证当 $x\in\left(0,\; \frac\pi2\right)$时,$f(x)>f(0)$.
\end{analyze}

\begin{proof}
    设$f(x)=\tan x- \sin x\quad x\in \left [ 0,\; \frac \pi 2\right )$

$\because\quad f'( x) = \sec ^{2}x- \cos x= \frac {1- \cos ^{3}x}{\cos ^{2}x}> 0$, 
其中$x\in\left(0,\; \frac\pi2\right)$.

$\therefore\quad f( x) =\tan x-\sin x$ 在区间$\left[0,\; \frac\pi2\right)$是增函数

即当$0<x<\frac{\pi}{2}$时,$f(x)>f(0)=0$. 

$\therefore\quad \tan x>\sin x$, 其中$x\in\left(0,\frac\pi2\right)$.
\end{proof}

\begin{ex}
\begin{enumerate}
    \item 求下列函数的单调区间:
\begin{multicols}{2}
\begin{enumerate}[(1)]
    \item $y=\frac{2x}{1+x^2}$
\item $y=x-e^{x}$
\item $y= \ln ( x+ \sqrt {1+ x^{2}})$ 
\item $y=(x-1)(x+1)^{3}$
\end{enumerate}
\end{multicols}

    
\item    证明不等式:
\begin{enumerate}[(1)]
    \item $\frac{1}{3}(x-1)>\sqrt[3]{x}-1$, 其中$x\in ( 1, + \infty )$
    \item $\ln(1+x)>x-\frac{1}{2}x^{2}$, 其中 $x\in(0,+\infty)$
\end{enumerate}
\end{enumerate} 
\end{ex}

\section{函数的极大值与极小值}
请看图12.6至图12.9:

\noindent
\begin{minipage}{.48\textwidth}
    \centering
\begin{tikzpicture}[>=stealth]
    \draw[->](-1,0)--(5,0)node[below]{$x$};
    \draw[->](0,-1)--(0,4)node[left]{$y$};
\node[below left]{$O$};
\draw[very thick](2,1)node[right]{极小} arc (0:90:2.2);
\draw[very thick](2,1) arc (180:90:2)node[above]{$y=f(x)$};
\draw[dashed](2,1)--(2,0)node[below]{$x_0$};
\node at (2,0){\small (\quad)};
\end{tikzpicture}    
\captionof{figure}{}
\end{minipage}\hfill
\begin{minipage}{.48\textwidth}
        \centering
\begin{tikzpicture}[>=stealth]
    \draw[->](-1,0)--(5,0)node[below]{$x$};
    \draw[->](0,-1)--(0,4)node[left]{$y$};
    \node[below left]{$O$};
\draw[domain=pi/2:pi*1.5, smooth, samples=100, very thick]plot(\x-.5, {sin(2*\x r)+2.5})node[below ]{$y=f(x)$};
\draw[dashed](.75*pi-.5,0)node[below]{$x_0$}--(.75*pi-.5, 1.5);
\node at (.75*pi-.5,0){\small (\quad)};
\draw[thick](.75*pi-1, 1.5)--(.75*pi, 1.5)node[right]{极小};
\end{tikzpicture}    
\captionof{figure}{}
\end{minipage}

\noindent
\begin{minipage}{.48\textwidth}
    \centering
\begin{tikzpicture}[>=stealth]
    \draw[->](-1,0)--(5,0)node[below]{$x$};
    \draw[->](0,-1)--(0,4)node[left]{$y$};
\node[below left]{$O$};
\draw[domain=.25:2.85, smooth, samples=100, very thick]plot(\x+.5, {.6*(\x-.8)*(\x-1.5)*(\x-2.5)+2})node[above]{$y=f(x)$};
\draw[dashed](1.6,2.12)--(1.6,0)node[below]{$x_0$};
\node at (1.6,0){\small (\quad)};
\draw[thick](1.6-.5,2.12)--node[above]{极大}(1.6+.5,2.12);

\end{tikzpicture}    
\captionof{figure}{}
\end{minipage}\hfill
\begin{minipage}{.48\textwidth}
        \centering
\begin{tikzpicture}[>=stealth]
    \draw[->](-1,0)--(5,0)node[below]{$x$};
    \draw[->](0,-1)--(0,4)node[left]{$y$};
    \node[below left]{$O$};
\draw[very thick](.25,1)--(1.8,3);
\draw[very thick](1.8,3)arc (-170:-20:1.5);
\node at (1.8,0){\small (\quad)};
\draw[dashed](1.8,0)node[below]{$x_0$}--(1.8,3)node[above]{极大};
\node at (3.7,2.5){$y=f(x)$};
\end{tikzpicture}    
\captionof{figure}{}
\end{minipage}

在图12.6和图12.7中,$f(x_0)$的值小于点$x_0$的附近中的
其它函数值;在图12.8和图12.9中,$f(x_0)$的值大于点$x_0$的附近中的其它函数值.

\begin{thm}
{定义} 设函数$y=f(x)$在点$x_0$处连续,并且$x_0$不是其定义区间的端点,$x$是点$x_0$的附近$(x_0-\delta,x_0+\delta)$中异于$x_0$的任意一点.
\begin{enumerate}[(1)]
\item 若恒有$f(x)>f(x_0)$,则称函数$f(x)$在点$x_0$处有一极小值$f(x_0)$, $x_0$称为函数$f(x)$的一个极小值点.
\item 若恒有$f(x)<f(x_0)$,则称函数$f(x)$在点$x_0$处有一极小值$f(x_0)$, $x_0$称为函数$f(x)$的一个极大值点.
\end{enumerate}
函数的极大值和极小值可统称为函数的极值.
\end{thm}


从图12.7和图12.8可看出,若$f(x)$在点$x_0$处有极值,且$f(x)$在点$x_0$可导,则函数$y=f(x)$的图象在点$(x_0,f(x_0))$处的切线平行于$x$轴. 即$f'(x_0)=0$.(证明从略).

若$f(x)$可导,把$f'(x)=0$的解称为函数$f(x)$的驻点.

显然,可导函数的极值点一定是驻点. 由于函数$y=x^3$的驻点$x=0$既不是函数$y=x^3$的极大值点,也不是其极小值点,所以,可导函数的驻点不一定是函数的极值点.


\noindent
\begin{minipage}{.48\textwidth}
    \centering
\begin{tikzpicture}[>=stealth]
    \draw[->](-1,0)--(4,0)node[below]{$x$};
    \draw[->](0,-1)--(0,4)node[left]{$y$};
\node[below left]{$O$};

\draw[domain=-1.5:1.5, smooth, samples=100, very thick]plot(\x+2, {-.5*\x*\x+2.5});
\draw[dashed](2,2.5)--(2,0)node[below]{$x_0$};
\node at (2,0){\small (\quad)};
\node at (2,-.75){(极大值点)};
\draw[thick](2+1,2.5)--node[above]{$f'(x_0)=0$}(2-1,2.5);
\node at (1,2)[left]{$f'(x)>0$};
\node at (3,2)[right]{$f'(x)<0$};

\draw[->](1,1.5)--+(.5,.5);
\draw[->](2.5,2)--+(.5,-.5);

\end{tikzpicture}    
\captionof{figure}{}
\end{minipage}\hfill
\begin{minipage}{.48\textwidth}
        \centering
\begin{tikzpicture}[>=stealth]
    \draw[->](-1,0)--(5,0)node[below]{$x$};
    \draw[->](0,-1)--(0,4)node[left]{$y$};
\node[below left]{$O$};

\draw[domain=-1.5:1.5, smooth, samples=100, very thick]plot(\x+2, {.5*\x*\x+2});

\draw[dashed](2,2)--(2,0)node[below]{$x_0$};
\node at (2,0){\small (\quad)};

\node at (2,0){\small (\quad)};
\node at (2,-.75){(极小值点)};


\draw[thick](2+1,2)--node[below, fill=white]{$f'(x_0)=0$}(2-1,2);
\node at (1,2.5)[left]{$f'(x)<0$};
\node at (3,2.5)[right]{$f'(x)>0$};

\draw[->](1,2.85)--+(.5,-.5);
\draw[->](2.5,2.35)--+(.5,.5);


\end{tikzpicture}    
\captionof{figure}{}
\end{minipage}

由图12.10 可知,对于可导函数$y=f(x)$, $f'(x_0)=0$, 若当 $x\in(x_0-\delta,x_0)$时,$f'(x)>0$;当 $x\in(x_0,x_0+\delta)$时,$f'(x)<0$, 则函数 $f(x)$在点 $x_0$ 处有极大值 $f(x_0)$.

由图 12.11 可知,对于可导函数 $y=f(x)$, $f'(x_0)=0$, 若当$x\in(x_{0}-\delta,x_{0})$时,$f'(x)<0$;当$x\in(x_{0},x_{0}+\delta)$时,$f'(x)>0$, 则函数 $f(x)$在点 $x_0$ 处有极小值 $f(x_0)$.

为此,我们给出求可导函数 $y=f(x)$的极值的一种方法:
\begin{enumerate}
\item 求 $f(x)$的导数 $f'(x)$;
\item 求$f(x)$在定义域内的驻点即求方程 $f'(x)=0$ 的根;
\item 考查$f'(x)$在驻点$x_{0}$左右的符号:
\begin{enumerate}[(1)]
    \item 若“左正右负”, 则$f(x)$在点$x_0$处有极大值$f(x_0)$; 
    \item 若“左负右正”, 则$f(x)$在点$x_0$处有极小值$f(x_0)$;
    \item 若“左右同号”, 则 $f(x)$在点 $x_0$ 处无极值.
\end{enumerate}
\end{enumerate}

\begin{example}
求函数$f(x)=\frac{1}{3}x^{3}-4x+4$的极值.
\end{example}

\begin{solution}
$f(x)=\frac{1}{3}x^{3}-4x+4\quad f'(x)=x^{2}-4=(x-2)(x+2)$

令$f'(x)=0$, 得$x_{1}=2,\quad x_{2}=-2$.
\begin{center}
    \begin{tabular}{c|ccccc}
\hline
$x$&$(-\infty,-2)$& $-2$& $(-2,2)$& 2&$(2,+\infty)$\\
\hline
$f'(x)$ &  $+$  &  0  & $-$   &  0  & $+$ \\ 
$f(x)$ &  $\nearrow$  &  极大值  &  $\searrow$  &  极小值  &  $\nearrow$\\ 
\hline
    \end{tabular}
\end{center}
\[\begin{split}
    \therefore\quad  y_{\text{极大}}&=f(-2)=\frac{1}{3}\x (-2)^3-4\x(-2)+4=9\frac{1}{3}\\
    y_{\text{极小}}&=f(2)=\frac{1}{3}\x 2^3-4\x 2+4=-1\frac{1}{3}\\
\end{split}\]
\end{solution}

\begin{example}
    求函数$f(x)=x^3-3x^2+3x+2$的极值.
\end{example}

\begin{solution}
    $f(x)=x^3-3x^2+3x+2$

$\because\quad f'(x)=3x^2-6x+3=3(x-1)^2\ge 0$

$\therefore\quad x=1$虽是$f(x)$的驻点,但不是$f(x)$的极值点

$\therefore\quad $函数$f(x)=x^3-3x^2+3x+2$无极值.
\end{solution}



\begin{example}
    求函数$f(x)=-x^{\tfrac{2}{3}}$的极值.
\end{example}

\begin{solution}
    $f(x)=-x^{\tfrac{2}{3}}$

$\because\quad    f'(x)=-\frac{2}{3\sqrt[3]{x}}\ne 0\quad (x\ne 0)$

$\therefore\quad f(x)$在点$x=0$处不可导,当$x\ne 0$时,$f(x)$无驻点.

又$\because\quad f(x)$在点$x=0$处连续. 当$x<0$时,$f'(x)>0$;$x>0$时,$f'(x)<0$.

$\therefore\quad $当$x=0$时,$f(x)$有极大值0.
\end{solution}

\begin{rmk}
    在用求导的方法求$f(x)$的极值时,对于不可导点,要考查$f(x)$在该点附近的单调性及在这个点的连续性,以确定此点是否为$f(x)$的极值点.
\end{rmk}


\begin{ex}
\begin{enumerate}
    \item 求下列函数的极值:
\begin{multicols}{2}
\begin{enumerate}[(1)]
\item $y=2x^3-6x^2-18x+7$
\item $y=x-\ln(1+x)$
\item $y=x+\sqrt{1-x}$
\item $y=\frac{3x^2+4x+4}{x^2+x+1}$
\end{enumerate}
\end{multicols}

    \item 求函数$y=x^x\; (x\ge 0.1)$的极值.
\end{enumerate}
\end{ex}

\section{函数的最大值与最小值}
我们知道,函数的最大值与最小值又可统称为函数的最值,求最值的问题在实际中是很多的,例如,在一定条件下使“用料最省”、“容积最大”、“功率最大”、“运费最少”等.本节所学的最值问题,是在求函数极值(即函数在一个点的邻域内的最值)的基础上,研究函数在给定区间上的最值问题.

在12.4节中介绍连续函数的性质时,我们曾指出,定义在闭区间$[a,b]$上的连续函数一定有最大值和最小值. 我们学了函数的极值后,只要求出函数$f(x)$在区间$[a,b]$内的一切极值和函数在区间端点的值$f(a)$及$f(b)$,其中最大者就是函数的最大值,而其中最小者即为函数的最小值(如图12.12).

\begin{figure}[htp]
    \centering
\begin{tikzpicture}[>=stealth]
    





\end{tikzpicture}
    \caption{}
\end{figure}


\begin{example}
求函数$f(x)=x^3-3x^2-9x+2$在区间$[-2,5]$上的最大值和最小值.
\end{example}

\begin{analyze}
    确定函数的最大、最小值的关键是比较函数的极值和区间端点$f(x)$的值的大小.
\end{analyze}

\begin{solution}
$f(x)=x^3-3x^2-9x+2,\qquad 
f'(x)=3x^2-6x-9=3(x^2-2x-3)$

令$f'(x)=0$,即$x^2-2x-3=0$

解这个二次方程,得$f(x)$的驻点为
$x_1=-1,\quad x_2=3$

\[\begin{split}
    \because\quad f(-1)&=-1-3+9+2=7\qquad \text{($x=-1$是极大值点)}\\
f(3)&=27-27-27+2=-25\qquad  \text{($x=3$是极小值点)}\\
f(-2)&=-8-12+18+2=0\\
f(5)&=125-75-45+2=7
\end{split}\]

$\therefore\quad $在区间$[-2,5]$上函数$f(x)$的最大值为7,最小值为$-25$.
\end{solution}

\begin{example}
    在测量某物理量的过程中,因仪器和观察的误差, 使得$n$次 测 量 分 别 得 到$x_1, x_2, x_3, x_4,\ldots , x_n$共$n$个数据,试求量$x$, 使得它与这$n$个值的差的平方和最小。
\end{example}

\begin{solution}
按题意,设
\[\begin{split}
    f(x)&=(x-x_{1})^{2}+(x-x_{2})^{2}+\cdots+(x-x_{n})^{2}\\
   f'(x)&=2(x-x_{1})+2(x-x_{2})+\cdots+2(x-x_{n})=2[nx-(x_{1}+x_{2}+\cdots+x_{n})]
\end{split}\]

令 $f'(x)=0$, 解得 $f(x)$的一个驻点为
$x=\frac{x_{1}+x_{2}+\cdots+x_{n}}{n}$

又$\because\quad$ 当$x< \frac {x_{1}+ x_{2}+ \cdots + x_{n}}n$时,$f'(x)<0$;当
$x>\frac{x_1+x_2+\cdots+x_n}n$时,$f'(x)>0$

$\therefore\quad x= \frac {x_{1}+ x_{2}+ \cdots + x_{n}}n$既为 $f(x)$极小值点又为 $f(x)$
的 最 小 值 点 . 故  $x= \frac {x_{1}+ x_{2}+ \cdots + x_{n}}n$为所求.
  \end{solution}

\begin{rmk}
    若由问题的实际情况可以断定:可导函数 $f(x)$在定义域开区间内存在最大值(或最小值),而且$f(x)$在此区间内的\textbf{驻点唯一},那么可立即断定这个驻点所对应的函数值为$f(x)$的最大值(或最小值). 这一点在解决某些实际问题时很有用.
\end{rmk}

\begin{example}
已 知 : 直 线 $y= 2x+ m\; ( m> 0)$ 与 抛 物 线 $y= - \frac 12x^{2} +6$ 交于 $A,B$ 两点, $P(2,4)$是抛物线上一定点,试求$\triangle PAB$面积的最大值.
\end{example}

\begin{solution}
$\begin{cases}
    y=2x+m& m>0\\
    y=-\frac{1}{2}x^{2}+6
\end{cases}$ 
消去$y$得
$$x^2+4x+2(m-6)=0$$

\[\begin{cases}
    \Delta =16-8(m-6)=8(8-m)>0\\
    m>0
\end{cases}\]
得  $0<m<8$

又$\because\quad | AB| = \sqrt {5}\left | x_{A}- x_{B}\right |=2\sqrt{10}\sqrt{8-m}$

$P$点到直线 $AB$ 的距离
$$d=\frac{|4-4+m|}{\sqrt{5}}=\frac{m}{\sqrt{5}}$$

$\therefore\quad \triangle PAB$的面积$S$ 为
$$S=\frac{1}{2}\cdot\frac{m}{\sqrt{5}}\cdot2\sqrt{10}\sqrt{8-m}=\sqrt{2(8m^{2}-m^{3})}$$
其中$0<m<8$

$$S^{\prime}=\frac{2(16m-3m^{2})}{2\sqrt{2(8m^{2}-m^{3})}}=\frac{m(16-3m)}{\sqrt{2(8m^{2}-m^{3})}}$$

    令$S'=0$,得$m=\frac{16}{3}$.

$\because\quad $在区间$(0,8)$内函数有唯一驻点$m=\frac{16}{3}$. 且$\triangle PAB$的面积$S$有最大值.

$\therefore\quad $当$m=\frac{16}{3}$时,$S_{\triangle ABC}$有最大值
\[S_{\max}= \frac{16}{3}\x \sqrt{2}\x\sqrt{8-\frac{16}{3}}=\frac{64\sqrt{3}}{9} \]
\end{solution}

\begin{example}
    从南到北的铁路干线经过甲、乙两城,两城之间的距离为150千米,某工厂位于乙城正西60千米,现要从甲城把货物运往工厂,在铁路上的运费为每千米4元,而沿公路上的运费为每千米6元,为了使货物从甲城运到工厂的运费最省,应该从铁路干线上的何处起修筑一条通往工厂的公路较为适宜?
\end{example}

\begin{solution}

\noindent
\begin{minipage}{.6\textwidth}
 \CTEXindent   如图12.13,设从铁路干线$S$处修一条通往工厂$F$处的公路($S$距乙城$x$千米),货物运费为$W$元,则
\[W=4(150-x)+6\sqrt{x^2+3600}\]
其中$0<x<150$
\[\begin{split}
    W'&=-4+\frac{6x}{\sqrt{x^2+3600}}\\
    &=\frac{6x-4\sqrt{x^2+3600}}{\sqrt{x^2+3600}}
\end{split}\]
\end{minipage}
\hfill
\begin{minipage}{.35\textwidth}
\centering
\begin{tikzpicture}[scale=.8]
\tkzDefPoints{0/0/A, 0/-5/B, 0/-2/S, -2.5/0/F}
\draw(B)node[right]{甲}--(A)node[right]{乙}--node[above]{$60$}(F)node[left]{$F$}--(S)node[right]{$S$};
\tkzMarkRightAngles[size=.2](B,A,F)
\node at  (0,-1)[right]{$x$};
\end{tikzpicture}
\captionof{figure}{}
\end{minipage}

令$W'=0$,则
\[\begin{split}
    3x&=2\sqrt{x^2+3600}\\
    5x^2&=4\x 3600\\
    x^2&=24^2\x 5
\end{split}\]

$\therefore\quad x=24\sqrt{5}\approx 53.666.$

$\because\quad $驻点唯一,且此问题有最小值.

$\therefore\quad x=24\sqrt{5}\approx 53.666$为运费函数的最小值点.    

答:可在铁路干线上,距乙城约53.666千米处修一条到工厂的公路,使运费最省.
\end{solution}

\begin{ex}
\begin{enumerate}
    \item 在本节(12.15节)中,例12.31、例12.32、例12.33的初等解法是什么?
    \item 用边长为60cm的正方形铁皮做一个无盖水箱,先在四角分别截去一个小正方形,然后把四边翻转$90^{\circ}$,再焊接而成. 问水箱底边的长应取多少,才能使水箱容积最大,最大容积是多少?
    \item 求函数$y=-3x^4+6x^2-1\; (-2\le x\le 2)$的最大值和最小值.
\end{enumerate}
\end{ex}

\section{函数的图象的凸性与拐点}

我们过去在作函数的图象时曾经遇到这样一个问题,同
是给定区间上的增(或减)函数,但函数的图象的几何性态却
不同,例如图 12.14 中的图象,曲线上升有四种情况:

一般地设函数 $y=f(x)$定义在区间$[a,b]$上,若对于任意
$x_{1},x_{2}\in[a,b]$且$x_1<x_{2}$,
\begin{enumerate}
    \item 若总有$f\left(\frac{x_{1}+x_{2}}{2}\right)>\frac{f(x_{1})+f(x_{2})}{2}$,
则称函数 $y=f(x)$的图象在区间$[a,b]$上凸;
\item 若总有$f\left(\frac{x_{1}+x_{2}}{2}\right)<\frac{f(x_{1})+f(x_{2})}{2}$,则称函数$y=
f(x)$的图象在区间$[a,b]$下凸(如图 12.15).
\end{enumerate}

\begin{figure}[htp]
    \centering
\begin{tikzpicture}[>=stealth]
\begin{scope}
\draw[->](-1,0)--(4,0)node[below]{$x$};
\draw[->](0,-1)--(0,4)node[left]{$y$};
\node[below left]{$O$};
\draw[domain=0:2, smooth, samples=100, very thick]plot(\x, {\x*\x})node[below right, text width=2cm, align=center]{$y=x^2$\\$(x>0)$};
\node at (2,-.75){(1)};

\end{scope}
\begin{scope}[xshift=6cm]
    \draw[->](-1,0)--(4,0)node[below]{$x$};
\draw[->](0,-1)--(0,4)node[left]{$y$};
\node[below left]{$O$};
\draw[domain=0:2, smooth, samples=100, very thick]plot({\x*\x},\x)node[above left]{$y=\sqrt{x}$};
\node at (2,-.75){(2)};
\end{scope}
\begin{scope}[yshift=-4cm, xshift=1.5cm]
    \draw[->](-2.5,0)--(2.5,0)node[below]{$x$};
    \draw[->](0,-2)node[below]{(3)}--(0,2)node[left]{$y$};
    \node[below right]{$O$};
\draw[domain=-pi/2:pi/2, smooth, very thick, samples=100]plot(\x, {sin(\x r)});
\foreach \x in {-pi/2, pi/2}
{
    \draw[dashed](\x,0)--(\x,\x*2/pi);
}
\node at (1.5,-1)[text width=3cm, align=center]{$y=\sin x$\\$\left(-\frac{\pi}{2}\le x\le \frac{\pi}{2}\right)$};
\end{scope}
\begin{scope}[yshift=-4cm, xshift=7.5cm]
    \draw[->](-2.5,0)--(2.5,0)node[below]{$x$};
    \draw[->](0,-2)node[below]{(4)}--(0,2)node[left]{$y$};
    \node[below right]{$O$};
    \draw[domain=-1.2:1.2, smooth, very thick, samples=100]plot(\x, {\x^3})node[below right]{$y=x^3$};
\end{scope}    
\end{tikzpicture}
    \caption{}
\end{figure}

\begin{figure}[htp]
    \centering
\begin{tikzpicture}[>=stealth]
\begin{scope}
    \draw[->](-1,0)--(4,0)node[below]{$x$};
\draw[->](0,-1)--(0,3.5)node[left]{$y$};
\node[below left]{$O$};
\draw[domain=.5:3.5, very thick, smooth]plot(\x, {-.23*(\x-3.5)^2+3});
\draw[dashed](.5,0)node[below]{$a$}--(.5,.93);
\draw[dashed](3.5,0)node[below]{$b$}--(3.5,3);
\draw[dashed](1,0)node[below]{$x_1$}--(1,1.563)node[above left]{$A$};
\draw[dashed](3,0)node[below]{$x_2$}--(3,2.943)node[above]{$B$};
\draw[thick](1,1.563)--(3,2.943);
\draw[thick](2,0)node[below]{$\frac{x_1+x_2}{2}$}--(2,2.48)node[above]{$C$};
\node at (2,2.25)[ right]{$D$};

\end{scope}
\begin{scope}[xshift=6cm]
    \draw[->](-1,0)--(4,0)node[below]{$x$};
\draw[->](0,-1)--(0,3.5)node[left]{$y$};
\node[below left]{$O$};
\draw[domain=.5:3.5, very thick, smooth]plot(\x, {.23*(\x-.5)^2+1});
\draw[dashed](.5,0)node[below]{$a$}--(.5,1);
\draw[dashed](3.5,0)node[below]{$b$}--(3.5,3.07);
\draw[dashed](1,0)node[below]{$x_1$}--(1,1.06)node[above left]{$A$};
\draw[dashed](3,0)node[below]{$x_2$}--(3,2.44)node[above]{$B$};
\draw[thick] (1,1.06)--(3,2.44);
\draw[thick](2,0)node[below]{$\frac{x_1+x_2}{2}$}--(2,1.75)node[above]{$C$};
\node at (2,1.52)[right]{$D$};
\end{scope}
\end{tikzpicture}
    \caption{}
\end{figure}

\begin{example}
 求证:函数$y=\log_a x\; (a>1)$的图象上凸.
\end{example}

\begin{proof}
任取$x_1,x_2\in (0,+\infty)$,且$x_1<x_2$.

$\because\quad $函数$y=\log_a x\; (a>1)$是增函数

$\therefore\quad \log_a\frac{x_1+x_2}{2}>\log_a \sqrt{x_1x_2}=\frac{\log_a x_1+\log_a x_2}{2}$

$\therefore\quad$函数$y=\log_a x\; (a>1)$的图象上凸.
\end{proof}

根据定义,证明曲线$y=f(x)$在给定区间内是否上凸(或下凸),一般不像例12.34这么容易,下面我们介绍一种如何利用导数来确定曲线$y=f(x)$凸性的方法.

对于在闭区间$[a,b]$上连续,在开区间内可导的函数$y=f(x)$,其图象如果是下凸的,从图12.16可以看出,曲线$y=f(x)$在点$(x,f(x))$处切线的上方(切点除外),并且随着$x$的增大,各点切线的斜率$f'(x)$也跟着增大. 又因为$f'(x)$的单调性可由$f'(x)$的导数,即$f''(x)$的
正、负来判定. 于是启发我们可用$f(x)$的二阶导数来判定函数$y=f(x)$图象的凸性.

\begin{figure}[htp]
    \centering
\begin{tikzpicture}[>=stealth]
    \draw[->](-1,0)--(5,0)node[below]{$x$};
    \draw[->](0,-1)--(0,3.5)node[left]{$y$};
    \node[below left]{$O$};
    \draw[domain=-1.5:2, very thick, smooth]plot(\x+2, {0.5*\x*\x+1});
\draw[dashed](.5,0)node[below]{$a$}--(.5, 2.125)node[above]{$A$};
\draw[dashed](1,0)node[below]{$x_1$}--(1, 1.5);
\draw[dashed](2.5,0)node[below]{$x_2$}--(2.5, 1.125);
\draw[dashed](3.5,0)node[below]{$x_3$}--(3.5, 2.125)node[left]{$y=f(x)$};
\draw[dashed](4,0)node[below]{$b$}--(4, 3)node[above]{$B$};

\draw[domain=1-.5:1+.5, smooth]plot(\x, {-\x+2.5});
\draw[domain=2.5-.75:2.5+.75, smooth]plot(\x, {0.5*\x-0.125});
\draw[domain=3.5-.75:3.5+.75, smooth]plot(\x, {1.5*\x-3.125});




\end{tikzpicture}
    \caption{}
\end{figure}

    

\begin{thm}
{定理} 设函数$y=f(x)$在开区间$(a,b)$内有二阶导数.
\begin{enumerate}[(1)]
\item 若对于任意的$x\in (a,b)$, $f''(x)>0$,则曲线$y=f(x)$在区间$(a,b)$内下凸;
\item 若对于任意的$x\in (a,b)$, $f''(x)<0$,则曲线$y=f(x)$在区间$(a,b)$内上凸.  
\end{enumerate}
\end{thm}

此定理只作如前的说明,它的证明留给读者上大学后完成.

\begin{example}
    求证:正弦曲线$y=\sin x$在$(0,\pi)$上是上凸.
\end{example}

\begin{proof}
$y=\sin x,\qquad y'=\cos x,\qquad y''=-\sin x$

$\because\quad $当$x\in(0,\pi)$时,$y''=-sinx<0$

$\therefore\quad $正弦曲线$y=\sin x$在区间$(0,\pi)$是上凸的.
\end{proof}

大家知道,对于函数$f(x)=\sin x\; (x\in (0,\pi))$, $x=\frac{\pi}{2}$是它的极大值点,$f'(x)$在点$\frac{\pi}{2}$附近的符号是“左正右负”与函数的图象上凸是一致的,因此,又可用$f''(x)$在该点值的正负来判定驻点是函数的极大值点还是极小值点. 下面给出一种利用二阶导数来判定函数极值的方法(证明留给读者).

\begin{thm}
    {定理} 设$f(x_0)=0$且$f''(x_0)$存在
\begin{enumerate}[(1)]
\item 若$f''(x_0)>0$,则$f(x)$在点$x_0$处有极小值$f(x_0)$;
\item 若$f''(x_0)<0$,则$f(x)$在点$x_0$处有极大值$f(x_0)$.
\end{enumerate}
\end{thm}


\begin{example}
    求函数$f(x)=\frac{1}{3}x^3-4x+4$的极值.
\end{example}

\begin{solution}
\[\begin{split}
    f(x)&=\frac{1}{3}x^3-4x+4\\
f'(x)&=x^2-4=(x-2)(x+2)\\
f''(x)&=2x
\end{split}\]
令$f'(x)=0$,解得$f(x)$的驻点为$x_1=2$、$x_2=-2$

$\because\quad f''(2)=4>0$

$\therefore\quad y_{\text{极小}}=f(2)=\frac{8}{3}-8+4=-\frac{4}{3}$

$\because\quad f''(-2)=-4<0$

$\therefore\quad y_{\text{极大}}=f(-2)=-\frac{8}{3}+8+4=\frac{28}{3}$.

对于函数$y=x^3$, 点$x=0$是曲线$y=x^3$由左半部分上凸转为右半部分下凸的分界点,并且$f'(0)=0$, $f''(0)=0$,像这样的点我们称它为曲线的拐点.
\end{solution}

\begin{thm}
    {定义} 设函数$y=f(x)$的图象上一点$P(x_0,f(x_0))$恰为曲线$y=f(x)$上凸与下凸部分的分界点,则称点$P(x_0, f(x_0))$为曲线$y=f(x)$的拐点,$x_0$称为函数$f(x)$的拐点.
\end{thm}



\begin{example}
    求函数$f(x)=\sin x$的拐点.
\end{example}

\begin{solution}
    \[f(x)=\sin x,\quad f'(x)=\cos x,\quad f''(x)=-\sin x\]
令$f''(x)=0$得$x=k\pi\; (k\in\Z)$

又$\because\quad f''(x)$在点$k\pi$的两侧异号

$\therefore\quad x=k\pi\; (k\in\Z)$为$f(x)=\sin x$的拐点.
\end{solution}

一般地,设函数$f(x)$在点$x_0$处的附近有二阶导数,并且:
\begin{enumerate}[(1)]
\item $f''(x_0)=0$;
\item $f''(x)$的值在点的两侧异号
\end{enumerate}
则点$P(x_0,f(x_0))$为函数$y=f(x)$图象的拐点.

类似于$f'(x_0)=0$是点$x_0$为$f(x)$的极值点的必要条件一样, $f''(x)=0$是点$x_0$为$f(x)$的拐点的必要条件,而不是充分条件. 例如,$f(x)=x^6$, $f''(x)=30x^4$,当$x\ne 0$时,$f''(x)>0$,故曲线$y=x^6$下凸,虽然$f''(0)=0$,但$x=0$不是函数的拐点,而是极小值点.

\begin{ex}
\begin{enumerate}
    \item 求曲线$y=\arctan x$的上凸、下凸区间及拐点.
    \item 求证函数$y=a^x$ ($a>0$且$a\ne 1$)的图象下凸.
    \item 求$f(x)=x^3-3x^2-9x+14$的极值及拐点.
\end{enumerate}
\end{ex}

\section{函数的图象绘制}
前面我们对函数的性质进行了进一步的研究,这就使我们能依据函数的性质,更准确地作出函数的图象. 今后在绘制
一个函数的图象之前,一般应依次对函数作如下分析,然后列表画图.
\begin{enumerate}
\item 确定函数的定义域和值域(可大致确定图象的范围);
\item 验证函数的奇偶性、周期性(可缩小研究函数的范围);
\item 找出函数图象的某些特殊点(例如图象与坐标轴的交点、间断点及其他特殊点);
\item 利用导数求出函数的单调区间、极值点和极值、图象的上凸或下凸区间、拐点等;
\item 考察函数的图象是否有渐近线(了解图象若无限延伸时,它的变化趋势).
\end{enumerate}





\begin{example}
    试作函数 $f(x)=\frac{x^{2}}{x+1}$的图象
\end{example}

\begin{solution}
    对于函数$f(x)=\frac{x^2}{x+1}$,
\begin{enumerate}[(1)]
    \item $x\in ( - \infty , - 1) \cup ( - 1, + \infty )$ 
    \item 当$x\in(-\infty,-1)$时,$f(x)<0$;当$x\in(-1,+\infty)$时$f(x)\geqslant0$;(当且仅当 $x=0$ 时,$f(x)=0$)
    \item $f'(x)=\frac{x^2+2x}{(x+1)^2}$

令 $f'\left(x\right)=0$, 解得驻点 $x_1=0,\quad x=-2$.
$$f''(x)=\frac{2(x+1)(x+1)^2-2(x+1)(x^2+2x)}{(x+1)^4}=\frac{2}{(x+1)^{3}}$$

$\because\quad f''( 0) > 0$

$\therefore\quad y_{\text{极 小 }}= f( 0) = 0$.

又$\because\quad f'' (-2)<0$

$\therefore\quad  y_{\text{极大}}=f(-2)=-4$.

\item $\because\quad$当$x\in ( - \infty , - 1)$时, $f''( x) < 0$,

$\therefore\quad $ 函数 $f(x)$的图象在区间$(-\infty,-1)$内上凸

$\because\quad $ 当$x\in(-1,+\infty)$时, $f''(x)>0$

$\therefore\quad$ 函数 $f(x)$的图象在区间$(-1,-\infty)$内下凸

\item $f(x)$的递增区间是$(-\infty,-2]$, $[0,+\infty)$; 递减区间
是$[-2,-1)$, $(-1,0]$

\item $\because\quad f( x) = \frac {x^{2}}{x+ 1}= \frac {( x^{2}- 1) + 1}{x+ 1}=(x-1)+\frac{1}{x+1}$

当 $x\to\infty$时,$f(x)\approx x-1$ 且 $f(x)\to\infty$

$\therefore\quad $ 当  $x\to \infty$时,曲线 $y=f(x)$上的点到直线 $y=x-1$
的距离$d\to0$.

又$\because\quad \Lim{x}{- 1^{+}}f( x) = + \infty,\quad \Lim{x}{- 1^{- }}f( x) = - \infty$,

$\therefore\quad $ 函数 $f(x)$的图象的渐近线方程是 $x=-1$ 和
$y=x-1$.

\item 由 以 上 分 析$f( x) \in ( - \infty , - 4) \cup [ 0, + \infty ) $.
\end{enumerate}
上述要点可汇总如下表:
\begin{center}\small
\begin{tabular}{cccccccc}
    \hline
$x$ & $(-\infty,-2)$ & $-2$ & $(-2,-1)$ & $-1$ & $(-1,0)$ & $0$ & $(0,+\infty)$\\
\hline
$f'(x)$   &    $+$     &  0       & $-$        &         & $-$        &   0      & $+$\\        
$f''(x)$   &   $-$      &     $-$    &    $-$     &         &      $+$   &   $+$      & $+$\\     
$f(x)$   & $-\infty\nearrow-4$& $-4$& $-4\searrow -\infty$&间断点&$+\infty\searrow 0$& 0& $0\nearrow +\infty$\\
&   &    极大值     &         &         &       &   极小值      &   \\
\hline
\end{tabular}
\end{center}
函数$f(x)=\frac{x^2}{x+1}$的图象如图12.17.
\begin{figure}[htp]
    \centering
\begin{tikzpicture}[>=stealth, scale=.6]
\draw[->](-5,0)--(5,0)node[below]{$x$};
\draw[->](0,-7.5)--(0,4)node[left]{$y$};
\draw[domain=-5:4, thick, smooth]plot(\x, \x-1)node[below right]{$y=x-1$};
\draw[thick](-1,-7)node[below]{$x=-1$}--(-1,4);
\draw[domain=-1+.23:3.5, smooth, very thick, samples=100]plot(\x, {\x-1+1/(\x+1)})node[above left]{$y=\frac{x^2}{x+1}$};
\draw[domain=-1-.23:-4.5, smooth, very thick, samples=100]plot(\x, {\x-1+1/(\x+1)});
\node [below left]{$O$};
\node at (0,-1)[right]{$-1$};
\node at (-1,0)[below left]{$-1$};
\node at (1,0)[below]{$1$};
\end{tikzpicture}
    \caption{}
\end{figure}

\begin{center}
\begin{tabular}{c|ccccc}
\hline
$x$ & $-3$ & $-\frac{3}{2}$ & $-\frac{1}{2}$  &  1 &2  \\[1.5ex]
\hline
$y$ & $-\frac{9}{2}$&$-\frac{9}{2}$&$\frac{1}{2}$&$\frac{1}{2}$&$\frac{4}{3}$\\[1.5ex]
\hline
\end{tabular}
\end{center}

\end{solution}

\begin{ex}
作下列函数的图象:
\begin{multicols}{2}
\begin{enumerate}
    \item $y=\frac{x}{x^2+1}$
    \item $y=x-\ln(x+1)$
\end{enumerate}
\end{multicols}
\end{ex}



\section*{习题五}
\begin{center}
    \bfseries A
\end{center}

\begin{enumerate}
    \item 求下列函数的单调区间:
\begin{multicols}{2}
\begin{enumerate}[(1)]
 \item $y=x^3-6x^2+9x-3$
\item $y= \sqrt [ 3] {x^2+ 1}$ 
\item $y=\frac{x}{x^{2}+1}$
\item $y=x-2\sin x\quad (0\leqslant x\leqslant 2\pi)$   
\end{enumerate}
\end{multicols}

\item    求下列函数的极值:
\begin{multicols}{2}
\begin{enumerate}[(1)]
\item $y=\frac{x}{2}+\frac{2}{x}$ 
\item $y=3x^4-4x^3$ 
\item $y=\frac{(x-2)(3-x)}{x^{2}}$
\item $y=e^x\sin x$    
\end{enumerate}
\end{multicols}

\item 求下列函数的最大值:
\begin{enumerate}[(1)]
    \item     $y= - x^{3}+ 3x^{2}+ 2\quad ( - 2\leqslant x\leqslant 3)$ 
    \item       $y=\frac{x-1}{x+1}$
    \item     $y= \frac {a^{2}}x+ \frac {b^{2}}{1- x}\quad ( 0< x< 1,\;  a, b\in \R^{+ })$ 
    \item       $y= \arctan \frac {1- x}{1+ x}\quad ( 0\leqslant x\leqslant 1)$
\end{enumerate}

\item 一面积为$8\x 5{\rm cm}^{2}$的长方形的纸板,在各角剪去相同的小正方形,把四边折起成一个无盖盒子,要使纸盒的容积最大,则剪去的小正方形的边长应多长?
\item 设抛物线 $y=2x-x^2$ 在$x$轴上方的部分与直线 $y=t$
($t>0$)交于两点$P$、$Q$,求以$OP$、$OQ$为邻边的平行四边形$POQR$的面积$S$的最大值.
\item 在高15cm,底面半径6cm的直圆锥内作内接的直圆柱(其底面在直圆锥的底面上),求:
\begin{multicols}{2}
\begin{enumerate}[(1)]
    \item 表面积的最大值;
    \item 体积的最大值.
\end{enumerate}
\end{multicols}
\item 半径为$R$、圆心角为$\alpha$弧度的扇形,卷成一个圆锥,求$\alpha$为多大时,圆锥的体积最大.
\item 求下列函数的图象的上凸或下凸区间及函数的拐点.
\begin{multicols}{2}
    \begin{enumerate}[(1)]
        \item $y=\sqrt{x^2+1}$
        \item $y=x+\sin x$
    \end{enumerate}
\end{multicols}

\item 作下列函数的图象:
\begin{multicols}{2}
    \begin{enumerate}[(1)]
        \item $y=\frac{1}{1-x^2}$
        \item $y=\frac{x}{x^2-1}$
    \end{enumerate}
\end{multicols}
\end{enumerate}

\begin{center}
    \bfseries B
\end{center}

\begin{enumerate}\setcounter{enumi}{9}
    \item 若函数$f(x)=x^3+3kx^2-kx+2$既无极大值也无极小值,求实数$k$的取值范围.
    \item   求函数$f(x)=|x|(x^2-3x)$在区间$(-1,4)$的最值.
    \item   证明下列各不等式.
\begin{enumerate}[(1)]
    \item $\cos x>1-\frac{x^2}{2}\qquad (x>0)$
    \item $x-\frac{x^2}{2}<\ln(1+x)<x\qquad (x>0)$
\end{enumerate}
\end{enumerate}

\section{本章小结}
\subsection{主要内容}
本章主要内容是导数与微分的概念、求导数和求微分的方法以及如何利用导数和微分研究函数.

\subsection{导数}
导数概念是微积分学的基本概念之一,函数$y=f(x)$的导数$f'(x)$就是
\[\lim_{\Delta x\to 0}\frac{\Delta y}{\Delta x}=\lim_{\Delta x\to 0}\frac{f(x+\Delta x)-f(x)}{\Delta x}=f'(x)\]
$f'(x)$表示$f(x)$在点$x$处对自变量的变化率,它的几何意义是曲线$y=f(x)$在点$(x,f(x))$处的切线的斜率.

\subsection{微分}
函数$y=f(x)$的微分$\dd y$就是$f'(x)\dd x$. 即$\dd y=f'(x)\dd x\; (\dd x=\Delta x\ne 0)$. 在这个定义下,$\frac{\dd y}{\dd x}=f'(x)$.

当$|\Delta x|$很小时,用$\dd y$替代$\Delta y$在近似计算中有广泛的应用.

利用近似公式$f(x)\approx f(x_0)+f'(x_0)(x-x_0)$推出的常用近似公式应掌握.

\subsection{求函数的导数或微分}
求函数的导数或微分的方法叫做微分法,除掌握根据定
义求导数的基本方法外,要熟练掌握求导的四则运算法则以及复合函数、隐函数的求导法则.

本章中求出的各基本初等函数的导数(或微分)公式是进行微分运算的依据,必须熟练掌握.

\subsection{导数的应用}
作为导数的应用,在本章中我们利用一阶、二阶导数讨论了函数的单调性、极值、最值、拐点及函数的图象的凸性等.

利用导数研究函数性质,使我们能够较为准确地作出函数的图象.

\section*{复习题十二}
\begin{center}
    \bfseries A
\end{center}
\begin{enumerate}
    \item 回答下列问题:
\begin{enumerate}[(1)]
\item 函数$f(x)$在点$x_0$处的导数是怎样定义的?$f'(x)$与$f'(x_0)$有什么不同?
\item 函数$f(x)$在点$x_0$处的导数的几何意义是什么?
\item $f(x)$在点$x_0$处可导与$f(x)$在点$x_0$处连续,它们之间有什么关系?
\item 函数$f(x)$在点$x_0$处的微分是怎样定义的?怎样求这个微分?
\item 写出各基本初等函数的微分公式.
\item 写出函数的四则运算的求导法则.
\item 求复合函数的导数时应注意什么?
\item 怎样求隐函数的导数?
\item 怎样确定函数的极值点和单调区间?
\item 怎样求函数的最值?
\item 怎样确定函数的图象的凹凸区间及拐点?
\end{enumerate}

    \item 函数$f(x)=|x^2-3x|$在$x=3$处是否可导?
    \item 若函数$f(x)$在$x=a$处$f'(a)$存在,求\[\lim_{h\to 0}\frac{f(a+h)-f(a-h)}{2h}\]
    \item 求下列函数的导数$\frac{\dd y}{\dd x}$.
\begin{multicols}{2}
 \begin{enumerate}[(1)]
    \item $y=(x^3-2)(x+2)$
    \item $y=(x-2)(x+1)(2x+1)$
    \item $y=(x^2+x+1)^4$
    \item $y=\frac{2x^2+3}{x(x^2+1)}$
    \item $y=\sin^2(2x^2+3x)$
    \item $y=\sqrt[3]{x^4-3x^3+2}$
    \item $y=x^{5x}$
    \item $y=\arctan(e^{3x})$
\end{enumerate}   
\end{multicols}

\item 求下列函数的二阶导数
\begin{multicols}{2}
\begin{enumerate}[(1)]
    \item $y=\frac{1}{x^3+1}$
    \item $y=e^{\sqrt{x}}$
    \item $y=\ln\sin x$
    \item $y=\tan x$
\end{enumerate}
\end{multicols}
\item 已知$y=\frac{x-3}{x+4}$,求证$2(y')^2=(y-1)y''$
\item 已知$y=e^{\sqrt{x}}+e^{-\sqrt{x}}$,求证$xy''+\frac{1}{2}y'-\frac{1}{4}y=0$

\item \begin{enumerate}[(1)]
    \item 已知$\begin{cases}
        x=\sin t\\ y=\cos 2t
    \end{cases}$,求$\frac{\dd y}{\dd x}$
    \item 求曲线$\begin{cases}
        x=\sin t\\y=\cos 2t
    \end{cases}$在$t=\frac{\pi}{2}$处的切线方程.
\end{enumerate}

\item 求曲线$\sqrt{x}+\sqrt{y}=3$在点$(1,4)$处的切线方程.
\item 设在抛物线$y=x^2+1$上的任意一点$P$的切线和抛物线
$y=x^2$交于$R$、$Q$两点,求证$P$为线段$RQ$的中点.
\item 求下列函数的单调区间和极值.
\begin{multicols}{2}
\begin{enumerate}[(1)]
    \item $y=(x-2)^5(2x+1)^4$
    \item $y=\sqrt[3]{(2x-3)(x-3)^2}$
    \item $y=x+\cos x$
    \item $y=\frac{3x+1}{\sqrt{5x^2+4}}$
\end{enumerate}
\end{multicols}

\item 求曲线$y^2=4-2x$上距原点最近的点$P(x,y)$.
\item 将边长为$a$的正三角形的铁板的三个角如图那样挖去全等的四边形以后,将边折起做成无盖的三棱柱状的盒子,欲使其体积最大,问图中的$x$应为多少?

\noindent
\begin{minipage}{.45\textwidth}
    \centering
\begin{tikzpicture}
\tkzDefPoints{0/1/A', 0/2/A}
\tkzDefPoint(-30:1){C'}
\tkzDefPoint(-30:2){C}
\tkzDefPoint(-150:1){B'}
\tkzDefPoint(-150:2){B}
\tkzDrawPolygon(A,B,C)
\tkzDrawPolygon(A',B',C')
\tkzLabelPoints[above](A)
\tkzLabelPoints[below left](B)
\tkzLabelPoints[below right](C)

\tkzDefPointBy[projection = onto A--B](A')  \tkzGetPoint{A1}
\tkzDefPointBy[projection = onto A--B](B')  \tkzGetPoint{B1}
\tkzDefPointBy[projection = onto C--B](C')  \tkzGetPoint{C1}
\tkzDefPointBy[projection = onto C--B](B')  \tkzGetPoint{B2}
\tkzDefPointBy[projection = onto A--C](A')  \tkzGetPoint{A2}
\tkzDefPointBy[projection = onto A--C](C')  \tkzGetPoint{C2}

\tkzMarkRightAngles[size=.1](A',A1,B B',B1,A B',B2,C C',C1,B C',C2,A A',A2,C)

\draw[pattern=north east lines](A)--(A1)--(A')--(A2)--cycle;
\draw[pattern=north east lines](B)--(B1)--(B')--(B2)--cycle;
\draw[pattern=north east lines](C)--(C1)--(C')--(C2)--cycle;
\draw[decorate, decoration={brace, amplitude=5pt}](C)--node[below=5pt]{$x$}(C1);
\draw[decorate, decoration={brace, amplitude=5pt}](C2)--node[right=3pt]{$x$}(C);



\end{tikzpicture}
\captionof*{figure}{第13题}
\end{minipage}\hfill
\begin{minipage}{.45\textwidth}
        \centering
\begin{tikzpicture}
\draw(0,0)node[below left]{$B$} rectangle (4,2.5);
\draw[pattern=north east lines](0,0) rectangle(.5,.5);
\draw[pattern=north east lines](0,2.5-.5) rectangle(.5,2.5);
\draw(.5,.5) rectangle (.5+1.5,.5+1.5);
\draw[pattern=north east lines](.5+1.5,.5+1.5) rectangle(4,2.5)node[above right]{$D$};
\draw[pattern=north east lines](.5+1.5,.5) rectangle(4,0)node[below right]{$C$};
\node at(0,2.5)[above left]{$A$};
\draw[dashed](2.5,.5)--(2.5,2);
\end{tikzpicture}
\captionof*{figure}{第14题}
\end{minipage}


\item 有一矩形的铁板$ABCD$,长与宽分别为16cm、10cm,若切去$A$、$B$为顶点的两个全等正方形,和以$C$、$D$为顶点的两个全等矩形(如图),用剩余
部分沿直线折起做成长方体,求此长方体的最大体积.
\item 作下列函数的图象:
\begin{multicols}{2}
\begin{enumerate}[(1)]
    \item $y=\left(\frac{1+x}{1-x}\right)^4$
    \item $y=x-\arctan x$
\end{enumerate}
\end{multicols}
\end{enumerate}










\backmatter

% \input{fig_tikz.tex}


\end{document}
