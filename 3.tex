\chapter{概率的初步知识}
概率论是一门从数量方面研究随机现象规律性的学科. 随着科学技术的迅猛发展,概率论有了严格的理论基础和丰硕的成果,并成为一门应用非常广泛的数学学科,它的理论和方法在自然科学、社会科学、工程技术、经济管理、工农业生产和国民经济的各个部门都得到了广泛的应用,并取得了很好的效果. 

本章仅就概率的初步知识作一介绍. 

\section{随机现象}
随机现象就是我们通常所说的偶然现象它在现实世界中大量地存在着. 它的对立面就是必然现象或确定性现象. 为了弄清随机现象的概念,我们先来看一看确定性现象的例子. 如纯水在一个大气压下加热到$100^{\circ}{\rm C}$必然沸腾;向上抛一重物必然下落;同性电荷必然相斥;异性电荷必然相吸等等都是确定性的现象. 所以确定性现象是在一定条件下,只有一种结果的现象. 即在一定条件下必然发生或必然不发生的现象. 

我们还经常遇到另一类现象.例如:在混有2件次品的10件产品中任意抽取1件,可能得到正品,也可能得到次品.抽取的结果不能在抽取以前确切地知道;又如向上抛掷一枚
硬币,落下后,可能是正面朝上,也可能是反面朝上,而抛掷的结果事先不能预言,这类现象的特点是:在一定的条件下,可能出现的结果不止一种,至于出现哪一种,事先又无法确定,我们把这类现象称为随机现象. 

随机现象具有不确定性,然而它还是有规律可循的,即它还有确定性的一面. 比如在相同条件下,多次重复抛掷同一枚硬币,就会发现“出现正面”的次数与抛掷次数的比接近$\frac{1}{2}$,又如掷骰子,可能出现1点,出现2点,……,出现6点.掷一次,不能预言出现的是几点,但多次重复掷时,就会发现它的规律,即出现$1,2,\ldots,6$各点的次数大约都是抛掷次数的$\frac{1}{6}$,也就是说还是有规律可循的,即有确定性的一面,而这种确定性(规律)就是我们要研究的内容. 

\section{随机试验和随机事件}

无论是随机现象或是确定性的现象,常常在人们所进行的试验或观察中呈现出来,我们把呈现随机现象的试验或观察叫做随机试验,简称试验. 在关于概率的理论中,总是通过研究随机试验来研究随机现象的,而所研究的随机试验具有以下特点:
\begin{enumerate}
\item 试验在相同的条件下,可以重复进行;
\item 每次试验的结果,具有多种可能性,并且能在试验之前就明确知道试验的所有可能结果;
\item 在每次试验之前,不能肯定这次试验将出现哪种结果,但可以肯定每次试验总是出现所有可能结果中的某一个. 
\end{enumerate}

我们所讨论的每一个随机试验中,试验的所有可能结果都应是明确知道的. 它的每一个结果就叫做一个样本点(或基本事件),全体样本点组成的集合叫做样本空间(或基本事件空间). 样本点常用$\omega$表示,样本空间常用$\Omega$表示. 这个集合的每一个子集叫做这个样本空间的一个随机事件,简称事件,用符号$A,B,C,\ldots$表示. 显然,基本事件(样本点)也是随机事件. 

我们看下面的例子:

\begin{example}
    抛一枚硬币,观察正、反面出现的情况.这个随机试验的所有可能结果有两个:正(抛得正面朝上),反(抛得反面朝上). 所以样本空间$=\{\text{正, 反}\}$,样本点有两个,一个是“正”,一个是“反”. 

这里,记$\omega_1=\text{正}$,$\omega_2=\text{反}$. 样本空间记$\Omega=\{\omega_1,\omega_2\}$. 
\end{example}
  
\begin{example}
    随机试验:“连续两次抛一枚硬币,观察它们正、反面出现的情况. ”写出这个随机试验的样本空间. 
\end{example}

\begin{solution}
    在这个随机试验中,所有的可能结果有4个:(正,正),(反,反),(反,正),(正,反). 
    
\[\therefore\quad \Omega=\{(\text{正,正}), (\text{反,反}), (\text{反,正}), (\text{正,反})\}\]
若记
\[\begin{split}
    \omega_1=(\text{正, 正})&\qquad \omega_2=(\text{反, 反})\\
    \omega_3=(\text{反, 正})&\qquad \omega_4=(\text{正, 反})\\
\end{split}\]
则样本空间也可抽象地记为
\[\Omega =\{\omega_1,\omega_2,\omega_3,\omega_4\}\]
\end{solution}

\begin{example}
    随机试验:“一个袋中装有两个红球和一个白球,从袋中任意取出两球,观察它们的颜色”. 试指出这个随机试验的样本点和样本空间. 
\end{example}

\begin{solution}
因为袋中共有3球,两红一白,每次取出两球,只关
  心这两球的颜色,而不用管是哪两个球,又两球同时被取出,所以不必考虑顺序. 故试验的所有可能结果只有两个,即有两个样本点. 若记$\omega_1=$“取出的两球都红球”,$\omega_2=$“取出的两球=一个红一个白”. 则样本空间$\Omega=\{\omega_1,\omega_2\}$. 
\end{solution}

\begin{example}
    从$1,2,3,4,5,6,7,8,9$这9个数字中任意取出两个数字. 
\begin{enumerate}[(1)]
\item 有几种取法?
\item 其中和是奇数的取法有几种?
\end{enumerate}
并说明在这种取法下“和是奇数”是否为随机事件?
\end{example}

\begin{solution}
\begin{enumerate}[(1)]
    \item 这是从9个不同的元素中每次取出2个不同元素的所有组合的种数,所以共有取法
    \[{\rm C}^2_9=\frac{9\x8}{2\x 1}=36\text{(种)}\]
    \item 和是奇数,两个加数须一奇,一偶.所以和是奇数的取法有
\[{\rm C}^1_5 \cdot {\rm C}^1_4=20\text{(种)}\]
\end{enumerate}


在这个问题里,一次试验就是从9个数中,任意取出两个数. 这种试验满足随机试验的三个特点:
\begin{enumerate}[(i)]
\item 试验在相同的条件下,可以重复进行;    
\item 每次试验(从9个不同的元素中取出2个不同的元素)具有多种可能性,并且在试验之前就明确知道试验的所有可能结果(36种取法);
\item 在每次取出2个数字之前,不能肯定它们的和是奇数还是偶数,但可以肯定,每一种取法的结果都在36种取法之内.
\end{enumerate}
所以“和是奇数”是一个随机事件. 
\end{solution}

特别,在一个随机试验中,每次试验一定发生的事情称为必然事件;每次试验中一定不会发生的事情称为不可能事件. 
例如在例10. 4的条件下,“和小于18的两个数”是必然事件,“和不小于18的两个数”是不可能事件,又如在前面提到的掷骰子试验中,“点数小于7”是必然事件,“点数不小于7”是不可能事件. 

应该看到,必然事件和不可能事件有着紧密的联系. 如果每一次试验中,某一结果必然发生(如上例中“点数小于7”),那么这一结果的反面(即“点数不小于7”)就一定不发生.不论必然事件、不可能事件,还是随机事件,都是相对于一定的试验条件而言的. 如果试验条件变了,事件的性质也会跟着发生变化. 例如在掷骰子的试验中,掷一颗骰子(条件)时,“点数小于7”是必然事件,掷两颗骰子(条件)时,“点数之和小于7”是随机事件,而掷十颗骰子(条件)时,“点数之和小于7”就是不可能事件了. 概率是研究随机事件的. 但是为使研究方便,我们把必然事件和不可能事件也看作随机事件(尽管它们不符合随机事件的含义),作为随机事件的极端情况. 

\section{随机事件的概率}
对于一个随机事件来说,虽然在一次试验中它是否发生,不能事先知道,但是如果大量地重复这一试验,就会发现,不同事件发生的可能性是有大小之分的. 这种可能性的大小,是事件本身固有的一种属性. 例如:掷一枚骰子,我们凭经验可以知道:\{出现偶数点\}与\{出现奇数点\}这两个事件发生的可能是相同的,而\{出现奇数点\}和\{出现3点\}这两个事件发生的可能性就不同,\{出现奇数点\}这个事件发生的可能性比\{出现三点\}发生的可能性要大. 为了定量地描述随机事件发生的
可能性的大小,我们先介绍频率的概念.

在相同的条件下,重复进行$n$次试验,若在$n$次试验中,事件$A$发生的次数(称为频数)为$\mu_A$,我们把比值$\mu_A/n$称为事件$A$在$n$次试验中发生的频率. 即
\[A_{\text{发生的频率}}=\frac{\text{频数}}{\text{试验次数}}=\frac{\mu_A}{n}\]

显然$0\le \frac{\mu_A}{n}\le 1$,它在一定程度上反映了事件$A$发生的可能性的大小.

我们再来看一下抛硬币的试验. 我们把前人的一些试验记录列成下表:
\begin{center}
\begin{tabular}{l|ccc}
\hline
    试验者 & 抛掷次数$n$ & 正面出现的次数$\mu_A$ & 频率$\mu_A/n$\\
\hline
棣莫根(de Morgan)&2048&1061&0.518\\
蒲丰(Buffon)&4040&2048&0.5069\\
皮尔逊(K. Pearson)&12000&6019&0.5016\\
皮尔逊(K. Pearson)&24000&12012&0.5005\\
维尼&30000&14994&0.4998\\
\hline
\end{tabular}
\end{center}

由上表可以看出,虽然一个事件的频率在一定程度上反映了这个事件发生可能性的大小,但它不是一个完全确定的数,因而无法用它来定量地描述这个事件发生的可能性的大小. 不过从上表还可以看到:同一事件“抛得正面出现”($A$)发生的频率虽然各不相同,但却都在一固定的数值0.5附近摆动,并且随着抛掷次数的增加,这种摆动的幅度越小,从而呈现出一定的稳定性,我们说事件A发生的频率稳定在0.5.于
是,0.5这个确定的数值就可以作为事件$A$发生可能性大小的一个客观的度量,我们称0.5这个数值为事件$A$的概率,记为$\Pr(A)=0.5$.

一般地,在相同条件下,重复进行$n$次试验,如果当$n$充分大时,事件$A$发生的频率$\mu_A/n$稳定在某一数值$P$附近摆动,而且一般说来,随着$n$的增大,这种摆动的幅度越来越小,则称数值$P$为事件$A$的概率,记作
\[\Pr(A)=P\]

由于对任何事件$A$,都有$0\le \frac{\mu_A}{n}\le 1$,所以$0\le \Pr(A)\le 1$. 显然,必然事件的概率是1,不可能事件的概率是0.

由定义知道,事件$A$的概率$\Pr(A)$,就是事件$A$在多次试验中,随着试验次数的增加,事件$A$的频率逐渐逼近,趋于稳定的那个数,它是事件$A$发生的可能性大小的定量的客观的描述. 这个定义直观地说明了概率的来源,但无法用这个定义来直接计算概率$\Pr(A)$. 实际上,人们采用一次大量实验的频率或一系列频率的平均值作为$\Pr(A)$的近似值(或估计值).


\section*{习题一}
\begin{center}
    \bfseries A
\end{center}

\begin{enumerate}
    \item 试举出两个随机现象的例子和其中的若干个随机事件,并且指出其中的必然事件和不可能事件.
    \item 随机试验:一个大箱子中装有5个型号相同的杯子,其中3个是一等品,2个是二等品,从中任取两个,观察它们是一等品,还是二等品,试写出它的样本空间$\Omega$.
    \item 指出下列事件是必然事件,不可能事件还是随机事件.
\begin{enumerate}[(1)]
\item 从54张扑克牌中任取一张,取出的恰好是方块5.
\item 从实数中任取两个数,取出的数$a$,$b$具有性质:
    $a\cdot b=b\cdot a$.
    \item 从三角形集合中任取一个三角形,这三角形的内角和大于$180^{\circ}$.
    \item 某电话总机在一分钟内接到5次呼唤. 
\end{enumerate}

    \item 对一批产品进行抽查,结果如下表所示.
\begin{enumerate}[(1)]
    \item 计算表中优等品的频率.
\item 写出优等品的频率接近的并在它附近摆动的那个常数.
\end{enumerate}

\begin{center}
\begin{tabular}{l|cccccc}
\hline
    抽取产品数$n$&50&    100&    200&    500&  1000
&    2000\\ \hline 
优等品数$\mu_A$&45&92&194&470&954&1902\\
优等品频率$\mu_A/n$\\
\hline
\end{tabular}
\end{center}

\end{enumerate}


\section{等可能性事件的概率}
随机试验的形式多种多样,内容可以千差万别,对此,我们可根据其特征的不同,来建立不同的数学模型,从而分类进行研究. 下面我们来研究最简单的一类随机试验,先看两个试验例子.
\begin{enumerate}[(1)]
\item 一盒灯泡一百个,要抽取一个检查灯泡的使用寿命,任意取一个,则一百个灯泡被抽取的机会相同.
\item 抛掷一枚匀称的硬币,可能出现正面与反面两种结果,显然这两种结果出现的可能性是相同的.
\end{enumerate}


这两个试验的共同特点是:
\begin{enumerate}[(1)]
\item 每次试验,只有有限个可能的试验结果,或说样本点总数为有限个.
\item 每次试验中,每个可能结果(即样本点)出现的可能性是相同的.
\end{enumerate}

凡是具有这两个特点的随机试验我们称之为古典的概率模型. 由于它只有有限个试验结果,且这些试验结果出现的可能性相同,所以也可以不通过重复试验,根据推理分析就能够求出某些事件的概率.

如上述例(1)中,从一百个灯泡中任取一个,由于一百个灯泡被抽取的机会相同.因此,可以认为,一个灯泡被抽取的可能性是$\frac{1}{100}$,就说一个灯泡被抽取的概率是$\frac{1}{100}$.

上例(2)中所说的掷硬币问题.因为掷一枚均匀的硬币,出现正面和出现反面这两种结果的可能性是相等的,所以可以认为,抛掷一次,正面出现的概率是$\frac{1}{2}$,反面出现的概率也是$\frac{1}{2}$,这样的分析和大量重复试验的结果是一致的.

又如,一个大箱子中装有5个型号相同的杯子,其中4个是一等品,1个是二等品,从中取出1个,取到一等品的概率是多少?

\begin{analyze}
    从箱子中任取一个,取到各个杯子的可能性是相同的,由于从5个杯子中任取1个,共有5种等可能的结果.又由于其中有4个一等品,从这5个杯子中取到一等品的结果有4种.因此可以认为取到一等品的概率是$\frac{4}{5}$.
\end{analyze}

这样,对这类随机试验我们有如下结论:

如果一次试验中共有$n$种等可能出现的结果,其中事件$A$包含的结果有$m$种,那么事件$A$的概率$\Pr(A)$是$\frac{m}{n}$.

也就是说,若随机试验属于古典概率模型,如果它的样本空间含有$n$个样本点,事件$A$含有$m$个样本点,则事件$A$的概率定义为
\begin{equation}
    \Pr(A)=\frac{m}{n}=\frac{\text{$A$包含的样本点数}}{\text{样本点总数}}\tag{1}
\end{equation}

这个定义与用频率来定义事件的概率是一致的. 如果我们进行大量重复试验,我们将会看到事件$A$发生的频率是稳定于$\frac{m}{n}$的.

\begin{example}
    在100个相同的球中,混有4个假货.现在从这100个球中任意取出一个球,这球是假货的概率是多少?
\end{example}

\begin{solution}
    从100个球中,任意取出1个,所以样本点的总数是100.每个球被取出的可能性是相同的,又因为100个球中混有4个假的,所以“从100个球中取出一个球是假货”这一事件包含有4个样本点,即$n=100$, $m=4$,所以所求的概率是
\[P=\frac{m}{n}=\frac{4}{100}=\frac{1}{25}\]
\end{solution}

\begin{example}
    在20件产品中,有15件一等品,5件二等品.从中任取4件,计算2件是一等品,2件是二等品的概率.
\end{example}

\begin{solution}
    由于是从20件中任取,每取4件都是等可能性的,所以样本点总数是${\rm C}_{20}^4$,记“任取4件,2件是一等品,2件是二等品”为事件$A$,则事件$A$含有的样本点数是${\rm C}_{15}^2\cdot {\rm C}_{5}^2$. 事件$A$的概率是
\[\Pr(A) = \frac{{\rm C}_{15}^2\cdot {\rm C}_{5}^2}{{\rm C}_{20}^4}=\frac{70}{323}\]

答:所求事件的概率是$\frac{70}{323}$.
\end{solution}

\section*{习题二}

\begin{center}
    \bfseries A
\end{center}

\begin{enumerate}
    \item 在45名同学中,有9名三好学生,从中任选一名学生,选到三好学生的概率是多少?
    \item 在10个乒乓球中,有8个是正品,2个是副品,从中任取两个,恰好都取到正品的概率是多少?
    \item 一个正方体的木块,各个侧面都涂有颜色,把它锯成1000个体积相同的小正方体.将这些小正方体充分地混合,求随机选取的一个小正方体恰有两个侧面涂有颜色的概率.在100张已编号的卡片(从1号到100号),从中任取1张,计算:
\begin{enumerate}[(1)]
    \item 卡片号是偶数的概率;
    \item 卡片号是7的倍数的概率.
\end{enumerate}

    \item  在7张卡片中,有4张负数卡片和3张正数卡片.从中任取2张作乘法计算,其积是负数的概率是多少?
    \item  某城市的电话号码由五个数字组成,每个数字可以是0到9这十个数字中的任一个,计算电话号码由五个不同数字组成的概率.
\end{enumerate}

\begin{center}
    \bfseries B
\end{center}

\begin{enumerate}\setcounter{enumi}{6}
    \item 4张同样的卡片上,分别写上数字$1,2,3,4$,从中任意抽出两张,求这两张卡片上两个数字是连续整数的概率.
    \item 号码锁有自左至右5个拨盘,每个拨盘上有1到9共9个数字,当这5个拨盘上的数字,组成某一个5位数(开锁号码)时,锁才能打开,如果不知道这个号码,求一次就能把锁打开的概率.
    \item 一枚骰子抛掷两次,求事件“两次掷得点数的和是8”的概率.
    \item 一枚硬币抛掷3次作为一次试验,试求
\begin{enumerate}[(1)]
    \item “恰有一次出现正面”的概率;
    \item “至少有一次出现正面”的概率.
\end{enumerate}

    \item $A$、$B$、$C$、$D$、$E$5个人站成一排,谁站在第几个位置是任意的. 计算:
\begin{enumerate}[(1)]
    \item $A$恰好站在正中间的概率;
    \item $B$、$C$两人恰好站在两端的概率.
\end{enumerate}

    \item 从数字1,2,3,4,5中任取3个,组成没有重复数字的三位数,计算:
\begin{enumerate}[(1)]
\item 这个三位数是5的倍数的概率;
\item 这个三位数是偶数的概率;
\item 这个三位数大于400的概率.
\end{enumerate}

\end{enumerate}

\section{事件间的关系}

在任何一个随机试验中,总有许多随机事件,其中有些比
较简单,有些就比较复杂. 它们之间又有着这样或那样的联系. 正确分析事件之间的联系,对计算事件的概率尤其是对复杂事件概率的计算是非常重要且关键的一步. 所以本节专门来研究事件间的一些关系,为进一步计算事件的概率作好准备.

\subsection{事件的包含关系}
设$A$、$B$为两个事件,若事件$A$发生时,事件$B$必发生,则称事件$B$包含事件$A$,或称事件$A$包含于事件$B$,记作$A\subseteq B$或$B\supseteq A$.

例如:连续两次抛一枚硬币,观察正、反面出现的情况. 我们知道在这个试验中,所有可能的结果有4个.它们是:$\omega_1=(\text{正}, \text{正})$,$\omega_2=(\text{正}, \text{反})$, $\omega_3=(\text{反}, \text{正})$, $\omega_4=(\text{反}, \text{反})$.

若令事件$A$为(两次出现正面)则$A=\{\omega_1\}$

令事件$B$为(第一次出现正面),则$B=\{\omega_1,\omega_2\}$. 这时,事件$A$发生时,事件$B$必然发生.

$\therefore\quad A\subseteq B$(或$B\supseteq A$)

\subsection{事件的等价关系}
若$A\subseteq B$且$B\subseteq A$(或说事件$A$发生,当且仅当事件$B$发生)则称事件$A$和事件$B$相等. 记作$A=B$.

在上例中令事件$A=$(两次抛掷,至少一次是正面),令事件$B=$(两次抛掷,出现反面不多于一次)则
\[A=\{\omega_1, \omega_2, \omega_3\},\qquad B=\{\omega_1, \omega_2, \omega_3\}\]

$\therefore\quad A=B$

\subsection{两事件的和}

例如:甲、乙两只猫去捉同一只老鼠,如果用$A$表示事件“猫甲捉到老鼠”,用$B$表示事件“猫乙捉到老鼠”,用$C$表示事件“猫捉到老鼠”.我们来看事件$C$如何用事件$A$和事件$B$ 表示.

事件$C$ 表示“猫捉到老鼠”,可能是猫甲捉到的,也可能是猫乙捉到的,或者,甲乙两猫同时捉到老鼠,可见事件$C$ 发生就是事件$A$发生或事件$B$发生. 记作$C=A+B$(或$A\cup B$).

我们把事件$C$叫做事件$A$与事件$B$的和. 它含有下面 3
部分内容.
\begin{enumerate}[(1)]
\item 事件$A$发生而$B$不发生;
\item 事件$B$发生而$A$不发生;
\item 事件$A,B$同时发生。  
\end{enumerate}

$A+B$也可以说成是:“事件$A$与事件$B$至少有一个发
生”.

事件的和可以推广到$n$个事件的情形:

如果事件$A_1,A_2,\ldots,A_n$至少有一个发生,这一事件记作
$B$, 则
$$B=A_{1}+A_{2}+\cdots+A_{n}$$

例如:连续两次抛一枚硬币的试验中,
\[\begin{split}
    A_{1}&=\{\omega_{1}\}\quad (\text{其中}\omega_1=(\text{正},\text{正}))\\ 
    A_{2}&=\{\omega_{2}\}\quad (\text{其中}\omega_2=(\text{正},\text{反}))\\ 
    A_{3}&=\{\omega_{3}\}\quad (\text{其中}\omega_3=(\text{反},\text{正}))\\ 
    A_{4}&=\{\omega_{4}\}\quad (\text{其中}\omega_4=(\text{反},\text{反}))\\ 
\end{split}\]
在这试验中“有正面向上”这一事件若用$B$表示. 则
$$B=A_{1}+A_{2}+A_{3}=\{\omega_{1},\omega_{2},\omega_{3}\}$$

\subsection{两事件的差}

设$A,B$为两个事件,$A$发生而$B$不发生也是一个事件,
称做事件$A$与事件$B$的差. 记作$A-B$.

在前面猫捉老鼠的例子中,$C-B$表示事件:猫甲提到老
鼠而且猫乙没有捉到老鼠”.

\subsection{事件的积}

设$A,B$为两个事件, $A,B$同时发生也是一个事件,称做
事件$A$与事件$B$的积(或交)记作$A\cdot B$或$AB$.

例:在射击比赛时,有一射手连续向一目标射击 2次. 我们把“第一次射击命中目标”叫做事件$A_1$, “第二次击中目标” 记作$A_2$,把“两次都击中目标”叫做事件$B$. 则
$$B=A_{1}\cdot A_{2}$$

\subsection{事件的逆}

“事件$A$不发生”也是一个事件,称为事件$A$的逆.(又称
做$A$的对立事件)记作$\overline{A}$.

例如:在上面射击比赛的例子中,“第二次射击命中目标” 叫做事件$A_2$. 若把事件“第二次射击未击中目标”叫做事件$C$,则 $C=\overline{A}_{2}$, 而事件 $A_1\overline{A_2}$ 表示事件“第一次射击命中目标, 第二次射击未命中目标.”

\subsection{互不相容}

在一次试验中,如果事件$A$和事件$B$不能同时发生(或
者说$AB$是不可能事件),则称事件$A$和事件$B$互不相容,记为$AB=\emptyset$. 这时,事件$A$和事件$B$也叫做互斥事件.

若事件$A$、$B$是互斥事件,$B$、$C$是互斥事件,$A$、$C$是互斥事件,换句话说,事件$A$,$B$,$C$中,任何两个都是互斥事件,这时我们说事件$A$,$B$,$C$彼此互斥(或互不相容). 一般地,如果事件$A_1,A_2,\ldots,A_n$中任何两个都是互斥事件,那么就说事件$A_1,A_2,\ldots,A_n$ 彼此互斥(或互不相容).

\section*{习题三}
\begin{center}
\bfseries A
\end{center}

\begin{enumerate}
    \item 一种圆柱形产品,只有当产品的长度和直径都合格时才算正品,否则就为次品,如果用$A_1$表示事件“长度合格”;$A_2$表示事件“直径合格”,试用$A_1,A_2$表示下列事件:
\begin{multicols}{2}
  \begin{enumerate}[(1)]
\item “产品为正品”.
\item “产品为次品”.
\end{enumerate}  
\end{multicols}

\item     10件产品中有7件正品3件次品,从中任意取出6件产品,若用$A$表示事件“取出的6件产品中至少有一件次品”;用$B$表示“取出的6件产品中次品不少于两件”.那么$\overline{A}$,$\overline{B}$各表示什么事件?
  \item   一枚硬币投掷两次,令$A_i=\text{“第i次正面朝上”}\; (i=1,2)$,试用$A_i\; (i=1,2)$表示下列事件:
\begin{enumerate}[(1)]
\item “两次都正面朝上”;
\item “至少有一次正面朝上”;
\item “至多有一次正面朝上”
\end{enumerate}

\item 制造某一零件需经过三道工序加工,只有当三道工序加工
均合格时,此零件才算正品,否则就为次品. 若用$A_i$表示事件“第$i$道工序加工合格” $(i=1,2,3)$. 
\begin{enumerate}[(1)]
    \item 试用$A_i\; (i=1,2,3)$表示“零件是次品”;
    \item 叙述事件$A_1A_2A_3$的含义;
    \item 说明在什么条件下$A_1A_2\subseteq A_3$.
\end{enumerate}

\item 某仪器由三个元件组成,用$A_i\; (i=1,2,3)$表示事件“第$i$个元件合格”,试用$A_i\; (i=1,2,2,3)$表示下列事件:
\begin{enumerate}[(1)]
\item “仪器合格”;
\item “仪器至多有一个元件不合格”;
\item “仪器仅有一个元件合格”;
\item “仪器至少有一个元件不合格”.
\end{enumerate}

\item 在某校高二年级中任选一名学生去参加一个会议.用$A$表示事件“被选学生是男生”,用$B$表示事件“被选学生是三好学生”,用$C$表示事件“被选学生是运动员”.
\begin{enumerate}[(1)]
    \item 叙述事件$AB\overline{C}$的意义;
    \item $ABC=C$在什么条件下成立?
    \item 什么时候关系式$C\subseteq B$是正确的?
    \item 什么时候$\overline{A}=B$成立?
\end{enumerate}

\item 设$A$,$B$,$C$是某个随机试验中的三个事件,试将下列事件用上面三个事件表示出来:
\begin{enumerate}[(1)]
\item 事件$A$发生;
\item 恰好事件$A$发生;
\item 事件$A$和$B$都发生而事件$C$不发生;
\item $A$,$B$,$C$三个事件都发生;
\item $A$,$B$,$C$三个事件中至少有一个事件发生;
\item $A$,$B$,$C$三个事件中至少有两个事件发生;
\item $A$,$B$,$C$三个事件中恰有一个事件发生;
\item $A$,$B$,$C$三个事件中恰有两个事件发生;
\item $A$,$B$,$C$三个事件中不多于一个事件发生;
\item $A$,$B$,$C$三个事件中不多于两个事件发生;
\item $A$,$B$,$C$三个事件都不发生.
\end{enumerate}
\end{enumerate}

\section{互斥事件有一个发生的概率}
\begin{example}
    100件产品中,有75件一等品,20件二等品,5件三等品.从其中任取1件,这件是一等品或二等品的概率是多少?
\end{example}

\begin{solution}
从100件产品中,任取一个产品有100种取法.即样本点总数为100.

从100件产品中,任取一个产品,这产品是一等品叫做事件$A$,因为有75件一等品,所以事件$A$含有的样本点数是75.

同理,把从100件产品中,任取一个产品,这产品是二等品叫做事件$B$,是三等产品叫做事件$C$,所以事件$A$、$B$、$C$彼此互斥,且事件B含有样本点数是20,事件C含有样本点的数是5.
因此,
\[\Pr(A)=\frac{75}{100},\qquad \Pr(B)=\frac{20}{100},\qquad \Pr(C)=\frac{5}{100}\]

事件“任意取出一件产品,该产品是一等品或二等品,是事件$A$与事件$B$的和:$A+B$.我们知道,从100件产品中,任意取出一件产品,这件产品是一等品或二等品”这个事件包含的样本点数是$75+20$. 所以$\Pr(A+B)=\frac{75+20}{100}=0.95$
\end{solution}

由$\frac{75+20}{100}=\frac{75}{100}+\frac{20}{100}$,我们看到:
\begin{equation}
\Pr(A+B)=\Pr(A)+\Pr(B) \tag{2}
\end{equation}

一般地,如果事件$A$,$B$互斥,那么事件“$A+B$”发生的概率,等于事件$A$和事件$B$分别发生的概率的和.

推广到$n$个彼此互斥事件的情形就是:

如果事件$A_1,A_2,\ldots,A_n$彼此互斥,那么事件“$A_1+A_2+\cdots+A_n$”发生的概率,等于这$n$个事件分别发生的概率的和,即
\begin{equation}
    \Pr(A_1+A_2+\cdots+A_n)=\Pr(A_1)+\Pr(A_2)+\cdots +\Pr(A_n)\tag{2$'$}
\end{equation}

\begin{example}
    在10件产品中,有7件一等品,3件二等品,从其中任取3件,至少有1件为一等品的概率是多少?
\end{example}

\begin{solution}
    把从10件产品中任取3件,其中恰有1件一等品记为事件$A_1$,恰有2件一等品记为事件$A_2$,恰有3件一等品记为事件$A_3$.则事件$A_1$, $A_2$, $A_3$的概率分别是:
\[\begin{split}
 \Pr(A_1)&=\frac{{\rm C}_7^1{\rm C}_3^2}{{\rm C}_{10}^3}=\frac{21}{120}\\
\Pr(A_2)&=\frac{{\rm C}_7^2{\rm C}_3^1}{{\rm C}_{10}^3}=\frac{63}{120}\\
\Pr(A_3)&=\frac{{\rm C}_7^3}{{\rm C}_{10}^3}=\frac{35}{120}  
\end{split}\]

根据题意,事件$A_1$, $A_2$, $A_3$彼此互斥.由公式($2'$)3件产品中至少1件为一等品的概率是:
\[\Pr(A_1+A_2+A_3)=\Pr(A_1)+\Pr(A_2)+\Pr(A_3)=\frac{21}{120}+\frac{63}{120}+\frac{35}{120}=\frac{119}{120}\]
答:从10件中任取3件,至少有1件为一等品的概率是$\frac{119}{120}$.
\end{solution}


因为对立事件是互斥事件,由(2)得
\begin{equation}
  \Pr(A)+\Pr(\overline{A})=\Pr(A+\overline{A})=1  \tag{3}
\end{equation}
即两个对立事件概率的和是1.

(3)式还可写做:
\begin{equation}
    \Pr(\overline{A})=1-\Pr(A)\quad \text{或}\quad \Pr(A)=1-\Pr(\overline{A}) \tag{$3'$}
\end{equation}

上述例(2)中,“任取3件,至少有1件为一等品”记作事件$B$,这个事件的对立事件是:“任取3件,3件全不是一等品”(应记作事件$B$).

$\because\quad \Pr(\overline{B})=\frac{{\rm C}_3^3}{{\rm C}_{10}^3}=\frac{1}{120}$

$\therefore\quad \Pr(B)=1-\Pr(\overline{B})=1-\frac{1}{120}=\frac{119}{120}$

\begin{rmk}
    对于两个对立事件,利用公式($3'$),可以把求其中一个事件的概率,转化为求它的对立事件的概率,往往比较简单.
\end{rmk}

\section*{习题四}
\begin{center}
    \bfseries A
\end{center}
\begin{enumerate}
    \item 在某一时期内,一条河流在某处的年最高水位在各个范围内的概率如下:
\begin{center}
\begin{tabular}{c|ccccc}
    \hline
    年最高水位 &低于10米& 10~12米&12~14米&14~16米&不低于16米\\
    \hline
    概率& 0.1&0.28&0.38&0.16&0.08\\
    \hline
\end{tabular}
\end{center}
计算在同一时期内,河流在此处的年最高水位在下列范围内的概率:
\begin{multicols}{2}
\begin{enumerate}[(1)]
\item 10~16米;
\item 低于12米;
\item 不低于14米;
\item 不低于12米.
\end{enumerate}
\end{multicols}

\item 某射手射击由三个区域组成的目标,击中第一个区域的概率是0.51,击中第二个区域的概率是0.32,求该射手在一次射击中,击中第一个区域或第二个区域的概率.
\item 一个袋子中有红球5个,白球4个,从中任取2个球,至少有一个为红球的概率是多少?
\item 100件商品中混有5件伪劣商品,
\begin{enumerate}[(1)]
\item 从这100件商品中任意取出50件,其中没有一件伪劣商品的概率是多少?
\item 从全部商品中任意取出50件,其中恰有一件伪劣商品的概率是多少?
\item 从全部商品中任意取出50件,伪劣商品不多于1件的概率是多少?  
\end{enumerate}

\item 50个产品中有46个合格品与4个废品,从中一次抽取3个,求其中有废品的概率.
\item 一部五卷本的文集,按任意的次序放到书架上,求
\begin{enumerate}[(1)]
\item 第一卷在左边或在右边的概率;    \item 第一卷和第五卷都在边上的概率;    \item 第三卷正好在正中间位置的概率.
\end{enumerate}

\item 在1000张彩券中,中奖的有10张,问持有5张彩券的人中奖的概率是多少?
\item 在$n$张彩券中,中奖的有$m$($m<n$)张,问持有$k$($k<n$)张彩券的人中奖的概率是多少?
\end{enumerate}

\section{相互独立事件同时发生的概率}
先看这样一个问题.甲袋里装有6个红球4个白球,乙袋里装有3个红球5个白球.从这两个袋里分别任意取出一个球,这两个球都是红球的概率是多少?

因为从一个袋里取出的球的颜色,对从另一个袋子取出的球的颜色没有影响,所以当我们把“从甲袋里取出的一个球是红球”叫做事件$A$,“从乙袋里取出一个球是红球”叫做事件$B$时,事件$A$(或$B$)是否发生对事件$B$(或$A$)发生没有影响,我们就说这两个事件在概率意义下是互相独立的,这样的两个事件叫做相互独立事件.

在这个问题里,事件$\overline{A}$是指“从甲袋里取出一个球是白球”,事件$\overline{B}$是指“从乙袋里取出一个球是白球”. 显然,事件$A$与$\overline{B}$,事件$\overline{A}$与$B$,事件$\overline{A}$和$\overline{B}$也都是相互独立的. 一般地,当事件$A$与事件$B$相互独立时,$A$与$\overline{B}$,$\overline{A}$与$B$,$\overline{A}$与$\overline{B}$也都是相互独立的.

“从两个袋子里分别任意取出一个球,这两个球都是红球”是一个事件,就是事件$AB$. 这个问题就是要求$A$、$B$同时发生的概率$\Pr(A\cdot B)$.从甲袋里取出一个球,有10种等可能的结果,从乙袋里取出一个球,有8种等可能的结果,于是从两个袋里分别取出一个球共有$10\x8$种结果,其中同时是红球的结果有$6\x3$种,所以从两袋里各取出一个球,都是红球的概率$\Pr(A\cdot B)=\frac{6\x3}{10\x 8}=\frac{6}{10}\cdot \frac{3}{8}$.

另一方面,从甲袋里任意取出一个球为红球的概率是
$\Pr(A)=\frac{6}{10}$,从乙袋里任意取出一个球为红球的概率是$\Pr(B)=\frac{3}{8}$.

从这例子中,我们看到
\begin{equation}
    \Pr(A\cdot B)=\Pr(A)\cdot \Pr(B)\tag{4}
\end{equation}

对于一般情况,当事件$A$与事件$B$相互独立时,这个等式也是成立的. 这就是说,\textbf{两个相互独立事件同时发生的概率,等于每个事件发生的概率的积}.

这个规律可以推广到$n$个事件的情况,就是:

\textbf{如果事件$A_1,A_2,\ldots,A_n$两两相互独立,那么这几个事件同时发生的概率,等于每个事件发生的概率的积,即}
\begin{equation}
    \Pr(A_1\cdot A_2\cdots A_n)=\Pr(A_1)\cdot \Pr(A_2)\cdots\Pr(A_n)  \tag{$4'$}    
\end{equation}

\begin{example}
甲、乙两人定点投篮各投一次,如果两人一次投进的概率都是0.7,计算
\begin{enumerate}[(1)]
 \item 两人都投进的概率;  
 \item 恰有一人投进的概率;
\item 至少有一人投进的概率.    
\end{enumerate}
 
\end{example}

\begin{analyze}
(1)甲、乙两人各投一次,甲是否投进对乙是否投进没有影响.反过来也是一样,所以甲(或乙)是否投进对乙(或甲)投进的概率没有影响. 即“甲投篮一次投进”与“乙投篮一次投进”是相互独立事件,依据公式可求出两个事件同时发生的概率.    
\end{analyze}

\begin{solution}
(1)把“甲投篮一次,投进”记为事件$A$,“乙投篮一次,投进”记为事件$B$. 所以“两人各投篮一次,都投进”就是事件$A\cdot B$,由题意知,$A$,$B$是相互独立事件. 

$\therefore\quad \Pr(A\cdot B)=\Pr(A)\cdot \Pr(B)
=0.7\x0.7=0.49$.

答:两人各投篮一次,都投进的概率是0.49.
\end{solution}

\begin{analyze}
(2)“两人各投一次,恰有一人投进“包括两种情况:一种是甲投进,乙未投进,即事件$A\overline{B}$发生;另一种情况是甲末投进,乙投进,即事件$\overline{A}B$发生,根据题意,这两种情况在各投篮一次的情况下不可能同时发生,即事件$A\cdot \overline{B}$和$\overline{A}\cdot B$互斥,所以根据公式和公式(4)可求出“恰有一人投进”的概率.
\end{analyze}

\begin{solution}
(2)
\[\begin{split}
    \Pr(AB+AB)&=\Pr(AB)+\Pr(AB)\\
&=\Pr(A)\cdot \Pr(\overline{B})+\Pr(\overline{A})\cdot \Pr(B)\\
&=0.7\x(1-0.7)+(1-0.7)\x0.7\\
&=0.21+0.21=0.42.    
\end{split}\]
答:恰有一人投进的概率是0.42.
\end{solution}

\begin{analyze}
(3)设“甲、乙两人各投篮一次,至少有一人投进”为事件$C$. 则$C$是事件$A\cdot B$, $\overline{A}\cdot B$, $A\cdot \overline{B}$的和,而后面这三个事件又是互斥的,根据公式($2'$)可求$\Pr(C)$.

又“两人各投篮一次,至少有一人投进”与事件“两人各投篮一次,都未投进”是对立事件,即事件$C$和事件$\overline{A}\cdot \overline{B}$是对立事件,所以又可用公式($3'$)求$\Pr(C)$.

从而有下面两种解法.    
\end{analyze}

\begin{solution}
\textbf{解法一:}\[\begin{split}
    \Pr(C)&=\Pr(A\cdot  B+\overline{A}\cdot B+A\cdot\overline{B})\\
    &=\Pr(A\cdot B)+\Pr(\overline{A}\cdot B)+\Pr(A\cdot \overline{B})\\
    &=\Pr(A)\cdot \Pr(B)+\Pr(\overline{A})\cdot \Pr(B)+\Pr(A)\cdot \Pr(\overline{B})\\
&=0.7\times0.7+0.3\times0.7+0.7\times0.3\\
&=0.49+0.21+0.21 =0.91.
\end{split}\]
\textbf{解法二:}
\[\begin{split}
    \Pr(C)=1-\Pr(\overline{A}\cdot\overline{B})
    &=1-\Pr(\overline{A})\cdot \Pr(\overline{B})\\
    &=1-0.3\times0.3=1-0.09=0.91
\end{split}\]
答:至少有一人投进的概率是0.91.
\end{solution}

\begin{example}
    在一段线路中并联着3个自动控制的开关(图10.1),只要其中一个开关能够闭合,线路就能正常工作,假定在某段时间内每个开关能够闭合的概率都是0.7,计算在这段时间内线路正常工作的概率.
\end{example}

\begin{analyze}
“这段时间内线路正常工作”这一事件即 3个开关
至少有一个能够闭合.

设这 3 个开关为$J_{A1}$, $J_{A2}$, $J_{A3}.$
$J_{A1}$闭合为事件 $B_1$, $J_{A2}$闭合为事件 $B_2$, $J_{A3}$闭合为事件$B_{3}$, 则 $\overline{B_{1}}$, $\overline{B_{2}}$, $\overline{B_{3}}$ 分别为 $J_{A1}$不闭合, $J_{A2}$不闭合, $J_{A3}$不闭合. 令“这段时间内线路正常工作”为事件$B$.
则
$$B=B_{1}B_{2}B_{3}+B_{1}B_{2}\overline{B_{3}}+B_{1}\overline{B_{2}}B_{3}+\overline{B_{1}}B_{2}B_{3}+B_{1}\overline{B_{2}}\overline{B_{3}}+\overline{B_1}B_2\overline{B_3}+\overline{B_1}\overline{B_2}B_3$$
那么 $\Pr(B)$等于右端各事件概率的和(因为它们互斥)

这样作太麻烦了. 能否有其他作法呢?

我们知道“3个开关中至少有一个能够闭合”这一事件的
对立事件是“3个开关都不闭合”. 即$\overline{B_1}\overline{B_2}\overline{B_3}$.

$\therefore\quad \Pr( B) = 1- \Pr( \overline {B_{1}}) P( \overline {B_{2}}) P( \overline {B_{3}}) $

从而得解法如下:
\end{analyze}


\begin{solution}
记“这段时间内线路正常工作”为事件$B$, 开关 $J_{A1}$,
$J_{A2}$, $J_{A3}$能够闭合分别记为事件$B_1$, $B_2$, $B_3$. 则
\[\begin{split}
    \Pr(B)&=1-\Pr(\overline{B_1})\Pr(\overline{B_2})\Pr(\overline{B_3})\\
    &=1-[1-\Pr(B_1)][1-\Pr(B_2)][1-\Pr(B_3)]\\
    &=1-0.3\times0.3\times0.3=1-0.027=0.973.
\end{split}\]
答:在这段时间内线路正常工作的概率是 0.973.
\end{solution}


\section*{习题五}
\begin{center}
    \bfseries A
\end{center}

\begin{enumerate}
    \item 甲、乙二人生产合格产品的概率分别是0.8和0.9,从他们生产的产品中各抽取一件,都抽到合格品的概率是多少?
    \item 某射手射击一次,击中目标的概率是0.8,他连续射击3次,第一次未击中,其余2次都击中的概率是多少?
    \item 电器$K_1,K_2,K_3$并联在电路中,已知电流在$K_1$处断路(事件$A$)的概率是0.3,在$K_2$处断路(事件$B$)的概率为0.4,在$K_3$处断路(事件$C$)的概率为0.6.求全电路断路的概率.
    \item 将一枚硬币连续抛掷4次,4次都出现反面的概率是多少?
    \item 电器$K_1,K_2,\ldots,K_9$串联在电路中,每个电器使用3000小时的概率为0.99,计算这个电路能使用3000小时的概率.
    \item 有三箱产品,每箱100个,每箱有一个产品不合格,从三箱中各抽取一个,计算
\begin{enumerate}[(1)]
    \item 3个中恰有一个不合格的概率;
    \item 3个中至少有一个不合格的概率.
\end{enumerate}
\item 四门大炮中,每门大炮在一次射击中命中目标的概率是0.7.求四门大炮在一次射击中,至少有一门炮命中目标的概率.
\end{enumerate}

\section{独立重复试验}
在实际问题中,常常会遇到这样的随机试验:它是一系列重复试验,其中每次试验结果与其它次试验结果无关,并且每次试验只有两个结果:一个结果的概率总是$P$,另一个的概率总是$1-P$.例如进行一系列射击,每次射击只有两个结果——命中目标与没有命中目标. 在射击条件不变时,可以认为命中目标的概率总是$P$,不命中目标的概率总是$1-P$.下面,我们就来研究这类试验(即通常所说的贝努利型).

\begin{example}
某射手向一目标连续射击3次,已知每次命中目标的概率都是0.8,试求恰好命中两次的概率.    
\end{example}

\begin{solution}
这个试验相当于下列试验进行 3次:“射手向目标射击一次,观察目标命中与否”. 这个试验只有两个可能的结果: $A$( “ 命 中 目 标 ” ) 和$\overline {A}$( “ 未 命 中 目 标 ” ) , 又 已 知$\Pr( A) = 0. 8$,

$\therefore\quad \Pr(\overline{A})=1-0.8=0.2$

因此射手向一目标连续射击 3 次所有可能的结果可表示为:
\begin{center}
\begin{tabular}{p{.6\textwidth}c}
    $\overline{A}_1\overline{A}_2\overline{A}_3$, & (${\rm C}_3^0$个)\\
   $A_{1} \overline{A}_{2} \overline{A}_{3}, \overline{A}_{1}A_{2} \overline{A}_{3}, \overline{A}_{1} \overline{A}_{2}A_{3}$,  & (${\rm C}_3^1$个)\\
   $A_{1}A_{2} \overline{A}_{3},A_{1} \overline{A}_{2}A_{3}, \overline{A}_{1} \overline{A}_{2}A_{3}$,
& (${\rm C}_3^2$个)\\
$A_1A_2A_3$ & (${\rm C}_3^3$个)
\end{tabular}
\end{center}
其中,$A_i\; (i=1,2,3)$表示第$i$ 次射击命中目标。

由于 3 次射击是独立进行的,即试验结果$A_1,A_2,A_3$是相互独立的,从而上面列举的每个可能结果的概率都可求出. 如
$$P(A_{1}\overline{A}_{2}A_{3})=P(A_{1})\cdot P( \overline{A}_{2})\cdot P(A_{3})=0.8\times0.2\times0.8=0.128$$

令$B=$“恰好命中目标两次”,

$\because\quad  P=0.8,\quad 1-P=0.2$

则$D=A_{1}A_{2}\overline{A}_{3}+A_{1}\overline{A}_{2}A_{3}+\overline{A}_{1}A_{2}A_{3}$
\[\begin{split}
    \therefore\quad P(B)&=P(A_{1}A_{2} \overline{A}_{3}+A_{1} \overline{A}_{2}A_{3}+ \overline{A}_{1}A_{2}A_{3})\\
    &=P(A_{1}A_{2} \overline{A}_{3})+P(A_{1} \overline{A}_{2}A_{3})+P( \overline{A}_{1}A_{2}A_{3})\\
    &=P^{2}(1-P)+P^{2}(1-P)+P^{2}(1-P)\\
    &={\rm C}_{3}^{2}P^{2}(1-P)\\
    &=3\times0.8^{2}\times0.2=0.384
\end{split}\]
\end{solution}

一般地,\textbf{如果在一次试验中某事件发生的概率是$P$,那么在$n$次独立重复试验中这个事件恰好发生$K$次的概率}
\begin{equation}
    P_n(K)={\rm C}^K_n P^K (1-P)^{n-K}\tag{5}
\end{equation}

\begin{example}
    某一批蚕豆种籽,如果一粒发芽的概率为80\%,播下5粒种籽,计算
\begin{enumerate}[(1)]
\item 其中恰有4粒发芽的概率;
\item 至少有4粒发芽的概率.
\end{enumerate}
\end{example}

\begin{analyze}
    播下5粒种籽,每一粒种籽只有两个结果:发芽或不发芽.所以本题属于独立重复试验,可用公式(5)解决.
\end{analyze}

\begin{solution}
\begin{enumerate}[(1)]
    \item 把“播下一粒种籽,发芽”记为事件A.播下5粒种籽相当于作5次独立重复试验,根据公式(5),播下5粒种籽恰有4粒发芽的概率是:
$$P_{5}(4)={\rm C}_{5}^{4}P^{4}(1-P)^{5-4}=5\times0.8^{4}\times0.2\approx0.41$$
答:播下5粒种籽恰有 4 粒发芽的概率约为 0.41.

\item 播下5粒种籽,至少有4粒发芽的概率,就是恰有 4
粒发芽的概率和 5 粒都发芽的概率之和.
\[\begin{split}
 P&=P_{5}(4)+P_{5}(5)\\
&\approx0.410+C_{5}^{5}\times0.8^{5}(1-0.8)^{5-5}\\
&\approx0.410+0.328\approx0.74
\end{split}\]
答:播下5粒种籽,至少有 4 粒发芽的概率约为 0.74
\end{enumerate}
\end{solution}

\section*{习题六}
\begin{center}
    \bfseries A
\end{center}

\begin{enumerate}
\item 生产一件产品,合格产品的概率是0.96,问生产这种产品
4件,其中恰有1件次品,恰有2件次品,至少有一件次品
的概率各是多少?
\item 4门大炮向同一目标各射击一次,每门炮击中目标的概率
都是 $P$, 试求恰有 3门大炮击中目标的概率。
\item 把一枚硬币连续地掷5次,试求5次中至多有两次正面朝
上的概率.
\item 某射手打靶,每次命中的概率为 0.7, 现在连续射击五次,
分别写出射手恰好击中5次,4次,3次,2次,1次,0次的概率的计算式子,并将它们与$(0.7+0.3)^{5}$的展开式的各项进行比较,你有什么结论?
\item 某气象站天气预报准确率为90\%, 计算
\begin{enumerate}
    \item 5次预报中恰有4次准确的概率;
    \item 5次预报中恰有2次不准确的概率.
\end{enumerate}
\end{enumerate}

\section{本章小结}
\subsection*{知识结构分析}
\begin{enumerate}
    \item 概念
\begin{center}
\begin{tikzpicture}[>=stealth, thick]
\node(A) at (-1,0)[right]{随机试验};
\node(A1) at (1,1)[right]{样本点};
\node(A2) at (3.5,1)[right]{随机事件};
\node(A3)[text width=.13\textwidth, align=center] at (6,1)[right]{随机事件\\的频率};
\node(A4)[text width=.13\textwidth, align=center] at (9,1)[right]{随机事件\\的概率};
\node(B1) at (1,-1)[right]{样本空间};
\node(B2)[text width=.5\textwidth] at (3,-1)[right]{随机事件的关系(包含关系;等价关系;\\事件的和,差,积,逆;互斥事件)};

\draw[->](A1)--(A2);
\draw[->](A2)--(A3);
\draw[->](A3)--(A4);

\draw[->](A2)--+(0,-1.5);

\draw[decorate, decoration={brace, amplitude=5pt}](1,-1)--(1,1);
\end{tikzpicture}
\end{center}

    \item 几个概率公式
\begin{enumerate}[(1)]
\item 古典概型:$\Pr(A)=\frac{m}{n}$.(其中$m$为事件$A$包含的样本点数,$n$为样本点总数)
\item $A$,$B$为互斥事件时,$\Pr(A+B)=\Pr(A)+\Pr(B)$.
\item $A$,$B$为对立事件时,$\Pr(A)+\Pr(B)=1$,
或$\Pr(B)=1-\Pr(A)$.
\item $A$,$B$为相互独立事件时,$\Pr(A\cdot B)=\Pr(A)\Pr(B)$.
\item $n$次独立重复试验. $P_n(K)={\rm C}^K_n P^K (1-P)^{n-K}$

(其中$P$为在一次试验中某事件发生的概率,$P_n(K)$为在$n$次独立重复试验中这个事件恰好发生$K$次的概率.)
\end{enumerate}
\end{enumerate}

\subsection{几点说明}
\begin{enumerate}
\item 在解具体问题时,首先要根据题意判断问题属于哪个类型,选择适合的概率公式;
\item 解决问题的关键是把所求事件用已知概率的事件去
表示. 所以正确分析事件之间的关系是非常重要的一步.
\item 恰当地利用对立事件的概率公式,往往使计算大大简化.
\end{enumerate}


\section*{复习题十}
\begin{center}
    \bfseries A
\end{center}
\begin{enumerate}
\item 袋中有5个白球,6个黑球,7个红球,从中任取一球,求取出的球是白球或红球的概率.
\item 用数字$1,2,3,5,6$任意组成没有重复数字的五位数,求这数为偶数的概率.
\item 100支电子元件中有6支次品,从中任意抽取20支,求20支中有次品的概率.
\item 靶子由1—10环组成,某射手射击时,命中1—4环的概率是0.2,命中5—8环的概率为0.4,脱靶的概率是0,求他命中9—10环的概率.
\item 在试验中四个两两互斥事件出现的概率分别是0.01,0.011, 0.007, 0.002.求试验中这四个事件之一出现的概率.
\item 某军训团的学员实弹射击中,有一批学员中,每一学员一次射击命中目标的概率都是0.3,要找几个这样的学员,同时射击一次,就可以使击中目标的概率大于0.95?
\item 某工人照看三台机器,如果在十分钟内,机器不需要工人照看的概率,第一台是0.9,第二台是0.8,第三台是0.7,假设各个机器是否需要照着相互之间没有影响.计算在这十分钟内,至少有一台机器需要工人照看的概率.
\item 在一副扑克牌(52张)中,有“黑桃,红桃,梅花,方块”这四种花色的牌各13张,从中任取4张,这4张牌的花色相同的概率是多少?这4张牌的花色没有任何两张相同的概率是多少?
\end{enumerate}

\begin{center}
    \bfseries B
\end{center}

\begin{enumerate}
   \setcounter{enumi}{8} 
\item 甲袋有10个白球5个红球,乙袋有5个白球,10个红球,从两个袋子内各任意摸出一个,
\begin{enumerate}[(1)]
    \item 得到一个白球,一个红球的概率是多少?
    \item 得到两个白球的概率是多少?
\end{enumerate}
\item 八把钥匙中有一把可以开锁,从中任取三把,求可以打开锁的概率.
\item 某计算中心备有4部计算机,在一天内每部机器被使用的概率为0.4.求在一天内至少有三部机器被使用的概率.
\item 将10人任意分成两组,每组5人,求甲、乙两人正好分在同一组的概率.
\item 甲、乙两个篮球运动员在罚球线投球的命中率分别是0.7和0.6,
\begin{enumerate}[(1)]
    \item 现在两人各投一次,求至少一人命中的概率.
    \item 每人投球3次,计算两人都恰好投进两球的概率.
\end{enumerate}
\item 如图,将6个相同的元件先两两串联成3组,再把这3组并联成一个系统,已知每个元件损坏的概率为$P$,且各个
元件损坏与否相互之间没有影响,求这个系统能正常工作的概率.   
\end{enumerate}

\begin{figure}[htp]
    \centering
\begin{tikzpicture}
\draw(0,0)--(1,0)--(5,0);
\draw(1,-1) rectangle (4,1);
\draw[fill=white](1.5,-1.1) rectangle (2,-.9);    
\draw[fill=white](2.5,-1.1) rectangle (3,-.9);    


\end{tikzpicture}
    \caption*{(第14题)}
\end{figure}

































